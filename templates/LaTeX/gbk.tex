% latex-chinese-template v0.1   http://ctex-kit.googlecode.com/
%
% See comments at the end of this file for more descriptions.

% Change cjkfonts to winfonts to use Microsoft fonts such as simsun, simkai etc.
% See http://tug.ctan.org/tex-archive/language/chinese/ctex/doc/
%  or http://ctex-kit.googlecode.com/svn/trunk/ctex/doc/
%  for details about options of ctex macro package.

\documentclass[GBK,cjkfonts,hyperref]{ctexart}
\ifxetex
  \XeTeXinputencoding "GBK"
\fi

\begin{document}

\title{\LaTeX~�����ĵ�ģ��}
\author{����}

\maketitle
\tableofcontents

\section{ǰ��}

���F������ã�CTeX~�ĵ��ࡣ

\section{����}

������ã�

\section{���}

�dz��ã�

\end{document}

%%%%%%%%%%%%%%%%%%% END %%%%%%%%%%%%%%%%%%%%%%%%%%%%%%%%%%%%%%%%%%%%%%%%

% How to compile:
%   install TeXLive 2009 as described below;
%   put Adobe fonts into ~/.fonts (Linux) or C:\WINDOWS\fonts (Windows)
%   run command sequences:
%       xelatex a.tex; xelatex a.tex
%           or
%       pdflatex a.tex; pdflatex a.tex
%           or
%       latex a.tex ; latex a.tex; dvipdfmx a.dvi
%           or (requires ghostscript)
%       latex a.tex ; latex a.tex; dvips a.dvi; ps2pdf a.ps
%
% If you use 'winfonts' or 'adobefonts' option with latex/pdflatex, copy
% sim*.ttf and simsun.ttc into $TEXMFHOME/fonts/truetype/sim/ directory.
% Run `kpsexpand '$TEXMFHOME'` to get the value of $TEXMFHOME.
%
% If you use 'winfonts' option with xelatex on Linux, copy sim*.ttf and
% simsun.ttc into ~/.fonts .
%
% See documents of ctex, zhmetrics, fontspec and xetex to learn how to
% use new fonts.

%%%%%%%%%%%%%%%%%%% ENVIRONMENT %%%%%%%%%%%%%%%%%%%%%%%%%%%%%%%%%%%%%%%%

% Tested with TeXLive 20090909 on Debian Squeeze:
%  xetex        r14602  3.1415926-2.2-0.9995.2
%  pdftex       r14549  3.1415926-1.40.10-2.2
%  ctex         r14619  0.93
%  zhmetrics    r14469
%  cjk          r15155  4.8.2
%  xecjk        r14567  2.3.5
%  arphic       r13822
%
% Way to install TeXLive 2009:
%  install-tl -repository http://ftp.ctex.org/mirrors/texlive/tlnet/  -profile texlive.profile
%
% Content of texlive.profile(without the leading "%" and white spaces):
%   # texlive.profile written on Fri Sep  4 07:00:20 2009 UTC
%   # It will NOT be updated and reflects only the
%   # installation profile at installation time.
%   selected_scheme scheme-basic
%   TEXDIR ~/texlive/2009
%   TEXDIRW ~/texlive/2009
%   TEXMFHOME ~/texmf
%   TEXMFLOCAL ~/texlive/texmf-local
%   TEXMFSYSCONFIG ~/texlive/2009/texmf-config
%   TEXMFSYSVAR ~/texlive/2009/texmf-var
%   binary_i386-linux 1
%   collection-basic 1
%   collection-fontsrecommended 1
%   collection-langcjk 1
%   collection-langlatin 1
%   collection-latex 1
%   collection-latexrecommended 1
%   collection-xetex 1
%   from_dvd 0
%   option_desktop_integration 1
%   option_doc 0
%   option_file_assocs 1
%   option_fmt 1
%   option_letter 0
%   option_path 0
%   option_post_code 1
%   option_src 0
%   option_sys_bin /usr/local/bin
%   option_sys_info /usr/local/info
%   option_sys_man /usr/local/man
%   option_w32_multi_user 1
%   option_write18_restricted 1

% View the generated pdf on Debian Squeeze(20090909) with:
%  xpdf (3.02-1.4+lenny1) + xpdf-chinese-simplified (20040727-1, depends cmap-adobe-gb1 (0+20051207-5))
%  evince (2.26.2-2) + poppler-data (0.2.1-5)
% !!! Remember installing poppler-data because it contains data about cmap-adobe-*.

%
% Hope these information useful for TeX newbies :-)
%   Liu Yubao <yubao.liu at gmail dot com>
%
% ChangeLog:
%   2009-09-11  Liu Yubao
%       * initial version, v0.1

