% Hist:	070824	First document out. All but Implementation written.
%		In the evening fixed some typos.
\documentclass[12pt]{ltxdoc}
%\usepackage[margin=1.5in]{geometry}
\usepackage{fontspec}
\defaultfontfeatures{Mapping=tex-text}
\usepackage[no-sscript]{xltxtra}
\usepackage{zhspacing}
\usepackage[fakebold]{zhfont}
\usepackage[noactive]{zhmath}
\usepackage[dvipdfmx]{hyperref}
\newfontfamily\zhcjkextafont{Sun-ExtA}
\newfontfamily\zhcjkextbfont{Sun-ExtB}
\zhspacing
\usepackage{fancyvrb}
\DefineVerbatimEnvironment{verbatim}{Verbatim}{baselinestretch=0.9}%,fontsize=\footnotesize}
\usepackage{pst-all}
\setlength\parskip{0.5em plus 0.5em minus 0.5em}
\newcommand\q[1]{``\hbox spread -0.2em{#1}''}
\newcommand\name[1]{\hbox{#1}}
\newcommand\fntname[1]{\texttt{#1}}
\newcommand\NFSS{\textsf{NFSS}}
\newcommand\mymacro[1]{\ifvmode\noindent\else\setbox0=\lastbox\fi\llap{$\bullet$}\cmd#1\\\indent}
\makeatletter
\@addtoreset{section}{part}
\makeatother
\begin{document}
\title{Typesetting Chinese in \XeTeX: |zhspacing| user's
manual\footnote{Revision: \zhspacingrevision.}}
\author{\textsc{Yin} Dian\footnote{Lab.\ of Network Comm. System \&
Control, Univ.\ of Sci.\ \& Tech.\ of China.}}
\date{2007.12.11.}
\maketitle
\tableofcontents
\newpage

\part{zhspacing.sty}
\section{Introduction}
\XeTeX\ is a new \TeX\ engine by Jonathan Kew and SIL International, which combines
\eTeX\ with pervasive Unicode support and advanced font support\footnote{Words
copied from the ConTeXt Wiki.}. Using \XeTeX, it is possible to typeset scripts
of any languages whose glyphs are contained in the font in use, without the
help of any extra packages, such as |CJK|\footnote{|CJK| is a LaTeX 2e macro
package which enables the use of CJK scripts (Chinese/Japanese/Korean) in
various encodings, written by Werner Lemberg.}.

However, \XeTeX\ itself does not solve all the problems. Some details in
Chinese typesetting have not been dealt with by \XeTeX, such as the automatic
font switch between Chinese and Western characters, the skip adjustment of
fullwidth punctuations, the automatic skip insertion between Chinese and
Western characters or math formulas, etc.

To meet the need of easy and high quality typesetting of Chinese documents
using \XeTeX, the package |zhspacing| is coded, utilizing the new primitive of
\cs{XeTeXinterchartoks} provided in \XeTeX\ version 0.997.

\section{Getting Started}
|zhspacing| can be used in both plain \XeTeX\ or \XeLaTeX. The usage is quite
straightforward. For instance, in plain \XeTeX, 
{\emptyskipscheme
\begin{verbatim}
	\input zhspacing.sty
	\zhspacing
	这是中文测 试。中文和English的混排。中
	文和$E=mc^2$的混排。
	\bye
\end{verbatim}
}
And in \XeLaTeX,
{\emptyskipscheme
\begin{verbatim}
	\documentclass{article}
	\usepackage{zhspacing}
	\zhspacing
	\begin{document}
	这是中文测 试。中文和English的混排。中
	文和$E=mc^2$的混排。
	\end{document}
\end{verbatim}
}
Both of the examples above produce the same result as follows,
\begin{center}
这是中文测 试。中文和English的混排。中
文和$E=mc^2$的混排。
\end{center}

As you can see, spaces after Chinese characters are always ignored. Moreover, a
noticable skip is inserted between Chinese characters and English characters as
well as math formulas. In fact, all of the following inputs can produce mixed
langauge output with skip automatically inserted between Chinese and English
characters,
\begin{center}
\emptyskipscheme |中Eng文|,\quad |中 Eng文|,\quad |中Eng 文|\quad and\quad |中 Eng 文|. 
\end{center}
And their corresponding output is,
\begin{center}
中Eng文,\quad 中 Eng文,\quad 中Eng 文\quad and\quad 中 Eng 文. 
\end{center}

Look close and you'll find, that the first and second input generates exactly
the same output, and so does the third and fourth. However, the skip between
|Eng| and |文| in the last two cases is wider than the skip between |中| and
|Eng|. That is because the space is produced by the space token after the
letter |g|, not the skip automatically inserted by |zhspacing|'s skip
mechanism.

The space ignoring and skip auto-inserting ability of |zhspacing| can make
writing documents much comfortable. You don't need to worry about the manual
insertion of skips such as adding the annoying |~| in |CJK|.

Now come to the topic of punctuation skip adjustment. Proper Chinese
typesetting requires consecutive fullwidth punctuations be compressed, and a
linebreak before or after a fullwidth punctuation will cut off the blank spaces
of this punctuation, making it align to the margin. |zhspacing| does have
solved these problems, as well as proper prohibitions(禁则). Here's an example.
\begin{center}
\begin{minipage}{21em}
他强调,“三个代表”重要思想是在新的历史条件下运用马克思主义的立场、观点和方法的
典范,是我们学习马克思主义的立场、观点和方法最现实、最生动的教材。“三个代表”
重要思想是与时俱进的理论。
\end{minipage}
\end{center}

\section{Advanced Usage}
\subsection{Fonts}
|zhspacing| uses an extensible way of selecting fonts. The rules can be
summarized as follows,
\begin{itemize}
	\item Western characters, i.e.\ those that are not CJKV ideographies nor
		CJKV punctuations, use the same font as default.
	\item Chinese characters use seperate fonts. Font changes in the
		document does not affect the font used to display Chinese,
		unless you are using \NFSS\ and change the font series or shape.
	\item On displaying basic Chinese ideographies, execute the command
		\cs{zhfont}.
	\item On displaying Chinese punctuations, execute the command
		\cs{zhpunctfont}.
	\item On displaying CJK Ext-A characters, execute the command
		\cs{zhcjkextafont}.
	\item On displaying CJK Ext-B characters, execute the command
		\cs{zhcjkextbfont}.
	\item When switching from non-Chinese characters to Chinese characters,
		execute \cs{zhs@savefont}. When switching back, execute
		\cs{zhs@restorefont}.
\end{itemize}

When importing |zhspacing| from plain \XeTeX, the default definitions of the
above commands are,
\begin{verbatim}
	\font\zhfont="SimSun" at 10pt
	\font\zhpunctfont="SimSun" at 10pt
	\def\zhcjkextafont{\message{CJK Ext-A}}
	\def\zhcjkextbfont{\message{CJK Ext-B}}
	\let\zhs@savefont=\begingroup
	\let\zhs@restorefont=\endgroup
\end{verbatim}

When imported from \XeLaTeX, the defaults are,
\begin{verbatim}
	\newfontfamily\zhfont[BoldFont=SimHei]{SimSun}
	\newfontfamily\zhpunctfont{SimSun}
	\def\zhcjkextafont{\message{CJK Ext-A}}
	\def\zhcjkextbfont{\message{CJK Ext-B}}
	\def\zhs@savefont{\zhs@savef@nt{old}}
	\def\zhs@restorefont{\zhs@restoref@nt{old}}
\end{verbatim}
Here \cs{zhs@savef@nt} and \cs{zhs@restoref@nt} are internal macros to save and
restore the \NFSS\ info of the current font to specified places.

CJK Ext-A/B fonts have not been defined by default in consideration that not
every user has installed the particular fonts. I recommend to use
\fntname{Sun-ExtA} and \fntname{Sun-ExtB} as the corresponding fonts. You can
define the ext-font macros manually in a similar way to the definition of
\cs{zhfont}.

An example of mixed typesetting of various classes of characters is as follows,
cited from \emph{The Tale of Kiều}\footnote{《金云翘传》,or \emph{Truyện Kiều},
a Vietnamese \emph{lục bát} poem written in \emph{chữ Nôm}. \emph{Chữ Nôm} is
a system of ideographies created by the Vietnamese laboring people, most of
which are encoded in CJK Ext-A\hskip -0.2em/B area, or not even included in
Unicode.}.
\begin{center}
\parbox{21em}{
\fontspec{Times New Roman}
𤾓𢆥𥪝𡎝𠊛些
Trăm năm, trong cõi người ta,
𡨸才𡨸命𡑛罗恄𠑬
Chữ tài, chữ mệnh, khéo là ghét nhau.
𣦆戈𠬠局𣷭橷
Trải qua một cuộc bể dâu,
仍條𥉩𧡊𦓡𤴬疸𢚸
Những điều trông thấy mà đau đớn lòng.
\footnotemark
}
\end{center}
\footnotetext{Some of the ideographies are replaced by similar characters due
to the limitation of character set. A picture of the original manuscript can be found at
\href{http://bbs.gophor.cn/cjkv/attachment/14_2_b44e05d28a68271.jpg}{\emph{this
link}}.}% or Fig.~\ref{fig:kieu}.}
\subsection{More on Fonts}
Sometimes it may be desirable to use different Chinese fonts for different
\NFSS\ font families in \LaTeX. This can be done by changing the definition of
\cs{zhs@savefont} and \cs{zhfont}, without modifying the |zhspacing| source. 

Here is the way to do it.\footnote{See zhfont.sty later for convenient ways.}
\begin{verbatim}
\makeatletter
\g@addto@macro\zhs@savefont{%
  \long\edef\zhs@tmpmacro{\f@family}%
  \def\zhs@curr@fam{0}%
  \ifx\zhs@tmpmacro\sfdefault
    \def\zhs@curr@fam{1}%
  \else\ifx\zhs@tmpmacro\ttdefault
    \def\zhs@curr@fam{2}%
  \fi\fi
  \edef\zhs@tmpmacro{\f@family}%
  \ifx\zhs@tmpmacro\sfdefault
    \def\zhs@curr@fam{1}%
  \else\ifx\zhs@tmpmacro\ttdefault
    \def\zhs@curr@fam{2}%
  \fi\fi
}
\newfontfamily\zhrmfont[BoldFont=SimHei,
  ItalicFont=KaiTi_GB2312]{SimSun}
\newfontfamily\zhsffont{SimHei}
\newfontfamily\zhttfont[BoldFont=SimHei]{KaiTi_GB2312}
\def\zhfont{\ifcase\zhs@curr@fam\zhrmfont\or\zhsffont
  \or\zhttfont\else\zhrmfont\fi}
\end{verbatim}
\subsection{Skips}
The skip mechanism in |zhspacing| is also flexible. |zhspacing| utilizes the
following skip commands. Note that they are defined as macros instead of skip
register values, in order that the skip varies according to font size changes.
\begin{description}
	\item[\mdseries\cs{skipzh}]
		Skip between adjacent Chinese characters.
	\item[\mdseries\cs{skipenzh}]
		Skip between a Chinese character and a Western character or a
		math formula.
	\item[\mdseries\cs{skipzhopen}]
		Skip before fullwidth opening punctuations, such as \q{“},
		\q{(}, \q{《}, etc.
	\item[\mdseries\cs{skipzhinteropen}]
		Skip before a fullwidth opening punctuation when preceded by
		another fullwidth punctuation.
	\item[\mdseries\cs{skipzhlinestartopen}]
		Skip before a fullwidth opening punctuation when it occurs at
		the start of a line.
	\item[\mdseries\cs{skipzhclose}]
		Skip after fullwidth closing punctuations, such as \q{”},
		\q{)}, \q{》}, etc.
	\item[\mdseries\cs{skipzhinterclose}]
		Skip after a fullwidth closing punctuation when followed by
		another fullwidth punctuation.
	\item[\mdseries\cs{skipzhlineendclose}]
		Skip after a fullwidth closing punctuation when it occurs at
		the end of a line.
	\item[\mdseries\cs{skipzhfullstop}]
		Skip after fullwidth fullstop punctuations, such as \q{;},
		\q{.}, \q{。}, etc.
	\item[\mdseries\cs{skipzhinterfullstop}]
		Skip after a fullwidth fullstop punctuation when followed by
		another fullwidth punctuation.
	\item[\mdseries\cs{skipzhlineendfullstop}]
		Skip after a fullwidth fullstop punctuation when it occurs at
		the end of a line.
	\item[\mdseries\cs{skipzhhalfstop}]
		Skip after fullwidth halfstop punctuations, such as \q{、},
		\q{,}, \q{:}, etc.
	\item[\mdseries\cs{skipzhinterhalfstop}]
		Skip after a fullwidth halfstop punctuation when followed by
		another fullwidth punctuation.
	\item[\mdseries\cs{skipzhlineendhalfstop}]
		Skip after a fullwidth halfstop punctuation when it occurs at
		the end of a line.
	\item[\mdseries\cs{skipnegzhlinestartopen}]
		Negative skip to \cs{skipzhlinestartopen}.
	\item[\mdseries\cs{skipnegzhlineendclose}]
		Negative skip to \cs{skipzhlineendclose}.
	\item[\mdseries\cs{skipnegzhlineendfullstop}]
		Negative skip to \cs{skipzhlineendfullstop}.
	\item[\mdseries\cs{skipnegzhlineendhalfstop}]
		Negative skip to \cs{skipzhlineendhalfstop}.
\end{description}

All of the skip commands above are defined in the pattern \\
\vadjust{}\hfil |\def\skipxxx{\hskip xxxxx}|.

|zhspacing| comes with three pre-defined skip schemes, namely
\cs{simsunskipscheme}, \cs{emptyskipscheme} and \cs{haltskipscheme}. The 
first scheme should be suitable for font \fntname{SimSun} and other popular
Chinese fonts used in China, which does not support OpenType features of
|halt|, and needs negative spaces be inserted before opening punctuations and
after closing or judou punctuations. The second scheme simply addes zero
length. And the last one should be fit for OpenType Chinese fonts supporting
|halt| feature such as Adobe Song Std, where positive spaces should be inserted
before or after certain punctuations. You can define your own skip schemes for
customization, of course.
\subsection{Vertical Chinese}
Vertical Chinese can be achieved by adding raw feature |vertical| to the
specified font. An example could be, \hfil
\setbox0=\hbox to 0pt{\vbox{\noindent
\newfontfamily\zhfont[RawFeature={vertical:},BoldFont=SimHei]{SimSun}
\newfontfamily\zhpunctfont[RawFeature={vertical:+vert:+vhal}]{[AdobeSongStd-Light.otf]}
\haltskipscheme
\rotatebox{-90}{\fbox{\raise 0.8em\hbox{我是中国人,我爱自己的祖国。}}}
}\hss}
\dp0=0pt \raise 0.8em\box0

\begin{verbatim}
  \documentclass[12pt]{article}
  \usepackage[dvipdfm]{graphicx}
  \usepackage{zhspacing}
  \zhspacing
  \begin{document}
  \newfontfamily\zhfont[RawFeature={vertical:}]{SimSun}
  \newfontfamily\zhpunctfont[RawFeature={vertical:
              +vert:+vhal}]{[AdobeSongStd-Light.otf]}
  \haltskipscheme
  \rotatebox{-90}{我是中国人,我爱自己的祖国。}
  \end{document}
\end{verbatim}

Note that in this example, in order to have proper vertical punctuations, we
set \cs{zhpunctfont} to use the Adobe one supporting |vert| feature, and change
the skip scheme to \cs{haltskipscheme} to match the |vhal| feature specified.

However, there exists some bug in typesetting vertical Chinese containing
punctuations using some fonts such as \fntname{KaiTi\_GB2312} etc. I've told
this bug to \name{jjgod} in May, but it seems that he hasn't solved the problem
yet.  Moreover, the baseline of vertical Chinese is not correct, so mixed
typesetting of Chinese and English in vertical mode generates ugly results, and
thus should be avoided. 
\subsection{Compatibility}
Theoretically, |zhspacing| should be compatible with all macro packages, except
those who change the definition of \cs{hskip} and \cs{penalty}, in which case
special treatment should be applied. I haven't found any conflict when using
common packages such as |hyperref| and |fancyhdr|. However, |ulem| redefineds
\cs{hskip} and \cs{penalty}, and causes unexpected output\footnote{I was using
\cs{begingroup} and \cs{endgroup} as the font saving/restoring commands.
However, I found |ulem| redefines \cs{hskip} and \cs{penalty} to end the
current hbox group and start a new hbox, and thus group mismatch occurs. So I
changed the font saving/restoring commands when used in \LaTeX, using \NFSS\ info
instead of groups. But in the new hbox group, no Chinese font is applied by
default so the Chinese characters disappears. Therefore I changed the
definitions of |ulem|'s new \cs{hskip} and \cs{penalty} in |zhulem|, and advice
users to use this modified one.}. Use |zhulem| provided along with
|zhspacing| instead.

\subsection{Working with the \texttt{ctex} package}
The |ctex| package, made by \href{http://www.ctex.org}{|ctex.org|}, is a widely
used Chinese document framework which hides the underlying Chinese processing
details to the user, which enables portability between different Chinese
processing packages or systems, such as |CCT| and |CJK|. \name{Jjgod} has made
a \XeTeX\ port of the |ctex| package, named |cxetex|, which can be found at
\url{http://www.newsmth.net/att.php?s.460.257121.452.bz2}.

To use |zhspacing| in |cxetex|, a slight modification of the file |ctexcjk.clo|
should be made. Change the corresponding lines to the followes (\% means the
original lines).
\begin{verbatim}
%\XeTeXlinebreaklocale "zh"
%\XeTeXlinebreakskip = 0pt plus 1pt minus 0.1pt
%\setmainfont[BoldFont={cwTeXHeiBold},
%             ItalicFont={cwTeXKaiItalic},
%             Mapping=tex-text]{cwTeXMing}
%\setsansfont[BoldFont={cwTeXHeiBold},Mapping=tex-text]{cwTeXYen}
%\setmonofont{cwTeXFangSongTT}
\usepackage{zhspacing}
\newfontfamily\zhfont[BoldFont=SimHei]{SimSun}
\newfontfamily\zhcjkextafont{Sun-ExtA}
\newfontfamily\zhcjkextbfont{Sun-ExtB}
\zhspacing
\end{verbatim}

You can surely change these lines to your own taste.

\section{Getting \texttt{zhspacing}}
|zhspacing| is hosted on an open-source project at googlecode. The project
homepage is \url{http://code.google.com/p/zhspacing/}. You can get the lastest
source from the download page or via SVN.

\goodbreak
\part{zhfont.sty}
\section{Introduction}
To simplify the complex font setting to use \NFSS\ in \LaTeX, the package
|zhfont| is coded. It also adds simple interface to define fonts with fake
slant, and also supports fake bold.
\section{Package Options}
|zhfont| currently has only one option --- |fakebold|. This option sets
Chinese show fake boldface when the current series is bold.
\section{Provided Macros}
\mymacro\zhrmfont
\mymacro\zhsffont
\mymacro\zhttfont
The font commands to be executed according to the current font family.
You can redefine them to customize your fonts.

\mymacro\setzhmainfont
\mymacro\setzhsansfont
\mymacro\setzhmonofont
Simply macro to set \cs{zh\{rm/sf/tt\}font} using \cs{newfontfamily}.

\mymacro\newfontfamilywithslant
Create a font family with fake slant. Take one argument as the font name. 

Example: |\newfontfamilywithslant\zhrmfont{SimSun}|

\mymacro\newfontfamilywithslantandbold
Create a font family with fake slant, and specify another font for its
boldface. Take two arguments as the font name and bold font name. 

Example:\\\quad |\newfontfamilywithslantandbold\zhrmfont{SimSun}{SimHei}|
\section{Examples}
The following example is generated with |zhfont| loaded with option
|fakebold|.

{\emptyskipscheme
\begin{verbatim}
\newfontfamilywithslant\zhrmfont{SimSun}
中文\textbf{测试}。\textit{中文\textbf{测试}。}
\newfontfamilywithslant\zhrmfont{FangSong_GB2312}
中文\textbf{测试}。\textit{中文\textbf{测试}。}
\end{verbatim}
}
\nobreak
{
\newfontfamilywithslant\zhrmfont{SimSun}
中文\textbf{测试}。\textit{中文\textbf{测试}。}
\newfontfamilywithslant\zhrmfont{FangSong_GB2312}
中文\textbf{测试}。\textit{中文\textbf{测试}。}
}
\goodbreak

\part{zhmath.sty}
\section{Introduction}
|zhmath| is a generic package which allows the Chinese characters to be
displayed in math formulas. It also allows changing the math font for alphas
and numbers.

|zhmath| only sets the basic Chinese characters to be displayable, not CJK
Ext-A/B ones, as they are seldom used and discouraged in math. You can use
\cs{hbox} to let |zhspacing| display them, however.
\section{Package Options}
In \LaTeX, |zhmath| has two options --- |active| and |noactive|. They set the
method to implement Chinese character displaying in math. The default is
|noactive|.

When |active| is set, the Chinese characters in math formulas are set to
active, which, when executed, display the corresponding characters in
\cs{zhmathfont}. This behavior is much like that of the package |mathcjk|. The
font size in different styles are also set in \cs{mathcjksizea}/|b|/|c|/|d|.

When |noactive| is set, the mathcodes of Chinese characters are set to use a
special math font family to display them. This way is more native than to use
active characters.

Both |active| and |noactive| have disadvantages. In |active| mode the Chinese
math font sizes are determined by \cs{mathcjksizea}/|b|/|c|/|d|, so they are
somewhat independent with the main text size, which means their size cannot
change when the main font size changes, which gives ugly results. However, in
|noactive| mode, although Chinese math font size varies with the main font
size, Chinese inside \cs{mathrm} etc cannot be displayed, because the math
font family are set to \cs{mathrm}'s. So choose the option to your own need.

In plain \TeX\ no package options are available.
\section{Provided Macros}
\mymacro\setzhmathfont
Set the math font used to display Chinese. Take one argument as the font
command. Preamble only.

\mymacro\zhmathfont
The font command which determines the math font for Chinese. However, if you
use |noactive| option in \LaTeX, or you're in plain \TeX, they are not
executed when Chinese in math occur, but when \cs{setzhmathfont} is invoked.
So you should always call |\setzhmathfont\zhmathfont| after the change.

\mymacro\usecustommathfonts
Enable the use of custom math fonts for alphas and numbers. The default is
Times New Roman.

\mymacro\setalphanummathfonts
Set the math font used to display alphas and numbers. Preamble only. Take one
argument as the font family in \LaTeX, and two argument as the font command
for alphas and numbers accordingly in plain \TeX.

\mymacro\zhmathalphanumfont
The font command which determines the math font for alphas and numbers in
\LaTeX. Should always call |\setalphanummathfonts\zhmathalphanumfont| after
it changes.

\mymacro\zhmathletterfont
\mymacro\zhmathnumberfont
The font command which determines the math font for alphas and numbers in plain
\TeX. After their change \cs{setalphanummathfonts} with corresponding
arguments should be called.
\section{Examples}
{\emptyskipscheme
\begin{verbatim}
{
$中文 in math 测_试$
\usecustommathfonts
\Large $能量E = mc^2$, $s_总 = {1 \over 2} at^2$
}
\end{verbatim}
}
\nobreak
{
$中文 in math 测_试$
\usecustommathfonts
\Large $能量E = mc^2$, $s_总 = {1 \over 2} at^2$
}
\goodbreak
\end{document}
