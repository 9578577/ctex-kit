%# -*- coding: utf-8 -*-
% ctex.tex: manual of ctex package

\documentclass{ltxdoc}
\usepackage[fntef,hyperref,UTF8]{ctexcap}
\hypersetup{pdfstartview=FitH,bookmarksnumbered}
\usepackage{texnames}
 \topmargin 0.5 true cm
 \oddsidemargin 1 true cm
 \evensidemargin 1 true cm
 \textheight 21 true cm
 \textwidth 14 true cm

\MakeShortVerb{\|}
\setcounter{StandardModuleDepth}{1}

\def\email#1{\href{mailto:#1}{\path{#1}}}

\newcommand{\ctex}{\texttt{ctex}}
\newcommand{\ctexorg}{\texttt{ctex.org}}

\newcommand{\ctexdefault}{\emph{这个是 \ctex{} 宏包的缺省模式。}}

\setlength{\parskip}{0.75ex plus .2ex minus .5ex}
\linespread{1.2}

\makeatletter
\def\parg#1{\mbox{$\langle${\it #1\/}$\rangle$}}
\def\@smarg#1{{\tt\string{}\parg{#1}{\tt\string}}}
\def\@marg#1{{\tt\string{}{\rm #1}{\tt\string}}}
\def\marg{\@ifstar\@smarg\@marg}
\def\@soarg#1{{\tt[}\parg{#1}{\tt]}}
\def\@oarg#1{{\tt[}{\rm #1}{\tt]}}
\def\oarg{\@ifstar\@soarg\@oarg}
\makeatother

\begin{document}

\title{\bf \ctex{} 宏包说明}
\author{\it \ctexorg\thanks{\url{http://www.ctex.org}}}
\date{\small 版本号:v1.02c \hskip 2\ccwd 修改日期:2011/03/11}
\maketitle


\begin{abstract}
\ctex{} 宏包提供了一个统一的中文 \LaTeX{} 文档框架,底层支持 CCT、CJK 和
 xeCJK 三种中文 \LaTeX{} 系统。\ctex{} 宏包提供了编写中文 \LaTeX{} 文档
常用的一些宏定义和命令。

\ctex{} 宏包需要 CCT 系统或者 CJK 宏包或者 xeCJK 宏包的支持。主要文件包括
 \texttt{ctexart.cls}、\texttt{ctexrep.cls}、\texttt{ctexbook.cls} 和
 \texttt{ctex.sty}、\texttt{ctexcap.sty}。

\ctex{} 宏包由 \ctexorg{} 制作并负责维护。
\end{abstract}

\tableofcontents

\section{简介}

这个宏包的部分原始代码来自于由王磊编写 \texttt{cjkbook.cls} 文档类,
还有一小部分原始代码来自于吴凌云编写的 \texttt{GB.cap} 文件。
原来的这些工作都是零零碎碎编写的,没有认真、系统的设计,
也没有用户文档,非常不利于维护和改进。2003 年,吴凌云用 \texttt{doc}
和 \texttt{docstrip} 工具重新编写了整个文档,并增加了许多新的功能。
2007 年,oseen 和王越在 \ctex{} 宏包基础上增加了对 UTF-8 编码的支持,
开发出了 \texttt{ctexutf8} 宏包。
2009 年 5 月,我们在 Google Code 建立了 ctex-kit 项目\footnote{\url{http://code.google.com/p/ctex-kit/}},
对 \ctex{} 宏包及相关宏包和脚本进行了整合,并加入了对 Xe\TeX{} 的支持。
该项目由 \ctexorg{} 社区的开发者共同维护,新版本号为 v0.9。
在开发新版本时,考虑到合作开发和调试的方便,
我们不再使用 \texttt{doc} 和 \texttt{docstrip} 工具,改为直接编写宏包文件。

最初 Knuth 设计开发 \TeX{} 的时候没有考虑到支持多国语言,
特别是多字节的中日韩语言。这使得 \TeX{} 以至后来的
\LaTeX{} 对中文的支持一直不是很好。即使在 CJK 解决了中文字符
处理的问题以后,中文用户使用 \LaTeX{} 仍然要面对许多困难。
最常见的就是中文化的标题。由于中文习惯和西方语言的不同,
使得很难直接使用原有的标题结构来表示中文标题。因此需要对
标准 \LaTeX{} 宏包做较大的修改。此外,还有诸如中文字号的对应
关系等等。\ctex{} 宏包正是尝试着解决这些问题。中间很多地方
用到了在 \ctexorg{} 论坛上的讨论结果,在此对参与讨论的
朋友们表示感谢。

\ctex{} 宏包由五个主要文件构成:\texttt{ctexart.cls}、\texttt{ctexrep.cls}、\texttt{ctexbook.cls}
和 \texttt{ctex.sty}、\texttt{ctexcap.sty}。\texttt{ctex.sty} 主要是提供整合的中文环境,可以配合大多数文档类使用。而
\texttt{ctexcap.sty} 则是在 \texttt{ctex.sty} 的基础上对 \LaTeX{} 的三个标准文档类的格式进行修改以符合中文习惯,该宏包只能配合这三个
标准文档类使用。\texttt{ctexart.cls}、\texttt{ctexrep.cls}、\texttt{ctexbook.cls}
则是 \texttt{ctex.sty}、\texttt{ctexcap.sty} 分别和三个标准文档类
结合产生的新文档类,除了包含 \texttt{ctex.sty}、\texttt{ctexcap.sty}
的所有功能,还加入了一些修改文档类缺省设置的内容(如使用五号字体为缺
省字体)。

\vskip 10pt
\emph{这份说明文档可以通过用 Xe\LaTeX{} 编译 \texttt{ctex.tex} 文件来得到。
编译说明文档需要先安装 \ctex{} 宏包。}


\section{使用帮助}

\ctex{} 宏包的使用十分简单。如果是使用 \ctex{} 的文档类,只需用
\texttt{ctexart}、\texttt{ctexrep} 或者 \texttt{ctexbook} 替换原来的
文档类就可以了。你也可以继续使用原来的文档类,而用 \texttt{ctex.sty} 或者
\texttt{ctexcap.sty} 宏包来配合使用,两者的效果是一样的
(除了不能修改一些文档设置如缺省字体大小)。

\subsection{使用 CJK 或 xeCJK}

这是 \ctex{} 宏包的缺省设置。\ctex{} 宏包会自动根据使用的 \TeX{} 引擎
调用 CJK 或者 xeCJK 宏包,你无需再自己调用。

此外,\ctex{} 宏包会在 |\begin{document}| 和 |\end{document}|
之间自动加入一个 CJK 环境,你无需再添加 CJK 环境。CJK 宏包的命令都可以
在 |\begin{document}| 和 |\end{document}| 之间正常使用。

例子1:使用文档类宏包
\begin{verbatim}
\documentclass{ctexart}
\begin{document}
中文宏包测试
\end{document}
\end{verbatim}

例子2:使用普通宏包
\begin{verbatim}
\documentclass{article}
\usepackage{ctex}
\begin{document}
中文宏包测试
\end{document}
\end{verbatim}

\subsection{使用 CCT}

注:CCT 方式\emph{不再建议}使用。

\ctex{} 宏包也可以配合新版的 CCT 使用,只需在使用 \ctex{} 宏包时加上 CCT 选项即可。
缺省 CCT 会使用 CJK 字库,因为这种字库方式比传统 CCT 字库更方便,兼容性也更好。
如果要使用传统 CCT 字库,则还要加上 CCTfont 选项。

例子3:使用 CJK 方式字库
\begin{verbatim}
\documentclass[CCT]{ctexart}
\begin{document}
中文宏包测试
\end{document}
\end{verbatim}

例子2:使用 CCT 方式字库
\begin{verbatim}
\documentclass[CCT,CCTfont]{ctexart}
\begin{document}
中文宏包测试
\end{document}
\end{verbatim}


\subsection{选项}

宏包的选项用于改变一些缺省风格的设置。缺省的设置已经针对中文
的习惯进行了尽量的修改,所以一般用户无需使用这些选项。
如果你觉得某些设置不合适,可以向作者反映。我们会考虑在后面的
版本中予以改进。我们也欢迎关于增加或者删减选项的建议。

\emph{除了 \ref{class-opts} 和 \ref{caption-opts} 的选项,
其余的选项都可以在所有文档类宏包和普通宏包上使用。}

\subsubsection{只能用于文档类的选项} \label{class-opts}

下面的选项可能会是最经常使用的。但是它们只能用于文档类
(\texttt{ctexart}、 \texttt{ctexrep} 和 \texttt{ctexbook})。
\begin{description}
\item[cs4size] 使用小四字号为缺省字体大小。
\item[c5size] 使用五号字为缺省字体大小。\ctexdefault
\end{description}

\subsubsection{只能用于文档类和 \texttt{ctexcap.sty} 的选项} \label{caption-opts}

下面这些则只可以在文档类宏包和 \texttt{ctexcap.sty} 上使用。
\begin{description}
\item[sub3section] 将 |\paragraph| 命令产生的标题改为 section 类格式。
此时 |\subparagraph| 命令产生的标题会具有原来 |\paragraph| 的格式。

\item[sub4section] 将 |\paragraph| 和 |\subparagraph| 命令产生的标题
都改为 section 类格式。
\end{description}

\subsubsection{中文编码选项}

下面的选项用于选择 ctex 宏包的内部编码。

\begin{description}
\item[GBK] 使用 GBK 编码。\ctexdefault

\item[UTF8] 使用 UTF-8 编码。
\end{description}

\emph{注意使用 Xe\TeX{} 引擎的情况下总是内部使用 UTF-8 编码,所以不必使用
这个选项,但这并不妨碍编写 GBK 编码的文档。}

\subsubsection{中文字库选项} \label{fontset}

下面的选项用于选择可用的中文字库。设置这些选项是考虑到不同的操作系统平台
提供的中文字库是不同的。不同的 \TeX{} 发行版可以根据目标操作系统平台和
提供的中文字库在 \texttt{ctexfonts.cfg} 文件中修改这些选项之一为缺省设置。
\begin{description}
\item[nofonts] 没有中文字库,此时没有中文字体命令可用。

\item[winfonts] 使用 Windows 的字体设置,默认为六种中易字体:宋体、仿宋、黑体、
楷体、隶书、幼圆(在使用 Xe\TeX{} 时只有前四种)。该选项的结果将和老版本
\ctex{} 宏包完全一致。这是默认设置。

\item[adobefonts] 在 xeCJK 模式中使用 Adobe 的四套字体:宋体、仿宋、黑体、楷体。
在 CJK 模式(即不使用 Xe\TeX{} 时)下,该选项将使用 winfonts 选项的设置。

\item[zhmap] 仅在 winfonts 模式下有效。使用 zhmetrics 宏包提供的字体文件映射,
将中文字库映射到相应的 ttf 字库文件。\ctexdefault

\item[nozhmap] 仅在 winfonts 模式下有效。使用系统提供的字体文件映射方式。
如果需要使用自定义的字体映射或者使用 Type1 字库,请使用该选项。
\end{description}

\subsubsection{CCT 引擎选项}

下面的选项用于选择底层的中文系统。缺省情况下,宏包会根据编译方式自动选择 CJK
或者 xeCJK 引擎。
\begin{description}
\item[CCT] 使用 CCT 代替 CJK 做为底层的中文支持系统。

\item[CCTfont] 使用传统的 CCT 字库方式,该选项会自动激活 CCT 选项。
\end{description}

\subsubsection{排版风格选项}

\begin{description}
\item[cap] 使用中文的标题样式,缺省格式由 \texttt{ctexcap.cfg} 配置文件
内的定义给出。对于 \texttt{ctex.sty},该选项只影响交叉引用中的数字和日期格式。
\ctexdefault

\item[nocap] 保留使用英文的标题样式。

\item[punct] 对中文标点的位置(宽度)进行调整。\ctexdefault

\item[nopunct] 不对中文标点的位置进行调整(每个标点占有相同的宽度)。

\item[space] 使用 CJK 的保留空格模式,保留中文字符间的空格(类似英文的
习惯)。你需要自己处理中文字符间的空格以及换行产生的空格(在行尾加上
 \% 符号可以避免),否则排版结果可能不符合中文习惯。这种模式可以通过
 |\CTEXnospace| 转换到 nospace 模式。

\item[nospace] 使用 CJK 的忽略空格模式,也就是 CJK* 环境的模式。
 CJK 会自动忽略中文字符间的空格,比较符合中文习惯。在这种模式下,
可以使用 \textasciitilde 来分隔中英文字符,产生的间距稍小于普通空格,
排版效果比较美观。这种模式可以通过 |\CTEXspace| 命令转换到 space 模式。
\ctexdefault

\item[indent] 使用中文的段首缩进模式,即缩进两个汉字宽度,同时每个段落
都缩进。\ctexdefault

\item[noindent] 使用原来的段首缩进模式,章节标题后的第一段不缩进。
\end{description}

\subsubsection{宏包兼容选项}

\begin{description}
\item[fancyhdr] 保持和 \texttt{fancyhdr} 宏包的兼容性。该选项将使得
 \texttt{fancyhdr} 宏包被自动调用。

\item[hyperref] 自动判断 \texttt{hyperref} 宏包的正确参数以避免产生乱码。如果在
  导言区用户没有自己调用 \texttt{hyperref},则该选项将使得 \texttt{hyperref} 宏
  包在导言区末尾被自动调用;如果需要对 \texttt{hyperref} 宏包做进一步的设置,则
  用户可以自己在 \texttt{ctex} 宏包后调用 \texttt{hyperref},并使用适当的选项或
  设置。

\item[fntef] 为 \texttt{CJKfntef} 宏包和 \texttt{CCTfntef} 宏包提供统一接口。
该选项将使得 \texttt{CJKfntef} 宏包或者 \texttt{CCTfntef} 宏包被自动调用。
\end{description}

\subsubsection{缺省选项}

总结一下:\ctex{} 宏包的缺省选项是 GBK zhmap nospace cap punct indent,\ctex{} 文档类
的缺省选项是 GBK zhmap nospace cap punct indent c5size。


\subsection{基本命令}

\ctex{} 宏包给用户提供一个通用的文档框架,使得用户可以自由地在不同的
底层中文系统间切换。为此,我们为 CJK 定制了一些模拟 CCT 的命令,
也对部分 CCT 命令进行了修改,使得两者保持一致。
此外,我们还定义了用于设置文档参数的高级设置命令。

\subsubsection{字体设置}

中文字体很多,但是常用的就那么几个。我们为 CJK 常用的六种中文
字体定义了简单易用的命令。它们是:

\DescribeMacro{\songti}
宋体: |\songti|, CJK 等价命令 |\CJKfamily{song}|

\DescribeMacro{\heiti}
黑体: |\heiti|, CJK 等价命令 |\CJKfamily{hei}|

\DescribeMacro{\fangsong}
仿宋: |\fangsong|, CJK 等价命令 |\CJKfamily{fs}|

\DescribeMacro{\kaishu}
楷书: |\kaishu|, CJK 等价命令 |\CJKfamily{kai}|

\DescribeMacro{\lishu}
隶书: |\lishu|, CJK 等价命令 |\CJKfamily{li}|

\DescribeMacro{\youyuan}
幼圆: |\youyuan|, CJK 等价命令 |\CJKfamily{you}|

\vskip 10pt
{\kaishu \TeX{} 系统中必须已经定义好这六种中文字体,并且使用和 \CTeX{} 套装中
一致的字体名称。(参见上面 CJK 等价命令的参数)

可用的字体命令还取决于使用的中文字库选项,参见 \ref{fontset} 一节的介绍。

上面的字体命令和 CCT 中的一致,但传统的 CCT 字库中没有隶书和
仿宋两种字体,需要用户自行安装定义。如果使用 CCT 时选择 CJK 字库方式,
则可以使用这两种中文字体。

上面的字体在不同的字体选项下有不同的设置,不一定都有定义。}

\subsubsection{字号、字距、字宽和缩进}

\DescribeMacro{\zihao}
中文字号的设置命令是 |\zihao|\marg*{字号},例如 |\zihao{3}|。
可以使用的参数有 16 个,小号字体在前面加负号表示,从大到小依次为
\begin{center}
\begin{tabular}{cccccccc}
\hline
初号 & 小初 & 一号 & 小一 & 二号 & 小二 & 三号 & 小三 \\
0 & -0 & 1 & -1 & 2 & -2 & 3 & -3 \\
\hline
四号 & 小四 & 五号 & 小五 & 六号 & 小六 & 七号 & 八号 \\
4 & -4 & 5 & -5 & 6 & -6 & 7 & 8 \\
\hline
\end{tabular}
\end{center}
\noindent 英文字体大小会始终保持和中文字体一致。

\DescribeMacro{\ziju}
汉字字距的调整使用命令 |\ziju|\marg*{字宽的倍数}。参数可以是任意的数字,
例如 |\ziju{5}| 设置汉字字距为当前汉字字宽的 5 倍, |\ziju{0.5}| 设置汉字
字距为当前汉字字宽的一半。这里的汉字字宽指的是实际汉字的宽度,
不包含当前字距。该命令不影响英文字距。

\DescribeMacro{\ccwd}
当前汉字的字宽保存在宏 |\ccwd| 中。字宽是相邻两个汉字中心的距离,
也就是说字距会被计算在内。

\DescribeMacro{\CTEXindent}
正常的缩进两个汉字字宽的距离,同时在汉字大小和字距改变的
情况都可以自动修改缩进距离。

\DescribeMacro{\CTEXnoindent}
取消缩进。

\DescribeMacro{\CTEXsetfont}
|\CTEXsetfont| 命令用于更新当前的中文字体信息,包括当前字距和缩进
距离。一般来说,用户无需使用这个命令。


\subsubsection{中文数字转换}

\DescribeMacro{\CTEXnumber}
使用 CJK 提供的 |\CJKnumber| 命令可以将阿拉伯数字转换为中文数字。
由于 \LaTeX{} 臭名昭著的脆弱命令的原因,当 |\CJKnumber| 被用在
章节标题等地方的时候,要么出现错误无法使用,要么无法达到预期目的,
例如在产生 PDF 书签的时候。于是我们定义了一个 |\CTEXnumber| 命令,
可以将产生的中文数字保存下来。该命令的格式为
\begin{quote}
|\CTEXnumber|\marg*{result}\marg*{number}
\end{quote}
其中 \parg{result} 必须是一个 \TeX{} 宏的名字,不需要预先定义。
例如
\begin{quote}
|\CTEXnumber{\test}{100002005}|
\end{quote}
则 |\test| 中的内容就是“一亿零二千零五”(不包括引号)。

\DescribeMacro{\CTEXdigits}
|\CTEXdigits| 命令和 |\CTEXnumber| 命令类似,用于代替 CJK 提供的
 |\CJKdigits| 命令。它和 |\CTEXnumber| 命令的不同之处在于转换后
结果是中文数字串,而不是按照中文习惯的数字。该命令的格式为
\begin{quote}
|\CTEXdigits|\marg*{result}\marg*{number}
\end{quote}
其中 \parg{result} 必须是一个 \TeX{} 宏的名字,不需要预先定义。
例如
\begin{quote}
|\CTEXdigits{\test}{100002005}|
\end{quote}
\CTEXdigits{\test}{100002005}
则 |\test| 中的内容就是“\test{}”(不包括引号)。

\DescribeMacro{\chinese}
对于经常需要转换的计数器,我们特别定义了一个 |\chinese| 命令。
该命令可以象罗马数字转换命令 |\roman|、 |\Roman| 一样使用。
具体格式是
\begin{quote}
|\chinese|\marg*{counter}
\end{quote}
其中 \parg{counter} 是一个 \LaTeX{} 计数器(counter),即由
 |\newcounter| 命令产生的,例如 |section|, |figure| 等。

\DescribeMacro{\Chinese}
\ctex{} 宏包会在每次使用 |\setcounter|, |\stepcounter| 或
|\addtocounter| 时利用 |\CTEXcounter{|\parg{counter}|}|
更新 |\chinese| 命令产生的汉字,如果计数器的修改没有用到上述命令
(如页码),就需要在 |\chinese| 前手工使用 |\CTEXcounter| 命令
更新。为此,\ctex{} 宏包提供了大写的 |\Chinese| 命令,作为
上述过程的简写。例如 |\Chinese{page}| 产生“\Chinese{page}”。
该命令不宜用在 |\section| 等命令的参数中。

\subsection{高级设置}

\DescribeMacro{\CTEXoptions}
\ctex{} 宏包中一般的设置通过 |\CTEXoptions| 命令完成。
这个命令的基本格式是
\begin{quote}
|\CTEXoptions|\oarg{\parg{key1}={\parg{val1}},
                  \parg{key2}={\parg{val2}}, ...}
\end{quote}
其中 \parg{key1}, \parg{key2} 是设置选项,
 \parg{val1}, \parg{val2} 则是对应选项的设置内容。
多个选项可以在一个语句中完成设置。

\DescribeMacro{\CTEXsetup}
部分设置如章节标题则通过 |\CTEXsetup| 命令完成。这个命令比
 |\CTEXoptions| 多一个参数,用于指定设置对象。
基本格式是
\begin{quote}
|\CTEXsetup|\oarg{\parg{key1}={\parg{val1}},
                  \parg{key2}={\parg{val2}}, ...}\marg*{type}
\end{quote}
其中 \parg{type} 是设置的对象类型,如 |part|, |chapter|, |section|,
|subsection|, |subsubsection|, |paragraph|, |subparagraph| 等。
 \parg{key1}, \parg{key2} 是设置选项,如 |name|, |number|, |format|,
|nameformat|, |numberformat|, |aftername|, |titleformat| 等。
 \parg{val1}, \parg{val2} 则是对应选项的设置内容。
同一个目标类型的多个选项可以在一个语句中完成设置。

{\kaishu
在 v0.7 版本之前,如果以上命令的参数中包含中文字符,则命令必须放在
|\begin{document}| 之后才能正常工作。从 v0.7 版本开始支持在导言区使用中文。
}

\subsubsection{章节标题设置}

普通章节标题的格式全部通过 |\CTEXsetup| 命令完成。
章节类型在 |\CTEXsetup| 命令的第二个参数中指定。

{\kaishu 在 v0.7 版本之前,如果使用了宏包选项 cap (缺省情况即是如此),则所有
对章节标题的修改必须在 |\begin{document}| 以后进行。原因是
缺省的中文标题设置文件 \texttt{ctexcap.cfg} 文件是在
 |\begin{document}| 之后才会自动装入,因而之前的修改都
会被覆盖而无效。这一限制对后面的附录标题以及其他标题设置
一样有效。从 v0.7 版本开始,\texttt{ctexcap.cfg} 文件
在宏包文件结束时就已经被装入,因此可以在导言区使用设置命令。}

\begin{description}

\item[name=\{\parg{prename},\parg{postname}\}]
该选项用于设置章节的名字,包括章节编号前后的词语,两个之间用逗号分开。
例如
\begin{quote}
|\CTEXsetup[name={第,节}]{section}|
\end{quote}
会使得 section 的标题使用形如“第1节”的名字。注意{\bf 不要}使用中文
的逗号。

该选项的缺省设置是
\begin{center}
\begin{tabular}{lll}
\hline\hline
   & 使用宏包选项 cap  & 使用宏包选项 nocap  \\
\hline
part & \{第,部分\} & \{Part\cs{space},\} \\
chapter & \{第,章\} & \{Chapter\cs{space},\} \\
section & 同右 & \{,\} \\
subsection & 同右 & \{,\} \\
subsubsection & 同右 & \{,\} \\
paragraph & 同右 & \{,\} \\
subparagraph & 同右 & \{,\} \\
\hline\hline
\end{tabular}
\end{center}

\item[number=\{\parg{number}\}]
该选项用于设置章节编号的数字样式。例如
\begin{quote}
|\CTEXsetup[number={\roman{section}}]{section}|
\end{quote}
会使得 section 的标题使用小写罗马数字作为编号。常用的数字样式命令有
\begin{description}
\item \cs{chinese}\marg*{counter}: 一, 二, 三, ...
\item \cs{arabic}\marg*{counter}: 1, 2, 3, ...
\item \cs{roman}\marg*{counter}: i, ii, iii, ...
\item \cs{Roman}\marg*{counter}: I, II, III, ...
\item \cs{alph}\marg*{counter}: a, b, c, ...
\item \cs{Alph}\marg*{counter}: A, B, C, ...
\end{description}

该选项的缺省设置是
\begin{center}
\begin{tabular}{lll}
\hline\hline
   & 使用宏包选项 cap  & 使用宏包选项 nocap  \\
\hline
part & \{\cs{chinese}\marg{part}\} & \{\cs{Roman}\marg{part}\} \\
chapter & \{\cs{chinese}\marg{chapter}\} & \{\cs{arabic}\marg{chapter}\} \\
section & 同右 & \{\cs{thesection}\} \\
subsection & 同右 & \{\cs{thesubsection}\} \\
subsubsection & 同右 & \{\cs{thesubsubsection}\} \\
paragraph & 同右 & \{\cs{theparagraph}\} \\
subparagraph & 同右 & \{\cs{thesubparagraph}\} \\
\hline\hline
\end{tabular}
\end{center}

\item[format=\{\parg{format}\}]
用于控制章节标题的全局格式,作用域为章节名字和随后的标题内容。
常用于控制章节标题的对齐方式。

该选项的缺省设置是
\begin{center} \small
\begin{tabular}{lll}
\hline\hline
   & 使用宏包选项 cap  & 使用宏包选项 nocap  \\
\hline
part (article) & \{\cs{centering}\} & \{\cs{raggedright}\} \\
part & \{\cs{centering}\} & \{\cs{centering}\} \\
chapter & \{\cs{centering}\} & \{\cs{raggedright}\} \\
section & \{\cs{Large}\cs{bfseries}\cs{centering}\} & \{\cs{Large}\cs{bfseries}\} \\
subsection & 同右 & \{\cs{large}\cs{bfseries}\} \\
subsubsection & 同右 & \{\cs{normalsize}\cs{bfseries}\} \\
paragraph & 同右 & \{\cs{normalsize}\cs{bfseries}\} \\
subparagraph & 同右 & \{\cs{normalsize}\cs{bfseries}\} \\
\hline\hline
\end{tabular}
\end{center}

\item[nameformat=\{\parg{nameformat}\}]
用于控制章节名字的格式,作用域为章节名字,包括编号。

该选项的缺省设置是
\begin{center}
\begin{tabular}{lll}
\hline\hline
   & 使用宏包选项 cap  & 使用宏包选项 nocap  \\
\hline
part (article) & 同右 & \{\cs{Large}\cs{bfseries}\} \\
part & 同右 & \{\cs{huge}\cs{bfseries}\} \\
chapter & 同右 & \{\cs{huge}\cs{bfseries}\} \\
section & 同右 & \{\} \\
subsection & 同右 & \{\} \\
subsubsection & 同右 & \{\} \\
paragraph & 同右 & \{\} \\
subparagraph & 同右 & \{\} \\
\hline\hline
\end{tabular}
\end{center}

\item[numberformat=\{\parg{numberformat}\}]
用于控制章节编号的格式。一般为空,当你需要编号的格式和前后的章节名字
不一样时使用。

\item[aftername=\{\parg{aftername}\}]
用于控制章节标题中章节名字和随后的标题内容之间的格式变换。
常用于控制标题内容是否另起一行。

该选项的缺省设置是
\begin{center}
\begin{tabular}{lll}
\hline\hline
   & 使用宏包选项 cap  & 使用宏包选项 nocap  \\
\hline
part (article) & \{\cs{quad}\} & \{\cs{par}\cs{nobreak}\} \\
part & 同右 & \{\cs{par}\cs{vskip} 20pt\} \\
chapter & \{\cs{quad}\} & \{\cs{par}\cs{vskip} 20pt\} \\
section & 同右 & \{\} \\
subsection & 同右 & \{\} \\
subsubsection & 同右 & \{\} \\
paragraph & 同右 & \{\} \\
subparagraph & 同右 & \{\} \\
\hline\hline
\end{tabular}
\end{center}

\item[titleformat=\{\parg{titleformat}\}]
用于控制标题内容的格式,作用域为章节标题内容。

该选项的缺省设置是
\begin{center}
\begin{tabular}{lll}
\hline\hline
   & 使用宏包选项 cap  & 使用宏包选项 nocap  \\
\hline
part (article) & \{\cs{Large}\cs{bfseries}\} & \{\cs{huge}\cs{bfseries}\} \\
part & \{\cs{huge}\cs{bfseries}\} & \{\cs{Huge}\cs{bfseries}\} \\
chapter & \{\cs{huge}\cs{bfseries}\} & \{\cs{Huge}\cs{bfseries}\} \\
section & 同右 & \{\} \\
subsection & 同右 & \{\} \\
subsubsection & 同右 & \{\} \\
paragraph & 同右 & \{\} \\
subparagraph & 同右 & \{\} \\
\hline\hline
\end{tabular}
\end{center}

\item[beforeskip=\{\parg{beforeskip}\}]
用于控制章节标题前的空距。

该选项的缺省设置是
\begin{center}
\begin{tabular}{lll}
\hline\hline
   & 使用宏包选项 cap  & 使用宏包选项 nocap  \\
\hline
part (article) & 同右 & \{4ex\} \\
part & 无效 & 无效 \\
chapter & 同右 & \{50pt\} \\
section & 同右 & \{-3.5ex plus -1ex minus -.2ex\} \\
subsection & 同右 & \{-3.25ex plus -1ex minus -.2ex\} \\
subsubsection & 同右 & \{-3.25ex plus -1ex minus -.2ex\} \\
paragraph & 同右 & \{3.25ex plus 1ex minus .2ex\} \\
subparagraph & 同右 & \{3.25ex plus 1ex minus .2ex\} \\
\hline\hline
\end{tabular}
\end{center}

在 section 及以下的标题中,使用负的距离表示标题后的段落不缩进
(如标准的英文 LaTeX 文档),否则缩进。标题上方真正的空距是该参数的绝对值。

\item[afterskip=\{\parg{afterskip}\}]
用于控制章节标题后的空距。

该选项的缺省设置是
\begin{center}
\begin{tabular}{lll}
\hline\hline
   & 使用宏包选项 cap  & 使用宏包选项 nocap  \\
\hline
part (article) & 同右 & \{3ex\} \\
part & 无效 & 无效 \\
chapter & 同右 & \{40pt\} \\
section & 同右 & \{2.3ex plus .2ex\} \\
subsection & 同右 & \{1.5ex plus .2ex\} \\
subsubsection & 同右 & \{1.5ex plus .2ex\} \\
paragraph & 同右 & \{-1em\} \\
subparagraph & 同右 & \{-1em\} \\
\hline\hline
\end{tabular}
\end{center}

在 section 及以下的标题中,正的距离表示向下留出的空距(如标准的 section 标题),
使用负的距离则表示向右留出的空距的负值(如标准的 paragraph 标题)。

\item[indent=\{\parg{indent}\}]
用于控制章节标题本身的缩进。

该选项的缺省设置是
\begin{center}
\begin{tabular}{lll}
\hline\hline
   & 使用宏包选项 cap  & 使用宏包选项 nocap  \\
\hline
part (article) & 同右 & \{0pt\} \\
part & 无效 & 无效 \\
chapter & 同右 & \{0pt\} \\
section & 同右 & \{0pt\} \\
subsection & 同右 & \{0pt\} \\
subsubsection & 同右 & \{0pt\} \\
paragraph & 同右 & \{0pt\} \\
subparagraph & 同右 & \{\cs{parindent}\} \\
\hline\hline
\end{tabular}
\end{center}

\end{description}


\subsubsection{部分修改标题格式}

如果只想修改标题格式中的某些参数而不是完全重新设置,可以使用带 + 号的
设置选项。例如
\begin{quote}
|\CTEXsetup[format+={\zihao{1}}]{section}|
\end{quote}
则 section 的标题使用一号字体,而其他格式设置保持不变。

标题格式相关的选项都支持这一功能,包括 |format|, |nameformat|, |numberformat|,
|aftername| 和 |titleformat|,而且对所有文档类型都有效。

\subsubsection{附录标题设置}

附录(appendix)的标题也使用 |\CTEXsetup| 命令进行设置,
第二个参数设为 |appendix|。但是只能使用 |name| 和 |number|
两个设置选项。在使用了 |\appendix| 命令之后,附录
的名字和编号会被自动使用。{\bf 附录的名字和前面的章节不同,
它只有一个部分,放在编号之前。}在 article 类文档中,
附录是用 section 实现的,而在 report 和 book 类文档中附录
使用的是 chapter 的设置。因此在设置附录的编号的时候要注意
使用正确的计数器。如果你要设置其他格式的附录标题,
可以根据使用的文档类直接用 section 或者 chapter 的设置命令来控制,
但是要记住把设置命令放在 |\appendix| (如果有的话)的后面,
否则会被 |\appendix| 命令的设置覆盖。

附录的缺省设置是
\begin{center}
\begin{tabular}{lll}
\hline\hline
   & 使用宏包选项 cap  & 使用宏包选项 nocap  \\
\hline
name (article) & 同右 & \{\} \\
name & \{附录\textasciitilde\} & \{Appendix\cs{space}\} \\
number (article) & 同右 & \{\cs{Alph}\marg{section}\} \\
number & 同右 & \{\cs{Alph}\marg{chapter}\} \\
\hline\hline
\end{tabular}
\end{center}


\subsubsection{其他标题设置}

除章节标题外其他标题的设置通过 |\CTEXoptions| 设置。包括

\begin{description}
\item[contentsname] 目录名
\item[listfigurename] 表格目录
\item[listtablename] 插图目录
\item[figurename] 图
\item[tablename] 表
\item[abstractname] 摘要
\item[indexname] 索引
\item[bibname] 参考文献
\end{description}

例如
\begin{quote}
|\CTEXoptions[indexname={总索引}]|
\end{quote}
把索引的名字改为“总索引”。


\subsubsection{其他设置}

\paragraph{设置 \cs{today} 的日期格式}
使用 |\CTEXoptions| 可以设置 |\today| 命令产生的日期格式。
支持的格式包括

\begin{enumerate}

\item 阿拉伯数字加中文年月日
\begin{quote}
|\CTEXoptions[today=small]|
\end{quote}
\CTEXoptions[today=small]
|\today| 生成的日期例子为“\today{}”。

\item 中文数字加中文年月日
\begin{quote}
|\CTEXoptions[today=big]|
\end{quote}
\CTEXoptions[today=big]
|\today| 生成的日期例子为“\today{}”。

\item \LaTeX{} 标准格式
\begin{quote}
|\CTEXoptions[today=old]|
\end{quote}
\CTEXoptions[today=old]
|\today| 生成的日期例子为“\today{}”。

\end{enumerate}


\paragraph{设置图表标题的分隔符}
使用 |\CTEXoptions| 可以设置 |\caption| 命令产生的图表标题的分隔符。
这个分隔符缺省是使用冒号 : 。可以通过命令
\begin{quote}
|\CTEXoptions[captiondelimiter={|\parg{string}|}]|
\end{quote}
设置为任意的单个字符或者字符串 \parg{string}。


\subsection{配置文件}

主要的配置文件有:
\begin{itemize}
\item \texttt{ctexopts.cfg}
\item \texttt{ctexcap.cfg}, \texttt{ctexcap-gbk.cfg}, \texttt{ctexcap-utf8.cfg}
\item \texttt{ctex.cfg}
\end{itemize}

\texttt{ctexopts.cfg} 用于设置缺省选项。例如可以根据系统中的可用中文字库,
设置 \ref{fontset} 一节中的几个选项之一为缺省选项。
该文件会在处理宏包选项之前装入。

\texttt{ctexcap.cfg} 是缺省中文标题格式的定义,当你使用 \texttt{cap}
选项时就会使用这里的定义。你可以把它改为你经常使用的格式,这样就不用
每次都在正文中修改了。 \texttt{ctexcap.cfg} 中的设置都可以通过宏包提供
的设置命令在正文中进行修改。与中文编码有关的定义分别在
 \texttt{ctexcap-gbk.cfg} 和 \texttt{ctexcap-utf8.cfg} 文件中。

最后,宏包还将读入 \texttt{ctex.cfg} 文件,该文件中的设置将覆盖其他配置
文件中的设置。用户可以在该文件中加入自己的定义。

在修改这些配置文件的时候,你可以修改系统目录中的文件,也可以拷贝一份放到
当前目录下,然后修改。TeX 会优先使用当前目录下的同名文件。这样你可以针对
不同的应用设置不同的缺省配置文件。


\section{版本更新}

\begin{description}

\item[v1.02c 2011/03/11]
修改 \texttt{hyperref} 选项的行为,使 \texttt{hyperref} 宏包可以在用户指定的位
置被调用,以解决个别兼容性问题。修正 XeTeX 编译书签中百分号出错的 BUG。

\item[v1.02b 2011/01/21]
修正使用一个中文书签 BUG,恢复依赖 expl3。

\item[v1.02 2010/10/17]
采用新的 hook 机制,去掉对 expl3 宏包的依赖性

\item[v1.01 2010/09/27]
处理 |format| 选项默认值的 BUG,将 |\subsection| 以下改为西文默认的两端对齐。

处理 |xunicode.sty| v0.95 版本导致 |\beth| 等数学符号在数学字体包中重定义的问题。

\item[v1.00 2010/09/18]
处理在一些系统中 |zhwinfonts.tex| 的 map 在浮动体中失效的问题。

\item[v0.99b 2010/07/11]
发现 BUG,把 |\chinese*| 改为 |\Chinese|。

\item[v0.99a 2010/07/11]
增加 |\chinese*| 命令,作为 |\CTEXcounter| 和 |\chinese| 混合体的
简写形式。

\item[v0.99 2010/07/08]
提早引入 expl3 宏包,解决 |\chinese| 命令在 expl3 下的不能正确使用的问题。

\item[v0.98 2010/06/19]
解决 xeCJK 默认代替 indentfirst 功能的问题。修正 xeCJK 模式下 winfonts
的中文字体。

\item[v0.97 2010/01/22]
修正 pdf\LaTeX{} 与 Xe\LaTeX{} 中生成 PDF 书签的问题。

\item[v0.96 2009/11/24]
添加 zhmap 与 nozhmap 选项控制 zhwinfonts 的载入。

\item[v0.95 2009/10/20]
移除对 CJKnumb 宏包的依赖。去除无用的 cjkfonts 选项,以 winfonts
作为默认值。

\item[v0.94 2009/09/13]
改进 hyperref 选项的支持

\item[v0.93 2009/07/10]
加入选项 hyperref 以支持自动用合适的参数调用 hyperref 宏包

\item[v0.92 2009/06/25]
Add backward compatibility to old ctexutf8 packages

\item[v0.91 2009/05/23]
调整宏包结构,增加对各种系统中文字库的支持选项:cjkfonts, winfonts,
adobefonts

\item[v0.9 2009/05/05]
整合 ctexutf8 宏包,增加对 UTF-8 编码的支持;
开始支持 Xe\TeX{} 中文系统(使用 xeCJK 宏包)

\item[v0.8a 2007/05/06]
增加 bold 字体的定义

\item[v0.8 2006/06/09]
将 ctex.sty 文件分割为 ctex.sty 和 ctexcap.sty,
后者只支持标准文档类增加对 |\stepcounter| 的重定义,以和 calc 宏包兼容

\item[v0.7f 2006/04/12]
采用修改 |\AtBeginDocument| 和 |\AtEndDocument| 命令的方式
来设置 CJK 环境,以减少宏包冲突

\item[v0.7e 2006/03/22]
改用 |\DeclareRobustCommand| 定义 |\CTEXnumber| 和 |\CTEXcounter|;
除去 |\CTEXdigits| 和 |\CTEX@getdigit| 命令带来的多余空格

\item[v0.7d 2005/12/28]
在 fntef 类宏包后使用 |\normalem| 恢复 |\em| 宏的缺省定义

\item[v0.7c 2005/12/20]
增加对 |\if@mainmatter| 的判断,以兼容 amsbook 宏包

\item[v0.7b 2005/12/09]
调整宏包导入位置,解决 fntef 类宏包早于相应中文宏包导入的问题

\item[v0.7a 2005/11/28]
将 ctex.cfg 文件的读取时间前移,使得导言中的设置命令优先

\item[v0.7 2005/11/25]
支持在导言区中使用中文和章节标题设置命令(感谢 tercelxy 的建议);
增加 CJKfntef 宏包和 CCTfntef 宏包的统一接口(感谢 chenyu\_21cn 的建议)

\item[v0.6b 2005/11/07]
将节以下编号和标题之间的空距定义转移到相应的 aftername 变量中

\item[v0.6a 2005/09/30]
增加对 |\CCT@set@fontsize| 的判断

\item[v0.6 2005/09/24]
针对 cct 0.6180 的修改,
|\set@fontsize|: cct 从 0.6180 开始将宏 |\oset@fontsize| 改为 |\CCT@set@fontsize|

\item[v0.5c 2004/09/29]
避免重复执行设置 CJK 环境结束语句

\item[v0.5b 2004/09/29]
改变设置 CJK 环境结束语句的 |\AtEndDocument| 执行的位置,以减少宏包冲突

\item[v0.5a 2004/09/06]
修改图表标题分隔符设置中的错误

\item[v0.5 2004/08/23]
General: Move Chinese definitions from ctex.cfg to ctex.def

\item[v0.4d 2004/08/14]
|\ps@fancy|: 增加对 mainmatter 的判断;
|\refstepcounter|: 修改 |\ref| 命令,不再包含除编号外的内容

\item[v0.4c 2004/07/26]
增加判断以避免嵌套定义 |\setcounter| 和 |\addtocounter|

\item[v0.4b 2004/07/13]
|\baselinestretch|: 把 |\baselinestretch| 从 1.2 改为 1.3

\item[v0.4a 2004/05/15]
|\CTEXdigits|: 增加 |\CTEXdigits| 命令;
|\ziju|: 修改 CCT 的字距命令使得缩进保持一致

\item[v0.4 2004/05/13]
General: 如果指定了标准的 \LaTeX{} 字体大小,则不使用中文字号;
中文字号定义改为直接使用 pt 为单位;
|\zihao|: 删除 |\CTEX@fontsize| 命令,改为直接使用 |\fontsize| 命令

\item[v0.3b 2004/05/11]
General: 增加 fancyhdr 选项

\item[v0.3a 2004/04/30]
General: 修改命令 |\CCTpuncttrue| 的拼写错误

\item[v0.3 2004/04/24]
General: 对页眉设置进行微调;
对中文标题的章节编号格式进行调整,去掉 |\S|;
修改为使用 |\chinese| 命令以避免产生错误;
修正 sub3section 和 sub4section 选项无效的问题;
增加对图表标题分隔符的设置;
|\ps@fancy|: 解决与 fancyhdr 的冲突

\item[v0.2d 2004/04/23]
General: Change option c5size to base on 10pt basic class;
补上字号定义中行间距参数中缺少的 |\CTEX@bp|;
修改缺省的字号大小

\item[v0.2c 2004/02/13]
General: Add CJKpunct as standard configuration;
|\ifCTEX@punct|: 增加判断是否调整中文标点宽度的选项

\item[v0.2b 2004/02/13]
General: 修改缺省的行距;
修改缺省的字号大小

\item[v0.2a 2004/02/11]
|\baselinestretch|: 增加对行距的设置;
|\CTEX@spaceChar|: 加快处理速度,改善和 CJKpunct 的兼容性

\item[v0.2 2004/01/16]
General: Add support for CCT;
增加部分修改标题格式设置的选项;
增加修改标题前后空距设置的选项;
|\CTEXsetfont|: |\CTEXfontinfo| 命令改为 |\CTEXsetfont|;
|\ziju|: 参数的单位由绝对距离改为相对于当前汉字大小的倍数

\item[v0.1f 2003/12/24]
|\refname|: 修正 article 类中参考文献标题没有使用中文的问题

\item[v0.1e 2003/11/05]
|\refstepcounter|: 修正 |\ref| 命令后多出空格的问题

\item[v0.1d 2003/09/27]
|\addtocounter|: 将对 |\setcounter| 和 |\addtocounter| 的修改放到导言的最后以和其他宏包兼容

\item[v0.1c 2003/08/19]
General: 去掉生成的 .out 文件里章的标题前的多余空格

\item[v0.1b 2003/08/17]
|\zihao|: 删除多余的 |\newcount| 命令

\item[v0.1a 2003/08/15]
General: 修正 ctex.sty 中无法使用 sub3section 和 sub4section 选项的问题

\item[v0.1 2003/08/15]
General: First beta release

\item[v0.0 2003/04/26] General: Initial version

\end{description}


\section{开发人员}

\begin{itemize}

\item 吴凌云 (\email{aloft@ctex.org})
\item 江疆 (\email{gzjjgod@gmail.com})
\item 王越 (\email{yuleopen@gmail.com})
\item 刘海洋 (\email{LeoLiu.PKU@gmail.com})
\item LiYanrui.m2 (LiYanrui)
\item 陈之初 (Neals)

\end{itemize}

\end{document}

