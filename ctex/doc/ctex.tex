% ctex.tex: manual of ctex package

\documentclass{ltxdoc}
\usepackage{ctex}
\usepackage{texnames}
 \topmargin 0.5 true cm
 \oddsidemargin 1 true cm
 \evensidemargin 1 true cm
 \textheight 21 true cm
 \textwidth 14 true cm

\MakeShortVerb{\|}
\setcounter{StandardModuleDepth}{1}

\newcommand{\ctex}{\texttt{ctex}}
\newcommand{\ctexorg}{\texttt{ctex.org}}

\setlength{\parskip}{0.75ex plus .2ex minus .5ex}
\renewcommand{\baselinestretch}{1.2}

\makeatletter
\def\parg#1{\mbox{$\langle${\it #1\/}$\rangle$}}
\def\@smarg#1{{\tt\string{}\parg{#1}{\tt\string}}}
\def\@marg#1{{\tt\string{}{\rm #1}{\tt\string}}}
\def\marg{\@ifstar\@smarg\@marg}
\def\@soarg#1{{\tt[}\parg{#1}{\tt]}}
\def\@oarg#1{{\tt[}{\rm #1}{\tt]}}
\def\oarg{\@ifstar\@soarg\@oarg}
\makeatother

\begin{document}

\title{\bf \ctex{} 宏包说明\thanks
  {本文件版本号为 \fileversion{},最后修改日期 \filedate{}。}}
\author{\it 吴凌云\thanks{aloft@ctex.org}}
\date{\small 打印日期:\today}
\maketitle


\begin{abstract}
\ctex{} 宏包提供了一个统一的中文 \LaTeX{} 文档框架,底层支持 CCT、CJK 和 XeTeX 三种中文 \LaTeX{} 系统。\ctex{} 宏包提供了编写中文 \LaTeX{} 文档常用的一些宏定义和命令。

\ctex{} 宏包需要 CCT 系统或者 CJK 宏包或者 xeCJK 宏包的支持。主要文件包括 \texttt{ctexart.cls}、\texttt{ctexrep.cls}、\texttt{ctexbook.cls} 和 \texttt{ctex.sty}、\texttt{ctexcap.sty}。

\ctex{} 宏包由 \ctexorg{} 制作并负责维护。
\end{abstract}

\tableofcontents

\section{简介}

这个宏包的部分原始代码来自于由王磊编写~\texttt{cjkbook.cls}~文档类,
还有一小部分原始代码来自于吴凌云编写的~\texttt{GB.cap}~文件。
原来的这些工作都是零零碎碎编写的,没有认真、系统的设计,
也没有用户文档,非常不利于维护和改进。所以我们用~\texttt{doc}~
和~\texttt{docstrip}~工具重新编写了整个文档,并增加了许多新的功能。

最初~Knuth~设计开发~\TeX{}~的时候没有考虑到支持多国语言,
特别是多字节的中日韩语言。这使得~\TeX{}~以至后来的
~\LaTeX{}~对中文的支持一直不是很好。即使在~CJK~解决了中文字符
处理的问题以后,中文用户使用~\LaTeX{}~仍然要面对许多困难。
最常见的就是中文化的标题。由于中文习惯和西方语言的不同,
使得很难直接使用原有的标题结构来表示中文标题。因此需要对
标准~\LaTeX{}~宏包做较大的修改。此外,还有诸如中文字号的对应
关系等等。~\ctex{}~宏包正是尝试着解决这些问题。中间很多地方
用到了在~\ctexorg{}~论坛上的讨论结果,在此对参与讨论的
朋友们表示感谢。

\ctex{}~宏包由五个主要文件构成:
~\texttt{ctexart.cls}、~\texttt{ctexrep.cls}、~\texttt{ctexbook.cls}~和
~\texttt{ctex.sty}、~\texttt{ctexcap.sty}。~\texttt{ctex.sty}~主要是提供整合的
中文环境,可以配合大多数文档类使用。而~\texttt{ctexcap.sty}~则是对~\LaTeX{}~
的三个标准文档类的格式进行修改以符合中文习惯,该宏包只能配合这三个标准文档类使用。
~\texttt{ctexart.cls}、~\texttt{ctexrep.cls}、~\texttt{ctexbook.cls}~则是
~\texttt{ctex.sty}、\texttt{ctexcap.sty}~分别和三个标准文档类结合产生的新文档类,
除了包含~\texttt{ctex.sty}、\texttt{ctexcap.sty}~的所有功能,还加入了一些修改文档类
缺省设置的内容(如使用五号字体为缺省字体)。

\vskip 10pt
{\kaishu
这份说明文档可以通过用~\LaTeX{}~编译~\texttt{ctex.dtx}~文件来得到。
编译说明文档需要~CJK~宏包和~\ctex{}~宏包。
为了生成正确的索引和版本记录,需要使用如下命令
\begin{verbatim}
makeindex -s gind.ist -o ctex.ind ctex.idx
makeindex -s gglo.ist -o ctex.gls ctex.glo
\end{verbatim}
}


\section{使用帮助}

\ctex{}~宏包的使用十分简单。如果是使用~\ctex{}~的文档类,只需用
~\texttt{ctexart}、~\texttt{ctexrep}~或者~\texttt{ctexbook}~替换原来的
文档类就可以了。你也可以继续使用原来的文档类,而用~\texttt{ctex.sty}~和
~\texttt{ctexcap.sty}~宏包来配合使用,两者的效果是一样的
(除了不能修改一些文档设置如缺省字体大小)。

\subsection{使用~CJK}

这是~\ctex{}~宏包的缺省设置。\ctex{}~宏包会自动调用~CJK~宏包,你无需再自己调用。
此外,\ctex{}~宏包会在~|\begin{document}|~和~|\end{document}|~
之间自动加入一个~CJK~环境,你无需再添加~CJK~环境。~CJK~宏包的命令都可以
在~|\begin{document}|~和~|\end{document}|~之间正常使用。

例子1:使用文档类宏包
\begin{verbatim}
\documentclass{ctexart}
\begin{document}
中文宏包测试
\end{document}
\end{verbatim}

例子2:使用普通宏包
\begin{verbatim}
\documentclass{article}
\usepackage{ctex}
\begin{document}
中文宏包测试
\end{document}
\end{verbatim}

\subsection{使用~CCT}
\ctex{}~宏包也可以配合新版的~CCT~使用,只需在使用~\ctex{}~宏包时加上~CCT~选项即可。
缺省~CCT~会使用~CJK~字库,因为这种字库方式比传统~CCT~字库更方便,兼容性也更好。
如果要使用传统~CCT~字库,则还要加上~CCTfont~选项。

例子3:使用~CJK~方式字库
\begin{verbatim}
\documentclass[CCT]{ctexart}
\begin{document}
中文宏包测试
\end{document}
\end{verbatim}

例子2:使用~CCT~方式字库
\begin{verbatim}
\documentclass[CCT,CCTfont]{ctexart}
\begin{document}
中文宏包测试
\end{document}
\end{verbatim}


\subsection{选项}

宏包的选项用于改变一些缺省风格的设置。缺省的设置已经针对中文
的习惯进行了尽量的修改,所以一般用户无需使用这些选项。
如果你觉得某些设置不合适,可以向作者反映。我们会考虑在后面的
版本中予以改进。我们也欢迎关于增加或者删减选项的建议。


下面的选项可能会是最经常使用的。但是它们只能用于文档类
(\texttt{ctexart}、~\texttt{ctexrep}~和~\texttt{ctexbook})。
\begin{description}
\item[cs4size] 使用小四字号为缺省字体大小。
\item[c5size] 使用五号字为缺省字体大小。{\heiti 这个是
~\ctex{}~文档类的缺省格式。}
\end{description}


下面这些则可以在文档类宏包和~\texttt{ctex.sty}~上使用。
\begin{description}
\item[CCT] 使用~CCT~代替~CJK~做为底层的中文支持系统。

\item[CCTfont] 使用传统的~CCT~字库方式,该选项会自动激活~CCT~选项。

\item[punct] 对中文标点的位置(宽度)进行调整。

\item[nopunct] 不对中文标点的位置进行调整(每个标点占有相同的宽度)。
\item[space] 使用~CJK~的保留空格模式,保留中文字符间的空格(类似英文的
习惯)。你需要自己处理中文字符间的空格以及换行产生的空格(在行尾加上
~\%~符号可以避免),否则排版结果可能不符合中文习惯。这种模式可以通过
~|\CTEXnospace|~转换到~nospace~模式。

\item[nospace] 使用~CJK~的忽略空格模式,也就是~CJK*~环境的模式。
~CJK~会自动忽略中文字符间的空格,比较符合中文习惯。在这种模式下,
可以使用~\textasciitilde~来分隔中英文字符,产生的间距稍小于普通空格,
排版效果比较美观。这种模式可以通过~|\CTEXspace|~命令转换到~space~模式。
{\heiti 这个是~\ctex{}~宏包的缺省模式。}

\item[cap] 使用中文的标题样式。{\heiti 这个是文档类宏包的缺省模式。}

\item[nocap] 保留使用英文的标题样式。

\item[indent] 使用中文的段首缩进模式,即缩进两个汉字宽度,同时每个段落
都缩进。{\heiti 这个是~\ctex{}~宏包的缺省模式。}

\item[noindent] 使用原来的段首缩进模式,章节标题后的第一段不缩进。

\item[fancyhdr] 保持和~\texttt{fancyhdr}~宏包的兼容性。该选项将使得
~\texttt{fancyhdr}~宏包被自动调用。

\item[amstex] 保持和~\AMSLaTeX{}~宏包的兼容性。

\item[fntef] 为~\texttt{CJKfntef}~宏包和~\texttt{CCTfntef}~宏包提供统一接口。
该选项将使得~\texttt{CJKfntef}~宏包或者~\texttt{CCTfntef}~宏包被自动调用。
\end{description}


下面这些则可以在文档类宏包和~\texttt{ctexcap.sty}~上使用。
\begin{description}
\item[cap] 使用中文的标题样式,缺省格式由~\texttt{ctexcap.cfg}~配置文件
内的定义给出。{\heiti 这个是文档类宏包的缺省模式。}

\item[nocap] 保留使用英文的标题样式。

\item[sub3section] 将~|\paragraph|~命令产生的标题改为~section~类格式。
此时~|\subparagraph|~命令产生的标题会具有原来~|\paragraph|~的格式。

\item[sub4section] 将~|\paragraph|~和~|\subparagraph|~命令产生的标题
都改为~section~类格式。
\end{description}


\vskip 10pt
{\kaishu
总结:\ctex{}~宏包的缺省选项是~nospace cap indent,文档类
的缺省选项是~nospace cap indent c5size。
}


\subsection{基本命令}

\ctex{}~宏包给用户提供一个通用的文档框架,使得用户可以自由地在不同的
底层中文系统间切换。为此,我们为~CJK~定制了一些模拟~CCT~的命令,
也对部分~CCT~命令进行了修改,使得两者保持一致。
此外,我们还定义了用于设置文档参数的高级设置命令。

\subsubsection{字体设置}

中文字体很多,但是常用的就那么几个。我们为~CJK~常用的六种中文
字体定义了简单易用的命令。它们是:

\DescribeMacro{\songti}
宋体:~|\songti|,~CJK~等价命令~|\CJKfamily{song}|

\DescribeMacro{\heiti}
黑体:~|\heiti|,~CJK~等价命令~|\CJKfamily{hei}|

\DescribeMacro{\fangsong}
仿宋:~|\fangsong|,~CJK~等价命令~|\CJKfamily{fs}|

\DescribeMacro{\kaishu}
楷书:~|\kaishu|,~CJK~等价命令~|\CJKfamily{kai}|

\DescribeMacro{\lishu}
隶书:~|\lishu|,~CJK~等价命令~|\CJKfamily{li}|

\DescribeMacro{\youyuan}
幼圆:~|\youyuan|,~CJK~等价命令~|\CJKfamily{you}|

\vskip 10pt
{\kaishu
\TeX{}~系统中必须已经定义好这六种中文字体,并且使用和~\CTeX{}~套装中
一致的字体名称。(参见上面~CJK~等价命令的参数)

上面的字体命令和~CCT~中的一致,但传统的~CCT~字库中没有隶书和
仿宋两种字体,需要用户自行安装定义。如果使用~CCT~时选择~CJK~字库方式,
则可以使用这两种中文字体。
}

\subsubsection{字号、字距、字宽和缩进}

\DescribeMacro{\zihao}
中文字号的设置命令是~|\zihao|\marg*{字号},例如~|\zihao{3}|。
可以使用的参数有~16~个,小号字体在前面加负号表示,从大到小依次为
\begin{center}
\begin{tabular}{cccccccc}
\hline
初号 & 小初 & 一号 & 小一 & 二号 & 小二 & 三号 & 小三 \\
0 & -0 & 1 & -1 & 2 & -2 & 3 & -3 \\
\hline
四号 & 小四 & 五号 & 小五 & 六号 & 小六 & 七号 & 八号 \\
4 & -4 & 5 & -5 & 6 & -6 & 7 & 8 \\
\hline
\end{tabular}
\end{center}
\noindent 英文字体大小会始终保持和中文字体一致。

\DescribeMacro{\ziju}
汉字字距的调整使用命令~|\ziju|\marg*{字宽的倍数}。参数可以是任意的数字,
例如~|\ziju{5}|~设置汉字字距为当前汉字字宽的~5~倍,~|\ziju{0.5}|~设置汉字
字距为当前汉字字宽的一半。这里的汉字字宽指的是实际汉字的宽度,
不包含当前字距。该命令不影响英文字距。

\DescribeMacro{\ccwd}
当前汉字的字宽保存在宏~|\ccwd|~中。字宽是相邻两个汉字中心的距离,
也就是说字距会被计算在内。

\DescribeMacro{\CTEXindent}
正常的缩进两个汉字字宽的距离,同时在汉字大小和字距改变的
情况都可以自动修改缩进距离。

\DescribeMacro{\CTEXnoindent}
取消缩进。

\DescribeMacro{\CTEXsetfont}
|\CTEXsetfont|~命令用于更新当前的中文字体信息,包括当前字距和缩进
距离。一般来说,用户无需使用这个命令。


\subsubsection{中文数字转换}

\DescribeMacro{\CTEXnumber}
使用~CJK~提供的~|\CJKnumber|~命令可以将阿拉伯数字转换为中文数字。
由于~\LaTeX{}~臭名昭著的脆弱命令的原因,当~|\CJKnumber|~被用在
章节标题等地方的时候,要么出现错误无法使用,要么无法达到预期目的,
例如在产生~PDF~书签的时候。于是我们定义了一个~|\CTEXnumber|~命令,
可以将产生的中文数字保存下来。该命令的格式为
\begin{quote}
|\CTEXnumber|\marg*{result}\marg*{number}
\end{quote}
其中~\parg{result}~必须是一个~\TeX{}~宏的名字,不需要预先定义。
例如
\begin{quote}
|\CTEXnumber{\test}{100002005}|
\end{quote}
则~|\test|~中的内容就是“一亿零二千零五”(不包括引号)。

\DescribeMacro{\CTEXdigits}
|\CTEXdigits|~命令和~|\CTEXnumber|~命令类似,用于代替~CJK~提供的
~|\CJKdigits|~命令。它和~|\CTEXnumber|~命令的不同之处在于转换后
结果是中文数字串,而不是按照中文习惯的数字。该命令的格式为
\begin{quote}
|\CTEXdigits|\marg*{result}\marg*{number}
\end{quote}
其中~\parg{result}~必须是一个~\TeX{}~宏的名字,不需要预先定义。
例如
\begin{quote}
|\CTEXnumber{\test}{100002005}|
\end{quote}
\CTEXdigits{\test}{100002005}
则~|\test|~中的内容就是“\test{}”(不包括引号)。

\DescribeMacro{\chinese}
对于经常需要转换的计数器,我们特别定义了一个~|\chinese|~命令。
该命令可以象罗马数字转换命令~|\roman|、~|\Roman|~一样使用。
具体格式是
\begin{quote}
|\chinese|\marg*{counter}
\end{quote}
其中~\parg{counter}~是一个~\LaTeX{}~计数器(counter),即由
~|\newcounter|~命令产生的,例如~|section|,~|figure|~等。


\subsection{高级设置}

\DescribeMacro{\CTEXoptions}
\ctex{}~宏包中一般的设置通过~|\CTEXoptions|~命令完成。
这个命令的基本格式是
\begin{quote}
|\CTEXoptions|\oarg{\parg{key1}={\parg{val1}},
                  \parg{key2}={\parg{val2}}, ...}
\end{quote}
其中~\parg{key1}, \parg{key2}~是设置选项,
~\parg{val1}, \parg{val2}~则是对应选项的设置内容。
多个选项可以在一个语句中完成设置。

\DescribeMacro{\CTEXsetup}
部分设置如章节标题则通过~|\CTEXsetup|~命令完成。这个命令比
~|\CTEXoptions|~多一个参数,用于指定设置对象。
基本格式是
\begin{quote}
|\CTEXsetup|\oarg{\parg{key1}={\parg{val1}},
                  \parg{key2}={\parg{val2}}, ...}\marg*{type}
\end{quote}
其中~\parg{type}~是设置的对象类型,如~|part|, |chapter|, |section|,
|subsection|, |subsubsection|, |paragraph|, |subparagraph|~等。
~\parg{key1}, \parg{key2}~是设置选项,如~|name|, |number|, |format|,
|nameformat|, |numberformat|, |aftername|, |titleformat|~等。
~\parg{val1}, \parg{val2}~则是对应选项的设置内容。
同一个目标类型的多个选项可以在一个语句中完成设置。

{\bf 如果以上命令的参数中包含中文字符,则命令必须放在
~|\begin{document}|~之后才能正常工作。}
\footnote{从~v0.7~版本开始支持在导言区使用中文。}


\subsubsection{章节标题设置}

普通章节标题的格式全部通过~|\CTEXsetup|~命令完成。
章节类型在~|\CTEXsetup|~命令的第二个参数中指定。
{\bf 如果使用了宏包选项~cap~(缺省情况即是如此),则所有
对章节标题的修改必须在~|\begin{document}|~以后进行。原因是
缺省的中文标题设置文件~\texttt{ctexcap.cfg}~文件是在
~|\begin{document}|~之后才会自动装入,因而之前的修改都
会被覆盖而无效。}这一限制对后面的附录标题以及其他标题设置
一样有效。\footnote{从~v0.7~版本开始,\texttt{ctexcap.cfg}~文件
在宏包文件结束时就已经被装入,因此可以在导言区使用设置命令。}

\begin{description}

\item[name=\{\parg{prename},\parg{postname}\}]
该选项用于设置章节的名字,包括章节编号前后的词语,两个之间用逗号分开。
例如
\begin{quote}
|\CTEXsetup[name={第,节}]{section}|
\end{quote}
会使得~section~的标题使用形如“第1节”的名字。注意{\bf 不要}使用中文
的逗号。

该选项的缺省设置是
\begin{center}
\begin{tabular}{lll}
\hline\hline
   & 使用宏包选项~cap~ & 使用宏包选项~nocap~ \\
\hline
part & \{第,部分\} & \{Part\cs{space},\} \\
chapter & \{第,章\} & \{Chapter\cs{space},\} \\
section & 同右 & \{,\} \\
subsection & 同右 & \{,\} \\
subsubsection & 同右 & \{,\} \\
paragraph & 同右 & \{,\} \\
subparagraph & 同右 & \{,\} \\
\hline\hline
\end{tabular}
\end{center}

\item[number=\{\parg{number}\}]
该选项用于设置章节编号的数字样式。例如
\begin{quote}
|\CTEXsetup[number={\roman{section}}]{section}|
\end{quote}
会使得~section~的标题使用小写罗马数字作为编号。常用的数字样式命令有
\begin{description}
\item \cs{chinese}\marg*{counter}: 一, 二, 三, ...
\item \cs{arabic}\marg*{counter}: 1, 2, 3, ...
\item \cs{roman}\marg*{counter}: i, ii, iii, ...
\item \cs{Roman}\marg*{counter}: I, II, III, ...
\item \cs{alph}\marg*{counter}: a, b, c, ...
\item \cs{Alph}\marg*{counter}: A, B, C, ...
\end{description}

该选项的缺省设置是
\begin{center}
\begin{tabular}{lll}
\hline\hline
   & 使用宏包选项~cap~ & 使用宏包选项~nocap~ \\
\hline
part & \{\cs{chinese}\marg{part}\} & \{\cs{Roman}\marg{part}\} \\
chapter & \{\cs{chinese}\marg{chapter}\} & \{\cs{arabic}\marg{chapter}\} \\
section & 同右 & \{\cs{thesection}\} \\
subsection & 同右 & \{\cs{thesubsection}\} \\
subsubsection & 同右 & \{\cs{thesubsubsection}\} \\
paragraph & 同右 & \{\cs{theparagraph}\} \\
subparagraph & 同右 & \{\cs{thesubparagraph}\} \\
\hline\hline
\end{tabular}
\end{center}

\item[format=\{\parg{format}\}]
用于控制章节标题的全局格式,作用域为章节名字和随后的标题内容。
常用于控制章节标题的对齐方式。

该选项的缺省设置是
\begin{center} \small
\begin{tabular}{lll}
\hline\hline
   & 使用宏包选项~cap~ & 使用宏包选项~nocap~ \\
\hline
part (article) & \{\cs{centering}\} & \{\cs{raggedright}\} \\
part & \{\cs{centering}\} & \{\cs{centering}\} \\
chapter & \{\cs{centering}\} & \{\cs{raggedright}\} \\
section & \{\cs{Large}\cs{bfseries}\cs{centering}\} & \{\cs{Large}\cs{bfseries}\} \\
subsection & \{\cs{large}\cs{bfseries}\cs{centering}\} & \{\cs{large}\cs{bfseries}\} \\
subsubsection & \{\cs{normalsize}\cs{bfseries}\cs{centering}\} & \{\cs{normalsize}\cs{bfseries}\} \\
paragraph & \{\cs{normalsize}\cs{bfseries}\cs{centering}\} & \{\cs{normalsize}\cs{bfseries}\} \\
subparagraph & \{\cs{normalsize}\cs{bfseries}\cs{centering}\} & \{\cs{normalsize}\cs{bfseries}\} \\
\hline\hline
\end{tabular}
\end{center}

\item[nameformat=\{\parg{nameformat}\}]
用于控制章节名字的格式,作用域为章节名字,包括编号。

该选项的缺省设置是
\begin{center}
\begin{tabular}{lll}
\hline\hline
   & 使用宏包选项~cap~ & 使用宏包选项~nocap~ \\
\hline
part (article) & 同右 & \{\cs{Large}\cs{bfseries}\} \\
part & 同右 & \{\cs{huge}\cs{bfseries}\} \\
chapter & 同右 & \{\cs{huge}\cs{bfseries}\} \\
section & 同右 & \{\} \\
subsection & 同右 & \{\} \\
subsubsection & 同右 & \{\} \\
paragraph & 同右 & \{\} \\
subparagraph & 同右 & \{\} \\
\hline\hline
\end{tabular}
\end{center}

\item[numberformat=\{\parg{numberformat}\}]
用于控制章节编号的格式。一般为空,当你需要编号的格式和前后的章节名字
不一样时使用。

\item[aftername=\{\parg{aftername}\}]
用于控制章节标题中章节名字和随后的标题内容之间的格式变换。
常用于控制标题内容是否另起一行。

该选项的缺省设置是
\begin{center}
\begin{tabular}{lll}
\hline\hline
   & 使用宏包选项~cap~ & 使用宏包选项~nocap~ \\
\hline
part (article) & \{\cs{quad}\} & \{\cs{par}\cs{nobreak}\} \\
part & 同右 & \{\cs{par}\cs{vskip} 20pt\} \\
chapter & \{\cs{quad}\} & \{\cs{par}\cs{vskip} 20pt\} \\
section & 同右 & \{\} \\
subsection & 同右 & \{\} \\
subsubsection & 同右 & \{\} \\
paragraph & 同右 & \{\} \\
subparagraph & 同右 & \{\} \\
\hline\hline
\end{tabular}
\end{center}

\item[titleformat=\{\parg{titleformat}\}]
用于控制标题内容的格式,作用域为章节标题内容。

该选项的缺省设置是
\begin{center}
\begin{tabular}{lll}
\hline\hline
   & 使用宏包选项~cap~ & 使用宏包选项~nocap~ \\
\hline
part (article) & \{\cs{Large}\cs{bfseries}\} & \{\cs{huge}\cs{bfseries}\} \\
part & \{\cs{huge}\cs{bfseries}\} & \{\cs{Huge}\cs{bfseries}\} \\
chapter & \{\cs{huge}\cs{bfseries}\} & \{\cs{Huge}\cs{bfseries}\} \\
section & 同右 & \{\} \\
subsection & 同右 & \{\} \\
subsubsection & 同右 & \{\} \\
paragraph & 同右 & \{\} \\
subparagraph & 同右 & \{\} \\
\hline\hline
\end{tabular}
\end{center}

\item[beforeskip=\{\parg{beforeskip}\}]
用于控制章节标题前的空距。

该选项的缺省设置是
\begin{center}
\begin{tabular}{lll}
\hline\hline
   & 使用宏包选项~cap~ & 使用宏包选项~nocap~ \\
\hline
part (article) & 同右 & \{4ex\} \\
part & 无效 & 无效 \\
chapter & 同右 & \{50pt\} \\
section & 同右 & \{-3.5ex plus -1ex minus -.2ex\} \\
subsection & 同右 & \{-3.25ex plus -1ex minus -.2ex\} \\
subsubsection & 同右 & \{-3.25ex plus -1ex minus -.2ex\} \\
paragraph & 同右 & \{3.25ex plus 1ex minus .2ex\} \\
subparagraph & 同右 & \{3.25ex plus 1ex minus .2ex\} \\
\hline\hline
\end{tabular}
\end{center}

在~section~及以下的标题中,使用负的距离表示标题后的段落不缩进
(如标准的英文~LaTeX~文档),否则缩进。标题上方真正的空距是该参数的绝对值。

\item[afterskip=\{\parg{afterskip}\}]
用于控制章节标题后的空距。

该选项的缺省设置是
\begin{center}
\begin{tabular}{lll}
\hline\hline
   & 使用宏包选项~cap~ & 使用宏包选项~nocap~ \\
\hline
part (article) & 同右 & \{3ex\} \\
part & 无效 & 无效 \\
chapter & 同右 & \{40pt\} \\
section & 同右 & \{2.3ex plus .2ex\} \\
subsection & 同右 & \{1.5ex plus .2ex\} \\
subsubsection & 同右 & \{1.5ex plus .2ex\} \\
paragraph & 同右 & \{-1em\} \\
subparagraph & 同右 & \{-1em\} \\
\hline\hline
\end{tabular}
\end{center}

在~section~及以下的标题中,正的距离表示向下留出的空距(如标准的~section~标题),
使用负的距离则表示向右留出的空距的负值(如标准的~paragraph~标题)。

\item[indent=\{\parg{indent}\}]
用于控制章节标题本身的缩进。

该选项的缺省设置是
\begin{center}
\begin{tabular}{lll}
\hline\hline
   & 使用宏包选项~cap~ & 使用宏包选项~nocap~ \\
\hline
part (article) & 同右 & \{0pt\} \\
part & 无效 & 无效 \\
chapter & 同右 & \{0pt\} \\
section & 同右 & \{0pt\} \\
subsection & 同右 & \{0pt\} \\
subsubsection & 同右 & \{0pt\} \\
paragraph & 同右 & \{0pt\} \\
subparagraph & 同右 & \{\cs{parindent}\} \\
\hline\hline
\end{tabular}
\end{center}

\end{description}


\subsubsection{部分修改标题格式}

如果只想修改标题格式中的某些参数而不是完全重新设置,可以使用带~+~号的
设置选项。例如
\begin{quote}
|\CTEXsetup[format+={\zihao{1}}]{section}|
\end{quote}
则~section~的标题使用一号字体,而其他格式设置保持不变。

标题格式相关的选项都支持这一功能,包括~|format|, |nameformat|, |numberformat|,
|aftername|~和~|titleformat|,而且对所有文档类型都有效。

\subsubsection{附录标题设置}

附录(appendix)的标题也使用~|\CTEXsetup|~命令进行设置,
第二个参数设为~|appendix|。但是只能使用~|name|~和~|number|~
两个设置选项。在使用了~|\appendix|~命令之后,附录
的名字和编号会被自动使用。{\bf 附录的名字和前面的章节不同,
它只有一个部分,放在编号之前。}在~article~类文档中,
附录是用~section~实现的,而在~report~和~book~类文档中附录
使用的是~chapter~的设置。因此在设置附录的编号的时候要注意
使用正确的计数器。如果你要设置其他格式的附录标题,
可以根据使用的文档类直接用~section~或者~chapter~的设置命令来控制,
但是要记住把设置命令放在~|\appendix|~(如果有的话)的后面,
否则会被~|\appendix|~命令的设置覆盖。

附录的缺省设置是
\begin{center}
\begin{tabular}{lll}
\hline\hline
   & 使用宏包选项~cap~ & 使用宏包选项~nocap~ \\
\hline
name (article) & 同右 & \{\} \\
name & \{附录\textasciitilde\} & \{Appendix\cs{space}\} \\
number (article) & 同右 & \{\cs{Alph}\marg{section}\} \\
number & 同右 & \{\cs{Alph}\marg{chapter}\} \\
\hline\hline
\end{tabular}
\end{center}


\subsubsection{其他标题设置}

除章节标题外其他标题的设置通过~|\CTEXoptions|~设置。包括

\begin{description}
\item[contentsname] 目录名
\item[listfigurename] 表格目录
\item[listtablename] 插图目录
\item[figurename] 图
\item[tablename] 表
\item[abstractname] 摘要
\item[indexname] 索引
\item[bibname] 参考文献
\end{description}

例如
\begin{quote}
|\CTEXoptions[indexname={总索引}]|
\end{quote}
把索引的名字改为“总索引”。


\subsubsection{其他设置}

\paragraph{设置~\cs{today}~的日期格式}
使用~|\CTEXoptions|~可以设置~|\today|~命令产生的日期格式。
支持的格式包括

\begin{enumerate}

\item 阿拉伯数字加中文年月日
\begin{quote}
|\CTEXoptions[today=small]|
\end{quote}
\CTEXoptions[today=small]
|\today|~生成的日期例子为“\today”。

\item 中文数字加中文年月日
\begin{quote}
|\CTEXoptions[today=big]|
\end{quote}
\CTEXoptions[today=big]
|\today|~生成的日期例子为“\today”。

\item \LaTeX{}~标准格式
\begin{quote}
|\CTEXoptions[today=old]|
\end{quote}
\CTEXoptions[today=old]
|\today|~生成的日期例子为“\today”。

\end{enumerate}


\paragraph{设置图表标题的分隔符}
使用~|\CTEXoptions|~可以设置~|\caption|~命令产生的图表标题的分隔符。
这个分隔符缺省是使用冒号~:~。可以通过命令
\begin{quote}
|\CTEXoptions[captiondelimiter={|\parg{string}|}]|
\end{quote}
设置为任意的单个字符或者字符串~\parg{string}。


\subsection{配置文件}

主要的配置文件有~\texttt{ctex.def}~和~\texttt{ctexcap.cfg}~以及几个字体
定义文件~\texttt{*.fd}。字体定义文件的内容请参考~\ref{sec:fontdef}~的
内容。

\texttt{ctex.def}~是一些中文字符串参数的定义,会被所有的宏包使用。
如果你想改用其他的中文字符,例如繁体字,可以修改这个文件。

\texttt{ctexcap.cfg}~是缺省中文标题格式的定义,当你使用~\texttt{cap}~
选项时就会使用这里的定义。你可以把它改为你经常使用的格式,这样就不用
每次都在正文中修改了。~\texttt{ctexcap.cfg}~中的设置都可以通过宏包提供
的设置命令在正文中进行修改。

最后,宏包还将读入 \texttt{ctex.cfg} 文件,该文件中的设置将覆盖其他配置
文件中的设置。用户可以在该文件中加入自己的定义。

在修改这些配置文件的时候,你可以修改系统目录中的文件,也可以拷贝一份放到
当前目录下,然后修改。TeX 会优先使用当前目录下的同名文件。这样你可以针对
不同的应用设置不同的缺省配置文件。

\subsection{用标准字体命令修改中文字体}


\subsubsection{字体定义文件} \label{sec:fontdef}

本节的内容用于生成中文字体定义文件,这些定义文件将被~\ctex{}~宏包
做为缺省字体设置装入,用于在改变英文字体时相应的改变中文字体。
~\texttt{c19rm.fd}~文件定义的字体将在使用~|\rm|~系列字体命令时使用。
类似的,~\texttt{c19sf.fd}~文件定义的字体将在使用~|\sf|~系列字体命
令时使用;~\texttt{c19tt.fd}~文件定义的字体将在使用~|\tt|~系列字体
命令时使用。


首先,使用~\LaTeX{}~的~\texttt{NFSS}~命令定义新的字体名称,
都使用~GBK~编码。

\changes{v0.8a}{2007/05/06}{增加~bold~字体的定义}
然后定义在各种情况下对应的真正汉字字体。
中文正常字体加黑都采用黑体代替,意大利体采用楷书代替,
意大利体加黑采用隶书代替。

\texttt{rm}~字体中的普通字体采用宋体:
\begin{verbatim}
\DeclareFontShape{C19}{rm}{m}{n}{<-> CJK * gbksong}{}
\DeclareFontShape{C19}{rm}{b}{n}{<-> CJK * gbkhei}{}
\DeclareFontShape{C19}{rm}{bx}{n}{<-> CJK * gbkhei}{}
\DeclareFontShape{C19}{rm}{m}{sl}{<-> CJK * gbksongsl}{}
\DeclareFontShape{C19}{rm}{b}{sl}{<-> CJK * gbkheisl}{}
\DeclareFontShape{C19}{rm}{bx}{sl}{<-> CJK * gbkheisl}{}
\DeclareFontShape{C19}{rm}{m}{it}{<-> CJK * gbkkai}{}
\DeclareFontShape{C19}{rm}{b}{it}{<-> CJKb * gbkkai}{\CJKbold}
\DeclareFontShape{C19}{rm}{bx}{it}{<-> CJKb * gbkkai}{\CJKbold}
\end{verbatim}

\texttt{sf}~字体中的普通字体采用幼圆:
\begin{verbatim}
\DeclareFontShape{C19}{sf}{m}{n}{<-> CJK * gbkyou}{}
\DeclareFontShape{C19}{sf}{b}{n}{<-> CJKb * gbkyou}{\CJKbold}
\DeclareFontShape{C19}{sf}{bx}{n}{<-> CJKb * gbkyou}{\CJKbold}
\DeclareFontShape{C19}{sf}{m}{sl}{<-> CJK * gbkyousl}{}
\DeclareFontShape{C19}{sf}{b}{sl}{<-> CJKb * gbkyousl}{\CJKbold}
\DeclareFontShape{C19}{sf}{bx}{sl}{<-> CJKb * gbkyousl}{\CJKbold}
\DeclareFontShape{C19}{sf}{m}{it}{<-> CJK * gbkyou}{}
\DeclareFontShape{C19}{sf}{b}{it}{<-> CJKb * gbkyou}{\CJKbold}
\DeclareFontShape{C19}{sf}{bx}{it}{<-> CJKb * gbkyou}{\CJKbold}
\end{verbatim}

\texttt{tt}~字体中的普通字体采用仿宋:
\begin{verbatim}
\DeclareFontShape{C19}{tt}{m}{n}{<-> CJK * gbkfs}{}
\DeclareFontShape{C19}{tt}{b}{n}{<-> CJKb * gbkfs}{\CJKbold}
\DeclareFontShape{C19}{tt}{bx}{n}{<-> CJKb * gbkfs}{\CJKbold}
\DeclareFontShape{C19}{tt}{m}{sl}{<-> CJK * gbkfssl}{}
\DeclareFontShape{C19}{tt}{b}{sl}{<-> CJKb * gbkfssl}{\CJKbold}
\DeclareFontShape{C19}{tt}{bx}{sl}{<-> CJKb * gbkfssl}{\CJKbold}
\DeclareFontShape{C19}{tt}{m}{it}{<-> CJK * gbkfs}{}
\DeclareFontShape{C19}{tt}{b}{it}{<-> CJKb * gbkfs}{\CJKbold}
\DeclareFontShape{C19}{tt}{bx}{it}{<-> CJKb * gbkfs}{\CJKbold}
\end{verbatim}

这些字体对应关系以后有可能根据用户意见做出调整。

\Finale

\setcounter{IndexColumns}{2}
\IndexPrologue{\section*{索引} {\it 意大利体的数字表示描述对应索引项的页码;
               带下划线的数字表示定义对应索引项的代码行号;
               罗马字体的数字表示使用对应索引项的代码行号。}}

\GlossaryPrologue{\section*{版本更新}}

\PrintIndex \PrintChanges

\end{document}

