%# -*- coding: utf-8 -*-
% $Id$
\documentclass[cs4size,a4paper,fancyhdr,fntef,UTF8,winfonts]{ctexbook}
\usepackage{ifpdf}
\usepackage{ifxetex}
\ifxetex
  \usepackage[xetex]{hyperref}
\else
  \ifpdf
    \usepackage[pdftex,unicode]{hyperref}
  \else% Assuming dvipdfm(x)
    \usepackage[dvipdfm,unicode]{hyperref}
  \fi
\fi

\usepackage{shortvrb, makeidx}

 \makeindex

 \MakeShortVerb{\|}

 \topmargin -0.5 true cm
 \oddsidemargin 0 true cm
 \evensidemargin 0 true cm
 \textheight 23 true cm
 \textwidth 16.5 true cm

 \setlength{\parskip}{0.75ex plus .2ex minus .5ex}
 \renewcommand{\baselinestretch}{1.2}


 \pagestyle{fancy}


 \newcommand{\ctex}{\texttt{ctex}}
 \newcommand{\ctexorg}{\texttt{ctex.org}}

 \newcommand{\TestFile}{测试文件}


 \CTEXoptions[today=big]
 \CTEXsetup[name={第,节},number={\chinese{section}},format+={\bfseries},beforeskip={-10ex plus -.1ex minus -.1ex},afterskip={1ex plus .1ex minus .1ex}]{section}
 \CTEXsetup[name={附件},number={\chinese{chapter}}]{appendix}

 \CTEXsetup[name={第,篇},nameformat={\centering\bfseries},titleformat={\zihao{0}\bfseries}]{part}

 \setcounter{secnumdepth}{4}
 \CTEXsetup[name={(,)},number={\arabic{paragraph}}]{paragraph}


\begin{document}

\title{宏包~ctex~\TestFile\thanks{just test}}
\author{吴凌云}

\maketitle

\tableofcontents


\part{前言}

\chapter{简介} \label{chapter:intro}

简介

\section{说明}

\ctex{}~宏包提供了编写中文~\LaTeX{}~文档常用的一些宏定义和命令。
需要~CJK~宏包的支持,由~\ctexorg{}~制作并负责维护。

本文件用于测试~\ctex{}~宏包的命令和功能。


\part{正文}

\chapter{基本命令}

基本命令


\section{字号}
测试命令:|\zihao| \index{zihao}

\begin{tabular}{l}
\zihao{0}  初号字 English \\
\zihao{-0} 小初号 English \\
\zihao{1}  一号字 English \\
\zihao{-1} 小一号 English \\
\zihao{2}  二号字 English \\
\zihao{-2} 小二号 English \\
\zihao{3}  三号字 English \\
\zihao{-3} 小三号 English \\
\zihao{4}  四号字 English \\
\zihao{-4} 小四号 English \\
\zihao{5}  五号字 English \\
\zihao{-5} 小五号 English \\
\zihao{6}  六号字 English \\
\zihao{-6} 小六号 English \\
\zihao{7}  七号字 English \\
\zihao{8}  八号字 English \\
\end{tabular}

\zihao{0}
初号字初号字初号字初号字初号字初号字初号字初号字初号字初号字
初号字初号字初号字初号字初号字初号字初号字初号字初号字初号字
初号字初号字初号字初号字初号字初号字初号字初号字初号字初号字

\zihao{-0}
小初号小初号小初号小初号小初号小初号小初号小初号小初号小初号
小初号小初号小初号小初号小初号小初号小初号小初号小初号小初号
小初号小初号小初号小初号小初号小初号小初号小初号小初号小初号

\zihao{1}
一号字一号字一号字一号字一号字一号字一号字一号字一号字一号字
一号字一号字一号字一号字一号字一号字一号字一号字一号字一号字
一号字一号字一号字一号字一号字一号字一号字一号字一号字一号字

\zihao{-1}
小一号小一号小一号小一号小一号小一号小一号小一号小一号小一号
小一号小一号小一号小一号小一号小一号小一号小一号小一号小一号
小一号小一号小一号小一号小一号小一号小一号小一号小一号小一号

\zihao{2}
二号字二号字二号字二号字二号字二号字二号字二号字二号字二号字
二号字二号字二号字二号字二号字二号字二号字二号字二号字二号字
二号字二号字二号字二号字二号字二号字二号字二号字二号字二号字

\zihao{-2}
小二号小二号小二号小二号小二号小二号小二号小二号小二号小二号
小二号小二号小二号小二号小二号小二号小二号小二号小二号小二号
小二号小二号小二号小二号小二号小二号小二号小二号小二号小二号

\zihao{3}
三号字三号字三号字三号字三号字三号字三号字三号字三号字三号字
三号字三号字三号字三号字三号字三号字三号字三号字三号字三号字
三号字三号字三号字三号字三号字三号字三号字三号字三号字三号字

\zihao{-3}
小三号小三号小三号小三号小三号小三号小三号小三号小三号小三号
小三号小三号小三号小三号小三号小三号小三号小三号小三号小三号
小三号小三号小三号小三号小三号小三号小三号小三号小三号小三号

\zihao{4}
四号字四号字四号字四号字四号字四号字四号字四号字四号字四号字
四号字四号字四号字四号字四号字四号字四号字四号字四号字四号字
四号字四号字四号字四号字四号字四号字四号字四号字四号字四号字
四号字四号字四号字四号字四号字四号字四号字四号字四号字四号字
四号字四号字四号字四号字四号字四号字四号字四号字四号字四号字

\zihao{-4}
小四号小四号小四号小四号小四号小四号小四号小四号小四号小四号
小四号小四号小四号小四号小四号小四号小四号小四号小四号小四号
小四号小四号小四号小四号小四号小四号小四号小四号小四号小四号
小四号小四号小四号小四号小四号小四号小四号小四号小四号小四号
小四号小四号小四号小四号小四号小四号小四号小四号小四号小四号

\zihao{5}
五号字五号字五号字五号字五号字五号字五号字五号字五号字五号字
五号字五号字五号字五号字五号字五号字五号字五号字五号字五号字
五号字五号字五号字五号字五号字五号字五号字五号字五号字五号字
五号字五号字五号字五号字五号字五号字五号字五号字五号字五号字
五号字五号字五号字五号字五号字五号字五号字五号字五号字五号字

\zihao{-5}
小五号小五号小五号小五号小五号小五号小五号小五号小五号小五号
小五号小五号小五号小五号小五号小五号小五号小五号小五号小五号
小五号小五号小五号小五号小五号小五号小五号小五号小五号小五号
小五号小五号小五号小五号小五号小五号小五号小五号小五号小五号
小五号小五号小五号小五号小五号小五号小五号小五号小五号小五号

\zihao{6}
六号字六号字六号字六号字六号字六号字六号字六号字六号字六号字
六号字六号字六号字六号字六号字六号字六号字六号字六号字六号字
六号字六号字六号字六号字六号字六号字六号字六号字六号字六号字
六号字六号字六号字六号字六号字六号字六号字六号字六号字六号字
六号字六号字六号字六号字六号字六号字六号字六号字六号字六号字

\zihao{-6}
小六号小六号小六号小六号小六号小六号小六号小六号小六号小六号
小六号小六号小六号小六号小六号小六号小六号小六号小六号小六号
小六号小六号小六号小六号小六号小六号小六号小六号小六号小六号
小六号小六号小六号小六号小六号小六号小六号小六号小六号小六号
小六号小六号小六号小六号小六号小六号小六号小六号小六号小六号

\zihao{7}
七号字七号字七号字七号字七号字七号字七号字七号字七号字七号字
七号字七号字七号字七号字七号字七号字七号字七号字七号字七号字
七号字七号字七号字七号字七号字七号字七号字七号字七号字七号字
七号字七号字七号字七号字七号字七号字七号字七号字七号字七号字
七号字七号字七号字七号字七号字七号字七号字七号字七号字七号字

\zihao{8}
八号字八号字八号字八号字八号字八号字八号字八号字八号字八号字
八号字八号字八号字八号字八号字八号字八号字八号字八号字八号字
八号字八号字八号字八号字八号字八号字八号字八号字八号字八号字
八号字八号字八号字八号字八号字八号字八号字八号字八号字八号字
八号字八号字八号字八号字八号字八号字八号字八号字八号字八号字

\normalsize

\section{字距}

测试命令:|\ziju| \index{ziju}

标准字距为零,改变字距后:|\ziju{1}|

{\ziju{1}
现在的字距为~1~个字,英文不受影响:\\
This is an English example。
}

\section{字宽}

测试命令:|\ccwd| \index{ccwd}

当前汉字字宽是~\the\ccwd,即盒子~\framebox[\ccwd]{\ }~的宽度。

改变字号大小后:|\zihao{3}|

{\zihao{3}
当前汉字字宽是~\the\ccwd,即盒子~\framebox[\ccwd]{\ }~的宽度。
}

改变字距后:|\ziju{1}|

{\ziju{1}
当前汉字字宽是~\the\ccwd,即盒子~\framebox[\ccwd]{\ }~的宽度。
}

\section{段首缩进}

测试命令:|\CTEXindent| \index{CTEXindent}

标准的中文段首缩进是两个汉字:

这是缩进的行。

\noindent 这是没有缩进的行。

改变字号大小后:|\zihao{3}|

{\zihao{3}
这是缩进的行。

\noindent 这是没有缩进的行。
}

改变字距后:|\ziju{1}|

{\ziju{1}
这是缩进的行。

\noindent 这是没有缩进的行。
}

\section{字体命令}

\begin{tabular}{ll}
 |\songti| & {\songti 宋体} \\
 |\heiti| & {\heiti 黑体} \\
 |\fangsong| & {\fangsong 仿宋} \\
 |\kaishu| & {\kaishu 楷书} \\
 |\lishu| & {\lishu 隶书} \\
 |\youyuan| & {\youyuan 幼圆}
\end{tabular}

\index{songti} \index{heiti} \index{fangsong} \index{kaishu} \index{lishu} \index{youyuan}

\section{字体框架}

\begin{center}
\begin{tabular}{lllll}

\hline
 Family & Series & Shape & 命令 & 例子 \\ \hline
 {\bf rm} &    &    & |\rmfamily| &
                      {\rmfamily English 中文字体} \\
          &    & it & |\rmfamily\itshape| &
                      {\rmfamily\itshape English 中文字体} \\
          &    & sl & |\rmfamily\slshape| &
                      {\rmfamily\slshape English 中文字体} \\ \cline{2-5}
          & bf &    & |\rmfamily\bfseries| &
                      {\rmfamily\bfseries English 中文字体} \\
          & bf & it & |\rmfamily\bfseries\itshape| &
                      {\rmfamily\bfseries\itshape English 中文字体} \\
          & bf & sl & |\rmfamily\bfseries\slshape| &
                      {\rmfamily\bfseries\slshape English 中文字体} \\ \hline

 {\bf sf} &    &    & |\sffamily| &
                      {\sffamily English 中文字体} \\
          &    & it & |\sffamily\itshape| &
                      {\sffamily\itshape English 中文字体} \\
          &    & sl & |\sffamily\slshape| &
                      {\sffamily\slshape English 中文字体} \\ \cline{2-5}
          & bf &    & |\sffamily\bfseries| &
                      {\sffamily\bfseries English 中文字体} \\
          & bf & it & |\sffamily\bfseries\itshape| &
                      {\sffamily\bfseries\itshape English 中文字体} \\
          & bf & sl & |\sffamily\bfseries\slshape| &
                      {\sffamily\bfseries\slshape English 中文字体} \\ \hline

 {\bf tt} &    &    & |\ttfamily| &
                      {\ttfamily English 中文字体} \\
          &    & it & |\ttfamily\itshape| &
                      {\ttfamily\itshape English 中文字体} \\
          &    & sl & |\ttfamily\slshape| &
                      {\ttfamily\slshape English 中文字体} \\ \cline{2-5}
          & bf &    & |\ttfamily\bfseries| &
                      {\ttfamily\bfseries English 中文字体} \\
          & bf & it & |\ttfamily\bfseries\itshape| &
                      {\ttfamily\bfseries\itshape English 中文字体} \\
          & bf & sl & |\ttfamily\bfseries\slshape| &
                      {\ttfamily\bfseries\slshape English 中文字体} \\ \hline
\end{tabular}
\end{center}


\chapter{高级命令}

\section{中文数字}

测试命令:|\CTEXnumber| |\CTEXdigits| \index{CTEXnumber} \index{CTEXdigits}

\CTEXnumber{\test}{100002005}
\typeout{\test}
\test

\CTEXdigits{\test}{100002005}
\typeout{\test}
\test

\section{中文引用}

测试命令:|\ref| |\ref*| \index{ref} \index{ref*}

我们引用的是:\ref{chapter:intro}。

我们引用的是:\ref*{chapter:intro}。(不带超链接)

\section{索引}

测试命令:|\index| \index{index}

\index{test1}
\index{test2}
\index{test3}
\index{test4}
\index{test5}
\index{test6}
\index{test7}
\index{test8}
\index{test9}
\index{test10}
\index{test11}
\index{test12}
\index{test13}
\index{test14}
\index{test15}
\index{test16}
\index{test17}
\index{test18}
\index{test19}
\index{test20}
\index{test21}
\index{test22}
\index{test23}
\index{test24}
\index{test25}
\index{test26}
\index{test27}
\index{test28}
\index{test29}
\index{test30}
\index{test31}
\index{test32}
\index{test33}
\index{test34}
\index{test35}
\index{test36}
\index{test37}
\index{test38}
\index{test39}
\index{test40}
\index{test41}
\index{test42}
\index{test43}
\index{test44}
\index{test45}
\index{test46}
\index{test47}
\index{test48}
\index{test49}
\index{test50}


\chapter{中文标点}

一千九百八十年间,西京城里出了桩异事,两个关系是死死的朋友,一日活得泼烦,去
了唐贵妃杨玉环的墓地凭吊,见许多游人都抓了一包坟丘的土携在怀里,甚感疑惑,询
问了,才知贵妃是绝代佳人,这土拿回去撒入花盆,花就十分鲜艳。这二人遂也刨了许
多,用衣包回,装在一只收藏了多年的黑陶盆里,只待有了好的花籽来种。没想,数天
之后,盆里兀自生出绿芽,月内长大,竟蓬蓬勃勃了一丛,但这草木特别,无人能识得
品类。抱了去城中孕璜寺的老花工请教,花工也是不识。恰有智祥大师经过,又请教大
师,大师还是摇头。其中一人却说:“常闻大师能卜卦预测,不妨占这花将来能开几
枝?”大师命另一人取一个字来,那人适持花工的剪刀在手,随口说出个“耳”字。大
师说:“花是奇花,当开四枝,但其景不久,必为尔所残也。”后花开果然如数,但形
状类似牡丹,又类似玫瑰。且一枝蕊为红色,一枝蕊为黄色,一枝蕊为白色,一枝蕊为
紫色,极尽娇美。一时消息传开每日欣赏者不绝,莫不叹为观止。两个朋友自然得意,
尤其一个更是珍惜,供养案头,亲自浇水施肥,殷勤务弄。不料某日醉酒,夜半醒来忽
觉得该去浇灌,竟误把厨房炉子上的热水壶提去,结果花被浇死。此人悔恨不已,索性
也摔了陶盆,生病睡倒一月不起。

\paragraph{第一段}
此事虽异,毕竟为一盆花而已,知道之人还并不广大,过后也便罢了。没想到了夏天,
西京城却又发生了一桩更大的人人都经历的异事。是这古历六月初七的晌午,先是太阳
还红堂堂地照着,太阳的好处是太阳照着而人却忘记了还有太阳在照着,所以这个城里
的人谁也没有往天上去看。街面的形势依旧是往日形势。有级别坐卧车的坐着卧车。没
级别的,但有的是钱,便不愿挤那公共车了,抖着票子去搭出租车。偏偏有了什么重要
的人物亲临到这里,数辆的警车护卫开道,尖锐的警笛就长声儿价地吼,所有的卧车,
出租车、公共车只得靠边慢行,扰乱了自行车长河的节奏。只有徒步的人只管徒步,你
踩着我的影子,我踩着他的影子,影子是不痛不痒的。突然。影子的颜色由深而浅,愈
浅愈短,一瞬间全然消失。人没有了阴影拖着,似乎人不是了人,用手在屁股后摸摸,
摸得一脸的疑惑。有人就偶尔往天上一瞅,立即欢呼:“天上有四个太阳了!”人们全
举了头往天上看,天上果然出现了四个太阳。四个太阳大小一般,分不清了新旧雌雄,
是聚在一起的,组成个丁字形。过去的经验里,天上是有过月亏和日蚀的,但同时有四
个太阳却没有遇过,以为是眼睛看错了;再往天上看,那太阳就不再发红,是白的,白
得像电焊光一样的白,白得还像什么?什么就也看不见了,完全的黑暗人是看不见了什
么的,完全的光明人竟也是看不见了什么吗?大小的车辆再不敢发动了,只鸣喇叭,人
却胡扑乱踏,恍惚里甚或就感觉身已不在街上了,是在看电影吧?放映机突然发生故障,
银幕上的图象消失了,而音响还在进行着。一个人这么感觉了,所有的人差不多也都这
么感觉了,于是寂静下来,竟静得死气沉沉,唯有城墙头上有人吹动的埙音最后要再吹
一声,但没有吹起,是力气用完,像风撞在墙角,拐了一下,消失了。人们似乎看不起
吹埙的人,笑了一下,猛地惊醒身处的现实,同时被寂静所恐惧,哇哇惊叫,各处便疯
倒了许多。

\paragraph{第二段}
这样的怪异持续了近半个小时,天上的太阳又恢复成了一个。待人们的眼睛逐渐看见地
上有了自己的影子,皆面面相觑,随之倒为人的狼狈有了羞愧,就慌不择路地四散。一
时又是人乱如蚁,却不见了指挥交通的警察。安全岛上,悠然独坐的竟是一个老头。老
头囚首垢面,却有一双极长的眉眼,冷冷地看着人的忙忙。这眼神使大家有些受不得,
终就愤怒了,遂喊警察呢?警察在哪儿,姓苏的警察就一边跑一边戴头上的硬壳帽子,
骂着老叫花子:“pi!”“pi”是西京城里骂“滚”的最粗俗的土话。老头听了,
拿手指在安全岛上写,写出来却是一个极文雅的上古词:避,就慢慢地笑了。随着笑起
来的是一大片,因为老头走下安全岛的时候、暴露了身上的衣服原是孕璜寺香客敬奉的
锦旗所制。前心印着“有求”两字,那双腿岔开,裤裆处是粗糙的大针脚一直到了后腰,
屁股蛋上左边就是个“必”字,右边就是个“应”字,老头并不知耻,却出口成章;说
出了一段谣儿来。这谣儿后来流传全城,其辞是:一类人是公仆,高高在上享清福。二
类人作“官倒”,投机倒把有人保。三类人搞承包,吃喝嫖赌全报销。四类人来租赁,
坐在家里拿利润。五类人大盖帽,吃了原告吃被告。六类人手术刀,腰里揣满红纸包。
七类人当演员,扭扭屁股就赚钱。八类人搞宣传,隔三岔五解个馋。九类人为教员,山
珍海味认不全。十类人主人翁,老老实实学雷锋。

此谣儿流传开来后,有人分析老头并不是个乞丐,或者说他起码是个教师,因为只有教
师才能编出这样的谣辞,且谣辞中对前几类人都横加指责,唯独为教师一类人喊苦叫屈。
但到底老头是什么人,无人再作追究。这一年里,恰是西京城里新任了一位市长,这市
长原籍上海,夫人却是西京土著,十数春秋,西京的每任市长都有心在这座古城建功立
业,但却差不多全是几经折腾,起色甚微,便铁打的营盘流水的官去了。新的市长虽不
悦意在岳父门前任职,苦于身在仕途,全然由不得自己,到任后就犯难该从何处举纲张
目。夫人属于贤内助,便召集了许多亲朋好友为其夫顾问参谋,就有了一个年轻人叫黄
德复的,说出了一段建议来:西京是十二朝古都,文化积淀深厚是资本也是负担。各层
干部和群众思维趋于保守,故长期以来经济发展比沿海省市远远落后,若如前几任的市
长那样面面俱抓,常因企业老化,城建欠帐大多、用尽十分力,往往只有三分效果,且
当今任职总是三年或五载就得调动,长远规划难以完成便又人事更新;与其这样,倒不
如抓别人不抓之业,如发展文化和旅游,短期内倒有政绩出现。市长大受启发,不耻下
问,竟邀这年轻人谈了三天三夜,又将其调离原来任职的学校来市府作了身边秘书。一
时间,上京索要拨款,在下四处集资,干了一宗千古不朽之宏业,即修复了西京城墙,
疏通了城河,沿城河边建成极富地方特色的娱乐场。又改建了三条大街:一条为仿唐建
筑街,专售书画、瓷器;一条为仿宋建筑街,专营全市乃至全省民间小吃;一条仿明、
清建筑街,集中了所有民间工艺品、土特产。但是,城市文化旅游业的大力发展,使城
市的流动人员骤然增多,就出现了许多治安方面的弊病,一时西京城被外地人称作贼城、
烟城、暗娼城。市民也开始滋生另一种的不满情绪。当那位囚首垢面的老头又在街头说
他的谣儿,身后总是厮跟了一帮闲汉,嚷道:“来一段,再来一段!”,老头就说了两
句:“说你行,你就行,不行也行。说不行,就不行,行也不行。”闲汉们听了,一齐
鼓掌。老头并没说这谣儿所指何人,闲汉们却对号入座,将这谣儿传得风快,自然黄德
复不久也听到了,便给公安局拨了电话,说老头散布市长的谣言,应予制止。公安局收
留了老头,一查,原是一位十多年上访痞子。为何是上访痞子?因是此人十多年前任民
办教师,转公办教师时受到上司陷害未能转成,就上访省府,仍未能成功,于是长住西
京,隔三间五去省府门口提意见,递状书,静坐耍赖,慢慢地欲进没有门路,欲退又无
台阶,精神变态,后来也索性不再上访。亦不返乡,就在街头流浪起来。公安局收审了
十天、查无大罪,又放出来,用车一气拉出城三百里地放下。没想这老头几天后又出现
在街头,却拉动了一辆架子车,沿街穿巷收拾破烂了。一帮闲汉自然拥他,唆使再说谣
儿,老头却吝啬了口舌,只吼很高很长的“破烂喽-!承包破烂-喽!”这叫声每日
早晚在街巷吼叫。常也有人在城墙头上吹埙,一个如狼嚎,一个鸣咽如鬼,两厢呼应,
钟楼鼓楼上的成百上千只鸟类就聒噪一片了。


\chapter{CJKfntef~和~CCTfntef}

\CTEXunderdot{此谣儿流传开来后}

\CTEXunderline{此谣儿流传开来后}

\CTEXunderdblline{此谣儿流传开来后}

\CTEXunderwave{此谣儿流传开来后}

\CTEXsout{此谣儿流传开来后}

\CTEXxout{此谣儿流传开来后}


\begin{CTEXfilltwosides}{3cm}
此谣 \\
儿流 \\
传开 \\
来后
\end{CTEXfilltwosides}


\chapter{其他测试}

其他测试

\chapter{其他测试}

其他测试

\chapter{其他测试}

其他测试

\chapter{其他测试}

其他测试

\chapter{其他测试}

其他测试

\chapter{其他测试}

其他测试

\chapter{其他测试}

其他测试

\chapter{其他测试}

其他测试

\chapter{其他测试}

其他测试

\chapter{其他测试}

其他测试

\chapter{其他测试}

其他测试

\chapter{其他测试}

其他测试

\chapter{其他测试}

其他测试

\chapter{其他测试}

其他测试

\chapter{其他测试}

其他测试

\chapter{其他测试}

其他测试

\chapter{其他测试}

其他测试

\chapter{其他测试}

其他测试

\chapter{其他测试}

其他测试

\chapter{其他测试}

其他测试

\chapter{其他测试}

其他测试

\chapter{其他测试}

其他测试

\chapter{其他测试}

其他测试

\chapter{其他测试}

其他测试

\chapter{其他测试}

其他测试

\chapter{其他测试}

其他测试

\chapter{其他测试}

其他测试

\chapter{其他测试}

其他测试

\chapter{其他测试}

其他测试

\chapter{其他测试}

其他测试

\chapter{其他测试}

其他测试

\chapter{其他测试}

其他测试

\chapter{其他测试}

其他测试

\chapter{其他测试}

其他测试

\chapter{其他测试}

其他测试

\chapter{其他测试}

其他测试

\chapter{其他测试}

其他测试

\chapter{其他测试}

其他测试

\chapter{其他测试}

其他测试

\chapter{其他测试}

其他测试

\chapter{其他测试}

其他测试

\chapter{其他测试}

其他测试

\chapter{其他测试}

其他测试

\chapter{其他测试}

其他测试

\chapter{其他测试}

其他测试

\chapter{其他测试}

其他测试

\chapter{其他测试}

其他测试

\chapter{其他测试}

其他测试

\chapter{其他测试}

其他测试

\chapter{其他测试}

其他测试

\chapter{其他测试}

其他测试

\chapter{其他测试}

其他测试

\chapter{其他测试}

其他测试

\chapter{其他测试}

其他测试

\chapter{其他测试}

其他测试

\chapter{其他测试}

其他测试

\chapter{其他测试}

其他测试

\chapter{其他测试}

其他测试

\chapter{其他测试}

其他测试

\chapter{其他测试}

其他测试

\chapter{其他测试}

其他测试

\chapter{其他测试}

其他测试

\chapter{其他测试}

其他测试

\chapter{其他测试}

其他测试

\chapter{其他测试}

其他测试

\chapter{其他测试}

其他测试

\chapter{其他测试}

其他测试

\chapter{其他测试}

其他测试

\chapter{其他测试}

其他测试

\chapter{其他测试}

其他测试

\chapter{其他测试}

其他测试

\chapter{其他测试}

其他测试

\chapter{其他测试}

其他测试

\chapter{其他测试}

其他测试

\chapter{其他测试}

其他测试


\appendix

\chapter{测试}

测试

\chapter{测试}

测试

\chapter{测试}

测试


\printindex

\end{document}
