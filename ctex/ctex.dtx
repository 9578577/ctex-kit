% \iffalse meta-comment
% !TeX program  = XeLaTeX
% !TeX encoding = UTF-8
%
% Copyright (C) 2003--2015
% CTEX.ORG and any individual authors listed elsewhere in this file.
% --------------------------------------------------------------------------
%
% This work may be distributed and/or modified under the
% conditions of the LaTeX Project Public License, either
% version 1.3c of this license or (at your option) any later
% version. This version of this license is in
%    http://www.latex-project.org/lppl/lppl-1-3c.txt
% and the latest version of this license is in
%    http://www.latex-project.org/lppl.txt
% and version 1.3 or later is part of all distributions of
% LaTeX version 2005/12/01 or later.
%
% This work has the LPPL maintenance status `maintained'.
%
% The Current Maintainers of this work are Leo Liu, Qing Lee and Liam Huang.
% --------------------------------------------------------------------------
%
%<*internal>
\iffalse
%</internal>
%<*readme>
ctex is a collection of macro packages and document classes
for LaTeX Chinese typesetting.

This package is licensed in LPPL.

The authors and contributors of this package are:

    * Wu Lingyun <aloft@ctex.org>
    * Jiang Jiang <gzjjgod@gmail.com>
    * Wang Yue <yuleopen@gmail.com>
    * Liu Haiyang <leoliu.pku@gmail.com>
    * Li Yanrui <liyanrui.m2@gmail.com>
    * Chen Zhichu <zhichu.chen@gmail.com>
    * Li Qing <sobenlee@gmail.com>
    * Liam Huang <liamhuang0205@gmail.com>

If you are interested in the process of development you
may observe

    https://github.com/CTeX-org/ctex-kit

Report feedback in the Issues section of ctex-kit project,
or in [ctex](http://bbs.ctex.org) forum.

This package consists of the file ctex.dtx, and the derived files

    ctex.pdf,
    ctex.ins,
    ctex.sty,
    ctexcap.sty,
    ctexsize.sty,
    ctexart.cls,
    ctexbook.cls,
    ctexrep.cls,
    ctex-article.def,
    ctex-book.def,
    ctex-report.def,
    ctexcap-gbk.cfg,
    ctexcap-utf8.cfg,
    ctex.cfg,
    ctexopts.cfg,
    ctex-engine-pdftex.def,
    ctex-engine-xetex.def,
    ctex-engine-luatex.def,
    c19rm.fd,
    c19sf.fd,
    c19tt.fd,
    c70rm.fd,
    c70sf.fd,
    c70tt.fd,
    ctex-fontset-windows.def,
    ctex-fontset-windowsnew.def,
    ctex-fontset-windowsold.def,
    ctex-fontset-adobe.def,
    ctex-fontset-fandol.def,
    ctex-fontset-mac.def,
    ctex-fontset-founder.def,
    ctex-fontset-ubuntu.def,
    ctexspa.def,
    ctexpunct.spa,
    ctexmakespa.tex,
    ctexspamacro.tex,
    zhwinfonts.tex,
    zhadobefonts.tex,
    zhfandolfonts.tex,
    zhfounderfonts.tex,
    zhubuntufonts.tex, and
    README (this file).

%</readme>
%<*internal>
\fi
\begingroup
  \def\temp{LaTeX2e}
\expandafter\endgroup\ifx\temp\fmtname\else
\csname fi\endcsname
%</internal>
%<*install>

\newread\inputcheck
\openin\inputcheck=ctex.ver
\ifeof\inputcheck
  \def\ctexPutVersion{\string\GetIdInfo$Id$}
\else
  \input ctex.ver
\fi
\closein\inputcheck

\input l3docstrip.tex

\def\checkOption<#1{%
  \ifcase
    \ifx*#10\else \ifx/#11\else
    \ifx+#12\else \ifx-#13\else
    \ifx<#14\else \ifx @#15\else
    \ifx.#16\else
    7\fi\fi\fi\fi\fi\fi\fi\relax
  \expandafter\starOption\or
  \expandafter\slashOption\or
  \expandafter\plusOption\or
  \expandafter\minusOption\or
  \expandafter\verbOption\or
  \expandafter\moduleOption\or
  \expandafter\expOption\or
  \expandafter\doOption\fi
  #1%
}
\def\expOption .#1>#2\endLine{%
  \maybeMsg{<.#1 . >}%
  \Evaluate{#1}%
  \begingroup
    \catcode`\\=0 \catcode`\{=1 \catcode`\}=2
    \xdef\expandedLineStuff{\scantokens{#2\noexpand}}%
  \endgroup
  \def\do##1##2##3{%
    \if1\Expr{##2}\StreamPut##1{\expandedLineStuff}\fi
    }%
  \activefiles
  }


\keepsilent
\askforoverwritefalse

\preamble

    Copyright (C) 2003-2015
    CTEX.ORG and any individual authors listed in the documentation.
------------------------------------------------------------------------------

    This work may be distributed and/or modified under the
    conditions of the LaTeX Project Public License, either
    version 1.3c of this license or (at your option) any later
    version. This version of this license is in
       http://www.latex-project.org/lppl/lppl-1-3c.txt
    and the latest version of this license is in
       http://www.latex-project.org/lppl.txt
    and version 1.3 or later is part of all distributions of
    LaTeX version 2005/12/01 or later.

    This work has the LPPL maintenance status `maintained'.

    The Current Maintainers of this work are Leo Liu, Qing Lee and Liam Huang.
------------------------------------------------------------------------------

\endpreamble

\postamble

    This package consists of the file  ctex.dtx,
                 and the derived files ctex.pdf,
                                       ctex.ins,
                                       ctex.sty,
                                       ctexcap.sty,
                                       ctexsize.sty,
                                       ctexart.cls,
                                       ctexbook.cls,
                                       ctexrep.cls,
                                       ctex-article.def,
                                       ctex-book.def,
                                       ctex-report.def,
                                       ctexcap-gbk.cfg,
                                       ctexcap-utf8.cfg,
                                       ctex.cfg,
                                       ctexopts.cfg,
                                       ctex-engine-pdftex.def,
                                       ctex-engine-xetex.def,
                                       ctex-engine-luatex.def,
                                       c19rm.fd,
                                       c19sf.fd,
                                       c19tt.fd,
                                       c70rm.fd,
                                       c70sf.fd,
                                       c70tt.fd,
                                       ctex-fontset-windows.def,
                                       ctex-fontset-windowsnew.def,
                                       ctex-fontset-windowsold.def,
                                       ctex-fontset-adobe.def,
                                       ctex-fontset-fandol.def,
                                       ctex-fontset-mac.def,
                                       ctex-fontset-founder.def,
                                       ctex-fontset-ubuntu.def,
                                       ctexspa.def,
                                       ctexpunct.spa,
                                       ctexmakespa.tex,
                                       ctexspamacro.tex,
                                       zhwinfonts.tex,
                                       zhadobefonts.tex,
                                       zhfandolfonts.tex,
                                       zhfounderfonts.tex,
                                       zhubuntufonts.tex, and
                                       README.
\endpostamble

\declarepostamble\emptypostamble
\endpostamble


\generate
  {
    \usedir{tex/latex/ctex}
    \file{ctex.sty}                    {\from{\jobname.dtx}{package,style}}
    \file{ctexcap.sty}                 {\from{\jobname.dtx}{package,ctexcap}}
    \file{ctexsize.sty}                {\from{\jobname.dtx}{package,ctexsize}}
    \file{ctexart.cls}                 {\from{\jobname.dtx}{class,article}}
    \file{ctexbook.cls}                {\from{\jobname.dtx}{class,book}}
    \file{ctexrep.cls}                 {\from{\jobname.dtx}{class,report}}
    \usepostamble\emptypostamble
    \file{ctex-article.def}            {\from{\jobname.dtx}{heading,article}}
    \file{ctex-book.def}               {\from{\jobname.dtx}{heading,book}}
    \file{ctex-report.def}             {\from{\jobname.dtx}{heading,report}}
    \file{ctexcap-gbk.cfg}             {\from{\jobname.dtx}{GBK}}
    \file{ctexcap-utf8.cfg}            {\from{\jobname.dtx}{UTF8}}
    \file{ctex.cfg}                    {\from{\jobname.dtx}{config}}
    \file{ctexopts.cfg}                {\from{\jobname.dtx}{ctexopts}}
    \file{ctex-engine-pdftex.def}      {\from{\jobname.dtx}{pdftex}}
    \file{ctex-engine-xetex.def}       {\from{\jobname.dtx}{xetex}}
    \file{ctex-engine-luatex.def}      {\from{\jobname.dtx}{luatex}}
    \file{c19rm.fd}                    {\from{\jobname.dtx}{rm,c19}}
    \file{c19sf.fd}                    {\from{\jobname.dtx}{sf,c19}}
    \file{c19tt.fd}                    {\from{\jobname.dtx}{tt,c19}}
    \file{c70rm.fd}                    {\from{\jobname.dtx}{rm,c70}}
    \file{c70sf.fd}                    {\from{\jobname.dtx}{sf,c70}}
    \file{c70tt.fd}                    {\from{\jobname.dtx}{tt,c70}}
    \file{ctex-fontset-windows.def}    {\from{\jobname.dtx}{fontset,windows}}
    \file{ctex-fontset-windowsnew.def} {\from{\jobname.dtx}{fontset,windowsnew}}
    \file{ctex-fontset-windowsold.def} {\from{\jobname.dtx}{fontset,windowsold}}
    \file{ctex-fontset-adobe.def}      {\from{\jobname.dtx}{fontset,adobe}}
    \file{ctex-fontset-fandol.def}     {\from{\jobname.dtx}{fontset,fandol}}
    \file{ctex-fontset-mac.def}        {\from{\jobname.dtx}{fontset,mac}}
    \file{ctex-fontset-founder.def}    {\from{\jobname.dtx}{fontset,founder}}
    \file{ctex-fontset-ubuntu.def}     {\from{\jobname.dtx}{fontset,ubuntu}}
    \file{ctexspa.def}
      {
        \from{\jobname.dtx}  {ctexspa}
        \from{ctexpunct.spa} {}
      }
    \file{ctexmakespa.tex}             {\from{\jobname.dtx}{spa,make}}
    \file{ctexspamacro.tex}            {\from{\jobname.dtx}{spa,macro}}
    \file{zhwinfonts.tex}              {\from{\jobname.dtx}{zhmap,windows}}
    \file{zhadobefonts.tex}            {\from{\jobname.dtx}{zhmap,adobe}}
    \file{zhfandolfonts.tex}           {\from{\jobname.dtx}{zhmap,fandol}}
    \file{zhfounderfonts.tex}          {\from{\jobname.dtx}{zhmap,founder}}
    \file{zhubuntufonts.tex}           {\from{\jobname.dtx}{zhmap,ubuntu}}
    \usedir{source/latex/ctex}
    \file{\jobname.ins}                {\from{\jobname.dtx}{install}}
    \nopreamble\nopostamble
    \usedir{doc/latex/ctex}
    \file{README.txt}                  {\from{\jobname.dtx}{readme}}
  }

\catcode32=12\space

\Msg{*************************************************************}
\Msg{*                                                           *}
\Msg{* To finish the installation you have to move the following *}
\Msg{* file into proper directories searched by TeX:             *}
\Msg{*                                                           *}
\Msg{* The recommended directory is TDS:tex/latex/ctex           *}
\Msg{*                                                           *}
\Msg{*     ctex.sty                                              *}
\Msg{*     ctexcap.sty                                           *}
\Msg{*     ctexsize.sty                                          *}
\Msg{*     ctexart.cls                                           *}
\Msg{*     ctexbook.cls                                          *}
\Msg{*     ctexrep.cls                                           *}
\Msg{*     ctex-article.def                                      *}
\Msg{*     ctex-book.def                                         *}
\Msg{*     ctex-report.def                                       *}
\Msg{*     ctexcap-gbk.cfg                                       *}
\Msg{*     ctexcap-utf8.cfg                                      *}
\Msg{*     ctex.cfg                                              *}
\Msg{*     ctexopts.cfg                                          *}
\Msg{*     ctex-engine-pdftex.def                                *}
\Msg{*     ctex-engine-xetex.def                                 *}
\Msg{*     ctex-engine-luatex.def                                *}
\Msg{*     c19rm.fd                                              *}
\Msg{*     c19sf.fd                                              *}
\Msg{*     c19tt.fd                                              *}
\Msg{*     c70rm.fd                                              *}
\Msg{*     c70sf.fd                                              *}
\Msg{*     c70tt.fd                                              *}
\Msg{*     ctex-fontset-windows.def                              *}
\Msg{*     ctex-fontset-windowsnew.def                           *}
\Msg{*     ctex-fontset-windowsold.def                           *}
\Msg{*     ctex-fontset-adobe.def                                *}
\Msg{*     ctex-fontset-fandol.def                               *}
\Msg{*     ctex-fontset-mac.def                                  *}
\Msg{*     ctex-fontset-founder.def                              *}
\Msg{*     ctex-fontset-ubuntu.def                               *}
\Msg{*     ctexspa.def                                           *}
\Msg{*     ctexmakespa.tex                                       *}
\Msg{*     ctexspamacro.tex                                      *}
\Msg{*     zhwinfonts.tex                                        *}
\Msg{*     zhadobefonts.tex                                      *}
\Msg{*     zhfandolfonts.tex                                     *}
\Msg{*     zhfounderfonts.tex                                    *}
\Msg{*     zhubuntufonts.tex                                     *}
\Msg{*                                                           *}
\Msg{* To produce the documentation run the file ctex.dtx        *}
\Msg{* through XeLaTeX.                                          *}
\Msg{*                                                           *}
\Msg{* Happy TeXing!                                             *}
\Msg{*                                                           *}
\Msg{*************************************************************}

\endbatchfile
%</install>
%<*internal>
\fi
%</internal>
%<*!(readme|install|zhmap|spa)>
%<*!(c19|c70|ctexspa)>
%<*driver|package|class>
\NeedsTeXFormat{LaTeX2e}
\RequirePackage{expl3}
%</driver|package|class>
%<.!(readme|install|zhmap|spa|c19|c70|ctexspa)>\ctexPutVersion
%<*driver>
\InputIfFileExists{ctex.ver}{}{
  \def\ctexGetVersionInfo{\GetIdInfo$Id$}}
\ctexGetVersionInfo
  {ctex source file}
\ProvidesExplFile{\ExplFileName.\ExplFileExtension}
%</driver>
%<style>  {Chinese adapter in LaTeX (CTEX)}
%<style>\ProvidesExplPackage{\ExplFileName}
%<ctexcap>  {Chinese adapter in LaTeX (CTEX)}
%<ctexcap>\ProvidesExplPackage{ctexcap}
%<ctexsize>  {Chinese font size definition (CTEX)}
%<ctexsize>\ProvidesExplPackage{ctexsize}
%<article&!heading>  {Chinese adapter for class article (CTEX)}
%<article&!heading>\ProvidesExplClass{ctexart}
%<book&!heading>  {Chinese adapter for class book (CTEX)}
%<book&!heading>\ProvidesExplClass{ctexbook}
%<report&!heading>  {Chinese adapter for class report (CTEX)}
%<report&!heading>\ProvidesExplClass{ctexrep}
%<article&heading>  {Heading modification for article (CTEX)}
%<article&heading>\ProvidesExplFile{ctex-article.def}
%<book&heading>  {Heading modification for book (CTEX)}
%<book&heading>\ProvidesExplFile{ctex-book.def}
%<report&heading>  {Heading modification for report (CTEX)}
%<report&heading>\ProvidesExplFile{ctex-report.def}
%<GBK>  {Caption with encoding GBK (CTEX)}
%<GBK>\ProvidesExplFile{ctexcap-gbk.cfg}
%<UTF8>  {Caption with encoding UTF8 (CTEX)}
%<UTF8>\ProvidesExplFile{ctexcap-utf8.cfg}
%<config>  {Configuration file (CTEX)}
%<config>\ProvidesExplFile{\ExplFileName.cfg}
%<ctexopts>  {Option configuration file (CTEX)}
%<ctexopts>\ProvidesExplFile{ctexopts.cfg}
%<pdftex>  {(pdf)LaTeX adapter (CTEX)}
%<pdftex>\ProvidesExplFile{ctex-engine-pdftex.def}
%<xetex>  {XeLaTeX adapter (CTEX)}
%<xetex>\ProvidesExplFile{ctex-engine-xetex.def}
%<luatex>  {LuaLaTeX adapter (CTEX)}
%<luatex>\ProvidesExplFile{ctex-engine-luatex.def}
%<windows>  {Windows fonts definition (CTEX)}
%<windows>\ProvidesExplFile{ctex-fontset-windows.def}
%<windowsnew>  {Windows fonts definition for Vista or later version (CTEX)}
%<windowsnew>\ProvidesExplFile{ctex-fontset-windowsnew.def}
%<windowsold>  {Windows fonts definition for XP or earlier version (CTEX)}
%<windowsold>\ProvidesExplFile{ctex-fontset-windowsold.def}
%<adobe>  {Adobe fonts definition (CTEX)}
%<adobe>\ProvidesExplFile{ctex-fontset-adobe.def}
%<fandol>  {Fandol fonts definition (CTEX)}
%<fandol>\ProvidesExplFile{ctex-fontset-fandol.def}
%<mac>  {Mac OS X fonts definition (CTEX)}
%<mac>\ProvidesExplFile{ctex-fontset-mac.def}
%<founder>  {Founder fonts definition (CTEX)}
%<founder>\ProvidesExplFile{ctex-fontset-founder.def}
%<ubuntu>  {Ubuntu fonts definition (CTEX)}
%<ubuntu>\ProvidesExplFile{ctex-fontset-ubuntu.def}
  {\ExplFileDate}{2.0}{\ExplFileDescription}
%</!(c19|c70|ctexspa)>
%<rm&c19>\ProvidesFile{c19rm.fd}%
%<sf&c19>\ProvidesFile{c19sf.fd}%
%<tt&c19>\ProvidesFile{c19tt.fd}%
%<rm&c70>\ProvidesFile{c70rm.fd}%
%<sf&c70>\ProvidesFile{c70sf.fd}%
%<tt&c70>\ProvidesFile{c70tt.fd}%
%<ctexspa>\ProvidesFile{ctexspa.def}%
%<c19|c70>  [2014/03/08 v2.0 Chinese font definition (CTEX)]
%<ctexspa>  [2014/06/12 v2.0 Space info for CJKpunct (CTEX)]
%</!(readme|install|zhmap|spa)>
%<*driver>
\ExplSyntaxOff
\let\ctexrevnum\ExplFileVersion
\expandafter\let\csname ver@thumbpdf.sty\endcsname\fmtversion
\documentclass[a4paper,full,numbered]{l3doc}
\usepackage[UTF8, punct = kaiming, heading,
  linespread = 1.2, sub3section]{ctex}
\ctexset{
  abstractname = 简介,
  indexname    = 代码索引,
}
\appto\abstract{\parindent=2\ccwd} ^^A l3doc.cls 设置列表环境中 \listparindent=\z@
\usepackage[toc]{multitoc}
\usepackage{geometry}
\usepackage{tabularx}
\usepackage{unicode-math}
\geometry{includemp,hmargin={0mm,15mm},vmargin={25mm,15mm},footskip=7mm}
\hypersetup{pdfstartview=FitH,bookmarksdepth=subsubsection}
\setcounter{secnumdepth}{4}
\setcounter{tocdepth}{2}
\newcommand*\email{\nolinkurl}
\setmainfont{TeX Gyre Pagella}
\setsansfont{CMU Sans Serif}
\setmonofont[
  UprightFont=* Light, BoldFont=* Bold,
  SlantedFont=* Light Oblique]{CMU Typewriter Text}
\setmathfont{texgyrepagella-math.otf}
\usepackage{xcolor}
\usepackage{caption}
\captionsetup{strut=off}
\makeatletter
%% <--- http://tex.stackexchange.com/a/40896
\patchcmd{\@addtocurcol}%
    {\vskip \intextsep}%
    {\edef\save@first@penalty{\the\lastpenalty}\unpenalty
     \ifnum \lastpenalty = \@M  % hopefully the OR penalty
        \unpenalty
     \else
        \penalty \save@first@penalty \relax % put it back
     \fi
      \ifnum\outputpenalty <-\@Mii
                         \addvspace\intextsep
                         \vskip\parskip
      \else
                         \addvspace\intextsep
      \fi}%
    {\typeout{*** SUCCESS ***}}{\typeout{*** FAIL ***}}
\patchcmd{\@addtocurcol}%
    {\vskip\intextsep \ifnum\outputpenalty <-\@Mii \vskip -\parskip\fi}%
    {\ifnum\outputpenalty <-\@Mii
       \aftergroup\vskip\aftergroup\intextsep
       \aftergroup\nointerlineskip
     \else
       \vskip\intextsep
     \fi}%
    {\typeout{*** SUCCESS ***}}{\typeout{*** FAIL ***}}
\patchcmd{\@getpen}{\@M}{\@Mi}
  {\typeout{*** SUCCESS ***}}{\typeout{*** FAIL ***}}
%% --->
\ifxetex
  \let\ctexdocverbaddon\xeCJKVerbAddon
\else
  \let\ctexdocverbaddon\relax
\fi
\setlist{noitemsep,topsep=\smallskipamount}
\newlist{optdesc}{description}{3}
\setlist[optdesc]{%
  font=\mdseries\ttfamily,align=right,
  labelsep=\ccwd,labelindent=-\ccwd,leftmargin=*}
\fvset{
  fontsize=\small,baselinestretch=1,numbersep=5pt,
  formatcom=\ctexdocverbaddon,
  listparameters=\setlength\topsep{\MacrocodeTopsep}}
\DefineVerbatimEnvironment{ctexexam}{Verbatim}{%
  gobble=4,
  frame=single,framesep=10pt,
  label=\rule{0pt}{12pt}\textnormal{\bfseries 例 \arabic{ctexexam}},
  listparameters=
    \refstepcounter{ctexexam}\ctexexamlabelref
    \appto\FV@EndList{\nointerlineskip}}
\define@key{FV}{labelref}{\def\ctexexamlabelref{\label{#1}}}
\let\ctexexamlabelref\empty
\newcounter{ctexexam}
\BeforeBeginEnvironment{function}{\par\nointerlineskip}
\AtEndEnvironment{function}
  {\par\xdef\ctexfixprevdepth{\prevdepth=\the\prevdepth\space}}
\AfterEndEnvironment{function}{\ctexfixprevdepth}
\AtBeginEnvironment{syntax}{\linespread{1}}
\preto\MacroFont{\linespread{1}}
\appto\MacroFont{\hyphenchar\font\m@ne\ctexdocverbaddon}
\preto\AltMacroFont{\linespread{1}}
\appto\AltMacroFont{\hyphenchar\font\m@ne\ctexdocverbaddon}
\def\Module#1{\mbox{\normalfont\sffamily\textlangle#1\textrangle}}
\ExplSyntaxOn
\cs_set_protected:Npn \__codedoc_special_index_aux:nnnnn #1#2#3#4#5
  {
    \__codedoc_special_index_set:Nn \l__codedoc_index_escaped_macro_tl {#2}
    \str_if_eq:onTF { \@currenvir } { macrocode }
      { \codeline@wrindex }
      {
        \HD@target
        \index
      }
      {
        \tl_if_empty:nF { #3 #4 }
          { #3 \actualchar #4 \levelchar }
        #1
        \actualchar
        {
          \token_to_str:N \verbatim@font \c_space_tl
          \l__codedoc_index_escaped_macro_tl
        }
        \encapchar
        hdclindex{\the\c@HD@hypercount}{#5}
      }
  }
\DeclareDocumentCommand \cs { O{} m }
  { \__codedoc_cmd_aux:no {#1} { \c__codedoc_backslash_tl #2 } }
\DeclareDocumentCommand \tn { O{} m }
  {
    \__codedoc_cmd_aux:no
      { index = TeX , replace = false , #1 }
      { \c__codedoc_backslash_tl #2 }
  }
\DeclareDocumentCommand \meta { +m }
  { \__codedoc_meta_aux:n {#1} }
\DeclareExpandableDocumentCommand \bookmarkcstn { O{} m }
  { \tl_to_str:n {#2} }
\cs_new:Npn \bookmarkmeta #1 { < \tl_to_str:n {#1} > }
\cs_generate_variant:Nn \__codedoc_cmd_aux:nn { no }
\AtBeginEnvironment { syntax }
  {
    \char_set_active_eq:NN \| \orbar
    \char_set_active_eq:NN \( \defaultvalaux
  }
\ExplSyntaxOff
\pdfstringdefDisableCommands{%
  \let\cs\bookmarkcstn
  \let\tn\bookmarkcstn
  \let\meta\bookmarkmeta}
\def\orbar{\textup{\textbar}}
\def\defaultval#1{\textbf{\textup{#1}}}
\def\defaultvalaux#1){\defaultval{#1}}
\def\TF{true\orbar false}
\def\TTF{\defaultval{true}\orbar false}
\def\TFF{true\orbar\defaultval{false}}
\protected\def\opt{\texttt}
\def\pdfTeX{\hologo{pdfTeX}}
\def\XeTeX{\hologo{XeTeX}}
\def\XeLaTeX{\hologo{XeLaTeX}}
\def\LuaLaTeX{\hologo{LuaLaTeX}}
\def\pdfLaTeX{\hologo{pdfLaTeX}}
\def\LaTeX{\hologo{LaTeX}}
\def\LaTeXe{\hologo{LaTeX2e}}
\def\LTXIII{\hologo{LaTeX3}}
\def\dvipdfmx{DVIPDFM\textit{x}}
\def\ctexkitrev#1{%
  \href{https://github.com/CTeX-org/ctex-kit/commit/#1}{\texttt{ctex-kit} rev. #1}}
\patchcmd\theCodelineNo{\sffamily\tiny}{\normalfont\sffamily\tiny}{}{}
\appto\GlossaryParms{%
  \def\@idxitem{\par\hangindent 2em }%
  \def\subitem{\@idxitem\hspace*{1em}}%
  \def\subsubitem{\@idxitem\hspace*{2em}}}
\patchcmd\l@section{2.5em}{1.5em}{}{}
\patchcmd\l@subsection{2.5em}{1.5em}{}{}
\patchcmd\changes@{\space}{\lbrack}{}{}
\patchcmd\@wrglossary{hdpindex}{hdclindex{\the\c@HD@hypercount}}{}{}
\newenvironment{defaultcapconfig}{%
  \MakePercentComment
  % ctexcap-utf8.cfg
% vim:ft=tex

\ProvidesFile{ctexcap-utf8.cfg}
  [2009/05/05 v0.9 ctex
   configuration file]

%% Chinese captions
%%
%% character set: GBK
%% encoding: EUC

%%%%%%%%%%%%%%%%%%%%%%%%%%%%%%%%
%% caption name
%%%%%%%%%%%%%%%%%%%%%%%%%%%%%%%%

\def\CTEX@contentsname{目录}
\def\CTEX@listfigurename{插图}
\def\CTEX@listtablename{表格}

\def\CTEX@figurename{图}
\def\CTEX@tablename{表}

\def\CTEX@abstractname{摘要}
\def\CTEX@indexname{索引}
\def\CTEX@bibname{参考文献}

\def\CTEX@prepart{第}
\def\CTEX@postpart{部分}
\def\CTEX@prechapter{第}
\def\CTEX@postchapter{章}
\def\CTEX@presection{}
\def\CTEX@postsection{}
\def\CTEX@presubsection{}
\def\CTEX@postsubsection{}
\def\CTEX@presubsubsection{}
\def\CTEX@postsubsubsection{}
\def\CTEX@preparagraph{}
\def\CTEX@postparagraph{}
\def\CTEX@presubparagraph{}
\def\CTEX@postsubparagraph{}

\ifCTEX@cls{article}{
  \def\CTEX@appendixname{}
}{
  \def\CTEX@appendixname{附录~}
}

%%%%%%%%%%%%%%%%%%%%%%%%%%%%%%%%
%% caption number
%%%%%%%%%%%%%%%%%%%%%%%%%%%%%%%%

\def\CTEX@thepart{\chinese{part}}
\def\CTEX@thechapter{\chinese{chapter}}

\def\CTEX@thesection{\thesection}
\def\CTEX@thesubsection{\thesubsection}
\def\CTEX@thesubsubsection{\thesubsubsection}
\def\CTEX@theparagraph{\theparagraph}
\def\CTEX@thesubparagraph{\thesubparagraph}

\ifCTEX@cls{article}{
  \def\CTEX@appendixnumber{\@Alph\c@section}
}{
  \def\CTEX@appendixnumber{\@Alph\c@chapter}
}

%%%%%%%%%%%%%%%%%%%%%%%%%%%%%%%%
%% caption format
%%%%%%%%%%%%%%%%%%%%%%%%%%%%%%%%

\ifCTEX@cls{article}{
  \def\CTEX@part@format{\centering}
  \def\CTEX@part@nameformat{\Large\bfseries}
  \def\CTEX@part@aftername{\quad}
  \def\CTEX@part@titleformat{\Large\bfseries}
  \def\CTEX@part@beforeskip{4ex}
  \def\CTEX@part@afterskip{3ex}
  \def\CTEX@part@indent{\z@}
}{
  \def\CTEX@part@format{\centering}
  \def\CTEX@part@nameformat{\huge\bfseries}
  \def\CTEX@part@aftername{\par\vskip 20\p@}
  \def\CTEX@part@titleformat{\huge\bfseries}
}

\def\CTEX@chapter@format{\centering}
\def\CTEX@chapter@nameformat{\huge\bfseries}
\def\CTEX@chapter@aftername{\quad}
\def\CTEX@chapter@titleformat{\huge\bfseries}
\def\CTEX@chapter@beforeskip{50\p@}
\def\CTEX@chapter@afterskip{40\p@}
\def\CTEX@chapter@indent{\z@}

\def\CTEX@section@format{\Large\bfseries\centering}
\def\CTEX@section@aftername{\quad}
\def\CTEX@section@beforeskip{-3.5ex \@plus -1ex \@minus -.2ex}
\def\CTEX@section@afterskip{2.3ex \@plus .2ex}
\def\CTEX@section@indent{\z@}

\def\CTEX@subsection@format{\large\bfseries\flushleft}
\def\CTEX@subsection@aftername{\quad}
\def\CTEX@subsection@beforeskip{-3.25ex \@plus -1ex \@minus -.2ex}
\def\CTEX@subsection@afterskip{1.5ex \@plus .2ex}
\def\CTEX@subsection@indent{\z@}

\def\CTEX@subsubsection@format{\normalsize\bfseries\flushleft}
\def\CTEX@subsubsection@aftername{\quad}
\def\CTEX@subsubsection@beforeskip{-3.25ex \@plus -1ex \@minus -.2ex}
\def\CTEX@subsubsection@afterskip{1.5ex \@plus .2ex}
\def\CTEX@subsubsection@indent{\z@}

\def\CTEX@paragraph@format{\normalsize\bfseries\flushleft}
\def\CTEX@paragraph@aftername{\quad}
\ifnum\c@CTEX@sectiondepth>2
  \def\CTEX@paragraph@beforeskip{-3.25ex \@plus -1ex \@minus -.2ex}
  \def\CTEX@paragraph@afterskip{1ex \@plus .2ex}
\else
  \def\CTEX@paragraph@beforeskip{3.25ex \@plus1ex \@minus .2ex}
  \def\CTEX@paragraph@afterskip{-1em}
\fi
\def\CTEX@paragraph@indent{\z@}

\def\CTEX@subparagraph@format{\normalsize\bfseries\flushleft}
\def\CTEX@subparagraph@aftername{\quad}
\ifnum\c@CTEX@sectiondepth>3
  \def\CTEX@subparagraph@beforeskip{-3.25ex \@plus -1ex \@minus -.2ex}
  \def\CTEX@subparagraph@afterskip{1ex \@plus .2ex}
\else
  \def\CTEX@subparagraph@beforeskip{3.25ex \@plus1ex \@minus .2ex}
  \def\CTEX@subparagraph@afterskip{-1em}
\fi
\ifnum\c@CTEX@sectiondepth>2
  \def\CTEX@subparagraph@indent{\z@}
\else
  \def\CTEX@subparagraph@indent{\parindent}
\fi

%%%%%%%%%%%%%%%%%%%%%%%%%%%%%%%%
%% other configurations
%%%%%%%%%%%%%%%%%%%%%%%%%%%%%%%%

\def\CTEX@caption@delimiter{: }

\endinput

%
  \ExplSyntaxOff
  \MakePercentIgnore}{}
\makeatother
\EnableCrossrefs
\CodelineIndex
\RecordChanges
\def\glossaryname{版本历史}
\GlossaryPrologue{\section*{\glossaryname}}
\IndexPrologue{%
  \section*{\indexname}
  \textit{意大利体的数字表示描述对应索引项的页码;
  带下划线的数字表示定义对应索引项的代码行号;
  罗马字体的数字表示使用对应索引项的代码行号。}}
\usepackage{makecell}
\begin{document}
  \DocInput{\jobname.dtx}
  \newgeometry{hmargin=15mm,vmargin={25mm,15mm},footskip=7mm}
  \PrintChanges
  \PrintIndex
\end{document}
%</driver>
%
% \fi
%
% \changes{v2.0}{2014/03/06}{应用 \LTXIII{} 重新整理代码。}
% \changes{v2.0}{2014/03/12}{删除 \file{c19gbsn.fd} 和 \file{c19gkai.fd}。}
%
%
% \CheckSum{4270}
%
% \CharacterTable
%  {Upper-case    \A\B\C\D\E\F\G\H\I\J\K\L\M\N\O\P\Q\R\S\T\U\V\W\X\Y\Z
%   Lower-case    \a\b\c\d\e\f\g\h\i\j\k\l\m\n\o\p\q\r\s\t\u\v\w\x\y\z
%   Digits        \0\1\2\3\4\5\6\7\8\9
%   Exclamation   \!     Double quote  \"     Hash (number) \#
%   Dollar        \$     Percent       \%     Ampersand     \&
%   Acute accent  \'     Left paren    \(     Right paren   \)
%   Asterisk      \*     Plus          \+     Comma         \,
%   Minus         \-     Point         \.     Solidus       \/
%   Colon         \:     Semicolon     \;     Less than     \<
%   Equals        \=     Greater than  \>     Question mark \?
%   Commercial at \@     Left bracket  \[     Backslash     \\
%   Right bracket \]     Circumflex    \^     Underscore    \_
%   Grave accent  \`     Left brace    \{     Vertical bar  \|
%   Right brace   \}     Tilde         \~}
%
% \GetFileInfo{\jobname.dtx}%
%
% \title{\bfseries \CTeX{} 宏包说明}
% \author{\href{http://www.ctex.org}{ctex.org}}
% \date{\filedate\qquad\fileversion\thanks{\ctexkitrev{\ctexrevnum}.}}
% \maketitle
%
% \begin{abstract}
% \CTeX{} 宏包是面向中文排版的通用 \LaTeX{} 排版框架,为中文 \LaTeX{} 文档
% 提供了汉字输出支持、标点压缩、字体字号命令、标题文字汉化、中文版式调整、数字
% 日期转换等支持功能,可适应论文、报告、书籍、幻灯片等不同类型的中文文档。
%
% \CTeX{} 宏包支持 \LaTeX{}、\pdfLaTeX{}、\XeLaTeX{}、\LuaLaTeX{} 等多种不同
% 的编译方式,并为它们提供了统一的界面。主要功能由宏包 \pkg{ctex} 或中文文档类
% \cls{ctexart}、\cls{ctexrep}、\cls{ctexbook} 实现。
% \end{abstract}
%
% \tableofcontents
% \clearpage
%
% \begin{documentation}
%
% \section{介绍}
%
% 这个宏包的部分原始代码来自于由王磊编写 \cls{cjkbook} 文档类,还有一小部分原
% 始代码来自于吴凌云编写的 \file{GB.cap} 文件。原来的这些工作都是零零碎碎编写的,没有认
% 真、系统的设计,也没有用户文档,非常不利于维护和改进。2003 年,吴凌云用 \pkg{doc} 和
% \pkg{DocStrip} 工具重新编写了整个文档,并增加了许多新的功能。2007 年,oseen 和王越在
% \CTeX{} 宏包基础上增加了对 UTF-8 编码的支持,开发出了 \pkg{ctexutf8} 宏包。
% 2009 年 5 月,我们在 Google Code 建立了 ctex-kit 项目^^A
% \footnote{\nolinkurl{http://code.google.com/p/ctex-kit/}},对 \CTeX{} 宏包及相关
% 宏包和脚本进行了整合,并加入了对 \XeTeX{} 的支持。在开发新版本时,考虑到合作开发和
% 调试的方便,我们不再使用 \pkg{doc} 和 \pkg{DocStrip} 工具,改为直接编写宏包文件。
% 2014 年 3 月,为了适应 \LaTeX{} 的最新发展,特别是 \LTXIII{} 的逐渐成熟,李清用
% \LTXIII{} 重构了整个宏包的代码,并重新使用 \pkg{doc} 和 \pkg{DocStrip} 工具进行代码
% 的管理,升级版本号为 2.0。2015 年 3 月,ctex-kit 项目迁移至
% \href{https://github.com/CTeX-org/ctex-kit}{GitHub}%
% \footnote{\url{https://github.com/CTeX-org/ctex-kit}}。
%
% 最初 Knuth 设计开发 \TeX{} 的时候没有考虑到支持多国语言,特别是多字节的中日韩
% 语言。这使得 \TeX{} 以至后来的 \LaTeX{} 对中文的支持一直不是很好。即使在
% \pkg{CJK} 宏包解决了中文字符处理的问题以后,中文用户使用 \LaTeX{} 仍然要面对许
% 多困难。最常见的就是中文化的标题。由于中文习惯和西方语言的不同,使得很难直接使
% 用原有的标题结构来表示中文标题,因此需要对标准 \LaTeX{} 宏包做较大的修改。此
% 外,还有诸如日期格式、首行缩进、中文字号的对应关系等等细节问题。\CTeX{} 宏
% 包正是为着解决这些 \LaTeX{} 文档中文化问题而产生的。另一方面,随着 \TeX{} 引擎
% 和 \LaTeX{} 宏包的不断开发,在 \LaTeX{} 中支持中文字符的方式也由早期的
% \pkg{CCT} 这类单一的专用系统发展为 \pdfTeX{} 引擎下的 \pkg{CJK}、\pkg{zhmCJK}
% 宏包、\XeTeX{} 引擎下的 \pkg{xeCJK} 宏包、\LuaTeX{} 引擎下的 \pkg{luatexja} 宏
% 包等多种方式,各种方式下都有不同的适用范围和用法细节,不同操作系统和语言环境
% 也带来不少细节差异,亟需一个统一的界面来访问不同的中文处理方式,使同一份文档能
% 在不同的环境下交换使用。\CTeX{} 宏包也在统一中文处理界面上做出了努力。
%
% 实现 \CTeX{} 宏包中间很多地方用到了在 \url{bbs.ctex.org} 论坛上的讨论结果,
% 在此对参与讨论的朋友们表示感谢。
%
%
% \section{简明教程}
%
% \subsection{使用 \CTeX{} 文档类}
%
% \CTeX{} 宏包使用起来十分简单。我们为 \LaTeX{} 的标准文档类
% \cls{article}、\cls{report}、\cls{book} 分别设计了对应的
% \cls{ctexart}、\cls{ctexrep} 和 \cls{ctexbook}。用户只需要使用对应的
% \CTeX{} 文档类替换原有的文档类即可获得中文支持和中文风格的版式。
%
% \subsubsection{使用 \XeLaTeX{} 或 \LuaLaTeX{} 编译}
%
% 使用 \XeLaTeX{} 或 \LuaLaTeX{} 编译时,必须将涉及到的所有源文件使用 UTF-8
% 编码保存。以下是一个简单的例子。
%
% \begin{ctexexam}
%   \documentclass{ctexart}
%   \begin{document}
%   中文文档类测试,保存为 UTF-8 编码,使用 XeLaTeX 或 LuaLaTeX 编译。
%   \end{document}
% \end{ctexexam}
%
% \subsubsection{使用 \LaTeX{} 或 \pdfLaTeX{} 编译}
%
% 在默认情况下,使用 \LaTeX{} 或 \pdfLaTeX{} 编译时,需将涉及到的所有源文件使用
% GBK 编码保存。以下是一个简单的例子。
%
% \begin{ctexexam}
%   \documentclass{ctexart}
%   \begin{document}
%   中文文档类测试,保存为 GBK 编码,使用 LaTeX 或 pdfLaTeX 编译。
%   \end{document}
% \end{ctexexam}
%
% 为 \CTeX{} 宏包或文档类添加 \opt{UTF8} 选项,则使用 \LaTeX{} 或 \pdfLaTeX{}
% 编译时,也可以让 \CTeX{} 宏包或文档类在 UTF-8 模式下工作。
%
% \begin{ctexexam}
%   \documentclass[UTF8]{ctexart}
%   \begin{document}
%   中文文档类测试,保存为 UTF-8 编码,使用 LaTeX 或 pdfLaTeX 编译。
%   \end{document}
% \end{ctexexam}
%
% 我们推荐总是将源文件保存为 UTF-8 编码,并使 \CTeX{} 宏包/文档类
% 工作在 UTF-8 模式下。
%
% \subsection{使用 \pkg{ctex} 宏包}
%
% 很多时候,用户并不是用 \LaTeX{} 的标准文档类。为此,我们设计了 \pkg{ctex} 宏包。
%
% 以下是使用 \cls{beamer} 文档类编写中文演示文稿的一个示例。
% \begin{ctexexam}
%   \documentclass{beamer}
%   \usepackage{ctex}
%   \begin{document}
%   \begin{frame}{中文演示文档}
%   \begin{itemize}
%     \item 保存为 UTF-8 编码,使用 XeLaTeX 或 LuaLaTeX 编译。
%     \item 保存为 GBK 编码,使用 LaTeX 或 pdfLaTeX 编译。
%   \end{itemize}
%   \end{frame}
%   \end{document}
% \end{ctexexam}
% 这里,\pkg{ctex} 包也可以加上 \opt{UTF8} 选项指定文档编码。这时,编译涉及到的所有
% 源文档都需要保存为 UTF-8 编码。
%
% 有些文档类是建立在 \LaTeX{} 标准文档类之上开发的。这时,给 \pkg{ctex} 宏包
% 加上 \opt{heading} 选项,可以将章节标题设置为中文风格。
% \begin{ctexexam}
%   \documentclass{ltxdoc}
%   \usepackage[UTF8, heading]{ctex}
%   \begin{document}
%   \section{简介}
%   中文化的 \LaTeX{} 手册。
%   \end{document}
% \end{ctexexam}
%
% \section{参考手册}
%
% \CTeX{} 宏包会根据用户使用的编译方式\footnote{\LaTeX、\pdfLaTeX、\XeLaTeX 以及
% \LuaLaTeX。}在底层选择不同的中文支持方式(见表 \ref{tab:chinese-support})。
%
% \begin{table}[htbp]
% \centering
% \begin{tabular}{ccc}
%   \toprule
%   (pdf)\LaTeX & \XeLaTeX & \LuaLaTeX \\
%   \midrule
%   \pkg{CJK} & \pkg{xeCJK} & \pkg{luatexja} \\
%   \bottomrule
% \end{tabular}
% \caption{\CTeX{} 宏包的中文支持方式}
% \label{tab:chinese-support}
% \end{table}
%
% \CTeX{} 宏包在用户使用 \XeLaTeX{} 及 \LuaLaTeX{} 编译时,使用(且仅能
% 使用)UTF-8 编码;而因为历史原因,在用户使用 \LaTeX{} 及 \pdfLaTeX{} 编译
% 时默认使用 GBK 编码。除非有特殊的需求,我们推荐用户使用 UTF-8 编码,并使用
% \XeLaTeX{} 或 \LuaLaTeX{} 编译。
%
% \subsection{依赖与安装}
%
% \CTeX{} 是一个 \LaTeX{} 宏包,只有一个源文件 \file{ctex.dtx}。它依赖下列宏包:
%
% \begin{itemize}
%   \item \pkg{expl3}、\pkg{xparse} 和 \pkg{l3keys2e} 宏包。它们属于 \pkg{l3kernel}
%   和 \pkg{l3packages} 集合,并依赖
%   \begin{itemize}
%     \item \pkg{etex} 宏包。
%   \end{itemize}
%   \item \pkg{ifpdf} 宏包,属于 \pkg{oberdiek} 集合。
%   \item \pkg{etoolbox} 宏包。
%   \item \pkg{everysel} 宏包,属于 \pkg{ms} 集合。
%   \item \pkg{zhnumber} 宏包。
%   \item[$\Rightarrow$] 以上是各种编译方式都必需的依赖项。
%   \item \pkg{CJK} 集合,它的下划线功能依赖 \pkg{ulem} 宏包。
%   \item \pkg{CJKpunct} 宏包。
%   \item \pkg{zhmetrics} 宏包。
%   \item \pkg{zhmCJK} 宏包。
%   \item[$\Rightarrow$] 以上是使用 \pdfLaTeX{} 或 \LaTeX{} + \dvipdfmx{} 的编译方式所需要
%   的依赖项,其中 \pkg{zhmCJK} 是可选的。
%   \item \pkg{xeCJK} 宏包,它还依赖
%   \begin{itemize}
%     \item \pkg{fontspec} 宏包,它还依赖
%     \begin{itemize}
%       \item \pkg{euenc} 宏包。
%       \item \pkg{xunicode} 宏包,它还依赖 \pkg{tipa} 宏包。
%     \end{itemize}
%     \item \pkg{everypage} 宏包。
%   \end{itemize}
%   \item[$\Rightarrow$] 以上是使用 \XeLaTeX{} 编译时的依赖项。
%   \item \pkg{luatexja} 集合,它还依赖
%   \begin{itemize}
%     \item \pkg{luaotfload} 宏包,它还依赖 \pkg{luatexbase} 宏包。
%     \item \pkg{xkeyval} 宏包。
%   \end{itemize}
%   \item[$\Rightarrow$] 以上是使用 \LuaLaTeX{} 编译时的依赖项。
% \end{itemize}
%
% \CTeX{} 正式发布的版本和依赖的各个宏包或者集合都会被发行版 \TeX{} Live 或者
% \hologo{MiKTeX} 收录,可以直接用它们提供的宏包管理器安装或者更新,不需要自行
% 下载安装。
%
% 由于种种原因,\pkg{zhmCJK} 目前还没有收入 \TeX{} Live 和 \hologo{MiKTeX} 发
% 行版。因此,如果希望使用 \pkg{zhmCJK} 作为底层中文支持方式,还需要自行安装该
% 宏包。\pkg{zhmCJK} 的安装较为复杂,可以从 CTAN 下载 \pkg{zhmCJK} 的
% \href{http://mirrors.ctan.org/install/language/chinese/zhmcjk.tds.zip}{TDS
% 安装包},按目录结构将文件复制到 \TeX{} 发行版的本地 TDS 根目录,然后刷新文件
% 名数据库。关于安装的详细内容,可以参照
% \href{http://mirrors.ctan.org/language/chinese/zhmcjk/zhmCJK.pdf}{宏包手册}%
% 中第 3 节的指导。
%
% \subsection{\CTeX{} 宏包与文档类}
%
% \CTeX{} 宏包的组成见表 \ref{tab:ctex}。
%
% \begin{table}[htbp]
% \centering
% \begin{tabularx}{\linewidth}{l>{\ttfamily}lX}
% \toprule
%   类别   & \textrm{文件} & 说明 \\
% \midrule
%   文档类 & ctexart.cls   & 标准文档类 \cls{article} 的中文化版本,一般适用于
%                            短篇幅的文章 \\
%          & ctexrep.cls   & 标准文档类 \cls{report} 的中文化版本,一般适用于
%                            中篇幅的报告 \\
%          & ctexbook.cls  & 标准文档类 \cls{book} 的中文化版本,一般适用于
%                            长篇幅的书籍 \\
% \midrule
%   格式   & ctex.sty      & 提供全部功能,但\emph{默认不开启章节标题设置功能},
%                            需要使用 \opt{heading} 选项来开启 \\
% \bottomrule
% \end{tabularx}
% \caption{\CTeX{} 宏包的组成}\label{tab:ctex}
% \end{table}
%
% 除此以外,\CTeX{} 宏包定义和调整中文字号的功能被提取到 \file{ctexsize.sty} 当中,
% 可以在 \CTeX{} 之外调用 \pkg{ctexsize} 宏包使用该功能。它可以使用随后介绍的宏包
% 选项 \opt{cs4size}、\opt{c5size} 和命令 \tn{zihao}。
%
% 我们推荐用户直接使用 \CTeX{} 宏包提供的三个中文文档类,而不是在使用 \LaTeX{}
% 标准文档类的基础上,再使用 \pkg{ctex} 宏包。\pkg{ctex} 宏包被设计为适配非标准
% 文档类(例如 \cls{beamer} 文档类);而它的 \opt{heading} 选项则被设计为适配
% 衍生自标准文档类的文档类(例如 \cls{ltxdoc} 文档类)。
%
% \subsection{宏包选项}
% \label{subs:options}
%
% 宏包的选项用于改变一些默认的设置。\CTeX{} 宏包和文档类的默认设置已经对中文
% 行文和排版习惯做了尽可能的配置,因此除非必要,普通用户应尽量避免对默认设置的修改。
% 如果你觉得某些默认设置不合适,或者需要增减选项,可以在项目主页上
% \href{https://github.com/CTeX-org/ctex-kit/issues}{提交 issue},
% 向我们反映,我们会酌情在后续版本中予以改进。
%
% \CTeX{} 宏包有部分选项以 \meta{key}|=|\meta{value} 的形式提供,剩余部分
% 是传统的选项。你应该在调用宏包的时候直接设置这些选项。在下面的说明中使用^^A
% \textbf{粗体}来表示 \CTeX{} 的默认设置。
%
% \CTeX{} 文档类基于 \LaTeX{} 的标准文档类。因此,它们除了可以使用下面介绍的
% 选项之外,还能够使用标准文档类的选项。例如,设置纸张大小和方向的
% \opt{a4paper} 和 \opt{landscape},设置单双面的 \opt{oneside} 和
% \opt{twoside} 等。\CTeX{} 会将这些选项传给标准文档类^^A
% \footnote{事实上,\LaTeX{} 在文档类中的选项是全局设定的,除了对使用的文档类有
% 影响外,也可能会影响到随后使用的宏包。如果这些宏包中有某些选项出现在文档类的
% 选项列表中,那么该选项将会被自动激活。}。
%
%
% \subsubsection{默认字号选项}
% \label{subsubs:options-class}
%
% \begin{function}{cs4size, c5size}
%   分别使用小四号字或五号字作为默认字体大小。\CTeX{} 默认设置 \opt{c5size}。
%   这两个选项在 \pkg{ctexsize} 宏包中也有定义。
% \end{function}
%
% \begin{function}{10pt, 11pt, 12pt}
%   可以使用标准文档类的同类选项(\opt{10pt}、\opt{11pt} 和 \opt{12pt})
%   抑制中文字号设置。
% \end{function}
%
% \opt{cs4size} 和 \opt{c5size} 选项还会将标准文档类中的字体大小命令调整为
% 中文字号(见表 \ref{tab:fontsize})。
%
% \begin{table}[htbp]
% \centering
% \setlength\tabcolsep{1em}
% \begin{tabular}{l*2{c>{\ttfamily}r}*3{>{\ttfamily}c}}
% \toprule
% & \multicolumn2c{|c5size|} & \multicolumn2c{|cs4size|} &
% \multicolumn1c{|10pt|} & \multicolumn1c{|11pt|} & \multicolumn1c{|12pt|} \\
% \cmidrule(lr){2-3} \cmidrule(lr){4-5}
% \cmidrule(lr){6-6} \cmidrule(lr){7-7} \cmidrule(lr){8-8}
% 字体命令        & 字号 & \textrm{bp} & 字号 & \textrm{bp}
%  & \textrm{pt} & \textrm{pt} & \textrm{pt} \\
% \midrule
% |\tiny|         & 七号 & 5.5  & 小六 & 6.5  & ~5 & ~6 & ~6 \\
% |\scriptsize|   & 小六 & 6.5  & 六号 & 7.5  & ~7 & ~8 & ~8 \\
% |\footnotesize| & 六号 & 7.5  & 小五 & 9~~  & ~8 & ~9 & 10 \\
% |\small|        & 小五 & 9~~  & 五号 & 10.5 & ~9 & 10 & 11 \\
% |\normalsize|   & 五号 & 10.5 & 小四 & 12~~ & 10 & 11 & 12 \\
% |\large|        & 小四 & 12~~ & 小三 & 15~~ & 12 & 12 & 14 \\
% |\Large|        & 小三 & 15~~ & 小二 & 18~~ & 14 & 14 & 17 \\
% |\LARGE|        & 小二 & 18~~ & 二号 & 22~~ & 17 & 17 & 20 \\
% |\huge|         & 二号 & 22~~ & 小一 & 24~~ & 20 & 20 & 25 \\
% |\Huge|         & 一号 & 26~~ & 一号 & 26~~ & 25 & 25 & 25 \\
% \bottomrule
% \end{tabular}
% \caption{标准字体命令与字号的对应}\label{tab:fontsize}
%\end{table}
%
% \subsubsection{中文标题选项}
% \label{subsubs:options-heading}
%
% \begin{function}{heading}
%   \begin{syntax}
%     heading = <\TFF>
%   \end{syntax}
%   是否启用 \pkg{ctex.sty} 宏包的标题格式与页眉设置功能。参见
%   \ref{subsubs:secstyle}~节、\ref{subsubs:pagestyle}~节。
% \end{function}
%
% \CTeX{} 宏包提供的三个文档类总是启用中文标题设置功能。\opt{heading} 选项
% 只在 \pkg{ctex.sty} 下有意义。如果在 \pkg{ctex.sty} 下启用该选项,将会检查使用
% 的是否为 \LaTeX{} 标准文档类。若是,则该选项将会使得 \pkg{ctex.sty} 宏包的行为
% 和 \CTeX{} 宏包提供的三个中文文档类\emph{完全}一致;否则,会根据 \tn{chapter}
% 是否有定义来使用 \cls{ctexbook} 或者 \cls{ctexart} 的标题设置。
%
% \begin{function}{sub3section}
%   \tn{paragraph} 默认的格式是标题与后面的正文放在同一段排版,而该选项将
%   \tn{paragraph} 标题改为类似 section 的格式,即标题单独占一行,与后面的正文
%   分开。这使 \tn{paragraph} 的效果类似于(并不存在的)有 3 个 |sub| 的
%   |\subsubsubsection|。此时 \tn{subparagraph} 命令产生的标题会具有原本
%   \tn{paragraph} 的格式。
%
%   具体格式可参考 \ref{subsubs:secstyle}~节中 \opt{afterskip} 等选项。
%
%   此选项通常需要配合将计数器 |secnumdepth| 的值为设置为 4。
% \end{function}
%
% \begin{function}{sub4section}
%   与 \opt{sub3section} 类似,而进一步,将 \tn{paragraph} 和 \tn{subparagraph}
%   都改为 section 类的格式,即标题单独占一行,与后面的正文分开。这使
%   \tn{subparagraph} 的效果类似于(并不存在的)有 4 个 |sub| 的
%   |\subsubsubsubsection|。
%
%   具体格式可参考 \ref{subsubs:secstyle}~节中 \opt{afterskip} 等选项。
%
%   此选项通常需要配合将计数器 |secnumdepth| 的值为设置为 4 或 5。
% \end{function}
%
% 这两个选项在文档类和启用 \opt{heading} 的 \pkg{ctex.sty} 下才有意义。
%
% \subsubsection{中文编码选项}
%
% 下面的选项用于选择 \CTeX{} 宏包内部的编码,以匹配用户编写文档使用的编码。
%
% \begin{function}{GBK, UTF8}
%   使用 (pdf)\LaTeX{} 编译时,\CTeX{} 的默认编码是 GBK。此时可以使用 UTF8 选项
%   将默认编码改为 UTF-8。而使用 \XeLaTeX{} 或 \LuaLaTeX{} 编译时,\CTeX{} 内部
%   总是使用 UTF-8 编码,所以不必指明这些选项。
% \end{function}
%
% \subsubsection{中文字库选项}
% \label{subsubs:options-CJK-font}
%
% 默认情况下,\CTeX{} 宏包会根据用户使用的操作系统\footnote{\CTeX{} 宏包现在
% 能够识别 Mac OS X 系统以及 Windows 系统。}自动选择设置的字体(见表
% \ref{tab:default-font-select})。
%
% 需要注意的是,在 \pdfTeX{} 引擎下,\CTeX{} 的默认字体设置只支持 \pdfLaTeX{} 或者
% \LaTeX{} + \dvipdfmx{} 的编译方式。如果需要使用 Dvips,就需要将下面介绍的
% \opt{fontset} 选项设置为 \opt{none},然后按照传统方式^^A
% \footnote{可以使用 \pkg{zhmetrics} 宏包提供的脚本
%   \href{https://github.com/CTeX-org/ctex-kit/blob/master/zhmetrics/zhmCJK.lua}
%        {\file{CTeXFonts.lua}}。}^^A
% 在本地安装好 CJK 字体。
%
% Mac OS X 系统下的华文字体在 (pdf)\LaTeX{} 下
% 不可用,只支持 \XeLaTeX{} 或者 \LuaLaTeX{} 的编译方式。Fandol 在 \pdfLaTeX{}
% 下不可用。
%
% \begin{table}[htbp]
% \centering
% \begin{tabular}{ccc}
%   \toprule
%   Mac OS X & Windows & 其他\\
%   \midrule
%   华文字体 & 中易字体 & Fandol 字体 \\
%   \bottomrule
% \end{tabular}
% \caption{\CTeX{} 宏包在默认选项下的字体}
% \label{tab:default-font-select}
% \end{table}
%
% 通常,由 \CTeX{} 宏包进行的自动配置已经足够使用,无需用户手工干预;但
% 是 \CTeX{} 仍然提供了一系列选项,供在 \CTeX{} 的自动选择机制因为
% 意外情况失效,或者在用户有特殊需求的情况下使用。\emph{除非必要,用户不
% 应使用这些选项。}
%
% \begin{function}{zhmap}
%   \begin{syntax}
%     zhmap = <\TTF|zhmCJK>
%   \end{syntax}
%   是否使用 \pkg{zhmetrics} 宏包提供的字体映射机制,将 \pkg{CJK} 中文字库通过
%   \tn{special} 命令映射到对应的 \texttt{.ttf} 字体文件。
% \end{function}
%
%   \opt{zhmCJK} 还将载入 \pkg{zhmCJK} 宏包。它基于 \pkg{zhmetrics} 机制和
%   \pkg{CJK} 宏包,提供了与 \pkg{xeCJK} 类似的用户界面,可以很方便的完成字体
%   安装设置工作。
%
%   如果需要使用自定义的字体映射文件,或者希望使用 Type1 字库,请禁用本选项。
%   此时,将会使用传统的字体映射机制,调用系统中的字体映射文件。
%
%   本选项只在使用 (pdf)\LaTeX{} 编译时有意义。
%
% \begin{function}{fontset}
%   \begin{syntax}
%     fontset =
%     <none|adobe|fandol|founder|mac|ubuntu|windows|windowsnew|windowsold...>
%   \end{syntax}
%   如果没有指定 \opt{fontset} 的值,\CTeX{} 宏包会根据用户使用的操作系统
%   配置相应的字体。目前 \CTeX{} 能够识别 Mac OS X 和 Windows 系统。
% \end{function}
%
% \CTeX{} 预定义了以下六种中文字库。它们之中除了 \opt{windows},在使用
% (pdf)\LaTeX 编译但没有启用 \opt{zhmap} 选项都没有定义,都会返回错误。
% 此时对于 \opt{windows} 来说,假设已经按照传统方式在本地安装好了
% CJK 字体。
%
% \begin{optdesc}
%   \item[adobe] 使用 Adobe 公司的四款中文字体,不支持 \pdfLaTeX。
%   \item[fandol] 使用 Fandol 中文字体\footnote{由马起园、苏杰、黄晨成等人开发
%   的开源中文字体,参见:\url{https://github.com/clerkma/fandol-fonts}。},
%   不支持 \pdfLaTeX。
%   \item[founder] 使用方正公司的中文字体。
%   \item[mac] 使用 Mac OS X 系统下的华文字体,只能用于 \XeLaTeX{} 和 \LuaLaTeX。
%   \item[ubuntu] 使用 Ubuntu 系统下的文泉驿和文鼎字体。
%   \item[windows] 使用简体中文 Windows 系统下的中文字体,自动判断 Windows 系
%   统版本,采用 |windowsnew| 或 |windowsold| 的设置。
%   \item[windowsnew] 使用简体中文 Windows XP 或之前系统下的中易字体。
%   \item[windowsold] 使用简体中文 Windows Vista 或之后系统下的中易字体和微软
%   雅黑字体。
% \end{optdesc}
%
% 如果不想使用 \CTeX{} 预定义的中文字库,可以设置 \opt{fontset} 为下述值之一。
%
% \begin{optdesc}
%   \item[none] 不配置中文字体,需要用户自己配置。
%   \item[\meta{name}] 这里 \meta{name} 为自定义的名字。
%   \CTeX{} 宏包将载入名为 |ctex-fontset-|\meta{name}|.def| 的文件作为字体配置
%   文件。因此,请先保证文件的存在。可以在当前工作目录或者本地 \texttt{TDS} 目录
%   树下合适位置建立一个名为 |ctex-fontset-|\meta{name}|.def| 的文件,在这个文件
%   里面自定义中文字体。然后通过使用 |fontset=|\meta{name} 选项来调用它。字体配置
%   文件的具体写法可以参考 \CTeX{} 宏包 \texttt{fontset} 目录下的字体配置文件。
% \end{optdesc}
%
% \subsubsection{排版风格选项}
% \label{subsubs:options-type-style}
%
% \begin{function}{cap}
%   \begin{syntax}
%     cap = <\TTF>
%   \end{syntax}
%   是否对标题中英文文字进行汉化(如“图”、“表”、“目录”、“参考文献”等,
%   见 \ref{subsubs:capname}~节)并设置章节标题的中文化格式(见
%   \ref{subsubs:secstyle}~节)。
%
%   \opt{cap} 选项控制的标题汉化总是生效的,而对章节标题的中文化格式设定仅在
%   \opt{heading} 选项(\ref{subsubs:options-heading}~节)打开时生效。
% \end{function}
%
% \begin{function}{punct}
%   \begin{syntax}
%     punct = <(quanjiao)|banjiao|kaiming|CCT|plain>
%   \end{syntax}
%   设置标点处理格式。预定义好的格式有:
% \end{function}
%
%   \begin{optdesc}
%     \item[quanjiao] 全角式:所有标点占一个汉字宽度,相邻两个标点占 1.5 汉字宽度;
%     \item[banjiao]  半角式:所有标点占半个汉字宽度;
%     \item[kaiming]  开明式:句末点号用全角,其他半角;
%     \item[CCT]   CCT 格式:所有标点符号的宽度略小于一个汉字宽度;
%     \item[plain] 原样(不调整标点间距)。
%   \end{optdesc}
%
% \begin{function}{space}
%   \begin{syntax}
%     space = <\TF|(auto)>
%   \end{syntax}
%   用于控制是否保留汉字后面的空格。
%
%   \begin{optdesc}
%     \item[true] 保留汉字后面的空格,类似英文的习惯。这时用户需要自己处理由换行
%     产生的空格(在行尾加上 |%| 可以避免),否则排版结果可能不符合中文习惯。
%     \item[false] 总是忽略掉汉字后面的空格。只在使用 (pdf)\LaTeX{} 编译时有意义。
%     不建议使用该选项。
%     \item[auto] 根据空格后面的情况决定是否保留:如果空格后面是汉字,则忽略该
%     空格,否则保留。
%   \end{optdesc}
%
%   目前该选项对 \pkg{xeCJK} 与 \pkg{CJK} 宏包有效,相当于
%   \tn{CJKspace}、\tn{CJKnospace} 等宏的作用。
%
%   \emph{使用 \LuaLaTeX{} 编译的时候,该选项无效:汉字间的空格以及汉字与西文字符
%   之间的空格总是有效,不会被忽略,但可以自动忽略掉由换行产生的空格。}
% \end{function}
%
% \begin{function}{linespread}
%   \begin{syntax}
%     linespread = <数值>
%   \end{syntax}
%   接受一个浮点数值,用于调整行距。初始值为 1.3,即设置基线距离
%   (\tn{baselineskip})为 $1.3\times 1.2=1.56$ 倍字体大小。
% \end{function}
%
% \begin{function}{autoindent}
%   \begin{syntax}
%     autoindent = <\TTF>
%   \end{syntax}
%   在字体大小发生变化时,是否自动调整段首缩进(\tn{parindent})的大小。
% \end{function}
%
% \subsubsection{宏包兼容选项}
% \label{subsubs:options-pkg}
%
% \begin{function}{fancyhdr}
%   \begin{syntax}
%     fancyhdr = <\TFF>
%   \end{syntax}
%   保持和 \pkg{fancyhdr} 宏包的兼容性。启用该选项将自动调用 \pkg{fancyhdr} 宏
%   包,并设置默认页面格式(page style)为 |fancy|。参见
%   \ref{subsubs:pagestyle}~节。
% \end{function}
%
% \begin{function}{fntef}
%   \begin{syntax}
%     fntef = <\TFF>
%   \end{syntax}
%   启用该选项将自动调用 \pkg{CJKfntef} 或 \pkg{xeCJKfntef} 宏包,同时禁止相关
%   宏包对 \tn{emph} 的影响,禁用彩色设置。
% \end{function}
%
% \begin{function}{hyperref}
%   \begin{syntax}
%     hyperref = <\TFF>
%   \end{syntax}
%   自动判断 \pkg{hyperref} 宏包的正确参数以避免产生乱码。如果在导言区用户没有自
%   己调用 \pkg{hyperref},则该选项将使得 \pkg{hyperref} 宏包在导言区末尾被自动
%   调用;如果需要对 \pkg{hyperref} 宏包做进一步的设置,则用户可以自己
%   在 \CTeX{} 宏包后使用 \tn{hypersetup} 命令进行适当设置。
% \end{function}
%
% \subsection{格式控制命令 \tn{ctexset}}
%
% \CTeX{} 宏包为用户提供了一个通用的文档框架,使得用户可以根据需求,自由地在
% 底层不同的中文支持方式之间切换。为此,我们定义了一些命令,使得用户可以用统一
% 的方式对文档进行控制。
%
% \begin{function}{\ctexset}
%   \begin{syntax}
%     \tn{ctexset} \Arg{键值列表}
%   \end{syntax}
% 是 \CTeX{} 宏包的通用控制命令,用来在宏包载入后控制宏包的各项功能。
% \tn{ctexset} 的参数是一个键值列表,以通用的接口完成各项设置。如非特别说明,
% 后文介绍的所有选项都使用 \tn{ctexset} 命令进行设置。
% \end{function}
%
% \tn{ctexset} 的参数是一组由逗号分隔的选项列表,列表中的选项通常是一个
% \meta{key}|=|\meta{value} 格式的定义。例如设置摘要与参考文献标题名称
% (\ref{subsubs:capname}~节)就可以使用:
% \begin{ctexexam}[labelref=exam:capname]
%   \ctexset{
%     abstractname={本文概要},
%     bibname={文\quad 献}
%   }
% \end{ctexexam}
%
% \tn{ctexset} 采用 \LaTeX3 风格的键值设置,因此支持不同类型的选项与层次化的选
% 项设置,相关示例见 \ref{subsubs:secstyle}~节。
%
% \subsection{中文文字支持}
%
% \CTeX{} 宏包为用户提供了一个通用的文档框架,使得用户可以根据需求,自由地在
% 底层不同的中文支持方式之间切换。因此,一份文档可以几乎不做修改地在不同的操作
% 系统设置、不同的 \TeX{} 引擎下编译,正常地输出中文文档,而无须考虑具体的底层
% 细节。
%
% 在不同的操作系统设置与 \TeX{} 引擎下,\CTeX{} 宏包需要在底层使用不同的中文支
% 持宏包,它们的字体、字号和文字修饰的命令可能各不相同。为此,我们定制了一些命
% 令,使得用户可以用统一的方式对文档中的文字格式进行控制。
%
% \subsubsection{中文字库与字体命令}
%
% 中文字库集合除了可以在宏包选项中设定,也可以在导言区使用 \tn{ctexset} 命令设
% 定。
% \begin{function}{fontset}
%   \begin{syntax}
%     fontset = <adobe|fandol|founder|mac|ubuntu|windows|...>
%   \end{syntax}
% 选择中文字库,与宏包选项意义相同(\ref{subsubs:options-CJK-font}~节)。但是
% 因为之前可能已经设置过一次中文字体,不能再使用 |none| 字库。
%
% 注意 \opt{fontset} 选项只能在导言区使用,不能在正文中使用。
% \end{function}
%
% \CTeX{} 宏包预定义的中文字库基本上都定义了以下四种常用的中文字体命令。
%
% \begin{optdesc}
%   \item[\tn{songti}] 宋体,CJK 等价命令 |\CJKfamily{zhsong}|。
%   \item[\tn{heiti}] 黑体,CJK 等价命令 |\CJKfamily{zhhei}|。
%   \item[\tn{fangsong}] 仿宋,CJK 等价命令 |\CJKfamily{zhfs}|。
%   \item[\tn{kaishu}] 楷书,CJK 等价命令 |\CJKfamily{zhkai}|。
% \end{optdesc}
%
% 需要注意的是 \tn{fangsong} 在 \opt{ubuntu} 字库下没有定义。而在
% \opt{windows} 和 \opt{founder} 字库下,还有 \tn{lishu} 和 \tn{youyuan}。
%
% \begin{optdesc}
%   \item[\tn{lishu}] 隶书,CJK 等价命令 |\CJKfamily{zhli}|。
%   \item[\tn{youyuan}] 圆体,CJK 等价命令 |\CJKfamily{zhyou}|。
% \end{optdesc}
%
% 在 \opt{windows} 字库下,还有 \tn{yahei}。
%
% \begin{optdesc}
%   \item[\tn{yahei}] 微软雅黑,CJK 等价命令 |\CJKfamily{zhyahei}|。
% \end{optdesc}
%
% \subsubsection{中文标点}
%
% 除了在宏包选项中用 \opt{punct} 设置标点格式,在导言区或正文中,也可以使用
% \tn{ctexset} 设置标点格式。
% \begin{function}{punct}
%   \begin{syntax}
%     punct = <(quanjiao)|banjiao|kaiming|CCT|plain>
%   \end{syntax}
%   设置标点处理格式。功能与宏包选项 \opt{punct} 类似
%   (\ref{subsubs:options-type-style}~节)。
% \end{function}
%
% \subsubsection{字号与间距}
%
% \begin{function}{\zihao}
%   \begin{syntax}
%     \tn{zihao} \Arg{字号}
%   \end{syntax}
%   用于调整字号大小。其中 \meta{字号} 的有效值共有 16 个,如表 \ref{tab:zihao}
%   所示。使用 \tn{zihao} 命令调整字体大小时,西文字号大小会始终和中文字号保持一致。
% \end{function}
%
% \begin{table}[htbp]
% \centering
% \catcode`*\active
% \def*{\phantom{.}}
% \def~{\phantom{0}}
% \tabcolsep=1em
% \begin{tabular}{>{\ttfamily}ccl}
% \toprule
% $\meta{字号}$ & 大小(bp) & 意义 \\
% \midrule
% ~0   & 42*~ & \zihao{0}    初号    \\
% -0   & 36*~ & \zihao{-0}   小初号  \\
% ~1   & 26*~ & \zihao{1}    一号    \\
% -1   & 24*~ & \zihao{-1}   小一号  \\
% ~2   & 22*~ & \zihao{2}    二号    \\
% -2   & 18*~ & \zihao{-2}   小二号  \\
% ~3   & 16*~ & \zihao{3}    三号    \\
% -3   & 15*~ & \zihao{-3}   小三号  \\
% ~4   & 14*~ & \zihao{4}    四号    \\
% -4   & 12*~ & \zihao{-4}   小四号  \\
% ~5   & 10.5 & \zihao{5}    五号    \\
% -5   & ~9*~ & \zihao{-5}   小五号  \\
% ~6   & ~7.5 & \zihao{6}    六号    \\
% -6   & ~6.5 & \zihao{-6}   小六号  \\
% ~7   & ~5.5 & \zihao{7}    七号    \\
% ~8   & ~5*~ & \zihao{8}    八号    \\
% \bottomrule
% \end{tabular}
% \caption{中文字号}\label{tab:zihao}
% \end{table}
%
% \begin{function}{\ziju}
%   \begin{syntax}
%     \tn{ziju} \Arg{中文字符宽度的倍数}
%   \end{syntax}
%   用于调整相邻汉字之间的间距,即(在正常中文行文中)前一个汉字的右边缘与后一个汉字
%   的左边缘之间的距离。其中参数可以是任意浮点数值;而中文字符宽度指的是实际汉字的
%   宽度,不包含当前字距。
%
%   这个命令会影响 \tn{ccwd} 的值,但不会影响英文字距。
% \end{function}
%
% \begin{function}{\ccwd}
%   当前汉字的字宽保存在长度寄存器 \tn{ccwd} 之中。汉字字宽是相邻两个汉字中心
%   之间的距离,包含字距在内。因此修改字距会间接修改字宽。
% \end{function}
%
% \begin{function}{\CTEXsetfont}
%   更新当前的中文字体信息,包括当前字距(\tn{ccwd})和段首缩进(\tn{parindent})。
%   一般来说,用户无需使用这个命令。
% \end{function}
%
% \begin{function}{space}
%   \begin{syntax}
%     space = <\TF|(auto)>
%   \end{syntax}
%   与宏包选项 \opt{space} 一样,用于在正文中控制是否保留汉字后面的空格。
%
%   例如,使用
%   \begin{ctexexam}
%   \ctexset{space=true}
%   汉字 分词 技术
%   \end{ctexexam}
%   将得到“{\ctexset{space=true}汉字 分词 技术}”。
% \end{function}
%
% \begin{function}{autoindent}
%   \begin{syntax}
%     autoindent = <\TTF>
%   \end{syntax}
%   与宏包选项 \opt{autoindent} 一样,在正文中控制在字体大小发生变化时,是否自
%   动调整段首缩进(\tn{parindent})的大小。
%
%   如果设置 \opt{autoindent} 为假,则方便设置不受字号变化的特殊段落缩进。如设
%   置每段缩进 4 个汉字宽:
%   \begin{ctexexam}
%   \ctexset{autoindent=false}
%   \setlength\parindent{4\ccwd}
%   \end{ctexexam}
% \end{function}
%
% \begin{function}{linestretch}
%   \begin{syntax}
%     linestretch = <数值或长度>
%   \end{syntax}
%   \opt{linestretch} 是一个比较特殊的选项,它用来设置汉字之间弹性间距的弹性程
%   度。如果有单位,则可以在选项中直接写;如果是数字,单位则是汉字宽度
%   \tn{ccwd} 的倍数。
%
%   如果行宽不是汉字宽度的整数倍,为了让段落左右两端对齐,自然就要求伸展汉字之
%   间的间距,而 \opt{linestretch} 选项就是设置每行总的允许伸行量。初始值是允
%   许每行伸行一个汉字的宽度 \tn{ccwd},并且此宽度能根据字号变化动态调整。
%
%   过小的 \opt{linestretch} 可能导致段落文字右侧可能参差不齐;较大的
%   \opt{linestretch} 选项则可以帮助拥有较长不可断行内容的复杂段落方便地断行,
%   而不会产生大量编译警告;但很大的 \opt{linestretch} 则会掩盖段落不良断行产
%   生的坏盒子警告。
%
%   如果将 \opt{linestretch} 选项的值设置为 \tn{maxdimen},则可以禁止按字号自
%   动修改每行的允许伸长量。此时汉字间的弹性间距则固定为 \tn{baselineskip} 的
%   $0.08$ 倍。
% \end{function}
%
% \subsection{文档汉化与中文版式}
%
% \subsubsection{中文数字转换}
%
% \CTeX{} 宏包的中文数字转换功能实际上是调用 \pkg{zhnumber} 宏包来完成。下面只
% 介绍一些基本的用法,更高级的用法可以查阅 \pkg{zhnumber} 宏包的文档。
%
% \begin{function}{\chinese}
%   \begin{syntax}
%     \tn{chinese} \Arg{counter}
%   \end{syntax}
%   \tn{chinese} 命令与 \tn{roman} 等命令的用法类似,作用在一个 \LaTeX{}
%   计数器上,将计数器的值以中文数字的形式输出。
% \end{function}
%
% \begin{function}{\zhnumber}
%   \begin{syntax}
%     \tn{zhnumber} \Arg{number}
%   \end{syntax}
%   以中文格式输出数字。这里的数字可以是整数、小数和分数。
% \end{function}
%
% \begin{function}{\zhdigits}
%   \begin{syntax}
%     \tn{zhdigits} \Arg{number}
%   \end{syntax}
%   将阿拉伯数字转换为中文数字串。
% \end{function}
%
% \begin{function}{\CTEXnumber}
%   \begin{syntax}
%     \tn{CTEXnumber} "\"<macro> \Arg{number}
%   \end{syntax}
%   |\|<macro> 必须是一个 \TeX{} 宏,不需预先定义。\tn{CTEXnumber} 通过
%   \tn{zhnumber} 将 \meta{number} 转为中文数字,最后将结果存储在 |\|<macro>
%   里。对 |\|<macro> 的定义是局部的,将它展开一次就可以得到转换结果。
% \end{function}
%
% 一般来说,并不需要使用 \tn{CTEXnumber},直接使用 \tn{zhnumber} 即可。但是,如果
% 在文档中需要多次使用同一个数字 \meta{number} 的中文形式,就可以先用
% \tn{CTEXnumber} 将结果保存起来备用,而不是每次使用时都用 \tn{zhnumber} 现场
% 转换一次。
%
% \begin{function}{\CTEXdigits}
%   \begin{syntax}
%     \tn{CTEXdigits} "\"<macro> \Arg{number}
%   \end{syntax}
%   \tn{CTEXdigits} 与 \tn{CTEXnumber} 类似,但其转换的结果是中文数字串,而不是
%   中文数字。
% \end{function}
%
% \subsubsection{日期汉化}
%
% \CTeX 宏包对显示当前日期的 \tn{today} 命令进行了汉化,使之以中文的方式显示今
% 天的日期。如编译本文档的日期就是“\today”。
%
% \begin{function}{today}
%   \begin{syntax}
%     today = <(small)|big|old>
%   \end{syntax}
% 该选项用来控制 \tn{today} 命令的输出格式:
% \begin{optdesc}
%   \item[small] \ctexset{today=small}
%     效果为“\today”。使用阿拉伯数字和汉字的日期格式。
%   \item[big] \ctexset{today=big}
%     效果为“\today”。使用全汉字的日期格式。
%   \item[old] \ctexset{today=old}
%     效果为“\today”。使用文档原来的(英文)日期格式。
% \end{optdesc}
% \end{function}
%
% 设置日期格式使用 \tn{ctexset} 命令完成,例如设置全汉字的日期格式:
% \begin{ctexexam}
%   \ctexset{today=big}
% \end{ctexexam}
%
% \CTeX 宏包的中文日期功能实际上也是调用 \pkg{zhnumber} 宏包完成的。如果需要更
% 多有关日期、时间的命令和更复杂的设置,可以查阅 \pkg{zhnumber} 宏包的文档。
%
% \subsubsection{文档标题汉化}
% \label{subsubs:capname}
%
% 这里主要介绍由宏包 \opt{cap} 选项(\ref{subsubs:options-type-style}~节)控制
% 的文档标题汉化功能。
%
% 设置文档标题名的示例可见例~\ref{exam:capname}。下面的选项(如
% \opt{contentsname})主要用来重新定义与选项同名的宏(如 \tn{contentsname})的
% 定义。
%
% \begin{defaultcapconfig}
%
% \begin{function}{contentsname}
%   \begin{syntax}
%     contentsname = <名字>
%   \end{syntax}
% 设置目录标题名 \tn{contentsname}。中文默认为“\contentsname”。
% \end{function}
%
% \begin{function}{listfigurename}
%   \begin{syntax}
%     listfigurename = <名字>
%   \end{syntax}
% 设置插图目录标题名 \tn{listfigurename}。中文默认为“\listfigurename”。
% \end{function}
%
% \begin{function}{listtablename}
%   \begin{syntax}
%     listtablename = <名字>
%   \end{syntax}
% 设置表格目录标题名 \tn{listtablename}。中文默认为“\listtablename”。
% \end{function}
%
% \begin{function}{figurename}
%   \begin{syntax}
%     figurename = <名字>
%   \end{syntax}
% 设置图片环境标题名 \tn{figurename}。中文默认为“\figurename”。
% \end{function}
%
% \begin{function}{tablename}
%   \begin{syntax}
%     tablename = <名字>
%   \end{syntax}
% 设置表格环境标题名 \tn{tablename}。中文默认为“\tablename”。
% \end{function}
%
% \begin{function}{abstractname}
%   \begin{syntax}
%     abstractname = <名字>
%   \end{syntax}
% 设置摘要 \env{abstract} 环境标题名 \tn{abstractname}。中文默认
% 为“\abstractname”。注意 \cls{book} 类没有摘要,该选项无效。
% \end{function}
%
% \begin{function}{indexname}
%   \begin{syntax}
%     indexname = <名字>
%   \end{syntax}
% 设置索引标题名 \tn{indexname}。中文默认为“\indexname”。
% \end{function}
%
% \begin{function}{appendixname}
%   \begin{syntax}
%     appendixname = <名字>
%   \end{syntax}
% 设置附录标题名 \tn{appendixname}。中文默认为“\appendixname”。
% \end{function}
%
% \begin{function}{bibname}
%   \begin{syntax}
%     bibname = <名字>
%   \end{syntax}
% 设置参考文献标题名 \tn{refname}(对 \cls{article})或 \tn{bibname}(对
% \cls{report} 和 \cls{book})。中文默认为“\refname”。
% \end{function}
%
% \end{defaultcapconfig}
%
% \subsubsection{章节标题格式定义}
% \label{subsubs:secstyle}
%
% \CTeX 宏包对 \LaTeX 的标准文档类(\cls{article}、\cls{report} 和
% \cls{book})进行了扩充。当以 \opt{heading} 选项调用 \CTeX 宏包时
% (\ref{subsubs:options-heading}~节),则会启用章
% 节标题的格式设置功能。本节就来介绍有关章节标题的格式选项,所有选项使用
% \tn{ctexset} 命令设置。
%
% \changes{v2.0}{2015/03/21}{\tn{CTEXsetup}, \tn{CTEXoptions} 是过时命令。}
% 章节标题的格式选项是分层设置的。顶层的选项是章节标题名称,次一级的选项是章节
% 标题的格式。章节标题名包括 |part|, |chapter|, |section|, |subsection|,
% |subsubsecton|, |paragraph|, |subparagraph|;而可用的格式包括 \opt{name},
% \opt{number}, \opt{format}, \opt{nameformat}, \opt{numberformat},
% \opt{aftername}, \opt{titleformat}, \opt{beforeskip}, \opt{afterskip},
% \opt{indent}, \opt{pagestyle} 等。但注意,对 \cls{article} 及其衍生的
% \cls{ctexart} 等文档类,没有 |chapter| 级别的标题。
%
% 多级选项之间用斜线分开,例如,\opt{part/name} 选项设置 \tn{part} 标题的在数
% 字前后的名称,而 \opt{section/number} 选项设置 \tn{section} 标题的数字类型。
%
% \begin{function}{part/name, chapter/name, section/name, subsection/name,
%   subsubsection/name, paragraph/name, subparagraph/name}
%   \begin{syntax}
%     name = \{<前名字>,<后名字>\}
%     name = \Arg{前名字}
%   \end{syntax}
%   设置章节的名字。名字可以分为前后两部分,即章节编号前后的词语,两个词之间用
%   一个半角逗号分开;也可以只有一部分,表示只有章节编号之前的名字。例如:
%   \begin{ctexexam}
%   \ctexset{
%     chapter/name = {第,章},
%     section/name = {\S},
%   }
%   \end{ctexexam}
%   会使得 \tn{chapter} 标题使用形如“第一章”的名字,而 \tn{section} 标题则使
%   用形如“\S1”的名字。
% \end{function}
%
% \begin{table}[htbp]
% \small\centering
% \begin{tabular}{llll}
% \toprule
% 标题名 & |cap=true| 的默认值 & |cap=false| 的默认值 & 注 \\
% \midrule
% part & |{第,部分}| & |{\partname\space}| & 原 \tn{partname} 为 Part \\
% chapter & |{第,章}| & |{\chaptername\space}|
%   & 原 \tn{chaptername} 为 Chapter \\
% section & |{}| & |{}| & \\
% subsection & |{}| & |{}| & \\
% subsubsection & |{}| & |{}| & \\
% paragraph & |{}| & |{}| & \\
% subparagraph & |{}| & |{}| & \\
% \bottomrule
% \end{tabular}
% \caption{\opt{name} 选项的默认设置}
% \end{table}
%
% \begin{function}{part/number, chapter/number, section/number, subsection/number,
%   subsubsection/number, paragraph/number, subparagraph/number}
%   \begin{syntax}
%     number = \Arg{数字输出命令}
%   \end{syntax}
%   设置章节编号的数字输出格式。\meta{数字输出命令} 通常是对应章节编号计数器的
%   输出命令,如 |\thesection| 或 |\chinese{chapter}| 之类。
%   \begin{ctexexam}
%   \ctexset{
%     section/number = \Roman{section}
%   }
%   \end{ctexexam}
%
%   \opt{number} 选项的定义同时将控制对章节计数器的交叉引用。在引用计数器时,
%   记录在 \LaTeX{} 辅助文件中的是 \opt{number} 选项的定义。
%
%   但是,\opt{number} 选项不会影响计数器本身的输出。即设置 |section/number|
%   不会影响 \tn{thesection} 的定义。(但该选项会影响 \tn{CTEXthesection} 的定
%   义,见后。)
% \end{function}
%
% \begin{table}[htbp]
% \small\centering
% \begin{tabular}{llll}
% \toprule
% 标题名 & |cap=true| 的默认值 & |cap=false| 的默认值
%   & 原 |\the|\meta{标题} 等价定义 \\
% \midrule% part & |\chinese{part}| & |\thepart| & |\Roman{part}| \\
% part & |\chinese{part}| & |\thepart| & |\Roman{part}| \\
% chapter & |\chinese{chapter}| & |\thechapter| & |\arabic{chapter}| \\
% section & 同右 & |\thesection| & |\arabic{section}| \\
% subsection & 同右 & |\thesubsection| & |\thesection.\arabic{subsection}| \\
% subsubsection & 同右 & |\thesubsubsection|
%   & |\thesubsection.\arabic{subsubsection}| \\
% paragraph & 同右 & |\theparagraph|
%   & |\thesubsubsection.\arabic{paragraph}| \\
% subparagraph & 同右 & |\thesubparagraph|
%   & |\theparagraph.\arabic{subparagraph}| \\
% \bottomrule
% \end{tabular}
% \caption{\opt{number} 选项的默认设置}
% \end{table}
%
% \begin{function}{\CTEXthepart, \CTEXthechapter, \CTEXthesection,
%   \CTEXthesubsection, \CTEXthesubsubsection, \CTEXtheparagraph,
%   \CTEXthesubparagraph}
%   以 |\CTEXthe| 开头的这组宏给出结合了 \opt{name} 与 \opt{number} 选项的章节
%   编号输出格式。例如在 \opt{cap} 选项下的默认章编号输出格式就是
%   \tn{CTEXthechapter},形如“第一章”。
%
%   这组宏在 \CTeX 文档类中将代替 \tn{thechapter} 等宏的作用,在章节中引用本章
%   节的完整编号。例如用于帮助定义自定义的目录格式、页眉格式等。
% \end{function}
%
% 使用 \tn{ctexset} 设置多级选项时,可以在同一个上级选项下设置多个下级选项。例
% 如同时设置 |section| 一级标题的 \opt{name} 与 \opt{number} 选项:
% \begin{ctexexam}
%   \ctexset{
%     section = {
%       name = {第,节},
%       number = \chinese{section}
%     }
%   }
% \end{ctexexam}
%
%
% \begin{function}{part/format, chapter/format, section/format,
%   subsection/format, subsubsection/format, paragraph/format,
%   subparagraph/format,
%   .../format+}
%   \begin{syntax}
%     format = \Arg{格式命令}
%     format+= \Arg{格式命令}
%   \end{syntax}
%   \opt{format} 选项用于控制章节标题的全局格式,作用域为章节名字和随后的标题
%   内容。可以用于控制章节标题的对齐方式、整体字体字号等格式。
%
%   带加号的 \opt{format+} 选项则用于在已有格式之后追加新的格式命令。
%
%   例如,设置章格式为无衬线字体左对齐,为节格式增加无衬线字体设置:
%   \begin{ctexexam}
%   \ctexset{
%     chapter/format=\sffamily\raggedright,
%     section/format+=\sffamily
%   }
%   \end{ctexexam}
% \end{function}
%
% \begin{table}[htbp]
% \small\centering
% \begin{tabular}{lll}
% \toprule
% 标题名 & |cap=true| 的默认值 & |cap=false| 的默认值 \\
% \midrule
% part (article) & |\centering| & |\raggedright| \\
% part & |\centering| & |\centering| \\
% chapter & |\centering| & |\raggedright| \\
% section & |\Large\bfseries\centering| & |\Large\bfseries| \\
% subsection & 同右 & |\large\bfseries| \\
% subsubsection & 同右 & |\normalsize\bfseries| \\
% paragraph & 同右 & |\normalsize\bfseries| \\
% subparagraph & 同右 & |\normalsize\bfseries| \\
% \bottomrule
% \end{tabular}
% \caption{\opt{format} 选项的默认设置}
% \end{table}
%
% \begin{function}{part/nameformat, chapter/nameformat, section/nameformat,
%   subsection/nameformat, subsubsection/nameformat, paragraph/nameformat,
%   subparagraph/nameformat,
%   .../nameformat+}
%   \begin{syntax}
%     nameformat = \Arg{格式命令}
%     nameformat+= \Arg{格式命令}
%   \end{syntax}
%   \opt{nameformat} 用于控制章节名字的格式,作用域为章节名字,包括编号。它一
%   般用于章节名(包括编号)与章节标题的字体、字号等设置不一致的情形。参见
%   \opt{titleformat} 选项。
%
%   \opt{nameformat+} 用于在已有的章节名字格式后附加内容。
% \end{function}
%
% \begin{table}[htbp]
% \small\centering
% \begin{tabular}{lll}
% \toprule
% 标题名 & |cap=true| 的默认值 & |cap=false| 的默认值 \\
% \midrule
% part (article) & 同右 & |\Large\bfseries| \\
% part & 同右 & |\huge\bfseries| \\
% chapter & 同右 & |\huge\bfseries| \\
% section & 同右 & |{}| \\
% subsection & 同右 & |{}| \\
% subsubsection & 同右 & |{}| \\
% paragraph & 同右 & |{}| \\
% subparagraph & 同右 & |{}| \\
% \bottomrule
% \end{tabular}
% \caption{\opt{nameformat} 选项的默认设置}
% \end{table}
%
% \begin{function}{part/numberformat, chapter/numberformat,
%   section/numberformat, subsection/numberformat, subsubsection/numberformat,
%   paragraph/numberformat, subparagraph/numberformat,
%   .../numberformat+}
%   \begin{syntax}
%     numberformat = \Arg{格式命令}
%     numberformat+= \Arg{格式命令}
%   \end{syntax}
%   \opt{numberformat} 选项用于控制章节编号的格式,作用域仅为编号数字本身。对
%   各级标题默认均为空,当你需要编号的格式和前后的章节名字不一样时可以使用。
%
%   \opt{numberformat+} 选项用于在原有编号格式后面附加格式命令。
%
%   例如,我们可以使用 \opt{numberformat} 特别强调章标题中的数字:
%   \begin{ctexexam}
%   \ctexset{
%     chapter/number = \arabic{chapter},
%     chapter/numberformat = \color{blue}\zihao{0}\itshape,
%   }
%   \end{ctexexam}
%   上面的代码在 |cap| 选项下可以做出类似这样的章标题效果:
%   \begin{center}
%   \huge\bfseries 第 \textit{\color{blue}\zihao{0}4} 章
%   \end{center}
% \end{function}
%
% \begin{function}{part/aftername, chapter/aftername, section/aftername,
%   subsection/aftername, subsubsection/aftername, paragraph/aftername,
%   subparagraph/aftername}
%   \begin{syntax}
%   aftername=\Arg{代码}
%   \end{syntax}
%   \opt{aftername} 选项的参数\meta{代码}将被插入到章节编号与其后的标题内容之
%   间,用于控制格式变换。常用于控制章节编号与标题内容之间的距离,或者控制标题
%   是否另起一行。
% \end{function}
%
% \begin{table}[htbp]
% \small\centering
% \begin{tabular}{lll}
% \toprule
% 标题名 & |cap=true| 的默认值 & |cap=false| 的默认值 \\
% \midrule
% part (article) & |\quad| & |\par\nobreak| \\
% part & 同右 & |\par\vskip 20pt| \\
% chapter & |\quad| & |\par\vskip 20pt| \\
% section & 同右 & |\quad| \\
% subsection & 同右 & |\quad| \\
% subsubsection & 同右 & |\quad| \\
% paragraph & 同右 & |\quad| \\
% subparagraph & 同右 & |\quad| \\
% \bottomrule
% \end{tabular}
% \caption{\opt{aftername} 选项的默认设置}
% \end{table}
%
% \begin{function}{part/titleformat, chapter/titleformat, section/titleformat,
%   subsection/titleformat, subsubsection/titleformat, paragraph/titleformat,
%   subparagraph/titleformat,
%   .../titleformat+}
%   \begin{syntax}
%     titleformat = \Arg{格式命令}
%     titleformat+= \Arg{格式命令}
%   \end{syntax}
%   \opt{titleformat} 选项用于控制标题内容的格式,作用域为章节标题内容。
%
%   \opt{titleformat+} 选项用于在原有标题格式后面附加格式命令。
% \end{function}
%
% \begin{table}[htbp]
% \small\centering
% \begin{tabular}{lll}
% \toprule
% 标题名 & |cap=true| 的默认值 & |cap=false| 的默认值 \\
% \midrule
% part (article) & |\Large\bfseries| & |\huge\bfseries| \\
% part & |\huge\bfseries| & |\Huge\bfseries| \\
% chapter & |\huge\bfseries| & |\Huge\bfseries| \\
% section & 同右 & |{}| \\
% subsection & 同右 & |{}| \\
% subsubsection & 同右 & |{}| \\
% paragraph & 同右 & |{}| \\
% subparagraph & 同右 & |{}| \\
% \bottomrule
% \end{tabular}
% \caption{\opt{titleformat} 选项的默认设置}
% \end{table}
%
% \begin{function}{part/beforeskip, chapter/beforeskip, section/beforeskip,
%   subsection/beforeskip, subsubsection/beforeskip, paragraph/beforeskip,
%   subparagraph/beforeskip}
%   \begin{syntax}
%   beforeskip = \Arg{弹性间距}
%   \end{syntax}
%   \opt{beforeskip} 选项用于设置章节标题前的垂直间距及章节标题后首段的缩进。
%
%   \meta{弹性间距} 的绝对值被用于设置标题间的垂直间距,而\meta{弹性间距}的正
%   负号用于设置标题后第一段的首行缩进。当参数是负值时,章节标题后的第一段按英
%   文文档的排版习惯,没有首行缩进;参数是正值时,则保留首行缩进。
%
%   使用 \opt{sub3section} 或 \opt{sub4section} 宏包选项(见
%   \ref{subsubs:options-heading}~节)后,\tn{paragraph} 与 \tn{subparagraph}
%   这两级标题会改为排在不同段,会影响 \opt{beforeskip} 选项的默认值。
% \end{function}
%
% \begin{table}[htbp]
% \small\centering
% \begin{tabular}{lll}
% \toprule
% 标题名 & |cap=true| 的默认值 & |cap=false| 的默认值 \\
% \midrule
% part (article) & |4ex| & |-4ex| \\
% part & 无效 & 无效 \\
% chapter & |50pt| & |-50pt| \\
% section & |3.5ex plus 1ex minus .2ex| & |-3.5ex plus -1ex minus -.2ex| \\
% subsection & |3.25ex plus 1ex minus .2ex| & |-3.25ex plus -1ex minus -.2ex| \\
% subsubsection & |3.25ex plus 1ex minus .2ex| & |-3.25ex plus -1ex minus -.2ex| \\
% paragraph & 同右 & |3.25ex plus 1ex minus .2ex| \\
% \qquad(sub3section) & |3.25ex plus 1ex minus .2ex| & |-3.25ex plus -1ex minus -.2ex| \\
% \qquad(sub4section) & 同上 & 同上 \\
% subparagraph & 同右 & |3.25ex plus 1ex minus .2ex| \\
% \qquad(sub4section) & |3.25ex plus 1ex minus .2ex| & |-3.25ex plus -1ex minus -.2ex| \\
% \bottomrule
% \end{tabular}
% \caption{\opt{beforeskip} 选项的默认设置}
% \end{table}
%
% \begin{function}{part/afterskip, chapter/afterskip, section/afterskip,
%   subsection/afterskip, subsubsection/afterskip, paragraph/afterskip,
%   subparagraph/afterskip}
%   \begin{syntax}
%   afterskip = \Arg{弹性间距}
%   \end{syntax}
%   \opt{afterskip} 选项控制章节标题与后面下方之间的距离。\meta{弹性间距}的正
%   负号确定标题与后面正文是否排在同一段。如果是正值则正文另起一段,\meta{弹性
%   间距}给出垂直间距;如果是负值则章节标题与正文第一段排在同一段,\meta{弹性
%   间距}的绝对值给出水平间距。
%
%   默认情况下,\tn{paragraph}、\tn{subparagraph} 两级标题是与后面正文排在同一
%   段的,\opt{afterskip} 选项取负数;但使用 \opt{sub3section} 或
%   \opt{sub4section} 宏包选项(见 \ref{subsubs:options-heading}~节)后,则这
%   两级标题会改为排在不同段。
%
%   \opt{afterskip} 选项的默认值,在 \opt{cap} 选项的不同取值下相同。
% \end{function}
%
% \begin{table}[htbp]
% \small\centering
% \begin{tabular}{ll}
% \toprule
% 标题名 & 默认值 \\
% \midrule
% part (article) & |3ex| \\
% part & 无效 \\
% chapter & |40pt| \\
% section & |2.3ex plus .2ex| \\
% subsection & |1.5ex plus .2ex| \\
% subsubsection & |1.5ex plus .2ex| \\
% paragraph & |-1em| \\
% \qquad(sub3section) & |1ex plus .2ex| \\
% \qquad(sub4section) & 同上 \\
% subparagraph & |-1em| \\
% \qquad(sub4section) & |1ex plus .2ex| \\
% \bottomrule
% \end{tabular}
% \caption{\opt{afterskip} 选项的默认设置}
% \end{table}
%
% \begin{function}{part/indent, chapter/indent, section/indent,
%   subsection/indent, subsubsection/indent, paragraph/indent,
%   subparagraph/indent}
%   \begin{syntax}
%   indent = \Arg{缩进间距}
%   \end{syntax}
%   \opt{indent} 选项用于设置章节标题本身的首行缩进。
%
%   \opt{indent} 选项的默认值,在 \opt{cap} 选项的不同取值下相同。
% \end{function}
%
% \begin{ctexexam}
%   \ctexset{section={
%       format=\Large\bfseries,
%       indent=20pt,
%     }
%   }
%   \section{首行缩进的标题}
%   \noindent 无缩进的正文。
% \end{ctexexam}
%
% \begin{table}[htbp]
% \small\centering
% \begin{tabular}{ll}
% \toprule
% 标题名 & 默认值 \\
% \midrule
% part (article) & |0pt| \\
% part & 无效 \\
% chapter & |0pt| \\
% section & |0pt| \\
% subsection & |0pt| \\
% subsubsection & |0pt| \\
% paragraph & |0pt| \\
% subparagraph & |\parindent| \\
% \qquad(sub3section) & |0pt| \\
% \qquad(sub4section) & |0pt| \\
% \bottomrule
% \end{tabular}
% \caption{\opt{indent} 选项的默认设置}
% \end{table}
%
% \begin{function}{part/pagestyle, chapter/pagestyle}
%   \begin{syntax}
%   pagestyle = \Arg{页面格式}
%   \end{syntax}
%   设置 \cls{book}/\cls{ctexbook} 或 \cls{report}/\cls{ctexrep} 文档类
%   中,\tn{part} 与 \tn{chapter} 标题所在页的页面格式(page style)。
% \end{function}
%
% \begin{table}[htbp]
% \small\centering
% \begin{tabular}{ll}
% \toprule
% 标题名 & 默认值 \\
% \midrule
% part (article) & 无效 \\
% part & |plain| \\
% chapter & |plain| \\
% \bottomrule
% \end{tabular}
% \caption{\opt{pagestyle} 选项的默认设置}
% \end{table}
%
%
% \begin{function}{appendix/name}
%   \begin{syntax}
%     name = \{<前名字>,<后名字>\}
%     name = \Arg{前名字}
%   \end{syntax}
%   设置附录章(对 \cls{book} 与 \cls{report})或附录节(对 \cls{article})的
%   名字。用法与普通章节 \opt{name} 选项类似。
%
%   注意该选项与 \opt{appendixname} 选项(\ref{subsubs:capname}~节)在意义上有
%   些重叠,但意义不同。\opt{appendixname} 选项只用来重定义
%   \tn{appendixname},而不管 \tn{appendixname} 如何使用;该选项则决定在章节标
%   题中输出的名字,可以调用 \tn{appendixname} 设置。
% \end{function}
%
% \begin{table}[htbp]
% \small\centering
% \begin{tabular}{llllll}
% \toprule
% 文档类 & 影响命令 & |cap=true| 的默认值 & 实际定义
%   & |cap=false| 的默认值 & 实际定义 \\
% \midrule
% article & \tn{section} & |{}| & & |{}| & \\
% book, report & \tn{chapter} & |\appendixname| & |附录|
%   & |\appendixname| & |Appendix| \\
% \bottomrule
% \end{tabular}
% \caption{\opt{appendix/name} 选项的默认设置}
% \end{table}
%
% \begin{function}{appendix/number}
%   \begin{syntax}
%     number = \Arg{数字输出命令}
%   \end{syntax}
%   设置附录章(对 \cls{book} 与 \cls{report})或附录节(对 \cls{article})编
%   号的数字输出格式。用法与普通章节的 \opt{number} 选项类似。
%
%   该选项也同时控制附录章节计数器的交叉引用。
%
%   与普通章节的 \opt{number} 选项类似,同样需要注意,该选项不会影响计数器本身
%   的输出,即不影响 \tn{thesection} 或 \tn{thechapter} 的定义。
% \end{function}
%
% \begin{table}[htbp]
% \small\centering
% \begin{tabular}{llllll}
% \toprule
% 文档类 & 影响命令 & 默认值 \\
% \midrule
% article & \tn{section} & |\Alph{section}| \\
% book, report & \tn{chapter} & |\Alph{chapter}| \\
% \bottomrule
% \end{tabular}
% \caption{\opt{appendix/number} 选项的默认设置}
% \end{table}
%
% \subsubsection{页面格式设置与汉化}
% \label{subsubs:pagestyle}
%
% 当使用了 \CTeX 的文档类或是用 \pkg{ctex.sty} 加载了 \opt{heading} 选项时,会
% 设置整个文档的页面格式(page style)为 |headings|,即相当于设置了
% \begin{verbatim}[gobble=2]
% \pagestyle{headings}
% \end{verbatim}
% 在页眉中显示当前章节的编号与标题。
%
% 同时,\CTeX 宏包也会对默认的 |headings| 页面格式进行修改,使之调用
% \tn{CTEXthechapter}、\tn{CTEXthesection} 等宏来正确显示中文的章节编号。
%
% \CTeX 宏包的默认页面格式设置是经过汉化的 |headings|,其基本效果如本文档所
% 示,只在页眉一侧显示章节编号和标题,另一侧显示页码。
%
% 更复杂的页面格式可以通过调用 \pkg{fancyhdr}、\pkg{titleps} 等宏包来设
% 置。\CTeX 宏包为这些自定义页面格式的包提供了以下宏供使用:
% \begin{itemize}
% \item \tn{CTEXthechapter}、\tn{CTEXthesection} 等章节编号(见
% \ref{subsubs:secstyle} 节)。它们用来代替英文文档类中的
% \tn{thechapter}、\tn{thesection} 等宏。
%
% \item \tn{leftmark}、\tn{rightmark},它们是在使用章节标题命令后,自动设置的
% 宏。它们实际是在与章节标题命令对应的标记命令
% \tn{chaptermark}、\tn{sectionmark} 中调用 \tn{markright} 或 \tn{markboth} 生
% 成的。
% \end{itemize}
% 有关 \LaTeX 页面标记的涵义与使用细节,已经超出了本文档讨论的范围。可以参考
% \cite[Chapter~23]{knuthtex1986}、\cite[\S4.3, \S4.4]{mittelbach2004} 等书籍。
%
% 这里举一个例子,说明通过重定义 \tn{sectionmark},在 \cls{ctexart} 文档类中的
% 标准 |headings| 页面格式下控制页眉的方式:
% \begin{ctexexam}
%   \documentclass{ctexart}
%   \pagestyle{headings}
%   \ctexset{section={
%       name={第,节},
%       number=\arabic{section},
%     }
%   }
%   \renewcommand\sectionmark[1]{%
%     \markright{\CTEXthesection——#1}}
%
%   \begin{document}
%
%   \section{天地玄黄}
%   \newpage
%
%   \section{宇宙洪荒}
%
%   \end{document}
% \end{ctexexam}
% 在上例中,我们设置了页眉的形式是用破折号分开的节编号与节标题,即“第 1 节
% ——天地玄黄”、“第 2 节——宇宙洪荒”。
%
% 而当给 \CTeX 宏包使用了 \opt{fancyhdr} 选项时,则会自动调用 \pkg{fancyhdr}
% 宏包,并设置其 |fancy| 页面格式使用 \tn{CTEXthechapter} 等宏显示中文章节编
% 号。这样,我们可以更为灵活地控制页眉页脚的格式。
%
%
% 关于 \pkg{fancyhdr} 的具体用法可以参见其宏包手册。通常也只要像在标准的英文文
% 档类中使用 \pkg{fancyhdr} 一样定义页眉页脚格式即可,并不需要额外的定义。
%
% 下面我则给出一个与前例类似而稍复杂的例子,展示如何在文档中设置页眉内容与页眉
% 的格式。
% \begin{ctexexam}
%   \documentclass[fancyhdr]{ctexart}
%   \ctexset{section={
%       name={第,节},
%       number=\arabic{section},
%     }
%   }
%   \fancyhf{}
%   \lhead{\textnormal{\kaishu\rightmark}}
%   \rhead{--\ \thepage\ --}
%   \pagestyle{fancy}
%   % \sectionmark 的重定义需要在 \pagestyle 之后生效
%   \renewcommand\sectionmark[1]{%
%     \markright{\CTEXthesection——#1}}
%
%   \begin{document}
%
%   \section{天地玄黄}
%   \newpage
%
%   \section{宇宙洪荒}
%
%   \end{document}
% \end{ctexexam}
% 本例的页眉效果大致如下(有页眉线):
% \begin{trivlist}\item
% \textnormal{\kaishu 第 1 节——天地玄黄}\hfill -- 1 --\par
% \smallskip\hrule
% \end{trivlist}
%
% \subsection{关于 \LuaLaTeX{} 下的中文支持方式}
%
% 在 \LuaLaTeX{} 下,\CTeX{} 宏包依赖 \pkg{luatexja} 宏包来完成中文支持。
% 该宏包是日本 \TeX{} 社区的北川弘典、前田一贵、八登崇之等人开发的,设计目的主要
% 是在 \LuaTeX{} 引擎下实现日本 p\TeX{} 引擎的(大部分)功能。它为了兼容 p\LaTeX
% 的使用习惯,对 \LaTeXe 的 \pkg{NFSS} 作了不少修改和扩充。这对于简体中文用户来说
% 不是必要的,因而 \CTeX{} 禁用了它在 \LaTeX{} 格式下的大部分设置,只保留了必要的
% 部分。同时修改了它的字体设置方式,使得相关命令与 \pkg{xeCJK} 宏包大致相同。
%
% \subsubsection{\LuaLaTeX{} 下替代字体的设置}
%
% \begin{function}{AlternateFont}
%   \begin{syntax}
%     \tn{setCJKfamilyfont} \Arg{family}
%     \  [
%     \    AlternateFont =
%     \      \{
%     \        \Arg{character range_1} \oarg{alternate font features_1} \Arg{alternate font name_1} ||
%     \        \Arg{character range_2} \oarg{alternate font features_2} \Arg{alternate font name_2} ||
%     \        ......
%     \      \} ,
%     \    <base font features>
%     \  ] \Arg{base font name}
%   \end{syntax}
%   在设置字体族 \meta{family} 的时候,同时设置该字体族在字符范围
%   \meta{character range_n} 内,对应字形的替代字体。
% \end{function}
%
% \begin{function}{CharRange}
%   \begin{syntax}
%     \tn{setCJKfamilyfont} \Arg{family}
%     \  [
%     \    CharRange = \Arg{character range} ,
%     \    <alternate font features>
%     \  ] \Arg{alternate font name}
%   \end{syntax}
%   只设置字体族 \meta{family} 在字符范围 \meta{character range} 内,对应字形的
%   替代字体。
% \end{function}
%
% 一个 \tn{setCJKfamilyfont} 里只能使用一次 \opt{CharRange} 或者
% \opt{AlternateFont},但可以将它们分开重叠使用。例如下面的方式是有效的。
%
% \begin{ctexexam}
%   \setCJKmainfont[AlternateFont={...}{...}, ...]{...}
%   \setCJKmainfont[CharRange={"4E00->"67FF,-2}, ...]{...}
%   \setCJKmainfont[CharRange={"6800->"9FFF}, ...]{...}
% \end{ctexexam}
%
% \begin{function}{declarecharrange}
%   \begin{syntax}
%     \tn{ctexset}
%     \  \{
%     \    declarecharrange =
%     \      \{
%     \        \Arg{name_1} \Arg{character range_1} ,
%     \        \Arg{name_2} \Arg{character range_2} ,
%     \        ...
%     \      \}
%     \  \}
%   \end{syntax}
%   预先声明字符范围。声明字符范围 \meta{name} 之后,它的名字 \meta{name} 可以
%   用在 \opt{AlternateFont} 和 \opt{CharRange} 选项的 \meta{character range}
%   之中,表示对应的字符范围。
% \end{function}
%
% 在声明字符范围 \meta{name} 的同时,还为 \tn{setCJKmainfont} 等字体设置命令定义
% 了选项 \meta{name},用于设置对应字符的替代字体:
% \begin{quote}\linespread{1}\small\ttfamily
%   \meta{name} = \oarg{alternate font features} \Arg{alternate font name}
% \end{quote}
% \meta{name} 选项可以与 \opt{AlternateFont} 共同使用,但不能与 \opt{CharRange}
% 一起使用。如果没有给 \meta{name} 设置值,则等价于设置 \opt{CharRange=\meta{name}},
% 即只设置 \meta{name} 对应的字符范围的替代字体。
%
% \begin{function}{clearalternatefont,resetalternatefont}
%   \begin{syntax}
%     \tn{ctexset}
%     \  \{
%     \    clearalternatefont = \Arg{family_1, family_2, ...} ,
%     \    resetalternatefont = \Arg{family_1, family_2, ...} ,
%     \    clearalternatefont ,
%     \    resetalternatefont
%     \  \}
%   \end{syntax}
%   清除与重置 CJK 字体族 \meta{family} 的替换字体设置。如果没有给定值,则作用于
%   当前 CJK 字体族。清除与重置操作总是全局的。
% \end{function}
%
% \subsection{\CTeX{} 宏包的配置文件}
%
% \CTeX{} 宏包提供了不同的配置文件,可以通过修改配置文件来改变 \CTeX{} 宏包的
% 默认行为。
%
% 在多数情况下,并不需要修改配置文件,\CTeX{} 宏包的默认设置已经能满足大多数用
% 户的需要。不恰当地修改 \CTeX{} 宏包的默认行为也可能导致同一文件在别处无法正
% 常编译或排版效果完全不同,因此修改应该慎重。
%
% 但在一些情况下,直接修改配置文件仍是必要的,例如:
% \begin{itemize}
% \item 系统没有安装默认设置的字体文件,无法编译。
% \item 需要经常编译来自其他系统的中文 \TeX{} 文件,但对方的操作系统或默认设置
% 与本机不同。
% \end{itemize}
%
% 与 \CTeX{} 宏包的源代码一样,配置文件采用 \LaTeX3 的语法编写。
%
% \CTeX{} 宏包的配置文件随宏包其他文件一起安装在 \TeX{} 系统 TDS 目录树中,文
% 件后缀是 \file{.cfg}。为了避免本地配置文件内容因 \CTeX{} 宏包的更新而丢失,
% 不要直接修改系统 TDS 目录树中的配置文件,而应该将系统自带的配置文件复制到本
% 地的或用户私有的 TDS 目录树中修改,并运行 |texhash| 命令刷新文件名数据库。
%
% 例如对于 \TeX{} Live,系统自带的配置文件就在 \TeX{} Live 安装目录下的
% \path{texmf-dist/tex/latex/ctex/config/} 子目录下,可以修改它的复本,保存在
% 本地 TDS 树的 \path{texmf-local/tex/latex/ctex/} 目录下,或者用户 TDS 树的
% \path{~/.texlive2015/texmf-var/tex/latex/ctex/} 目录下,作为本地/用户专有的
% 配置文件。复制配置文件后需要运行 |texhash| 命令使本地配置文件生效。
%
% \hologo{MiKTeX} 的配置文件也保存在类似的目录结构中,\hologo{MiKTeX} 管理的
% 几个 TDS 根目录可以在 \hologo{MiKTeX} Options 设置项中查看到,这里不再赘述。
%
% 除了修改本地 \TeX{} 系统中的配置文件,对于特定文档,也可以将修改过的配置文件
% 保存在文档的工作目录下。此时配置文件就只对工作目录下的所有文档生效。
%
% \subsubsection{修改宏包默认选项}
%
% 配置文件 \file{ctexopts.cfg} 可以用来修改宏包的默认选项。随系统安装的配置文
% 件除了文件信息声明外没有实际的内容,但在注释中给出了一个简单的示例,只要取消
% 注释就可以生效。
%
% \begin{ctexexam}
%   % 系统自带 ctexopts.cfg 注释中的示例语句,固定默认字体集为 windowsnew。
%   % 该设置可以用在安装了 Windows 字体的非 Windows 系统中。
%   \keys_set:nn { ctex / option } { fontset = windowsnew }
% \end{ctexexam}
% 如上例所示,宏包选项通常使用 \LaTeX3 的 \cs{keys_set:nn} 命令完成键值设置,
% 第一个参数是固定的子模块 |ctex/option|,第二个参数中是用户定义的新的默认宏包
% 选项。
%
% \file{ctexopts.cfg} 中的设置将在 \CTeX{} 宏包的开始处,定义过宏包选项之后,
% \tn{ProcessKeysOptions} 命令之前生效。最好只使用此配置文件修改宏包默认选项。
%
% \subsubsection{宏包载入后的配置}
%
% 配置文件 \file{ctex.cfg} 将在宏包的末尾被载入生效。可以用它完成任意的设置,
% 或是覆盖已有的定义。随系统安装的配置文件除版本信息外没有实际内容,注意配置文
% 件中也使用 \LaTeX3 语法。
%
% \begin{ctexexam}
%   % 简单的 ctex.cfg 内容示例。
%   % 修改默认的页面格式设置。
%   \pagestyle{plain}
% \end{ctexexam}
%
% \begin{ctexexam}
%   % 略复杂的 ctex.cfg 内容示例:禁止段末孤字成行。
%   % 在使用 XeTeX 编译时,打开 xeCJK 的 CheckSingle 选项。
%   \xetex_if_engine:T
%     {
%       \xeCJKsetup { CheckSingle }
%     }
%   % 在使用 LuaTeX 编译时,设置 LuaTeX-ja 的 jcharwidowpenalty 参数。
%   \luatex_if_engine:T
%     {
%       \ltjsetparameter { jcharwidowpenalty = 10000 }
%     }
% \end{ctexexam}
%
% \subsubsection{配置标题中文翻译}
%
% 由于 \CTeX{} 宏包需要同时支持 GBK 和 UTF-8 两种编码,因此对标题的中文翻译写
% 在两个配置文件当中:\file{ctexcap-gbk.cfg} 和 \file{ctexcap-utf8.cfg}。两个
% 文件的设置相同,只是编码不同。
%
% 为了同一文档在不同电脑上编译效果的一致性,通常不建议修改默认的中文翻译。
%
% \subsubsection{自定义字体集}
%
% \ref{subsubs:options-CJK-font}~节介绍的用于 |fontset| 选项的自定义字库文件,
% 类似于 \CTeX{} 宏包的配置文件,也应该与其他本地配置文件一起保存在本地
% \texttt{TDS} 目录树下,并可以配合 \file{ctexopts.cfg} 等配置文件使用。
%
% \subsection{杂项}
%
% \begin{function}{\CTeX}
% 用于显示 \CTeX 的标志。
% \end{function}
%
% \section{对旧版本的兼容性}
%
% \subsection{\CTeX\ 0.8a 及以前的版本}
%
% 在 ctex-kit 项目成立之前,\CTeX 宏包的最后一个版本是 \CTeX\
% 0.8a(2007/05/06)。
%
% 第 2 版未考虑对这些很早版本的兼容性。
%
% \subsection{\CTeX\ 0.9--\CTeX\ 1.0d}
%
% 在 2009 年在 ctex-kit 项目成立后,新增了 \XeTeX{} 引擎的支持,并增加了不少控
% 制字体的命令和选项。
%
% 这里主要介绍新版本 \CTeX 宏包相对 1.02d 版本(2014/06/09)的兼容性。
%
% 第 2 版的 \CTeX 宏包已尽力保证对 1.0x 版本的兼容性,原有为 1.0x 编写的代码,
% 在第 2 版的 \CTeX 宏包下保证仍能编译,并且在大多数情况下保持编译效果不变。
%
% \CTeX 宏包在 0.8a 以前的版本支持以 CCT 作为底层中文支持方式,从 0.9 版之后即
% 不再推荐使用,只保留向后兼容。在 \CTeX 宏包第 2 版中则完全不再支持 CCT。
%
% 下面这些是在旧版本 \CTeX 宏包中存在,而在新版本中已不建议使用的选项和命令,
% 在未来版本中可能会删去它们的支持。
%
% 在多数情况下它们的功能仍将保留,但也有部分选项命令功能已失效。
%
% \begin{function}{CCT, CCTfont}
% 相关选项已删除。
% \end{function}
%
% \begin{function}{indent, noindent}
% 使用 \opt{indent} 宏包选项会载入 \pkg{indentfirst} 宏包。过时选项。
%
% \opt{noindent} 什么也不做,也不会使章节首行停止缩进。需要使用 \tn{ctexset}
% 命令设置章节的 \opt{beforeskip} 选项为正数才能完成原来 \opt{noindent} 选项的
% 工作。另外,设置 |cap=false| 可以保持原有英文文档类的效果,使章节首行停止缩
% 进。过时选项。
% \end{function}
%
% \begin{function}{zhmap, nozhmap}
% \opt{zhmap} 宏包选项增加了参数,扩充了功能,除了支持真假值参数外,还支持选择
% \pkg{zhmCJK} 作为底层中文处理宏包。(\ref{subsubs:options-CJK-font}~节)
%
% \opt{nozhmap} 选项相当于 |zhmap=false|。过时选项。
% \end{function}
%
% \begin{function}{winfonts, adobefonts, nofonts}
% 宏包选项 \opt{winfonts} 相当于 |fontset=windows|,\opt{adobefonts} 相当于
% |fontset=adobe|,\opt{nofonts} 相当于 |fontset=none|。这几个选项是过时选项,
% 对于新文档,应使用 \opt{fontset} 选项设置不同字体集。
%
% 另外,第 2 版 \CTeX 宏包的默认字体不再是 Windows 系统字体,而是根据检测到的
% 操作系统选择使用 Windows、Mac 的系统字体还是 Fandol 字体
% (\ref{subsubs:options-CJK-font}~节)。
% \end{function}
%
% \begin{function}{punct, nopunct}
% 旧版本中宏包 \opt{punct} 选项没有参数,现在可以用参数设定标点风格
% (\ref{subsubs:options-type-style}~节)。原有无参形式的 \opt{punct} 选项相当
% 于 |punct=quanjiao|。
%
% \opt{nopunct} 选项相当于 |punct=plain|。过时选项。
% \end{function}
%
% \begin{function}{cap, nocap}
% 新版本宏包 \opt{cap} 选项增加了真假值参数。
% (\ref{subsubs:options-type-style}~节)
%
% \opt{nocap} 选项相当于 |cap=false|,成为过时选项。
% \end{function}
%
% \begin{function}{space, nospace}
% 新版本宏包 \opt{space} 选项增加真假值参数。
% (\ref{subsubs:options-type-style}~节)
%
% \opt{nospace} 选项相当于 |space=false|,成为过时选项。
% \end{function}
%
% \begin{function}{fntef}
% 相比旧版本,新版本宏包选项会禁用 \pkg{CJKfntef} 或 \pkg{xeCJKfntef} 的彩色设
% 置。(\ref{subsubs:options-pkg}~节)
% \end{function}
%
% \begin{function}{\CTEXunderdot, \CTEXunderline, \CTEXunderdblline,
%   \CTEXunderwave, \CTEXsout, \CTEXxout, \CTEXfilltwosides}
% 在调用 \opt{fntef} 宏包选项的同时,旧版本 \CTeX{} 宏包由于需要支持 CCT 系
% 统,会将以 |\CJK| 开头的 \tn{CJKunderline} 等宏换名为以 |\CTEX| 开头的
% \tn{CTEXunderline} 等宏。此功能在新版本的 \CTeX{} 宏包中已失去意义。
%
% 此外,在 \pdfTeX{} 引擎下,用于设置格式的 \tn{CJKunderdotbasesep} 等宏也被换
% 名为 \tn{CTEXunderdotbasesep} 等宏。
%
% 在新版本中,上述由 \opt{fntef} 衍生的相关宏都成为过时命令。
% \end{function}
%
% \begin{function}{\CTEXindent}
% 更新 \tn{ccwd} 宽度后设置 |\parindent=2\ccwd|。过时命令。
% \end{function}
%
% \begin{function}{\CTEXnoindent}
% 设置 |\parindent=0pt|。过时命令。
% \end{function}
%
% \begin{function}{\CTEXsetup}
% \begin{syntax}
%   \tn{CTEXsetup}\oarg{选项}\Arg{标题}
% \end{syntax}
% 相当于设置了
% \texttt{\tn{ctexset}\{ \meta{标题} = \Arg{选项} \}}。
% 过时命令。
% \end{function}
%
% \begin{function}{\CTEXoptions}
% \begin{syntax}
%   \tn{CTEXoptions}\Arg{选项}
% \end{syntax}
% 相当于设置了
% \texttt{\tn{ctexset}\Arg{选项}}。
% 过时命令。
% \end{function}
%
% \begin{function}{captiondelimiter}
% 原为 \tn{CTEXoptions} 命令的选项,用于控制 \tn{caption} 编号后面的标点。此选
% 项已过时,并在新版本的 \CTeX 宏包中失效。
%
% 可以使用 \pkg{caption} 宏包的 \opt{labelsep} 选项来完成同样的功能。
% \begin{ctexexam}
%   % 代替 \CTEXoptions[captiondelimiter={:}]
%   \usepackage{caption}
%   \captionsetup{labelsep=colon}
% \end{ctexexam}
% \end{function}
%
%
% \subsection{\CTeX\ 1.02c 以后的 SVN 开发版}
%
% \CTeX 宏包在 1.02c 版本(2011/03/11)之后在 Google code 上的 SVN 开发版本,
% 内部版本号一直升到 1.11 版,但从未正式发布。SVN 开发版在 1.02c 版本的基础上
% 新增的功能在第 2 版中大多继承了过来,但新增的命令与选项都不再保持兼容。
%
% \CTeX 宏包第 2 版不保证对未发布的 SVN 开发版兼容。
%
% \section{开发人员}
%
% \begin{itemize}
% \item 吴凌云 (\email{aloft@ctex.org})
% \item 江疆 (\email{gzjjgod@gmail.com})
% \item 王越 (\email{yuleopen@gmail.com})
% \item 刘海洋 (\email{LeoLiu.PKU@gmail.com})
% \item 李延瑞 (\email{LiYanrui.m2@gmail.com})
% \item 陈之初 (\email{zhichu.chen@gmail.com})
% \item 李清 (\email{sobenlee@gmail.com})
% \item 黄晨成 (\email{liamhuang0205@gmail.com})
% \end{itemize}
%
% \begin{thebibliography}{9}
% \bibitem[{Knuth(1986c)}]{knuthtex1986}
% \textsc{Donald~Ervin Knuth}.
% \newblock \textit{The {{\TeX{}book}}}, \textit{Computers \& Typesetting},
%   volume~A.
% \newblock Addison-Wesley, 1986c
%
% \bibitem[{Mittelbach and Goossens(2004)}]{mittelbach2004}
% \textsc{Frank Mittelbach} and \textsc{Michel Goossens}.
% \newblock \textit{The {{\LaTeX}} Companion}.
% \newblock Tools and Techniques for Computer Typesetting. Boston:
%   Addison-Wesley, second edition, 2004
%
% \end{thebibliography}
%
% \end{documentation}
%
%
% \StopEventually{}
%
%
%\begin{implementation}
% \clearpage
% \section{代码实现}
%
%    \begin{macrocode}
%<@@=ctex>
%    \end{macrocode}
%
%    \begin{macrocode}
%<*ctexcap>
\PassOptionsToPackage { heading = true } { ctexcap }
\RequirePackageWithOptions { ctex }
%</ctexcap>
%    \end{macrocode}
%
%    \begin{macrocode}
%<*class|style|ctexsize>
\RequirePackage { xparse , l3keys2e }
%</class|style|ctexsize>
%    \end{macrocode}
%
% 检查 \pkg{expl3} 和 \pkg{l3keys2e} 的版本。
%    \begin{macrocode}
%<*class>
\msg_new:nnnn { ctex } { l3-too-old }
  { Support~package~`#1'~too~old. }
  {
    Please~update~an~up~to~date~version~of~the~bundles\\\\
    `l3kernel'~and~`l3packages'\\\\
    using~your~TeX~package~manager~or~from~CTAN.
  }
\@ifpackagelater { expl3 } { 2014/07/20 } { }
  { \msg_error:nnn { ctex } { l3-too-old } { expl3 } }
\@ifpackagelater { l3keys2e } { 2014/05/05 } { }
  { \msg_error:nnn { ctex } { l3-too-old } { l3keys2e } }
%</class>
%    \end{macrocode}
%
%    \begin{macrocode}
%<*class|style>
\RequirePackage { etoolbox , ifpdf , fix-cm , everysel }
%    \end{macrocode}
%
% \subsection{内部函数与变量}
%
% \begin{variable}[internal]
% {\l_@@_tmp_tl,\g_@@_tmp_bool,\l_@@_tmp_int,\l_@@_tmp_dim,\l_@@_tmp_box}
%    \begin{macrocode}
\tl_new:N \l_@@_tmp_tl
\bool_new:N \g_@@_tmp_bool
\int_new:N \l_@@_tmp_int
\dim_new:N \l_@@_tmp_dim
\box_new:N \l_@@_tmp_box
\seq_new:N \l_@@_tmp_seq
%    \end{macrocode}
% \end{variable}
%
% 对旧版本的宏包给出错误信息。
%    \begin{macrocode}
\msg_new:nnnn { ctex } { package-too-old }
  { Support~package~`#1'~too~old. }
  {
    Please~update~an~up~to~date~version~of~the~package~`#1'\\
    using~your~TeX~package~manager~or~from~CTAN.
  }
%    \end{macrocode}
%
% \begin{macro}[internal]{\ctex_lua_now_x:n}
% 最新的 \pkg{expl3} 去掉了 \pkg{l3luatex} 模块,因而 \cs{lua_now_x:n} 不再有定义。
%    \begin{macrocode}
\cs_new_eq:NN \ctex_lua_now_x:n \luatex_directlua:D
%    \end{macrocode}
% \end{macro}
%
% \begin{macro}[internal]{\ctex_if_pdfmode:TF}
% \tn{ifpdf} 的简单 wrapper。
%    \begin{macrocode}
\ifpdf
  \cs_new_eq:NN \ctex_if_pdfmode:TF \use_i:nn
\else:
  \cs_new_eq:NN \ctex_if_pdfmode:TF \use_ii:nn
\fi:
%    \end{macrocode}
% \end{macro}
%
% \begin{macro}[internal]{\ctex_if_preamble:TF}
% 测试是否在 \LaTeXe{} 的导言区。\LaTeXe{} 中的 \tn{@onlypreamble} 命令可以处
% 理宏参数,使其在 \env{document} 环境后被重定义为 \tn{@notprerr},而又由于
% \tn{@onlypreamble} 本身也被这样处理过,因此可以测试 \tn{@onlypreamble} 是否
% 与 \tn{@notprerr} 相同来确定是否在导言区。
%    \begin{macrocode}
\prg_new_conditional:Npnn \ctex_if_preamble: { TF }
  {
    \if_meaning:w \@onlypreamble \@notprerr
      \prg_return_false:
    \else:
      \prg_return_true:
    \fi:
  }
%    \end{macrocode}
% \end{macro}
%
% \begin{macro}[internal]{\ctex_file_input:n}
% 使用 \tn{@pushfilename} 和 \tn{@popfilename} 是为了让文件可以不受当前
% \LTXIII{} 语法环境的影响。
%    \begin{macrocode}
\cs_new_protected_nopar:Npn \ctex_file_input:n #1
  { \@pushfilename \file_input:n {#1} \@popfilename }
\@onlypreamble \ctex_file_input:n
%    \end{macrocode}
% \end{macro}
%
% \begin{macro}[internal]{\ctex_parse_name:NN}
% 用 \tn{DeclareRobustCommand} 定义的宏或者由 \tn{newcommand} 或 \tn{newrobustcmd}
% 定义的带一个可选参数的宏第一次展开的结果都不是其实际定义,实际定义被保存在另外的
% 宏中。由这些命令定义的宏的第一次展开结果可以有下面的形式(细节可查阅 \pkg{xpatch}
% 的文档):
% \begin{verbatim}[numbers=left,gobble=4]
%   \protect␣\xaa␣␣                    % \DeclareRobustCommand\xaa[1]{...}
%   \protect␣\xab␣␣                    % \DeclareRobustCommand\xab[1][]{...}
%   \@protected@testopt␣\xac␣\\xac␣{}  % \newcommand\xac[1][]{...}
%   \@testopt␣\\xad␣{}                 % \newrobustcmd\xad[1][]{...}
%   \x@protect␣\1\protect␣\1␣␣         % \DeclareRobustCommand\1[1]{...}
%   \x@protect␣\2\protect␣\2␣␣         % \DeclareRobustCommand\2[1][]{...}
%   \@protected@testopt␣\3\\3␣{}       % \newcommand\3[1][]{...}
%   \@testopt␣\\4␣{}                   % \newrobustcmd\4[1][]{...}
% \end{verbatim}
% \pkg{etoolbox} 的 \tn{patchcmd} 的主要原理是先对宏的 \tn{meaning} 作字符串
% 替换,然后再用 \tn{scantokens} 来重建它。我们希望对宏的实际定义打补丁,为此需要
% 先得到对应的名字。\pkg{letltxmacro}、\pkg{show2e} 和 \pkg{xpatch} 宏包中都有
% 类似的工作。我们不想依赖 \pkg{xpatch},主要是因为它与同作者的 \pkg{regexpatch}
% 宏包共用了主要函数的名字,从而将导致用户不能使用 \pkg{regexpatch}。
%    \begin{macrocode}
\cs_new_protected:Npn \ctex_parse_name:NN #1#2
  { \ctex_parse_name:NNx #1#2 { \cs_to_str:N #2 } }
\group_begin:
\char_set_lccode:nn { `\< } { `\{ }
\char_set_lccode:nn { `\/ } { `\\ }
\char_set_lccode:nn { `\A } { `\t }
\tl_map_function:nN { \A \E \S \O \P } \char_set_catcode_other:N
\tex_lowercase:D
  {
    \group_end:
    \cs_new_protected:Npn \ctex_parse_name:NNn #1#2#3
      {
        \bool_if:nTF { \cs_if_exist_p:c { #3 ~ } || \cs_if_exist_p:c { /#3 } }
          {
            \group_begin:
            \use:x
              {
                \@@_parse_name:nNNNnN { \token_get_replacement_spec:N #2 }
                  \exp_not:N #2 \exp_not:c { #3 ~ } \exp_not:c { /#3 } {#3}
              } #1
          }
          { #1#2 }
      }
    \cs_new_protected:Npn \@@_parse_name:nNNNnN #1#2#3#4#5#6
      {
        \group_end:
        \exp_args:Nc #6
          {
            \str_case:nnTF {#1}
              {
                { \protect #3 } { }
                { \x@protect #2 \protect #3 } { }
              }
              {
                \str_if_eq_x:nnTF { \exp_not:n { /@protected@ #3 /#3 } }
                  {
                    \exp_last_unbraced:Nf \@@_parse_name:w
                    \token_get_replacement_spec:N #3 AESAOPA ~ < \q_stop
                  }
                  { /#5 ~ } { #5 ~ }
              }
              {
                \str_case:onTF { \@@_parse_name:w #1 AESAOPA ~ < \q_stop }
                  {
                    { /@protected@ #2 #4 } { }
                    { /@ #4 } { }
                  }
                  { /#5 } {#5}
              }
          }
      }
    \cs_new:Npn \@@_parse_name:w #1 AESAOPA ~ #2 < #3 \q_stop { #1#2 }
  }
\cs_generate_variant:Nn \ctex_parse_name:NNn { NNx }
%    \end{macrocode}
% \end{macro}
%
% \begin{macro}[internal]
% {\ctex_patch_cmd:NnnTF,\ctex_preto_cmd:NnTF,\ctex_appto_cmd:NnTF}
% 在打补丁前先解析实际名字。
%    \begin{macrocode}
\cs_new_protected:Npn \ctex_patch_cmd:NnnTF { \ctex_parse_name:NN \patchcmd }
\cs_new_protected:Npn \ctex_preto_cmd:NnTF  { \ctex_parse_name:NN \pretocmd }
\cs_new_protected:Npn \ctex_appto_cmd:NnTF  { \ctex_parse_name:NN \apptocmd }
%    \end{macrocode}
% \end{macro}
%
% \begin{macro}[internal]{\ctex_patch_cmd:Nnn}
% 参数记号 |#| 作为宏的参数被读入时,总是会双写,会影响随后的字符串替换。需要先
% 将它转换为普通符号。并且在补丁的时候关闭 \LTXIII{} 语法。
%    \begin{macrocode}
\cs_new_protected:Npn \ctex_patch_cmd:Nnn
  {
    \group_begin:
    \char_set_catcode_other:N \#
    \@@_patch_cmd:Nnn
  }
\cs_new_protected:Npn \@@_patch_cmd:Nnn #1#2#3
  {
    \group_end:
    \group_begin:
    \ExplSyntaxOff
    \ctex_patch_cmd:NnnTF #1 {#2} {#3}
      {
        \cs_gset_eq:NN \@@_tmp:w #1
        \group_end:
        \cs_set_eq:NN #1 \@@_tmp:w
        \cs_undefine:N \@@_tmp:w
      }
      { \group_end: \ctex_patch_failure:N #1 }
  }
\cs_new_protected:Npn \ctex_patch_failure:N #1
  { \msg_warning:nnx { ctex } { patch-failure } { \token_to_str:N #1 } }
\msg_new:nnn { ctex } { patch-failure }
  {
    Patching~command~`#1'~failed.\\
    ctex~may~not~work~as~expected.
  }
%    \end{macrocode}
% \end{macro}
%
% \begin{macro}[internal]{\ctex_default_pt:n}
% 最新版本的 \pkg{expl3} 已经不允许 \cs{dim_to_pt:n} 的参数带额外的单位。然而我们
% 需要这个特性实现可展的 \tn{@defaultunits}。
%    \begin{macrocode}
\cs_new:Npn \ctex_default_pt:n #1
  {
    \exp_after:wN \@@_default_pt:w
      \dim_use:N \etex_dimexpr:D #1 pt \scan_stop: \q_stop
  }
\group_begin:
  \char_set_catcode_other:N \P
  \char_set_catcode_other:N \T
\tex_lowercase:D
  {
    \group_end:
    \cs_new:Npn \@@_default_pt:w #1 PT #2 \q_stop { #1 PT }
  }
%    \end{macrocode}
% \end{macro}
%
% \begin{variable}[internal]{\l_@@_encoding_tl}
% (pdf)\LaTeX{} 初始化编码为 GBK,其它则是 UTF8。
%    \begin{macrocode}
\tl_new:N \l_@@_encoding_tl
\tl_set:Nx \l_@@_encoding_tl
  { \pdftex_if_engine:TF { GBK } { UTF8 } }
%    \end{macrocode}
% \end{variable}
%
% \begin{variable}[internal]{\g_@@_section_depth_flag}
% 若大于 |3|,则 \tn{paragraph} 和 \tn{subparagraph} 标题单独占一行;若为 |3|,则
% \tn{paragraph} 单独占一行。
%    \begin{macrocode}
\cs_new_eq:NN \g_@@_section_depth_flag \c_two
%    \end{macrocode}
% \end{variable}
%
% \begin{variable}[internal]{\g_@@_zhmCJK_bool}
% 是否使用 \pkg{zhmCJK} 宏包。
%    \begin{macrocode}
\bool_new:N \g_@@_zhmCJK_bool
%    \end{macrocode}
% \end{variable}
%
% \begin{macro}[internal]{\ctex_zhmap_case:nnn}
% 参数 |#1| 是 \pkg{zhmCJK} 的内容,|#2| 是 \pkg{zhmetrics}。
%    \begin{macrocode}
\cs_new_eq:NN \ctex_zhmap_case:nnn \use_ii:nnn
%    \end{macrocode}
% \end{macro}
%
% \begin{macro}[internal]{\ctex_at_end:n}
% 区分 \tn{AtEndOfClass} 和 \tn{AtEndOfPackage},虽然它们的意思都是一样的。
%    \begin{macrocode}
%<class>\cs_new_protected_nopar:Npn \ctex_at_end:n { \AtEndOfClass }
%<style>\cs_new_protected_nopar:Npn \ctex_at_end:n { \AtEndOfPackage }
%    \end{macrocode}
% \end{macro}
%
% \begin{variable}[internal]{\g_@@_std_options_clist}
% 保存传递给标准文档类的选项。
%    \begin{macrocode}
%<*class>
\clist_new:N \g_@@_std_options_clist
%</class>
%    \end{macrocode}
% \end{variable}
%
% 对无效选项给出警告。
%    \begin{macrocode}
\msg_new:nnn { ctex } { invalid-option }
  { Option~`\l_keys_key_tl'~is~invalid~in~current~mode. }
\msg_new:nnn { ctex } { invalid-value }
  { Value~`#1'~is~invalid~for~the~key~`\l_keys_key_tl'. }
%    \end{macrocode}
%
% 对过时选项或命令给出警告。
%    \begin{macrocode}
\msg_new:nnn { ctex } { deprecated-option }
  { Option~ `\l_keys_key_tl'~ is~ deprecated.\\ #1 }
\msg_new:nnn { ctex } { deprecated-command }
  { Command~ #1 is~ deprecated.\\ #2 }
\msg_new:nnn { ctex } { deprecated-environment }
  { Environment~ `#1'~ is~ deprecated.\\ #2 }
%    \end{macrocode}
%
%    \begin{macrocode}
%</class|style>
%<*class|style|ctexsize>
%    \end{macrocode}
%
% \begin{variable}[internal]{\g_@@_font_size_flag}
% |0| 表示修改默认字体大小为五号,|1| 为小四号,其它值则不作修改。
%    \begin{macrocode}
\cs_new_eq:NN \g_@@_font_size_flag \c_zero
%    \end{macrocode}
% \end{variable}
%
% \subsection{宏包选项}
%
%   \begin{macrocode}
\keys_define:nn { ctex / option }
  {
%    \end{macrocode}
%
% \begin{macro}{c5size,cs4size}
%    \begin{macrocode}
    c5size  .code:n = { \cs_gset_eq:NN \g_@@_font_size_flag \c_zero } ,
    cs4size .code:n = { \cs_gset_eq:NN \g_@@_font_size_flag \c_one } ,
    c5size  .value_forbidden: ,
    cs4size .value_forbidden: ,
%<ctexsize>  }
%    \end{macrocode}
% \end{macro}
%
%    \begin{macrocode}
%</class|style|ctexsize>
%<*class|style>
%    \end{macrocode}
%
% \changes{v2.0}{2014/04/23}{新增 \opt{linespread} 选项。}
%
% \begin{macro}{linespread}
% 行距初始值为 $1.3\times 1.2=2.56$ 倍字体大小。
%    \begin{macrocode}
    linespread  .fp_set:N = \l_@@_line_spread_fp ,
    linespread .initial:n = { 1.3 } ,
    linespread .value_required: ,
%    \end{macrocode}
% \end{macro}
%
% \changes{v2.0}{2014/03/13}{新增 \opt{autoindent} 选项。}
%
% \begin{macro}{autoindent}
% 自动调整段落的首行缩进功能。
%    \begin{macrocode}
    autoindent .bool_set:N = \l_@@_autoindent_bool ,
    autoindent  .initial:n = { true } ,
%    \end{macrocode}
% \end{macro}
%
% \changes{v2.0}{2015/03/21}{\opt{indent}, \opt{noindent} 是过时选项。}
% \begin{macro}{indent}
% 仅为兼容性保留,已过时。
%    \begin{macrocode}
    indent .code:n =
      {
        \msg_warning:nnn { ctex } { deprecated-option }
          {
            The~ indentfirst~ package~ will~ be~ loaded~ but~ the~
            functionality~ may~ be~ removed~ in~ later~ version.
            It's~ better~ to~ set~ the~ heading~ styles~ via~ beforeskip~
            options.~
          }
        \bool_set_true:N \l_@@_indent_bool
      } ,
    indent .value_forbidden: ,
    noindent .code:n =
      {
        \msg_warning:nnn { ctex } { deprecated-option }
          {
            The~ functionality~ has~ been~ removed.~
            It's~ better~ to~ set~ the~ heading~ styles~ via~ beforeskip~
            options.
          }
        \bool_set_false:N \l_@@_indent_bool
      } ,
    noindent .value_forbidden: ,
%    \end{macrocode}
% \end{macro}
%
% \begin{macro}{GBK,UTF8}
%   \begin{macrocode}
    GBK  .code:n = { \tl_set:Nn \l_@@_encoding_tl { GBK } } ,
    UTF8 .code:n = { \tl_set:Nn \l_@@_encoding_tl { UTF8 } } ,
    GBK  .value_forbidden: ,
    UTF8 .value_forbidden: ,
%    \end{macrocode}
% \end{macro}
%
% \changes{v2.0}{2014/03/08}{新增 \opt{fontset} 选项。}
% \changes{v2.0}{2015/03/21}{\opt{nofonts}, \opt{adobefonts}, \opt{winfonts}
% 是过时选项。}
%
% \begin{macro}{fontset}
% 初始值为空。若用户未指定,则根据操作系统载入对应字体配置,可以区分 Windows、
% Mac~OS~X 和其它。
%   \begin{macrocode}
    fontset    .tl_gset:N = \g_@@_fontset_tl ,
    fontset    .value_required: ,
    nofonts    .code:n =
      {
        \msg_warning:nnn { ctex } { deprecated-option }
          {
            Option~ `fontset=none'~ is~ set.~ It~ is~ better~ to~ use~
            fontset~ option.
          }
        \keys_set:nn { ctex / option } { fontset = none }
      } ,
    adobefonts .code:n =
      {
        \msg_warning:nnn { ctex } { deprecated-option }
          {
            Option~ `fontset=adobe'~ is~ set.~ It~ is~ better~ to~ use~
            fontset~ option.
          }
        \keys_set:nn { ctex / option } { fontset = none }
      } ,
    winfonts   .code:n =
      {
        \msg_warning:nnn { ctex } { deprecated-option }
          {
            Option~ `fontset=windows'~ is~ set.~ It~ is~ better~ to~ use~
            fontset~ option.
          }
        \keys_set:nn { ctex / option } { fontset = none }
      } ,
    nofonts    .value_forbidden: ,
    winfonts   .value_forbidden: ,
    adobefonts .value_forbidden: ,
%    \end{macrocode}
% \end{macro}
%
% \changes{v2.0}{2014/03/08}{新增 \opt{zhmCJK} 支持选项。}
% \changes{v2.0}{2015/03/22}{\opt{nozhmap} 是过时选项。}
%
% \begin{macro}{zhmap}
%   \begin{macrocode}
    zhmap .choice: ,
    zhmap .default:n = { true } ,
    zhmap / zhmCJK .code:n =
      {
        \bool_gset_true:N \g_@@_zhmCJK_bool
        \cs_gset_eq:NN \ctex_zhmap_case:nnn \use_i:nnn
      } ,
    zhmap / true   .code:n =
      {
        \bool_gset_false:N \g_@@_zhmCJK_bool
        \cs_gset_eq:NN \ctex_zhmap_case:nnn \use_ii:nnn
      } ,
    zhmap / false  .code:n =
      {
        \bool_gset_false:N \g_@@_zhmCJK_bool
        \cs_gset_eq:NN \ctex_zhmap_case:nnn \use_iii:nnn
      } ,
    nozhmap   .code:n =
      {
        \msg_warning:nnn { ctex } { deprecated-option }
          { Option~ `zhmap=false'~ is~ set. }
        \keys_set:nn { ctex / option } { zhmap = false }
      } ,
    nozhmap   .value_forbidden: ,
%    \end{macrocode}
% \end{macro}
%
% \changes{v2.0}{2014/04/11}{\opt{punct} 选项可以设置标点格式。}
% \changes{v2.0}{2015/03/21}{\opt{nopunct} 是过时选项。}
%
% \begin{macro}{punct}
% 设置标点符号输出格式。
%   \begin{macrocode}
    punct   .tl_set:N = \l_@@_punct_tl ,
    punct  .default:n = { quanjiao } ,
    punct  .initial:n = { quanjiao } ,
    nopunct   .code:n =
      {
        \msg_warning:nnn { ctex } { deprecated-option }
          { Option~ `punct=plain'~ is~ set.  }
        \keys_set:nn { ctex / option } { punct = plain }
      } ,
    nopunct   .value_forbidden: ,
%    \end{macrocode}
% \end{macro}
%
% \changes{v2.0}{2015/03/22}{\opt{nospace} 是过时选项。}
% \begin{macro}{space}
%   \begin{macrocode}
    space .choices:nn =
      { true , auto , false }
      {
        \exp_args:Nx \ctex_at_end:n
          { \keys_set:nn { ctex } { space = \l_keys_choice_tl } }
      } ,
    space  .default:n = { true } ,
    nospace   .code:n =
      {
        \msg_warning:nnn { ctex } { deprecated-option }
          { Option~ `space=false'~ is~ set. }
        \keys_set:nn { ctex / option } { space = false }
      } ,
    nospace   .value_forbidden: ,
%    \end{macrocode}
% \end{macro}
%
% \changes{v2.0}{2014/03/08}{\pkg{ctex.sty} 新增 \opt{heading} 选项。}
%
% \begin{macro}{heading}
%   \begin{macrocode}
    heading .bool_set:N = \l_@@_heading_bool ,
%    \end{macrocode}
% \end{macro}
%
% \changes{v2.0}{2015/03/22}{\opt{nocap} 是过时选项。}
% \begin{macro}{cap}
%    \begin{macrocode}
    cap .bool_set:N = \l_@@_caption_bool ,
    cap  .initial:n = { true } ,
    nocap   .code:n =
      {
        \msg_warning:nnn { ctex } { deprecated-option }
          { Option~ `cap=false'~ is~ set. }
        \keys_set:nn { ctex / option } { cap = false }
      } ,
    nocap   .value_forbidden: ,
%    \end{macrocode}
% \end{macro}
%
% \begin{macro}{sub3section,sub4section}
% \begin{macrocode}
    sub3section .code:n =
      { \cs_gset_eq:NN \g_@@_section_depth_flag \c_three } ,
    sub4section .code:n =
      { \cs_gset_eq:NN \g_@@_section_depth_flag \c_four } ,
    sub3section .value_forbidden: ,
    sub4section .value_forbidden: ,
%    \end{macrocode}
% \end{macro}
%
% \begin{macro}{fntef,fancyhdr,hyperref}
%   \begin{macrocode}
    fntef    .bool_set:N = \l_@@_fntef_bool ,
    fancyhdr .bool_set:N = \l_@@_fancyhdr_bool ,
    hyperref .bool_set:N = \l_@@_hyperref_bool
  }
%    \end{macrocode}
% \end{macro}
%
%    \begin{macrocode}
%</class|style>
%<*class|style|ctexsize>
%    \end{macrocode}
%
% \begin{macro}{10pt,11pt,12pt}
% 使 \pkg{ctex} 和 \pkg{ctexsize} 可以接受文档类的全局选项,不修改默认字体大小。
% 在文档类下还将参数传给标准文档类。
%    \begin{macrocode}
\clist_map_inline:nn { 10pt , 11pt , 12pt }
  {
    \keys_define:nn { ctex / option }
      {
        #1 .code:n =
%<*!class>
          { \cs_gset_eq:NN \g_@@_font_size_flag \c_minus_one } ,
%</!class>
%<*class>
          {
            \cs_gset_eq:NN \g_@@_font_size_flag \c_minus_one
            \clist_gput_right:Nn \g_@@_std_options_clist {#1}
          } ,
%</class>
        #1 .value_forbidden:
      }
  }
%    \end{macrocode}
% \end{macro}
%
% 将未知选项传给标准文档类。
%    \begin{macrocode}
%<*class>
\keys_define:nn { ctex / option }
  {
    unknown .code:n =
      { \clist_gput_right:No \g_@@_std_options_clist { \CurrentOption } }
  }
%</class>
%    \end{macrocode}
%
%    \begin{macrocode}
%<!ctexsize>\ctex_file_input:n { ctexopts.cfg }
%    \end{macrocode}
%
%    \begin{macrocode}
\ProcessKeysOptions { ctex / option }
%    \end{macrocode}
%
%    \begin{macrocode}
%</class|style|ctexsize>
%<*class|style>
%    \end{macrocode}
%
% 五号字使用标准文档类的 |10pt| 字体大小设置,小四号字则使用 |12pt|。
%    \begin{macrocode}
%<*class>
\if_case:w \g_@@_font_size_flag
  \clist_gput_right:Nn \g_@@_std_options_clist { 10pt }
\or:
  \clist_gput_right:Nn \g_@@_std_options_clist { 12pt }
\fi:
%    \end{macrocode}
%
% 使用 \tn{PassOptionsToClass} 是为了预防可能存在的选项冲突。
%    \begin{macrocode}
%<*article>
\PassOptionsToClass { \g_@@_std_options_clist } { article }
\LoadClass { article }
%</article>
%<*book>
\PassOptionsToClass { \g_@@_std_options_clist } { book }
\LoadClass { book }
%</book>
%<*report>
\PassOptionsToClass { \g_@@_std_options_clist } { report }
\LoadClass { report }
%</report>
%</class>
%    \end{macrocode}
%
%    \begin{macrocode}
\tl_set_eq:Nc \l_@@_tmp_tl { ver@ \@currname . \@currext }
%<*class>
\cs_new_eq:cN { ver@ctex.     \@pkgextension } \l_@@_tmp_tl
\cs_new_eq:cN { ver@ctexcap.  \@pkgextension } \l_@@_tmp_tl
\cs_new_eq:cN { ver@ctexsize. \@pkgextension } \l_@@_tmp_tl
%</class>
%<*style>
\msg_new:nnnn { ctex } { ctexsize-loaded }
  { Package~`ctexsize'~can~not~be~loaded~before~`ctex'. }
  {
    `ctexsize'~is~actually~a~part~of~`ctex'.\\
    It~is~not~necessary~to~load~it~separately.
  }
\@ifpackageloaded { ctexsize }
  { \msg_error:nn { ctex } { ctexsize-loaded } }
  { \cs_new_eq:cN { ver@ctexsize. \@pkgextension } \l_@@_tmp_tl }
%</style>
%    \end{macrocode}
%
% \subsection{用户设置接口}
%
% \changes{v2.0}{2014/03/18}{新增统一设置接口 \tn{ctexset}。}
%
% \begin{macro}{\ctexset}
%    \begin{macrocode}
\NewDocumentCommand \ctexset { +m }
  { \keys_set:nn { ctex } {#1} }
%    \end{macrocode}
% \end{macro}
%
% \changes{v2.0}{2015/03/21}{\tn{CTEXsetup}, \tn{CTEXoptions} 是过时命令。}
% \begin{macro}{\CTEXsetup,\CTEXoptions}
% 过时命令。
%    \begin{macrocode}
\NewDocumentCommand \CTEXsetup { +o > { \TrimSpaces } m }
  {
    \msg_warning:nnnn { ctex } { deprecated-command } { \CTEXsetup }
      { \ctexset~ {~ #2~ =~ {~ #1~ }~ }~ is~ set. }
    \IfNoValueF {#1} { \keys_set:nn { ctex / #2 } {#1} }
  }
\NewDocumentCommand \CTEXoptions { +o }
  {
    \msg_warning:nnnn { ctex } { deprecated-command } { \CTEXoptions }
      { \ctexset~ {~ #1~ }~ is~ set. }
    \IfNoValueF {#1} { \keys_set:nn { ctex } {#1} }
  }
%    \end{macrocode}
% \end{macro}
%
% \subsection{引擎支持}
%
% \begin{macro}[internal]{\hypersetup}
%    \begin{macrocode}
\bool_if:NT \l_@@_hyperref_bool
  {
    \cs_if_exist:NF \hypersetup
      {
        \cs_new_protected:Npn \hypersetup #1
          { \PassOptionsToPackage {#1} { hyperref } }
      }
    \hypersetup { colorlinks = true }
    \AtEndPreamble { \RequirePackage { hyperref } }
  }
%    \end{macrocode}
% \end{macro}
%
%    \begin{macrocode}
\pdftex_if_engine:TF
  {
    \tl_set:Nx \l_@@_encoding_tl { \l_@@_encoding_tl }
    \ctex_file_input:n { ctex-engine-pdftex.def }
  }
  {
    \tl_set:Nn \l_@@_encoding_tl { UTF8 }
    \xetex_if_engine:TF
      { \ctex_file_input:n { ctex-engine-xetex.def } }
      { \ctex_file_input:n { ctex-engine-luatex.def } }
  }
%    \end{macrocode}
%
%    \begin{macrocode}
%</class|style>
%<*pdftex>
%    \end{macrocode}
%
% \subsubsection{\pkg{ctex-engine-pdftex.def}}
%
% 首先检查选项,决定是否载入 \pkg{zhmCJK} 宏包。
%    \begin{macrocode}
\if_bool:N \g_@@_zhmCJK_bool
  \PassOptionsToPackage { encoding = \l_@@_encoding_tl } { zhmCJK }
  \RequirePackage { zhmCJK }
%    \end{macrocode}
% 不载入 \pkg{zhmCJK} 宏包时直接调用 \pkg{CJK} 及相关宏包。
%    \begin{macrocode}
\else:
  \str_if_eq:onTF { \l_@@_encoding_tl } { GBK }
    { \RequirePackage { CJK } }
    { \RequirePackage { CJKutf8 } }
  \RequirePackage { CJKpunct , CJKspace }
%    \end{macrocode}
%
% \begin{macro}[internal]{\ctex_load_zhmap:nnnn}
% 载入 \pkg{zhmetrics} 的字体映射文件,同时设置 \tn{CJKrmdefault} 等。
%    \begin{macrocode}
  \cs_new_protected_nopar:Npn \ctex_load_zhmap:nnnn #1#2#3#4
    {
      \tl_set:Nn \CJKrmdefault {#1}
      \tl_set:Nn \CJKsfdefault {#2}
      \tl_set:Nn \CJKttdefault {#3}
      \AtBeginDvi { \file_input:n {#4} }
      \AtBeginDocument
        { \cs_if_exist_use:NT \AtBeginShipoutFirst { { \file_input:n {#4} } } }
    }
  \@onlypreamble \ctex_load_zhmap:n
%    \end{macrocode}
% \end{macro}
%
%    \begin{macrocode}
  \tl_if_exist:NF \CJKfamilydefault
    { \tl_const:Nn \CJKfamilydefault { \CJKrmdefault } }
  \tl_if_exist:NF \CJKrmdefault { \tl_new:N \CJKrmdefault }
  \tl_if_exist:NF \CJKsfdefault { \tl_new:N \CJKsfdefault }
  \tl_if_exist:NF \CJKttdefault { \tl_new:N \CJKttdefault }
  \ctex_preto_cmd:NnTF \rmfamily { \CJKfamily { \CJKrmdefault } } { }
    { \ctex_patch_failure:N \rmfamily }
  \ctex_preto_cmd:NnTF \sffamily { \CJKfamily { \CJKsfdefault } } { }
    { \ctex_patch_failure:N \sffamily }
  \ctex_preto_cmd:NnTF \ttfamily { \CJKfamily { \CJKttdefault } } { }
    { \ctex_patch_failure:N \ttfamily }
  \ctex_preto_cmd:NnTF \normalfont { \CJKfamily { \CJKfamilydefault } }
    { \cs_set_eq:NN \reset@font \normalfont }
    { \ctex_patch_failure:N \normalfont }
%    \end{macrocode}
%
% \pkg{zhmCJK} 判断结束。
%    \begin{macrocode}
\fi:
%    \end{macrocode}
%
% \begin{macro}[internal]{\ctex_CJK_input:n,\CJK@input}
% \pkg{breqn} 包可能会在正文中将 |^| 的 \tn{catcode} 改为 $12$ 或 $13$,这将
% 破坏 \pkg{CJK} 对汉字的首字节的定义(\tn{CJK@loadBinding} 和
% \tn{CJK@loadEncoding})。因此需要确保载入 \file{.enc} 和 \file{.bdg} 文件时,
% |^| 的 \tn{catcode} 为 $7$。
%    \begin{macrocode}
\cs_new_protected_nopar:Npn \ctex_CJK_input:n #1
  {
    \use:x
      {
        \char_set_catcode_other:n            { 60 } % <
        \char_set_catcode_letter:n           { 64 } % @
        \char_set_catcode_math_superscript:n { 94 } % ^
        \int_set_eq:NN \tex_endlinechar:D \c_minus_one
        \file_input:n {#1}
        \int_set:Nn \tex_endlinechar:D { \int_use:N \tex_endlinechar:D }
        \char_set_catcode:nn { 60 } { \char_value_catcode:n { 60 } }
        \char_set_catcode:nn { 64 } { \char_value_catcode:n { 64 } }
        \char_set_catcode:nn { 94 } { \char_value_catcode:n { 94 } }
      }
  }
\cs_set_eq:NN \CJK@input \ctex_CJK_input:n
%    \end{macrocode}
% \end{macro}
%
% \begin{macro}[internal]{\ctex_plane_to_utfxvibe:Nn,\CJK@surr}
% \changes{v2.0}{2014/04/08}{解决与 \tn{nouppercase} 的冲突。}
% \pkg{fancyhdr} 宏包的 \tn{nouppercase} 会将 \tn{uppercase} 定义为 \tn{relax},而
% \tn{CJK@surr} 需要用它将 \tn{CJK@plane} 转化成大写字母,这就造成了冲突^^A
% \footnote{\url{https://github.com/CTeX-org/ctex-kit/issues/146}}。
% 我们在这里给出 \tn{CJK@surr} 的一个不依赖 \tn{uppercase} 的实现。
%    \begin{macrocode}
\if_cs_exist:N \CJK@surr
  \cs_new_protected_nopar:Npn \ctex_plane_to_utfxvibe:Nn #1#2
    {
      \tl_set:Nx \l_@@_tmp_tl {#2}
      \int_set:Nn \l_@@_tmp_int
        { \exp_args:No \int_from_hex:n { \l_@@_tmp_tl } }
      \int_compare:nNnTF \l_@@_tmp_int < \c_two_hundred_fifty_six
        { \tl_gset:Nx #1 { \int_to_Hex:n { \l_@@_tmp_int } } }
        {
          \int_sub:Nn \l_@@_tmp_int { \c_two_hundred_fifty_six }
          \tl_gset:Nx #1
            {
              \int_to_Hex:n
                { \int_div_truncate:nn { \l_@@_tmp_int } { \c_four } + "D800 }
              \int_to_Hex:n
                { \int_mod:nn { \l_@@_tmp_int } { \c_four } + "DC }
            }
        }
    }
  \cs_set_eq:NN \CJK@surr \ctex_plane_to_utfxvibe:Nn
\fi:
%    \end{macrocode}
% \end{macro}
%
% \pkg{CJKpunct} 宏包会在 \tn{AtBeginDocument} 的里设置标点格式为 \opt{quanjiao}。
%    \begin{macrocode}
\AtBeginDocument
  {
    \str_if_eq_x:nnF { \l_@@_punct_tl } { quanjiao }
      { \punctstyle { \l_@@_punct_tl } }
  }
%    \end{macrocode}
%
% 启用中文字符功能,将汉字的首字节设置为活动字符,并对这些字符初始化。
% \tn{CJK@makeActive} 与 \tn{CJK@@enc} 应该先于 \file{ctexcap-gbk.cfg} 等文件
% 的载入。注意 \tn{CJK@@enc} 需要调用补丁后的 \tn{CJK@input}。使用
% \pkg{zhmCJK} 时,此功能已经被启用。
%    \begin{macrocode}
\bool_if:NF \g_@@_zhmCJK_bool
  {
    \CJK@makeActive
%<@@=>
    \CJK@@enc
  }
%    \end{macrocode}
%    \begin{macrocode}
%<@@=ctex>
%    \end{macrocode}
%
% 在导言区结束时调用 \tn{CJK@envStart} 启用完整的中文功能。
%
% \tn{CJK@envStart} 的定义是
% \begin{verbatim}
%   \def\CJK@envStart#1#2#3{
%     \CJK@upperReset
%     \ifCJK@lowercase@
%       \CJK@lowerReset
%     \fi%
%     \CJK@makeActive%
%     \CJK@global\let\CJK@selectFamily \CJK@selFam
%     \CJK@global\let\CJK@selectEnc \CJK@selEnc%
%     \def\CJK@@@enc{#2}
%     \ifx\CJK@@@enc \@empty
%       \PackageInfo{CJK}{
%         no encoding parameter given,\MessageBreak
%         waiting for \protect\CJKenc\space commands}
%     \else
%       \CJKenc{#2}
%     \fi
%     \CJKfontenc{#2}{#1}
%     \CJKfamily{#3}
%     \def\CJK@series{\f@series}
%     \def\CJK@shape{\f@shape}%
%     \csname CJKhook\endcsname}
% \end{verbatim}
% \tn{CJK@upperReset} 可能会有一定风险,因此我们直到导言区末尾才使用
% \tn{CJK@envStart}。这样可以避免将 \env{CJK} 环境内置入 \env{document} 环境的
% 最里层,最后也就不需要 \tn{clearpage}。
%    \begin{macrocode}
\exp_args:Nx \AtEndPreamble
  {
    \exp_not:N \CJK@envStart
      { } { \l_@@_encoding_tl } { \exp_not:N \CJKfamilydefault }
    \exp_not:N \CJKtilde
  }
%    \end{macrocode}
%
% \begin{macro}[internal]{\CJK@ignorespaces,\ctex_auto_ignorespaces:}
% 默认忽略汉字之间的空格。并关闭名字空间,保存 |\CJK@@ignorespaces| 的定义,方便使用。
%    \begin{macrocode}
%<@@=>
\cs_set_eq:NN \CJK@ignorespaces \CJK@@ignorespaces
\cs_new_eq:NN \ctex_auto_ignorespaces: \CJK@@ignorespaces
%    \end{macrocode}
% 恢复名字空间,要把它放在一个 \env{macrocode} 环境中,\cls{l3doc} 才能正确工作。
%    \begin{macrocode}
%<@@=ctex>
%    \end{macrocode}
% \end{macro}
%
% \begin{macro}[internal]{\ctex_punct_set:n}
% 设置 CJK 族对应到实际的字体。|#1| 是 \opt{fontset} 的名字。
%    \begin{macrocode}
\cs_new_protected_nopar:Npn \ctex_punct_set:n #1
  {
    \clist_map_inline:Nn \c_@@_punct_family_clist
      {
        \cs_if_free:cF { CJKpunct@ #1 ##1 @spaces }
          {
            \cs_set_eq:cc
              { CJKpunct@ ##1 @spaces }
              { CJKpunct@ #1 ##1 @spaces }
          }
      }
  }
\clist_const:Nn \c_@@_punct_family_clist
  {
    zhsong , zhhei , zhfs , zhkai , zhli , zhyou ,
    zhsongb , zhheil , zhheib , zhyoub , zhyahei , zhyaheib
  }
%    \end{macrocode}
% \end{macro}
%
% \begin{macro}[internal]{\ctex_punct_map_family:nn}
% CJK 族 |#1| 使用族 |#2| 的边界信息。
%    \begin{macrocode}
\cs_new_protected_nopar:Npn \ctex_punct_map_family:nn #1#2
  {
    \cs_if_free:cF { CJKpunct@ #2 @spaces }
      { \cs_set_eq:cc { CJKpunct@ #1 @spaces } { CJKpunct@ #2 @spaces } }
  }
%    \end{macrocode}
% \end{macro}
%
% \begin{macro}[internal]{\ctex_punct_map_bfseries:nn}
% CJK 族 |#1| 的 \tn{bfseries} 使用族 |#2| 的边界信息。
%    \begin{macrocode}
\cs_new_protected_nopar:Npn \ctex_punct_map_bfseries:nn #1#2
  {
    \clist_map_inline:nn {#1}
      {
        \ctex_punct_map_series:nnn { ##1 } { b } {#2}
        \ctex_punct_map_series:nnn { ##1 } { bx } {#2}
      }
  }
\cs_new_protected_nopar:Npn \ctex_punct_map_series:nnn #1#2#3
  {
    \CJKpunctmapfamily { C19 } {#1} {#2} { m }  {#3}
    \CJKpunctmapfamily { C19 } {#1} {#2} { it } {#3}
    \CJKpunctmapfamily { C19 } {#1} {#2} { sl } {#3}
    \CJKpunctmapfamily { C70 } {#1} {#2} { m }  {#3}
    \CJKpunctmapfamily { C70 } {#1} {#2} { it } {#3}
    \CJKpunctmapfamily { C70 } {#1} {#2} { sl } {#3}
  }
%    \end{macrocode}
% \end{macro}
%
% \begin{macro}[internal]{\ctex_punct_map_itshape:nn}
% CJK 族 |#1| 的 \tn{itshape} 使用族 |#2| 的边界信息。
%    \begin{macrocode}
\cs_new_protected_nopar:Npn \ctex_punct_map_itshape:nn #1#2
  {
    \CJKpunctmapfamily { C19 } {#1} { m }  { it } {#2}
    \CJKpunctmapfamily { C19 } {#1} { b }  { it } {#2}
    \CJKpunctmapfamily { C19 } {#1} { bx } { it } {#2}
    \CJKpunctmapfamily { C70 } {#1} { m }  { it } {#2}
    \CJKpunctmapfamily { C70 } {#1} { b }  { it } {#2}
    \CJKpunctmapfamily { C70 } {#1} { bx } { it } {#2}
  }
%    \end{macrocode}
% \end{macro}
%
% 载入边界信息文件。
%     \begin{macrocode}
\ctex_file_input:n { ctexspa.def }
%    \end{macrocode}
%
%    \begin{macrocode}
%</pdftex>
%<*xetex>
%    \end{macrocode}
%
% \subsubsection{\pkg{ctex-engine-xetex.def}}
%
%    \begin{macrocode}
\RequirePackage { xeCJK }
\exp_args:Nx \xeCJKsetup
  {
    LoadFandol   = false ,
    AutoFakeBold = true ,
    PunctStyle   = \l_@@_punct_tl
  }
%    \end{macrocode}
%
% 最新版本的 \pkg{fontspec} 默认对 \tn{rmfamily} 和 \tn{sffamily} 设置
% |Ligatures=TeX|,对 \tn{ttfamily} 设置 |WordSpace={1,0,0}| 和
% |PunctuationSpace=WordSpace|。
%    \begin{macrocode}
\@ifpackagelater { fontspec } { 2014/05/25 } { }
  { \msg_error:nnn { ctex } { package-too-old } { fontspec } }
%    \end{macrocode}
%
%    \begin{macrocode}
%</xetex>
%<*luatex>
%    \end{macrocode}
%
% \subsubsection{\pkg{ctex-engine-luatex.def}}
%
% \changes{v2.0}{2014/03/08}{通过 \pkg{luatexja} 宏包支持 \LuaLaTeX。}
%
% \pkg{luatexja} 为了兼容 p\LaTeX 的使用习惯,对 \LaTeXe 的 \pkg{NFSS} 作了不少
% 修改和扩充,这对于简体中文用户来说不是必要的。我们在这里禁用它。
%    \begin{macrocode}
\msg_new:nnn { ctex } { luatexja-loaded }
  {
    Package~`luatexja'~can~not~be~loaded~before~`ctex'.\\
    Loading~file~`#1'~will~abort!
  }
\@ifpackageloaded { luatexja }
  { \msg_critical:nnx { ctex } { luatexja-loaded } { \g_file_current_name_tl } }
  { \cs_new_eq:cN { ver@ltj-latex.\@pkgextension } \ExplFileDate }
%    \end{macrocode}
%
%    \begin{macrocode}
\RequirePackage { luatexja } [ 2013/05/14 ]
\RequirePackage { fontspec }
%    \end{macrocode}
%
%    \begin{macrocode}
\@ifpackagelater { fontspec } { 2014/05/25 } { }
  { \msg_error:nnn { ctex } { package-too-old } { fontspec } }
%    \end{macrocode}
%
% \paragraph{\pkg{luatexja} 的默认设置}
%
%    \begin{macrocode}
\ExplSyntaxOff
%    \end{macrocode}
%
% 以下设置抄录自 \file{lltjdefs.sty}。
%    \begin{macrocode}
\ltjdefcharrange{1}{"80-"36F, "1E00-"1EFF}
\ltjdefcharrange{2}{"370-"4FF, "1F00-"1FFF}
\ltjdefcharrange{3}{%
  "2000-"243F, "2500-"27BF, "2900-"29FF, "2B00-"2BFF}
\ltjdefcharrange{4}{%
   "500-"10FF, "1200-"1DFF, "2440-"245F, "27C0-"28FF, "2A00-"2AFF,
  "2C00-"2E7F, "4DC0-"4DFF, "A4D0-"A82F, "A840-"ABFF, "FB00-"FE0F,
  "FE20-"FE2F, "FE70-"FEFF, "10000-"1FFFF, "E000-"F8FF} % non-Japanese
\ltjdefcharrange{5}{"D800-"DFFF, "E0000-"E00FF, "E01F0-"10FFFF}
\ltjdefcharrange{6}{%
  "2460-"24FF, "2E80-"2EFF, "3000-"30FF, "3190-"319F, "31F0-"4DBF,
  "4E00-"9FFF, "F900-"FAFF, "FE10-"FE6F, "20000-"2FFFF, "E0100-"E01EF}
\ltjdefcharrange{7}{
  "1100-"11FF, "2F00-"2FFF, "3100-"31EF, "A000-"A4CF, "A830-"A83F,
  "AC00-"D7FF}
\ltjdefcharrange{8}{"A7, "A8, "B0, "B1, "B4, "B6, "D7, "F7}
\ltjsetparameter{jacharrange={-1, +2, +3, -4, -5, +6, +7, +8}}
\directlua{for x=128,255 do luatexja.math.is_math_letters[x] = true end}
%    \end{macrocode}
%
% 以下设置抄录自 \file{ltj-latex.sty}。
%    \begin{macrocode}
\directlua{
  local s = kpse.find_file('ltj-kinsoku.lua', 'tex')
  luatexja.stack.charprop_stack_table[0] = s and dofile(s) or {}
}
\ltjsetparameter{kanjiskip=0pt plus 0.4pt minus 0.4pt,
  xkanjiskip=.25\zw plus 1pt minus 1pt,
  autospacing, autoxspacing, jacharrange={-1},
  yalbaselineshift=0pt, yjabaselineshift=0pt,
  jcharwidowpenalty=500, differentjfm=paverage
}
%    \end{macrocode}
%
%    \begin{macrocode}
\ExplSyntaxOn
%    \end{macrocode}
%
% \paragraph{\pkg{luatexja} 的补丁}
%
%    \begin{macrocode}
%<@@=ctex_ltj>
%    \end{macrocode}
%
% 在 \LaTeX{} 下,\pkg{luatexja} 对 \pkg{fontspec}、\pkg{xunicode}、\pkg{unicode-math}
% 和 \pkg{listings} 打了补丁。其中前三个是把 \tn{char} 换成 \tn{ltjalchar},确保
% 字符是 ALchar 类。我们这里用 \pkg{xunicode-addon} 来处理 \pkg{xunicode}。
%    \begin{macrocode}
\RequirePackage { xunicode-addon }
\AtBeginUTFCommand
  {
    \group_begin:
    \ctex_lua_now_x:n { tex.globaldefs = 0 }
    \ltj@allalchar
  }
\AtEndUTFCommand { \group_end: }
%    \end{macrocode}
% 对 \pkg{fontspec} 沿用 \pkg{luatexja} 的补丁。
%    \begin{macrocode}
\RequirePackage { lltjp-fontspec }
%    \end{macrocode}
% \pkg{lltjp-unicode-math} 让数学符号命令成为普通的文字宏。为了避免它被展开,应该
% 用 \tn{protected} 来定义。
%    \begin{macrocode}
\group_begin:
\char_set_catcode_other:n { \c_zero }
\cs_new_protected:Npn \@@_um_char:Nw #1 = #2 \q_nil
  {
    \group_begin:
      \char_set_lccode:nn { \c_zero } {#2}
      \tex_lowercase:D
        {
          \group_end:
          \cs_gset_protected_nopar:Npn #1
            {
              \mode_if_math:TF { ^^@ }
                { {
                    \ctex_lua_now_x:n { tex.globaldefs = 0 }
                    \ltj@allalchar ^^@
                } }
            }
        }
    \ltjsetmathletter {#2}
  }
\group_end:
\AfterPreamble
  {
    \cs_if_free:NF \um_cs_set_eq_active_char:Nw
      { \cs_set_eq:NN \um_cs_set_eq_active_char:Nw \@@_um_char:Nw }
  }
%    \end{macrocode}
% 对 \pkg{listings} 的补丁是让代码环境支持 JAchar 类。\pkg{luatexja} 的补丁会将
% 代码目录标题改为日文,我们不需要。
%    \begin{macrocode}
\AfterPreamble
  {
    \@ifpackageloaded { listings }
      {
        \use:x
          {
            \exp_not:N \RequirePackage { lltjp-listings }
            \tl_set:Nn \exp_not:N \lstlistingname
              { \exp_not:o { \lstlistingname } }
            \tl_set:Nn \exp_not:N \lstlistlistingname
              { \exp_not:o { \lstlistlistingname } }
          }
      } { }
  }
%    \end{macrocode}
%
% \paragraph{字体切换方式}
%
% \begin{macro}[internal]{\ctex_ltj_select_font:,\CJK@family}
% \tn{CJK@family} 保存的是当前 CJK 实际的字体族名,如果为空表示没有设置过字体。
%    \begin{macrocode}
\cs_new_protected_nopar:Npn \ctex_ltj_select_font:
  {
    \cs_if_exist_use:cF { \l_@@_current_font_tl }
      { \tl_if_empty:NF \CJK@family { \@@_select_font_aux: } }
  }
\tl_new:N \CJK@family
\tl_new:N \l_@@_current_font_tl
\tl_set:Nn \l_@@_current_font_tl
  { \CJK@encoding / \CJK@family / \f@series / \f@shape / \f@size }
%    \end{macrocode}
% \end{macro}
%
% \begin{macro}[internal]{\@@_select_font_aux:}
% 使用 \tn{pickup@font} 取得字体名称前,总需要先设置 \tn{font@name}。在这里将
% \tn{f@family} 换成 CJK 字体族,并确保编码正确。
%    \begin{macrocode}
\cs_new_protected_nopar:Npn \@@_select_font_aux:
  {
    \group_begin:
      \tl_set_eq:NN \f@encoding \CJK@encoding
      \tl_set_eq:NN \f@family \CJK@family
      \@@_push_fontname:n { \use:c { \curr@fontshape / \f@size } }
      \ctex_ltj_pickup_font:
    \group_end:
    \font@name
    \@@_pop_fontname:
%    \end{macrocode}
% 当字形未定义的时候,\textsf{NFSS} 就会启动替换机制(\tn{wrong@fontshape})。
% 第一次启动后,\cs{l_@@_current_font_tl} 还是没有定义。为此,我们再次选择字体,
% 确保它有定义和指向正确的 \texttt{font.id}。这对 \opt{AlternateFont} 的设置
% 特别重要。
%    \begin{macrocode}
    \cs_if_exist:cF { \l_@@_current_font_tl }
      { \@@_select_font_aux: }
  }
\cs_new_protected_nopar:Npn \@@_push_fontname:n #1
  {
    \cs_gset_eq:NN \@@_save_fontname:w \font@name
    \cs_gset_nopar:Npx \font@name {#1}
  }
\cs_new_protected_nopar:Npn \@@_pop_fontname:
  { \cs_gset_eq:NN \font@name \@@_save_fontname:w }
%    \end{macrocode}
% \end{macro}
%
% \begin{macro}[internal]{\ctex_ltj_pickup_font:}
% 替换 \tn{define@newfont} 内部调用的 \tn{extract@font} 和 \tn{do@subst@correction}。
%    \begin{macrocode}
\cs_new_protected_nopar:Npn \ctex_ltj_pickup_font:
  {
    \exp_after:wN \cs_if_exist:NF \font@name
      {
        \group_begin:
          \cs_set_eq:NN \extract@font \ctex_ltj_extract_font:
          \cs_set_eq:NN \do@subst@correction \ctex_ltj_subst_font:
          \define@newfont
        \group_end:
      }
  }
\cs_new_eq:NN \pickup@jfont \ctex_ltj_pickup_font:
%    \end{macrocode}
% \end{macro}
%
% \begin{macro}[internal]{\ctex_ltj_extract_font:}
% \pkg{luatexja} 的 \tn{globaljfont} 在 \pkg{luatexja-core} 中定义:
% \begin{verbatim}
%   %%%%%%%% \jfont\CS={...:...;jfm=metric;...}, \globaljfont
%   \protected\def\jfont{\afterassignment\ltj@@jfont\directlua{luatexja.jfont.jfontdefX(false)}}
%   \protected\def\globaljfont{%
%   \afterassignment\ltj@@jfont\directlua{luatexja.jfont.jfontdefX(true)}}
%   \def\ltj@@jfont{\directlua{luatexja.jfont.jfontdefY()}}
% \end{verbatim}
% \texttt{jfontdefX} 函数的作用是把 \tn{CS} 定义为其后的字体,\texttt{jfontdefY}
% 的作用是更新 \texttt{JFM} 和记录相关字体信息。最后的工作是:
% \begin{verbatim}
%   tex.sprint(cat_lp, global_flag .. '\\protected\\expandafter\\def\\csname '
%     .. cstemp  .. '\\endcsname{\\ltj@curjfnt=' .. fn .. '\\relax}')
% \end{verbatim}
% \tn{CS} 的作用就是把 \tn{ltj@curjfnt} 设置为刚才定义的字体的 \texttt{font.id}。
%    \begin{macrocode}
\cs_new_protected_nopar:Npn \ctex_ltj_extract_font:
  {
    \get@external@font
    \ctex_ltj_if_alternate_shape_exist:nT { \curr@fontshape }
      {
        \tl_set:Nx \external@font
          { \exp_after:wN \@@_patch_external_font:w \external@font }
      }
    \exp_after:wN \globaljfont \font@name \external@font \scan_stop:
%    \end{macrocode}
% 这里 \tn{font@name} 不会直接改变当前字体,而 \tn{DeclareFontFamily} 和
% \tn{DeclareFontShape} 的最后一个参数通常要使用 \tn{font} 来引用当前字体。
% 为此,我们在分组内启用之前定义的字体,以便能得到正确的 \tn{font}。对字体参数的
% 赋值总是全局的,不会受到分组的影响。
%    \begin{macrocode}
    \font@name
    \ctex_lua_now_x:n { font.current(tex.attribute['ltj@curjfnt']) }
    \use:c { \f@encoding + \f@family }
    \use:c { \curr@fontshape }
  }
%    \end{macrocode}
% \end{macro}
%
% \begin{macro}[internal]{\ctex_ltj_subst_font:}
% \tn{do@subst@correction} 在设置通过 \texttt{sub} 或者 \texttt{ssub} 函数定义的
% 字体时会用到。如果没有设置 \opt{SlantedFont},\pkg{fontspec} 会设置
% \tn{itdefault} 作为 \tn{sldefault} 的替代字形,因而会用到这个函数。它的本来定义是:
% \begin{verbatim}
%   \def\do@subst@correction{%
%       \xdef\subst@correction{%
%          \font@name
%          \global\expandafter\font
%            \csname \curr@fontshape/\f@size\endcsname
%            \noexpand\fontname\font
%           \relax}%
%       \aftergroup\subst@correction
%   }
% \end{verbatim}
% 我们在这里不需要定义新字体,而是设置对应字体的命令。
%    \begin{macrocode}
\cs_new_protected_nopar:Npn \ctex_ltj_subst_font:
  {
    \ctex_ltj_if_alternate_shape_exist:nF { \curr@fontshape }
      {
        \group_begin:
        \tl_set_eq:NN \CJK@family \f@family
        \cs_if_exist:cF { \l_@@_current_font_tl  }
          {
            \cs_gset_protected_nopar:Npx \subst@correction
              {
                \cs_new_eq:NN
                  \exp_not:c { \l_@@_current_font_tl }
                  \font@name
              }
            \group_insert_after:N \group_insert_after:N
            \group_insert_after:N \subst@correction
          }
        \group_end:
      }
  }
%    \end{macrocode}
% \end{macro}
%
% \begin{macro}[internal,TF]{\ctex_ltj_if_alternate_shape_exist:n}
% 即 \tn{luatexja} 中的 |\ltj@@does@alt@set|,判断是否存在替代字体。
%    \begin{macrocode}
\prg_new_conditional:Npnn \ctex_ltj_if_alternate_shape_exist:n #1 { T , F , TF }
  {
    \ctex_lua_now_x:n { luatexja.jfont.does_alt_set ('\luatexluaescapestring {#1}') }
      \prg_return_true: \else: \prg_return_false: \fi:
  }
%    \end{macrocode}
% \end{macro}
%
% \begin{macro}[aux]{\@@_patch_external_font:w}
% 若对字体的定义完全相同,则它们有相同的 \texttt{font.id}。因此如果字形是由
% \textsf{NFSS} 的替换机制定义的,它们就有相同的 \texttt{font.id}。
% |print_aftl_address| 函数的定义是
% \begin{verbatim}
%   function print_aftl_address()
%     tex.sprint(cat_lp, ';ltjaltfont' .. tostring(aftl_base):sub(8))
%   end
% \end{verbatim}
% 主要目的是,如果当前字形有替代字体,则往字形的定义中加入一些标志,确保
% \texttt{font.id} 唯一。
%    \begin{macrocode}
\cs_new_nopar:Npn \@@_patch_external_font:w #1 ~ at
  { #1 \ctex_lua_now_x:n { luatexja.jfont.print_aftl_address() } ~ at }
%    \end{macrocode}
% \end{macro}
%
% \begin{macro}[internal]{\ctex_ltj_select_alternate_font:}
% 在 \tn{selectfont} 中更新替代字体。
%    \begin{macrocode}
\cs_new_protected_nopar:Npn \ctex_ltj_select_alternate_font:
  {
    \ctex_ltj_if_alternate_shape_exist:nT { \l_@@_current_shape_tl }
      {
        \ctex_lua_now_x:n
          {
            luatexja.jfont.output_alt_font_cmd
              ('\luatexluaescapestring { \l_@@_current_shape_tl }')
          }
        \ctex_lua_now_x:n { luatexja.jfont.pickup_alt_font_a ('\f@size') }
      }
  }
\tl_new:N \l_@@_current_shape_tl
\tl_set:Nn \l_@@_current_shape_tl
  { \CJK@encoding / \CJK@family / \f@series / \f@shape }
%    \end{macrocode}
% \end{macro}
%
% \begin{macro}[internal]{\ltj@pickup@altfont@aux}
% 被用在函数 |output_alt_font_cmd| 中,作用是定义替代字体。
%    \begin{macrocode}
\cs_new_protected_nopar:Npn \ltj@pickup@altfont@aux #1
  {
    \cs_if_exist:cF { #1/\f@size }
      {
        \group_begin:
          \use:x { \exp_not:N \split@name #1 / \f@size } \@nil
          \@@_push_fontname:n { \use:c { \curr@fontshape / \f@size } }
          \ctex_ltj_pickup_font:
        \group_end:
        \@@_pop_fontname:
      }
  }
%    \end{macrocode}
% \end{macro}
%
%    \begin{macrocode}
%<@@=>
%    \end{macrocode}
%
% \begin{macro}[internal]{\ltj@pickup@altfont@copy}
% 被用在函数 |pickup_alt_font_a| 中。|ltj@@getjfontnumber| 的作用是将字体命令
% |#1| 对应的 \texttt{font.id} 保存到 \tn{ltj@tempcntc} 中。
%    \begin{macrocode}
\cs_new_protected_nopar:Npn \ltj@pickup@altfont@copy #1#2
  {
    \ltj@@getjfontnumber #1
    \ctex_lua_now_x:n
      {
        luatexja.jfont.pickup_alt_font_b
          ( \the\ltj@tempcntc, '\luatexluaescapestring {#2}' )
      }
  }
%    \end{macrocode}
% \end{macro}
%
%    \begin{macrocode}
\ExplSyntaxOff
%    \end{macrocode}
%
% 以下内容抄录自 \file{lltjfont.sty},目的是让汉字可以在数学环境中直接使用。
%    \begin{macrocode}
\def\ltj@@IsFontJapanese#1{%
  \directlua{luatexja.jfont.is_kenc(string.match(
      '\luatexluaescapestring{#1}', '[^/]+'))}}
{\catcode`M=12%
\gdef\ltj@@mathJapaneseFonts#1M#2#3\relax{\ltj@@IsFontJapanese{#3}}}
\let\ltj@@al@getanddefine@fonts=\getanddefine@fonts
\def\ltj@@ja@getanddefine@fonts#1#2{%
  \xdef\font@name{\csname \string#2/\tf@size\endcsname}%
  \pickup@jfont\let\textfont@name\font@name
  \xdef\font@name{\csname \string#2/\sf@size\endcsname}%
  \pickup@jfont\let\scriptfont@name\font@name
  \xdef\font@name{\csname \string#2/\ssf@size\endcsname}%
  \pickup@jfont
  \edef\math@fonts{\math@fonts\ltj@setpar@global%
    \ltj@@set@stackfont#1,\textfont@name:{MJT}%
    \ltj@@set@stackfont#1,\scriptfont@name:{MJS}%
    \ltj@@set@stackfont#1,\font@name:{MJSS}%
  }%
}
\def\getanddefine@fonts#1#2{%
  \ltj@tempcnta=#1\ltj@@IsFontJapanese{\string#2}%
  \ifin@\let\ltj@temp=\ltj@@ja@getanddefine@fonts%
  \else \let\ltj@temp=\ltj@@al@getanddefine@fonts\fi
  \ltj@temp{#1}{#2}%
}
\def\use@mathgroup#1#2{\relax\ifmmode
  \math@bgroup
    \expandafter\ifx\csname M@\f@encoding\endcsname#1\else
    #1\fi\ltj@tempcnta=#2 \expandafter\ltj@@mathJapaneseFonts\string#1\relax%
    \ifin@\jfam#2\relax\else\mathgroup#2\relax\fi
  \expandafter\math@egroup\fi}%
%    \end{macrocode}
%
%    \begin{macrocode}
\let\@@italiccorr=\/
%    \end{macrocode}
%
%    \begin{macrocode}
%<@@=ctex_ltj>
%    \end{macrocode}
%
%    \begin{macrocode}
\ExplSyntaxOn
%    \end{macrocode}
%
% \begin{macro}[internal]{\ctex_mono_jfm:n,\l_@@_jfm_tl}
% \pkg{luatexja} 中与标点格式 \opt{plain} 对应的 \texttt{JFM} 是 \opt{mono}。
%    \begin{macrocode}
\cs_new_protected_nopar:Npn \ctex_mono_jfm:n #1
  {
    \str_if_eq:nnTF {#1} { plain }
      { \tl_set:Nn \l_@@_jfm_tl { mono } }
      { \tl_set:Nn \l_@@_jfm_tl {#1} }
  }
\tl_new:N \l_@@_jfm_tl
\cs_generate_variant:Nn \ctex_mono_jfm:n { o }
\ctex_mono_jfm:o { \l__ctex_punct_tl }
%    \end{macrocode}
% \end{macro}
%
% \begin{macro}[internal]{\CJK@encoding,\@@_change_encoding:}
% 在 \LaTeX 下,\pkg{luatexja} 依赖字体编码来实现特殊设置。例如上述的
% |\ltj@@IsFontJapanese| 就是通过判断编码来实现的,它在设置数学字体时会用到。所以
% 不应该与西文共用 \texttt{EU2}。定义字体族 song 为 \tn{CJK@encoding} 的默认替换
% 字体。下划线 |_| 不在 \tn{nfss@catcodes} 里,可以放心使用。
%    \begin{macrocode}
\tl_const:Nn \CJK@encoding { LTJY3 }
\DeclareFontEncoding { \CJK@encoding } { } { }
\DeclareFontSubstitution { LTJY3 } { song } { \mddefault } { \updefault }
\ctex_lua_now_x:n { luatexja.jfont.add_kyenc_list('\CJK@encoding') }
\cs_new_protected_nopar:Npn \@@_change_encoding:
  { \tl_set_eq:NN \g_fontspec_encoding_tl \CJK@encoding }
\DeclareFontFamily { \CJK@encoding } { song } { }
\DeclareFontShape { \CJK@encoding } { song } { \mddefault } { \updefault }
  { <-> psft:SimSun:cid=Adobe-GB1-5;jfm=\l_@@_jfm_tl } { }
\DeclareFontShape { \CJK@encoding } { song } { \bfdefault } { \updefault }
  { <-> psft:SimHei:cid=Adobe-GB1-5;jfm=\l_@@_jfm_tl } { }
\tl_const:Nn \c_@@_math_tl { CJKmath }
\DeclareSymbolFont { \c_@@_math_tl }
  { \CJK@encoding } { song } { \mddefault } { \updefault }
\SetSymbolFont { \c_@@_math_tl } { bold }
  { \CJK@encoding } { song } { \bfdefault } { \updefault }
\int_const:Nn \c_@@_math_fam_int { \use:c { sym \c_@@_math_tl } }
\jfam \c_@@_math_fam_int
%    \end{macrocode}
% \end{macro}
%
% \paragraph{字体族的定义与使用}
%
% 这是 \pkg{luatexja-fontspec} 中新增的一些字体选项。
%    \begin{macrocode}
\newfontfeature { CID }     {    cid = #1 }
\newfontfeature { JFM }     {    jfm = #1 }
\newfontfeature { JFM-var } { jfmvar = #1 }
%    \end{macrocode}
%
% 在新版本的 \pkg{fontspec} 中,\cs{__fontspec_namewrap:n} 变成了私有函数。
%    \begin{macrocode}
\keys_define:nn { fontspec-preparse-external }
  {
    NoEmbed .code:n =
      { \cs_set_eq:NN \__fontspec_namewrap:n \@@_noembed_wrap:n }
  }
\cs_new:Npn \@@_noembed_wrap:n #1 { psft: #1 }
%    \end{macrocode}
%
% \begin{macro}[internal]{\ctex_ltj_set_family:nnn}
% 将自定义的字体族名与 \pkg{fontspec} 实际设置的名字对应起来。
%    \begin{macrocode}
\cs_new_protected:Npn \ctex_ltj_set_family:nnn #1#2#3
  {
    \group_begin:
    \clist_clear:N \l_@@_char_range_clist
    \seq_clear:N \l_@@_alternate_seq
    \tl_set:Nn \l_@@_base_CJKfamily_tl {#1}
    \keys_set_known:nnN { ctex_ltj / fontspec } {#2} \l_@@_tmp_tl
    \clist_set:No \l_@@_font_options_clist { \l_@@_tmp_tl }
    \ctex_ltj_set_alternate_family:nnF {#1} {#3}
      {
        \prop_gput:Nnn \g_@@_family_font_name_prop {#1} {#3}
        \prop_gput:Nno \g_@@_family_font_options_prop
          {#1} { \l_@@_font_options_clist }
        \@@_update_family_uid:N \l_@@_font_options_clist
        \@@_use_global_options:N \l_@@_font_options_clist
        \@@_gset_family_cs:nn {#1} {#3}
      }
    \group_end:
  }
\cs_new_protected:Npn \ctex_ltj_set_family:xxx #1#2#3
  { \use:x { \ctex_ltj_set_family:nnn {#1} {#2} {#3} } }
\tl_new:N \l_@@_base_CJKfamily_tl
\clist_new:N \l_@@_font_options_clist
\cs_new_protected_nopar:Npn \@@_use_global_options:N #1
  {
    \clist_concat:NNN #1 \g_@@_default_features_clist #1
    \clist_put_left:Nx #1 { JFM = \l_@@_jfm_tl }
  }
%    \end{macrocode}
% \end{macro}
%
% \begin{variable}[internal]
%  {\g_@@_family_name_prop,\g_@@_family_font_name_prop,\g_@@_family_font_options_prop}
% 分别保存 \pkg{fontspec} 设置的字体族名、字体名称和字体选项。
%    \begin{macrocode}
\prop_new:N \g_@@_family_name_prop
\prop_new:N \g_@@_family_font_name_prop
\prop_new:N \g_@@_family_font_options_prop
%    \end{macrocode}
% \end{variable}
%
% \begin{macro}[internal]{\@@_check_family:n}
% 删除重复的定义,清除替代字体的先前设置。
%   \begin{macrocode}
\cs_new_protected_nopar:Npn \@@_check_family:n #1
  {
    \prop_gpop:NnNT \g_@@_family_font_name_prop {#1} \l_@@_tmp_tl
      {
        \cs_undefine:c { \@@_family_csname:n {#1} }
        \cs_undefine:c { \@@_alternate_cs:n {#1} }
        \prop_gpop:NnNT \g_@@_family_name_prop {#1} \l_@@_base_family_tl
          {
            \use:c { \@@_alternate_cs:n { clear / #1 } }
            \cs_undefine:c { \@@_alternate_cs:n { clear / #1 } }
            \cs_undefine:c { \@@_alternate_cs:n { reset / #1 } }
            \prop_gremove:Nn \g_@@_reset_alternate_prop {#1}
          }
        \msg_warning:nnxx { ctex } { redefine-family } {#1} { \l_@@_tmp_tl }
      }
  }
\tl_new:N \l_@@_tmp_tl
\msg_new:nnn { ctex } { redefine-family }
  { Redefining~CJKfamily~`\@@_msg_family_map:n {#1}'~(#2). }
%    \end{macrocode}
% \end{macro}
%
% \begin{macro}[internal]{\@@_gset_family_cs:nn}
% 在设置字体时,实际上并不是马上就定义。而是只保存相关参数,在通过 \tn{CJKfamily}
% 第一次使用时才定义。需要注意将编码改为 \tn{CJK@encoding}。
%    \begin{macrocode}
\cs_new_protected_nopar:Npn \@@_gset_family_cs:nn #1#2
  {
    \cs_gset_protected_nopar:cpx { \@@_family_csname:n {#1} }
      {
        \group_begin:
        \@@_change_encoding:
        \exp_not:n { \cs_set_eq:NN \CJKfamily \use_none:n }
        \exp_not:n { \fontspec_set_family:Nnn \g_@@_fontspec_family_tl }
          { \exp_not:o { \l_@@_font_options_clist } } {#2}
        \prop_gput:Nno \exp_not:N \g_@@_family_name_prop {#1}
          { \exp_not:N \g_@@_fontspec_family_tl }
        \tl_gset_eq:NN \exp_not:N \g_@@_fontspec_family_tl
          \exp_not:N \g_@@_fontspec_family_tl
        \@@_set_alternate_family:n {#1}
        \group_end:
      }
  }
\tl_new:N \l_@@_base_family_tl
\tl_new:N \g_@@_fontspec_family_tl
\cs_new_nopar:Npn \@@_family_csname:n #1 { ctex_ltj/family/#1 }
\cs_new_protected_nopar:Npn \@@_set_alternate_family:n #1
  {
    \tl_set:Nn \l_@@_base_CJKfamily_tl {#1}
    \tl_set_eq:NN \l_@@_base_family_tl \g_@@_fontspec_family_tl
    \cs_if_exist_use:c { \@@_alternate_cs:n { reset / #1 } }
    \cs_if_exist_use:c { \@@_alternate_cs:n {#1} }
  }
\cs_new:Npn \@@_alternate_cs:n #1 { ctex_ltj/alternate_family/#1 }
%    \end{macrocode}
% \end{macro}
%
% \begin{macro}[internal]{\CJKfamily}
% 切换字体。
%    \begin{macrocode}
\NewDocumentCommand \CJKfamily { m }
  { \ctex_ltj_switch_family:x {#1} \tex_ignorespaces:D }
\cs_new_protected_nopar:Npn \ctex_ltj_switch_family:n #1
  {
    \ctex_ltj_family_if_exist:xNTF {#1} \CJK@family
      {
        \tl_set:Nn \l_ctex_ltj_family_tl {#1}
        \selectfont
      }
      { \@@_family_unknown_warning:n {#1} }
  }
\tl_new:N \l_ctex_ltj_family_tl
\cs_generate_variant:Nn \ctex_ltj_switch_family:n { x }
%    \end{macrocode}
% \end{macro}
%
% \begin{macro}[internal,TF]{\ctex_ltj_family_if_exist:n}
% 判断 CJK 字体族 |#1| 是否存在,若存在则把实际族名保存到 |#2| 中。
%    \begin{macrocode}
\prg_new_protected_conditional:Npnn \ctex_ltj_family_if_exist:xN #1#2 { T , F , TF }
  {
    \prop_get:NxNTF \g_@@_family_name_prop {#1} #2
      { \prg_return_true: }
      {
        \cs_if_exist_use:cTF { \@@_family_csname:n {#1} }
          {
            \tl_set_eq:NN #2 \g_@@_fontspec_family_tl
            \prg_return_true:
          }
          { \prg_return_false: }
      }
  }
\cs_generate_variant:Nn \prop_get:NnNTF { Nx }
%    \end{macrocode}
% \end{macro}
%
% \begin{macro}[internal]{\@@_family_unknown_warning:n}
%    \begin{macrocode}
\cs_new_protected_nopar:Npn \@@_family_unknown_warning:n #1
  {
    \prop_if_empty:NF \g_@@_family_font_name_prop
      {
        \seq_if_in:NnF \g_@@_unknown_family_seq {#1}
          {
            \seq_gput_right:Nn \g_@@_unknown_family_seq {#1}
            \msg_warning:nnn { ctex } { family-unknown } {#1}
          }
      }
  }
\seq_new:N \g_@@_unknown_family_seq
\msg_new:nnn { ctex } { family-unknown }
  {
    Unknown~CJK~family~`\@@_msg_family_map:n {#1}'~is~being~ignored.\\
    Try~to~use~`\@@_msg_def_family_map:n {#1}'~to~define~it.
  }
\cs_new_nopar:Npn \@@_msg_def_family_map:n #1
  {
    \str_case_x:nnF {#1}
      {
        \CJKrmdefault { \token_to_str:N \setCJKmainfont }
        \CJKsfdefault { \token_to_str:N \setCJKsansfont }
        \CJKttdefault { \token_to_str:N \setCJKmonofont }
      }
      { \token_to_str:N \setCJKfamilyfont \{ #1 \} }
    [...]\{...\}
  }
\cs_new_nopar:Npn \@@_msg_family_map:n #1
  {
    \str_case_x:nnF {#1}
      {
        \CJKrmdefault { \token_to_str:N \CJKrmdefault }
        \CJKsfdefault { \token_to_str:N \CJKsfdefault }
        \CJKttdefault { \token_to_str:N \CJKttdefault }
      }
      {#1}
  }
%    \end{macrocode}
% \end{macro}
%
% \begin{macro}[internal]{\ctex_ltj_fontspec:nn}
%    \begin{macrocode}
\cs_new_protected_nopar:Npn \ctex_ltj_fontspec:nn #1#2
  {
    \prop_get:NnNTF \g_@@_fontspec_prop
      { CJKfontspec/#1/#2/id } \l_ctex_ltj_family_tl
      { \ctex_ltj_switch_family:x { \l_ctex_ltj_family_tl } }
      {
        \int_gincr:N \g_@@_family_int
        \@@_fontspec:xnn
          { CJKfontspec ( \int_use:N \g_@@_family_int ) }
          {#1} {#2}
      }
  }
\cs_new_protected_nopar:Npn \ctex_ltj_fontspec:xx #1#2
  { \use:x { \ctex_ltj_fontspec:nn {#1} {#2} } }
\cs_new_protected_nopar:Npn \@@_fontspec:nnn #1#2#3
  {
    \bool_if:NT \l_@@_add_alternate_bool
      {
        \cs_if_free:cF
          { \@@_alternate_cs:n { reset / \l_ctex_ltj_family_tl } }
          {
            \cs_gset_eq:cc
              { \@@_alternate_cs:n { reset / #1 } }
              { \@@_alternate_cs:n { reset / \l_ctex_ltj_family_tl } }
            \cs_gset_eq:cc
              { \@@_alternate_cs:n { clear / #1 } }
              { \@@_alternate_cs:n { clear / \l_ctex_ltj_family_tl } }
          }
        \bool_set_false:N \l_@@_add_alternate_bool
      }
    \prop_gput:Nnn \g_@@_fontspec_prop { CJKfontspec/#2/#3/id } {#1}
    \ctex_ltj_set_family:nnn {#1} {#2} {#3}
    \ctex_ltj_switch_family:n {#1}
  }
\cs_generate_variant:Nn \@@_fontspec:nnn { x }
\prop_new:N \g_@@_fontspec_prop
%    \end{macrocode}
% \end{macro}
%
% \begin{macro}[internal]
% {\ctex_ltj_add_font_features:n,\ctex_ltj_add_font_features:nn}
%    \begin{macrocode}
\cs_new_protected_nopar:Npn \ctex_ltj_add_font_features:n #1
  { \ctex_ltj_add_font_features:xn { \l_ctex_ltj_family_tl } {#1} }
\cs_new_protected_nopar:Npn \ctex_ltj_add_font_features:nn #1#2
  {
    \prop_get:NnNTF \g_@@_family_font_name_prop
      {#1} \l_@@_tmp_tl
      {
        \prop_get:NnN \g_@@_family_font_options_prop
          {#1} \l_@@_font_options_clist
        \clist_put_right:Nn \l_@@_font_options_clist {#2}
        \bool_set_true:N \l_@@_add_alternate_bool
        \ctex_ltj_fontspec:xx
          { \exp_not:o { \l_@@_font_options_clist } }
          { \exp_not:o { \l_@@_tmp_tl } }
      }
      { \msg_warning:nn { ctex } { addCJKfontfeature-ignored } }
  }
\bool_new:N \l_@@_add_alternate_bool
\cs_generate_variant:Nn \ctex_ltj_add_font_features:n  { x }
\cs_generate_variant:Nn \ctex_ltj_add_font_features:nn { x }
\msg_new:nnn { ctex } { addCJKfontfeature-ignored }
  {
    \token_to_str:N \addCJKfontfeature (s)~ignored.\\
    It~cannot~be~used~with~a~font~that~wasn't~selected~by~ctex.
  }
%    \end{macrocode}
% \end{macro}
%
% \begin{macro}[internal]
% {\setCJKfamilyfont,\newCJKfontfamily,\CJKfontspec,\addCJKfontfeatures}
%    \begin{macrocode}
\NewDocumentCommand \setCJKfamilyfont { m O { } m }
  { \ctex_ltj_set_family:xxx {#1} {#2} {#3} }
\NewDocumentCommand \newCJKfontfamily { o m O { } m }
  {
    \tl_set:Nx \l_@@_tmp_tl
      { \IfNoValueTF {#1} { \cs_to_str:N #2 } {#1} }
    \cs_new_protected_nopar:Npx #2
      { \ctex_ltj_switch_family:n { \l_@@_tmp_tl } }
    \ctex_ltj_set_family:xxx { \l_@@_tmp_tl } {#3} {#4}
  }
\NewDocumentCommand \CJKfontspec { O { } m }
  {
    \ctex_ltj_fontspec:xx {#1} {#2}
    \tex_ignorespaces:D
  }
\NewDocumentCommand \addCJKfontfeatures { m }
  {
    \ctex_ltj_add_font_features:x {#1}
    \tex_ignorespaces:D
  }
\cs_new_eq:NN \addCJKfontfeature \addCJKfontfeatures
%    \end{macrocode}
% \end{macro}
%
% \begin{macro}[internal]
% {\setCJKmainfont,\setCJKsansfont,\setCJKmonofont,
%  \setCJKmathfont,\defaultCJKfontfeatures}
%    \begin{macrocode}
\NewDocumentCommand \setCJKmainfont { O { } m }
  {
    \ctex_ltj_set_family:xxx { \CJKrmdefault } {#1} {#2}
    \normalfont
  }
\cs_new_eq:NN \setCJKromanfont \setCJKmainfont
\NewDocumentCommand \setCJKsansfont { O { } m }
  {
    \ctex_ltj_set_family:xxx { \CJKsfdefault } {#1} {#2}
    \normalfont
  }
\NewDocumentCommand \setCJKmonofont { O { } m }
  {
    \ctex_ltj_set_family:xxx { \CJKttdefault } {#1} {#2}
    \normalfont
  }
\NewDocumentCommand \setCJKmathfont { O { } m }
  { \ctex_ltj_set_family:xxx { \c_@@_math_tl } {#1} {#2} }
\NewDocumentCommand \defaultCJKfontfeatures { m }
  { \clist_gset:Nn \g_@@_default_features_clist {#1} }
\clist_new:N \g_@@_default_features_clist
\@onlypreamble \setCJKmainfont
\@onlypreamble \setCJKsansfont
\@onlypreamble \setCJKmonofont
\@onlypreamble \setCJKmathfont
\@onlypreamble \setCJKromanfont
\@onlypreamble \defaultCJKfontfeatures
%    \end{macrocode}
% \end{macro}
%
%    \begin{macrocode}
\tl_if_exist:NF \CJKfamilydefault
  { \tl_const:Nn \CJKfamilydefault { \CJKrmdefault } }
\tl_if_exist:NF \CJKrmdefault { \tl_const:Nn \CJKrmdefault { rm } }
\tl_if_exist:NF \CJKsfdefault { \tl_const:Nn \CJKsfdefault { sf } }
\tl_if_exist:NF \CJKttdefault { \tl_const:Nn \CJKttdefault { tt } }
\ctex_preto_cmd:NnTF \rmfamily { \CJKfamily { \CJKrmdefault } } { }
  { \ctex_patch_failure:N \rmfamily }
\ctex_preto_cmd:NnTF \sffamily { \CJKfamily { \CJKsfdefault } } { }
  { \ctex_patch_failure:N \sffamily }
\ctex_preto_cmd:NnTF \ttfamily { \CJKfamily { \CJKttdefault } } { }
  { \ctex_patch_failure:N \ttfamily }
\ctex_preto_cmd:NnTF \normalfont { \CJKfamily { \CJKfamilydefault } }
  { \cs_set_eq:NN \reset@font \normalfont }
  { \ctex_patch_failure:N \normalfont }
%    \end{macrocode}
%
% \begin{macro}[internal]{\ctex_ltj_ensure_default_family:}
% 在导言区结束确认 \tn{CJKfamilydefault} 确实存在。
%    \begin{macrocode}
\cs_new_protected_nopar:Npn \ctex_ltj_ensure_default_family:
  {
    \prop_if_empty:NF \g_@@_family_font_name_prop
      {
        \ctex_ltj_family_if_exist:xNF { \CJKfamilydefault } \l_@@_tmpa_tl
          {
            \str_if_eq_x:nnTF { \CJKfamilydefault } { \CJKrmdefault }
              { \use:n }
              {
                \ctex_ltj_family_if_exist:xNTF { \CJKrmdefault } \l_@@_tmpa_tl
                  { \tl_gset:Nn \CJKfamilydefault { \CJKrmdefault } \use_none:n }
                  { \use:n }
              }
              {
                \prop_map_inline:Nn \g_@@_family_font_name_prop
                  {
                    \prop_map_break:n
                      { \tl_gset_rescan:Nnn \CJKfamilydefault { } { ##1 } }
                  }
              }
          }
        \normalfont
        \ctex_ltj_update_mathfont:
      }
  }
%    \end{macrocode}
% \end{macro}
%
% \begin{macro}[internal]{\ctex_ltj_update_mathfont:}
% 更新数学字体为实际的字体。
%    \begin{macrocode}
\cs_new_protected_nopar:Npn \ctex_ltj_update_mathfont:
  {
    \ctex_ltj_family_if_exist:xNTF { \c_@@_math_tl } \l_@@_tmp_tl
      { \ctex_ltj_update_mathfont:n { \l_@@_tmp_tl } }
      {
        \ctex_ltj_family_if_exist:xNT { \CJKfamilydefault } \l_@@_tmp_tl
          { \ctex_ltj_update_mathfont:n { \l_@@_tmp_tl } }
      }
  }
\cs_new_protected_nopar:Npn \ctex_ltj_update_mathfont:n #1
  {
    \tl_const:Nx \c_@@_math_family_tl {#1}
    \DeclareSymbolFont { \c_@@_math_tl } { \CJK@encoding }
      { \c_@@_math_family_tl } { \mddefault } { \updefault }
    \cs_if_free:cTF
      { \CJK@encoding/\c_@@_math_family_tl/\bfdefault/\updefault }
      {
        \SetSymbolFont { \c_@@_math_tl } { bold } { \CJK@encoding }
          { \c_@@_math_family_tl } { \mddefault } { \updefault }
      }
      {
        \SetSymbolFont { \c_@@_math_tl } { bold } { \CJK@encoding }
          { \c_@@_math_family_tl } { \bfdefault } { \updefault }
      }
  }
%    \end{macrocode}
% \end{macro}
%
% \paragraph{替代字体的设置}
%
% \begin{macro}{AlternateFont,CharRange}
% 设置替代字体的选项。
%    \begin{macrocode}
\keys_define:nn { ctex_ltj / fontspec }
  {
    AlternateFont  .code:n = \ctex_ltj_set_alternate_seq:n {#1} ,
    AlternateFont  .value_required: ,
    CharRange .clist_set:N = \l_@@_char_range_clist ,
    CharRange .value_required:
  }
%    \end{macrocode}
% \end{macro}
%
% \begin{macro}[internal]{\ctex_ltj_set_alternate_seq:n}
% 我们使用 \verb=||= 作为替代字体序列的分隔标志。它可能被设置为活动字符,为此
% 需要先“消毒”,同时过滤掉空元素。
%    \begin{macrocode}
\group_begin:
\char_set_catcode_active:N \/
\char_set_lccode:nn { `\/ } { `\| }
\tex_lowercase:D
  {
    \group_end:
    \cs_new_protected:Npn \ctex_ltj_set_alternate_seq:n #1
      {
        \clist_if_empty:NT \l_@@_char_range_clist
          {
            \tl_set:Nn \l_@@_tmp_tl {#1}
            \tl_replace_all:Nnn \l_@@_tmp_tl { // } { || }
            \seq_set_split:NnV \l_@@_tmp_seq { || } \l_@@_tmp_tl
            \seq_set_filter:NNn \l_@@_tmp_seq \l_@@_tmp_seq
              { ! \tl_if_blank_p:n { ##1 } }
            \seq_concat:NNN \l_@@_alternate_seq
              \l_@@_alternate_seq \l_@@_tmp_seq
          }
      }
  }
\seq_new:N \l_@@_tmp_seq
\seq_new:N \l_@@_alternate_seq
%    \end{macrocode}
% \end{macro}
%
% \begin{macro}[internal]{\ctex_ltj_set_alternate_family:nnF}
% 如果在字体的选项中设置了 \opt{CharRange},则只设置替代字体。
%    \begin{macrocode}
\cs_new_protected_nopar:Npn \ctex_ltj_set_alternate_family:nnF #1#2#3
  {
    \clist_if_empty:NTF \l_@@_char_range_clist
      {
        \@@_check_family:n {#1}
        \seq_if_empty:NF \l_@@_alternate_seq
          { \ctex_ltj_save_alternate_seq:cn { \@@_alternate_cs:n {#1} } {#2} }
        #3
      }
      { \ctex_ltj_set_alternate_family:nn {#1} {#2} }
  }
%    \end{macrocode}
% \end{macro}
%
% \begin{macro}[internal]
% {\ctex_ltj_save_alternate_seq:Nn,\ctex_ltj_save_alternate_seq:Nnnwn}
% 保存由 \opt{AlternateFont} 设置的替代字体序列。
%    \begin{macrocode}
\cs_new_protected_nopar:Npn \ctex_ltj_save_alternate_seq:Nn #1#2
  {
    \seq_map_inline:Nn \l_@@_alternate_seq
      { \ctex_ltj_save_alternate_seq:Nnnwnw #1 {#2} ##1 { } \q_stop }
  }
\cs_generate_variant:Nn \ctex_ltj_save_alternate_seq:Nn { c }
\NewDocumentCommand \ctex_ltj_save_alternate_seq:Nnnwnw
  { m m m +O{ } m u{ \q_stop } }
  {
    \clist_set:Nn \l_@@_char_range_clist {#3}
    \clist_set:Nn \l_@@_alternate_options_clist {#4}
    \@@_use_global_options:N \l_@@_alternate_options_clist
    \tl_if_blank:nTF {#5}
      { \tl_set:Nn \l_@@_tmp_tl {#2} }
      {
        \tl_set:Nn \l_@@_tmp_tl {#5}
        \tl_replace_all:Nnn \l_@@_tmp_tl { * } {#2}
      }
    \use:x
      {
        \ctex_ltj_save_alternate_family:Nnnn \exp_not:N #1
          { \exp_not:o { \l_@@_char_range_clist } }
          { \exp_not:o { \l_@@_alternate_options_clist } }
          { \exp_not:o { \l_@@_tmp_tl } }
      }
  }
\clist_new:N \l_@@_alternate_options_clist
%    \end{macrocode}
% \end{macro}
%
% \begin{macro}[internal]{\ctex_ltj_set_alternate_family:nn}
% 设置选项 \opt{CharRange} 范围内的替代字体。如果已经定义了主字体,我们也马上
% 定义替代字体,否则只保存起来备用。
%    \begin{macrocode}
\cs_new_protected_nopar:Npn \ctex_ltj_set_alternate_family:nn #1#2
  {
    \@@_update_family_uid:N \l_@@_font_options_clist
    \@@_use_global_options:N \l_@@_font_options_clist
    \ctex_ltj_set_alternate_family:coonn
      { \@@_alternate_cs:n {#1} }
      { \l_@@_char_range_clist }
      { \l_@@_font_options_clist } {#2} {#1}
  }
\cs_new_protected_nopar:Npn \ctex_ltj_set_alternate_family:Nnnnn #1#2#3#4#5
  {
    \prop_get:NnNT \g_@@_family_name_prop {#5} \l_@@_base_family_tl
      { \ctex_ltj_set_alternate_family:nnn {#2} {#3} {#4} }
    \ctex_ltj_save_alternate_family:Nnnn #1 {#2} {#3} {#4}
  }
\cs_generate_variant:Nn \ctex_ltj_set_alternate_family:Nnnnn { coo }
%    \end{macrocode}
% \end{macro}
%
% \begin{macro}[internal]{\ctex_ltj_save_alternate_family:Nnnn}
% 保存替代字体序列的定义,以备定义主字体时使用。
%    \begin{macrocode}
\cs_new_protected_nopar:Npn \ctex_ltj_save_alternate_family:Nnnn #1#2#3#4
  {
    \cs_if_exist:NF #1 { \cs_set_eq:NN #1 \prg_do_nothing: }
    \cs_gset_protected_nopar:Npx #1
      { \exp_not:o { #1 \ctex_ltj_set_alternate_family:nnn {#2} {#3} {#4} } }
  }
%    \end{macrocode}
% \end{macro}
%
% \begin{macro}[internal]{\ctex_ltj_set_alternate_family:nnn}
% 实际定义替代字体族。
%    \begin{macrocode}
\cs_new_protected_nopar:Npn \ctex_ltj_set_alternate_family:nnn #1#2#3
  {
    \group_begin:
    \@@_change_encoding:
    \cs_set_eq:NN \CJKfamily \use_none:n
    \ctex_ltj_swap_cs:NN
      \DeclareFontShape@ \ctex_ltj_declare_alternate_shape:nnnnnn
    \tl_set:Nn \l_@@_char_range_clist {#1}
    \fontspec_set_family:Nnn \l_@@_alternate_family_tl {#2} {#3}
    \group_end:
  }
\tl_new:N \l_@@_alternate_family_tl
%    \end{macrocode}
% \end{macro}
%
% \begin{macro}[internal]{\ctex_ltj_swap_cs:NN}
% 交换两个控制序列的意义。
%    \begin{macrocode}
\cs_new_protected:Npn \ctex_ltj_swap_cs:NN #1#2
  {
    \cs_set_eq:NN \@@_tmp:w #1
    \cs_set_eq:NN #1 #2
    \cs_set_eq:NN #2 \@@_tmp:w
    \cs_undefine:N \@@_tmp:w
  }
%    \end{macrocode}
% \end{macro}
%
% \begin{macro}[internal]{LTJFONTUID,\@@_update_family_uid:N}
% \pkg{fontspec} 在一个字体族的选项和字体名称相同的时候,就不定义新字体。为了
% 避免混淆替代字体的设置,我们新定义一个虚拟的选项 \opt{LTJFONTUID},确保
% \pkg{fontspec} 对 CJK 字体族总是定义新字体。
%    \begin{macrocode}
\keys_define:nn { fontspec } { LTJFONTUID .code:n = }
\cs_new_protected_nopar:Npn \@@_update_family_uid:N #1
  {
    \int_gincr:N \g_@@_family_int
    \clist_put_right:Nx #1 { LTJFONTUID = \int_use:N \g_@@_family_int }
  }
\int_new:N \g_@@_family_int
%    \end{macrocode}
% \end{macro}
%
% \begin{macro}[internal]{\ctex_ltj_declare_alternate_shape:nnnnnn}
% 在定义替代字体的字形时,通过字符范围与主字体的对应字形关联起来。
% \tn{DeclareFontShape@} 一个有六个参数,我们只需要使用它的第三个参数 \meta{series}
% 和第四个参数 \meta{shape}。
%    \begin{macrocode}
\cs_new_protected:Npn \ctex_ltj_declare_alternate_shape:nnnnnn #1#2#3#4#5#6
  {
    \ctex_ltj_declare_alternate_shape:nnnnnn {#1} {#2} {#3} {#4} {#5} {#6}
    \ctex_ltj_set_alternate_shape:Nnnnnnn \l_@@_char_range_clist
      { \l_@@_base_family_tl } {#3} {#4}
      { \l_fontspec_family_tl } {#3} {#4}
  }
%    \end{macrocode}
% \end{macro}
%
% \begin{macro}[internal]{\ctex_ltj_set_alternate_shape:Nnnnnnn}
% 与 \pkg{luatexja} 的 \tn{DeclareAlternateKanjiFont} 的功能类似,区别是固定编码
% 为 \tn{CJK@encoding}。这个设置总是全局的。
%    \begin{macrocode}
\cs_new_protected_nopar:Npn \ctex_ltj_set_alternate_shape:Nnnnnnn #1#2#3#4#5#6#7
  {
    \clist_map_inline:Nn #1
      {
        \prop_get:NnNTF \g_@@_char_range_prop { ##1 } \l_@@_char_range_tl
          {
            \ctex_ltj_set_alternate_shape:nnN { #2/#3/#4 } { #5/#6/#7 }
              \l_@@_char_range_tl
          }
          { \ctex_ltj_set_alternate_shape:nnn { #2/#3/#4 } { #5/#6/#7 } { ##1 } }
      }
    \@@_save_alternate_shape:cnn
      { \@@_alternate_cs:n { clear / \l_@@_base_CJKfamily_tl } }
      { luatexja.jfont.clear_alt_font_latex }
      { '\luatexluaescapestring { \CJK@encoding/#2/#3/#4 }' }
  }
%    \end{macrocode}
% \end{macro}
%
% \begin{macro}[internal]{\ctex_ltj_set_alternate_shape:nnn}
% 我们使用 \texttt{->} 而不是像 \pkg{luatexja} 一样使用 \texttt{-} 作为区间的
% 分隔符。\pkg{luatexja} 支持使用负数来引用由 \texttt{JFM} 设置的字符类。如果
% 使用 \texttt{-} 作为分隔符,那么负数单独使用时,就需要把它放在两层花括号之内
% (例如 |{{-1}}|),或者使用类似 |{-1}-{-1}| 的形式才不会解释错误。
%    \begin{macrocode}
\NewDocumentCommand \ctex_ltj_set_alternate_shape:nnn
  { m m > { \SplitArgument { \c_one } { -> } } m }
  { \ctex_ltj_set_alternate_shape:nnnn {#1} {#2} #3 }
\cs_new_protected_nopar:Npn \ctex_ltj_set_alternate_shape:nnnn #1#2#3#4
  {
    \ctex_ltj_set_alternate_shape:n
      {
        \IfNoValueTF {#4}
          { \int_eval:n {#3} , \int_eval:n {#3} , }
          {
            \int_eval:n { \tl_if_blank:nTF {#3} { "80 } {#3} } ,
            \int_eval:n { \tl_if_blank:nTF {#4} { "10FFFF } {#4} } ,
          }
        '\luatexluaescapestring { \CJK@encoding/#2 }' ,
        '\luatexluaescapestring { \CJK@encoding/#1 }'
      }
  }
\cs_new_protected_nopar:Npn \ctex_ltj_set_alternate_shape:n #1
  {
    \ctex_lua_now_x:n { luatexja.jfont.set_alt_font_latex ( #1 ) }
    \@@_save_alternate_shape:cnn
      { \@@_alternate_cs:n { reset / \l_@@_base_CJKfamily_tl } }
      { luatexja.jfont.set_alt_font_latex } {#1}
  }
%    \end{macrocode}
% \end{macro}
%
% \begin{macro}[internal]{\ctex_ltj_set_alternate_shape:nnN}
% 若字符范围预先由 \texttt{declarecharrange} 声明,则可以直接使用。
%    \begin{macrocode}
\cs_new_protected_nopar:Npn \ctex_ltj_set_alternate_shape:nnN #1#2#3
  {
    \tl_map_inline:Nn #3
      {
        \ctex_ltj_set_alternate_shape:n
          {
            ##1 ,
            '\luatexluaescapestring { \CJK@encoding/#2 }' ,
            '\luatexluaescapestring { \CJK@encoding/#1 }'
          }
      }
  }
%    \end{macrocode}
% \end{macro}
%
% \begin{macro}[aux]{\@@_save_alternate_shape:Nnn}
% 将实际设置的替换字形保存起来用于清除或恢复。
% 暂时令 \cs{l_@@_base_family_tl} 为 \cs{scan_stop:} 是让它不被展开,使得替换
% 字体的设置可以在 \tn{addCJKfontfeature} 中直接使用。
%    \begin{macrocode}
\cs_new_protected_nopar:Npn \@@_save_alternate_shape:Nnn #1#2#3
  {
    \group_begin:
    \cs_if_exist:NF #1 { \cs_set_eq:NN #1 \prg_do_nothing: }
    \cs_set_eq:NN \l_@@_base_family_tl \scan_stop:
    \cs_set_eq:NN \luatexluaescapestring \scan_stop:
    \cs_gset_protected_nopar:Npx #1
      { \exp_not:o {#1} \exp_not:N \ctex_lua_now_x:n { #2 ( #3 ) } }
    \group_end:
  }
\cs_generate_variant:Nn \@@_save_alternate_shape:Nnn { c }
%    \end{macrocode}
% \end{macro}
%
% \begin{macro}{clearalternatefont,resetalternatefont}
% 清除和重置操作总是全局的。
%    \begin{macrocode}
\keys_define:nn { ctex }
  {
    clearalternatefont    .code:n =
      { \clist_map_function:xN {#1} \ctex_ltj_clear_alternate_font:n } ,
    resetalternatefont    .code:n =
      { \clist_map_function:xN {#1} \ctex_ltj_reset_alternate_font:n } ,
    clearalternatefont .default:n = \l_ctex_ltj_family_tl ,
    resetalternatefont .default:n = \l_ctex_ltj_family_tl
  }
\cs_new_protected_nopar:Npn \ctex_ltj_clear_alternate_font:n #1
  {
    \group_begin:
      \ctex_ltj_family_if_exist:xNTF {#1} \l_@@_base_family_tl
        {
          \cs_if_exist_use:cT { \@@_alternate_cs:n { clear / #1 } }
            {
              \prop_gput:Nno \g_@@_reset_alternate_prop
                {#1} { \l_@@_base_family_tl }
              \tl_set_eq:NN \CJK@family \l_@@_base_family_tl
              \selectfont
            }
        }
        { \@@_family_unknown_warning:n {#1} }
    \group_end:
  }
\cs_new_protected_nopar:Npn \ctex_ltj_reset_alternate_font:n #1
  {
    \group_begin:
      \prop_gpop:NnNT \g_@@_reset_alternate_prop {#1} \CJK@family
        {
          \tl_set_eq:NN \l_@@_base_family_tl \CJK@family
          \use:c { \@@_alternate_cs:n { reset / #1 } }
          \selectfont
        }
    \group_end:
  }
\prop_new:N \g_@@_reset_alternate_prop
\cs_generate_variant:Nn \clist_map_function:nN { x }
%    \end{macrocode}
% \end{macro}
%
% \begin{macro}{declarecharrange}
% 预先声明字符范围。
%    \begin{macrocode}
\keys_define:nn { ctex }
  {
    declarecharrange .code:n = \ctex_ltj_declare_char_range:x {#1} ,
    declarecharrange .value_required:
  }
\cs_new_protected_nopar:Npn \ctex_ltj_declare_char_range:n #1
  { \clist_map_inline:nn {#1} { \@@_declare_char_range:nn ##1 } }
\cs_generate_variant:Nn \ctex_ltj_declare_char_range:n { x }
\cs_new_protected_nopar:Npn \@@_declare_char_range:nn #1#2
  { \use:x { \ctex_ltj_declare_char_range:nn { \tl_trim_spaces:n {#1} } } {#2} }
%    \end{macrocode}
% \end{macro}
%
% \begin{macro}[internal]{\ctex_ltj_declare_char_range:nn,\g_@@_char_range_prop}
% |#1| 是名字,|#2| 是范围。
%    \begin{macrocode}
\cs_new_protected_nopar:Npn \ctex_ltj_declare_char_range:nn #1#2
  {
    \tl_clear:N \l_@@_char_range_tl
    \clist_map_function:nN {#2} \ctex_ltj_save_char_range:n
    \prop_gput:Nno \g_@@_char_range_prop {#1} { \l_@@_char_range_tl }
    \ctex_ltj_def_char_range_key:n {#1}
    \tl_clear:N \l_@@_char_range_tl
  }
\tl_new:N \l_@@_char_range_tl
\prop_new:N \g_@@_char_range_prop
%    \end{macrocode}
% \end{macro}
%
% \begin{macro}[internal]{\ctex_ltj_save_char_range:n}
% 预先解释字符区间的意义。
%    \begin{macrocode}
\NewDocumentCommand \ctex_ltj_save_char_range:n
  { > { \SplitArgument { \c_one } { -> } } m }
  { \ctex_ltj_save_char_range:nn #1 }
\cs_new_protected_nopar:Npn \ctex_ltj_save_char_range:nn #1#2
  {
    \tl_put_right:Nx \l_@@_char_range_tl
      { {
          \IfNoValueTF {#2}
            { \int_eval:n {#1} , \int_eval:n {#1} }
            {
              \int_eval:n { \tl_if_blank:nTF {#1} { "80 } {#1} } ,
              \int_eval:n { \tl_if_blank:nTF {#2} { "10FFFF } {#2} }
            }
      } }
  }
%    \end{macrocode}
% \end{macro}
%
% \begin{macro}[internal]{\ctex_ltj_def_char_range_key:n}
% 在字体设置选项中定义字符范围键。
%    \begin{macrocode}
\cs_new_protected_nopar:Npn \ctex_ltj_def_char_range_key:n #1
  {
    \keys_if_exist:nnF { ctex_ltj / fontspec } {#1}
      {
        \keys_define:nn { ctex_ltj / fontspec }
          { #1 .code:n = \ctex_ltj_char_range_key:nn {#1} { ##1 } }
      }
  }
%    \end{macrocode}
% \end{macro}
%
% \begin{macro}[internal]{\ctex_ltj_char_range_key:nn}
% 如果字符范围键没有值,则只设置的这个字符范围内的替代字体。
%    \begin{macrocode}
\cs_new_protected:Npn \ctex_ltj_char_range_key:nn #1#2
  {
    \tl_if_blank:nTF {#2}
      { \tl_set:Nn \l_@@_char_range_clist {#1} }
      {
        \clist_if_empty:NT \l_@@_char_range_clist
          {
            \tl_set:Nn \l_@@_tmp_tl { {#1} }
            \@@_char_range_parse_feature:w #2 \q_stop
          }
      }
  }
%    \end{macrocode}
% \end{macro}
%
% \begin{macro}[internal]{\@@_char_range_parse_feature:w}
% 可以使用加方括号的方式,通过文件名来调用字体。这容易与字体选项混淆。例如,需要
% 将 |[simsun.ttc]| 设置为 \opt{range} 的主字体,就需要使用
% |range={{[simsun.ttc]}}| 或者 |[]{[simsun.ttc]}|。下面的目的是,支持直接使用
% |[simsun.ttc]| 和 |[...][simsun.ttc]|。
%    \begin{macrocode}
\NewDocumentCommand \@@_char_range_parse_feature:w
  { +o o u { \q_stop } }
  {
    \exp_args:NNf \tl_put_right:Nn \l_@@_tmp_tl
      {
        \IfNoValueTF {#1} { {#3} }
          {
            \IfNoValueTF {#2}
              { \tl_if_blank:nTF {#3} { { [#1] } } { [ {#1} ] {#3} } }
              { [ {#1} ] { [#2] } }
          }
      }
    \seq_put_right:No \l_@@_alternate_seq { \l_@@_tmp_tl }
  }
%    \end{macrocode}
% \end{macro}
%
% \paragraph{其它设置}
%
% \begin{macro}[internal]{\ctex_ltj_update_xkanjiskip:,\l_@@_xkanjiskip_skip}
% \tn{ltjsetparameter} 对 \opt{xkanjiskip} 是即时赋值。\tn{zw} 与字体相关,因此
% 需要每次 \tn{selectfont} 的时候更新一次 \opt{xkanjiskip}。如果用户设置过
% \opt{xkanjiskip},就不更新。注意,同 \TeX{} 的 \tn{baselineskip} 一样,如果在
% 一个段落内多次设置了 \opt{kanjiskip} 或 \opt{xkanjiskip},最后的设置会影响
% 全段。
%    \begin{macrocode}
\cs_new_protected_nopar:Npn \ctex_ltj_update_xkanjiskip:
  {
    \skip_if_eq:nnT
      { \ltjgetparameter { xkanjiskip } } { \l_@@_xkanjiskip_skip }
      {
        \skip_set:Nn \l_@@_xkanjiskip_skip { .25 \zw plus 1pt minus 1pt }
        \ltjsetparameter { xkanjiskip = \l_@@_xkanjiskip_skip }
      }
  }
%    \end{macrocode}
% 注意,此时还没有实际设置字体,所以 \tn{zw} 是 \opt{0pt}。
%    \begin{macrocode}
\skip_new:N \l_@@_xkanjiskip_skip
\skip_set:Nn \l_@@_xkanjiskip_skip { .25 \zw plus 1pt minus 1pt }
%    \end{macrocode}
% \end{macro}
%
% 在抄录环境中禁用 \opt{autospacing} 和 \opt{autoxspacing}。然而,\pkg{luatexja}
% 还是会使 JAchar 自动折行。没有看到有简单的禁用折行的办法,可能需要设置所有的
% JAchar 的 \opt{prebreakpenalty} 或 \opt{postbreakpenalty} 为 \texttt{10000}:
% \begin{verbatim}
%   \directlua
%     {
%       luatexja.isglobal = tex.globaldefs > 0 and "global" or ""
%       for i = 0x80, 0x10FFFF do
%         if luatexja.charrange.jcr_table_main[i] > 0 and
%            luatexja.charrange.jcr_table_main[i] < 218 and
%            luatexja.charrange.is_japanese_char_curlist(i) then
%           luatexja.stack.set_stack_table(luatexja.stack_table_index.PRE + i, 10000)
%         end
%       end
%     }
% \end{verbatim}
%    \begin{macrocode}
\AtBeginDocument
  {
    \ctex_appto_cmd:NnTF \verbatim@font { \CTEX@verbatim@font@hook }
      { } { \ctex_patch_failure:N \verbatim@font }
  }
\cs_new_protected_nopar:Npn \CTEX@verbatim@font@hook
  { \ltjsetparameter { autospacing = false , autoxspacing = false } }
%    \end{macrocode}
%
%    \begin{macrocode}
%<@@=ctex>
%    \end{macrocode}
%
%    \begin{macrocode}
%</luatex>
%<*pdftex|xetex|luatex>
%    \end{macrocode}
%
% \subsubsection{调整 \tn{CJKfamilydefault}}
%
% \begin{macro}[internal]{\ctex_update_default_family:}
% 在导言区结束,如果 \tn{CJKfamilydefault} 没有被更改,则在此时根据西文字体的情况
% 更新 \tn{CJKfamilydefault}。\pkg{xeCJK} 已经有这个功能,不需要再调整。
%    \begin{macrocode}
%<*pdftex|luatex>
\cs_new_protected_nopar:Npn \ctex_update_default_family:
  {
    \tl_if_eq:NNT \CJKfamilydefault \l_@@_family_default_init_tl
      {
        \group_begin:
          \cs_set_eq:NN \@@_family_default_wrap:n \exp_not:n
          \tl_gset:Nx \CJKfamilydefault
            {
              \str_case:onF { \familydefault }
                {
                  { \rmdefault } { \exp_not:N \CJKrmdefault }
                  { \sfdefault } { \exp_not:N \CJKsfdefault }
                  { \ttdefault } { \exp_not:N \CJKttdefault }
                }
                { \CJKfamilydefault }
            }
        \group_end:
      }
%    \end{macrocode}
% 使用 \LuaLaTeX{} 时,自动调整得到的 \tn{CJKfamilydefault} 可能没有定义,需要确认
% 它的存在性。使用 \pkg{CJK} 宏包时,\texttt{C19rm} 等总是有定义的,不需要确认。
%    \begin{macrocode}
%<*luatex>
    \ctex_ltj_ensure_default_family:
%</luatex>
  }
\AtEndPreamble { \ctex_update_default_family: }
%    \end{macrocode}
% \end{macro}
%
% \begin{macro}[internal]{\l_@@_family_default_init_tl}
% 往 \tn{CJKfamilydefault} 中加入标志,用于判断它是否被更改。
%    \begin{macrocode}
\tl_new:N \l_@@_family_default_init_tl
\cs_new_eq:NN \@@_family_default_wrap:n \use:n
\tl_set:Nx \l_@@_family_default_init_tl
  {
    \exp_not:N \@@_family_default_wrap:n
      { \exp_not:o { \CJKfamilydefault } }
  }
\tl_gset_eq:NN \CJKfamilydefault \l_@@_family_default_init_tl
%</pdftex|luatex>
%    \end{macrocode}
% \end{macro}
%
% \subsubsection{操作系统的判断}
%
% \changes{v2.0}{2014/04/16}{自动检测操作系统,载入对应的字体配置。}
%
% \begin{macro}[internal]{\ctex_detected_platform:}
% 在 \LuaTeX{} 下直接用调用 |os.name| 来判断。
%    \begin{macrocode}
\cs_new_protected_nopar:Npn \ctex_detected_platform:
%<*luatex>
  {
    \tl_gset:Nx \g_@@_fontset_tl
      {
        \ctex_lua_now_x:n
          {
            if ~ os.name == 'windows' then ~
              tex.sprint ( 'windows' )
            elseif ~ os.name == 'macosx' then ~
              tex.sprint ( 'mac' )
            else ~
              tex.sprint ( 'fandol' )
            end
          }
      }
  }
%</luatex>
%    \end{macrocode}
% \pdfTeX{} 和 \XeTeX{} 下则依据 \file{/dev/null} 和 \file{nul:} 的存在性以及
% 文件系统的大小写敏感性来判断。Mac~OS~X 的大小写敏感性在安装时是可选的。为了
% 保险起见,这里的判断很繁琐,最多要进行 4 次文件操作!
%    \begin{macrocode}
%<*xetex|pdftex>
  {
    \file_if_exist:nTF { /dev/null }
      {
        \file_if_exist:nTF { nul: }
          {
            \file_if_exist:nTF { \c_@@_engine_file_tl }
              { \ctex_if_macosx:TF { mac } { windows } }
              { \ctex_if_macosx:TF { mac } { fandol } }
          }
          { \ctex_if_macosx:TF { mac } { fandol } }
      }
      { \tl_gset:Nn \g_@@_fontset_tl { windows } }
  }
\tex_uppercase:D \exp_after:wN
  {
    \exp_after:wN \tl_const:Nn \exp_after:wN \c_@@_engine_file_tl
    \exp_after:wN { \g_file_current_name_tl }
  }
%    \end{macrocode}
% \end{macro}
%
% \begin{macro}[internal]{\ctex_if_macosx:TF}
% 以 \file{/mach_kernel} 为特征文件判断 Mac~OS~X。
%    \begin{macrocode}
\cs_new_protected_nopar:Npn \ctex_if_macosx:TF #1#2
  {
    \file_if_exist:nTF { \c_@@_macosx_file_tl }
      { \tl_gset:Nn \g_@@_fontset_tl {#1} }
      { \tl_gset:Nn \g_@@_fontset_tl {#2} }
  }
\tl_const:Nn \c_@@_macosx_file_tl { /mach_kernel }
%</xetex|pdftex>
%    \end{macrocode}
% \end{macro}
%
% \subsubsection{\pkg{hyperref} 的支持选项}
%
% 处理各个引擎下的 PDF 中文书签问题。在 \pdfTeX{} 下使用 \texttt{GBK} 编码,
% \dvipdfmx{} 驱动可以直接用它的 \tn{special} 命令,其它模式用
% \pkg{xCJK2uni} 宏包处理。使用 \texttt{UTF-8} 编码时,\pkg{CJKutf8} 已经处理了
% 书签问题,但仍需要设置 \opt{pdfencoding} 为 \opt{unicode},目的是在书签的
% 开头写入 BOM (|\376\377|),提示这是 \texttt{UTF-16BE} 字节流。
%    \begin{macrocode}
%<*pdftex>
\cs_if_free:NF \hypersetup
  {
    \hypersetup { driverfallback = dvipdfmx }
    \str_if_eq:onTF { \l_@@_encoding_tl } { GBK }
      {
        \hypersetup { CJKbookmarks = true }
        \ctex_if_pdfmode:TF
          { \RequirePackage { xCJK2uni } }
          {
            \AtEndPreamble
              {
                \str_if_eq:onTF { \Hy@driver } { hdvipdfm }
                  {
                    \AtBeginShipoutFirst
                      { \special { pdf:tounicode~GBK-EUC-UCS2 } }
                  }
                  { \RequirePackage { xCJK2uni } }
              }
          }
      }
      { \hypersetup { pdfencoding = unicode } }
  }
%</pdftex>
%    \end{macrocode}
% 在 \XeTeX{} 下,\pkg{hyperref} 在处理带有非 ASCII 字符和 |\%| 的书签时有问题^^A
% \footnote{\url{https://github.com/CTeX-org/ctex-kit/issues/39}}。
% 事实上,\pkg{hyperref} 在驱动文件 \file{hxetex.def} 中设置了
% \tn{Hy@unicodetrue},从而书签总是会被 \tn{HyPsd@ConvertToUnicode} 转化成
% \texttt{UTF-16BE} 编码的形式(抄录自 \tn{pdfstringdef}的定义):
% \begin{verbatim}
%       \ifHy@unicode
%         \HyPsd@ConvertToUnicode#1%
%         \ifx\HyPsd@pdfencoding\HyPsd@pdfencoding@auto
%           \ltx@IfUndefined{StringEncodingConvertTest}{%
%           }{%
%             \EdefUnescapeString\HyPsd@temp#1%
%             \ifxetex
%               \let\HyPsd@UnescapedString\HyPsd@temp
%               \StringEncodingConvertTest\HyPsd@temp\HyPsd@temp
%                                         {utf16be}{ascii-print}{%
%                 \EdefEscapeString\HyPsd@temp\HyPsd@temp
%                 \global\let#1\HyPsd@temp
%                 \HyPsd@EscapeTeX#1%
%                 \Hy@unicodefalse
%               }{%
%                \HyPsd@ToBigChars#1%
%               }%
% \end{verbatim}
% 通过宏包选项 \opt{pdfencoding=unicode} 设置 \tn{HyPsd@pdfencoding} 为
% \opt{unicode},可以避免随后再将书签从 \texttt{UTF-16BE} 字节流转化回正常
% 字符(其中使用的 \tn{HyPsd@ToBigChars} 没有考虑书签中含有 |\%| 的情况)。
% Heiko Oberdiek 在 \file{README} 中说明了将书签转化回正常字符的意图:避免
% XDVIPDFMX 的警告^^A
% \footnote{\url{http://project.ktug.org/dvipdfmx/mailman/dvipdfmx/2009-December/000153.html}}:
% \begin{verbatim}
%   ** WARNING ** Failed to convert input string to UTF16...
% \end{verbatim}
% \XeTeX{} 的维护者 Khaled Hosny 已经注意到了这个问题^^A
% \footnote{\url{http://tug.org/pipermail/tex-live/2013-December/034613.html}}。
% 需要注意的是,\file{hxetex.def} 重载了宏包选项 \opt{unicode},目的是不能设置
% 它为 \opt{false},但也导致它不会改变 \tn{HyPsd@pdfencoding}。如果
% \pkg{hyperref} 先于 \CTeX{} 被载入,那么 \opt{unicode} 选项是没有意义的。
% 因此要通过意义相同但在 \XeTeX{} 下更保险的 \opt{pdfencoding} 选项来设置。
% 为了与 \XeTeX 下的行为一致(使用 \tn{HyPsd@LoadUnicode} 载入 \file{puenc.def}),
% 在 \LuaTeX{} 下也启用这个选项。
%    \begin{macrocode}
%<*xetex|luatex>
\cs_if_exist_use:NT \hypersetup { { pdfencoding = unicode } }
%</xetex|luatex>
%    \end{macrocode}
%
% \subsubsection{\opt{fntef} 选项}
%
% \changes{v2.0}{2015/03/25}{关闭 \opt{fntef} 选项下的彩色设置。}
% \changes{v2.0}{2015/03/25}{\tn{CTEXunderdot}, \tn{CTEXunderline},
% \tn{CTEXunderdblline}, \tn{CTEXunderwave}, \tn{CTEXsout}, \tn{CTEXxout} 是过
% 时命令;\env{CTEXfilltwosides} 是过时环境。}
%
% 载入 \pkg{CJKfntef} 或 \pkg{xeCJKfntef} 并做适当格式设置。有关 |\CTEX| 开头
% 的宏定义是过时命令,仅做兼容性保留。
%
%    \begin{macrocode}
%<*luatex>
\msg_new:nnn { ctex } { fntef-not-available }
  { Option~`fntef'~is~not~available~in~LuaLaTeX. }
%</luatex>
\if_bool:N \l_@@_fntef_bool
%<pdftex>  \RequirePackage { CJKfntef } \normalem
%<xetex>  \RequirePackage { xeCJKfntef }
%<luatex>  \msg_warning:nn { ctex } { fntef-not-available }
  \clist_map_inline:nn
    { underdot , underline , underdblline , underwave , sout , xout }
%<*pdftex|xetex>
    {
%<pdftex>      \tl_clear:c { CJK#1color }
%<xetex>      \keys_set:nn { xeCJK / options } { #1 / format = { } }
      \cs_new_protected_nopar:cpx { CTEX#1 }
        {
          \msg_warning:nnnn { ctex } { deprecated-command } { \exp_not:c { CTEX#1 } }
            { You~ can~ use~ the~ command~ with~ prefix~ \exp_not:N \CJK~ instead. }
          \exp_not:c { CJK#1 }
        }
    }
  \cs_new_protected_nopar:Npn { \CTEXfilltwosides }
    {
      \msg_warning:nnnn { ctex } { deprecated-environment } { CTEXfilltwosides }
        { You~ can~ use~ `CJKfilltwosides'~ environment~ instead. }
      \CJKfilltwosides
    }
  \cs_new_protected_nopar:Npn { \endCTEXfilltwosides } { \endCJKfilltwosides }
%</pdftex|xetex>
%<*luatex>
    { \cs_new_eq:cN { CTEX#1 } \use:n }
  \cs_new_eq:NN \CTEXfilltwosides \use_none:n
  \cs_new_eq:NN \endCTEXfilltwosides \prg_do_nothing:
%</luatex>
%<*pdftex>
  \clist_map_inline:nn
    {
      underdotbasesep ,   underdotsep ,     underlinebasesep ,
      underlinesep ,      underdbllinesep , underdbllinebasesep ,
      underwavebasesep ,  underwavesep ,    southeight ,
      underdotcolor ,     underwavecolor ,  underlinecolor ,
      underdbllinecolor , soutcolor ,       xoutcolor
    }
    {
      \cs_new_eq:cc { CTEX#1 } { CJK#1 }
      \cs_set_nopar:cpx { CJK#1 } { \exp_not:c { CTEX#1 } }
    }
%</pdftex>
\fi:
%    \end{macrocode}
%
% \subsubsection{\tn{ccwd} 的更新}
%
% \begin{macro}[internal]{\ctex_update_ccwd:,\ccwd}
%    \begin{macrocode}
\cs_new_protected_nopar:Npn \ctex_update_ccwd:
%<*pdftex|xetex>
  {
    \hbox_set:Nn \l_@@_tmp_box { \CJKglue }
    \dim_set:Nn \ccwd { \box_wd:N \l_@@_tmp_box + \f@size \p@ }
  }
%</pdftex|xetex>
%<*luatex>
  { \skip_set:Nn \ccwd { \ltjgetparameter { kanjiskip } + \zw } }
%</luatex>
\dim_new:N \ccwd
%    \end{macrocode}
% \end{macro}
%
% \begin{macro}[internal]{\ctex_update_ccglue:}
% 更新字间距。
%    \begin{macrocode}
\cs_new_protected_nopar:Npn \ctex_update_ccglue:
%<*pdftex>
  {
    \cs_set_protected_nopar:Npn \CJKglue
      { \skip_horizontal:N \l_@@_ccglue_skip }
  }
%</pdftex>
%<*xetex>
  { \xeCJKsetup { CJKglue = { \skip_horizontal:N \l_@@_ccglue_skip } } }
%</xetex>
%<*luatex>
  { \ltjsetparameter { kanjiskip = \l_@@_ccglue_skip } }
%</luatex>
\skip_new:N \l_@@_ccglue_skip
%    \end{macrocode}
% \end{macro}
%
% \begin{macro}[internal,pTF]{\ctex_if_ccglue_touched:}
% 检查用户是否修改过汉字间距。
%    \begin{macrocode}
\prg_new_conditional:Npnn \ctex_if_ccglue_touched: { TF }
  {
%<*pdftex|xetex>
    \if_meaning:w \CJKglue \@@_ccglue:
      \prg_return_false: \else: \prg_return_true: \fi:
%</pdftex|xetex>
%<*luatex>
    \skip_if_eq:nnTF { \l_@@_ccglue_skip } { \ltjgetparameter { kanjiskip } }
      { \prg_return_false: } { \prg_return_true: }
%</luatex>
  }
%    \end{macrocode}
% 注意下面的标记不能用 \verb"%<pdftex|xetex>",它会导致旧版本的 \pkg{l3docstrip}
% 不能替换 \texttt{@@}。
%    \begin{macrocode}
%<*pdftex|xetex>
\ctex_at_end:n { \cs_new_eq:NN \@@_ccglue: \CJKglue }
%</pdftex|xetex>
%    \end{macrocode}
% \end{macro}
%
% \begin{macro}[internal]{\ctex_update_em_unit:}
% 将当前汉字的宽度保存到 \tn{ccwd} 中备用。不采用 \texttt{1 em},因为这时的
% \texttt{1 em} 实际上来自西文字体的信息,未必等于汉字的宽度,这似乎在传统的
% \file{.tfm} 字体上表现更明显。在 \pdfTeX{} 和 \XeTeX{} 下,直接使用 |\f@size\p@|
% 作为汉字的宽度,这应该对大多数汉字字体都成立,但不适用于诸如“方正兰亭黑长”之类
% 的特殊字体。在 \XeTeX{} 可以用 \tn{fontcharwd} 来改进。而在 \pdfTeX{} 下,若使用
% \pkg{zhmetrics} 技术,所有的汉字共享同一个 \file{.tfm},\tn{fontcharwd} 也就没有
% 意义。在 \LuaTeX{} 下,\pkg{luatexja} 总是按照 JFM 中的设置输出汉字的宽度,可以
% 直接用 \tn{zw} 作为汉字宽度。
%    \begin{macrocode}
\cs_new_protected_nopar:Npn \ctex_update_em_unit:
%<pdftex|xetex>  { \dim_set:Nn \ccwd { \f@size \p@ } }
%<luatex>  { \dim_set_eq:NN \ccwd \zw }
%    \end{macrocode}
% \end{macro}
%
% \subsubsection{其它}
%
% \begin{macro}[internal]{\ctex_add_to_selectfont:n,\CTEX@selectfont@hook}
% \tn{EverySelectfont} 直到文档开始时才有效。为了 \tn{ccwd} 和 \pkg{luatexja} 的
% 字体设置在导言区也可用,我们还需要在这里手工修改 \tn{selectfont}。\pkg{everysel}
% 宏包会用 \tn{CheckCommand} 来检查 \tn{selectfont} 是否为标准定义。我们修改了
% \tn{selectfont},所以会给出一个警告。为了消除这个警告,在它检查之前,还原本来
% 定义。
%    \begin{macrocode}
\cs_new_protected:Npn \ctex_add_to_selectfont:n #1
  {
    \cs_set_protected_nopar:Npx \CTEX@selectfont@hook
      { \exp_not:o { \CTEX@selectfont@hook #1 } }
  }
\cs_new_eq:NN \CTEX@selectfont@hook \prg_do_nothing:
\cs_new_eq:Nc \@@_save_selectfont: { selectfont ~ }
\use:n
  {
    \ExplSyntaxOff
    \ctex_preto_cmd:NnTF \selectfont { \CTEX@selectfont@hook }
      {
        \tl_put_left:Nn \@EverySelectfont@Init
          { \cs_set_eq:cN { selectfont ~ } \@@_save_selectfont: }
      }
      { \ctex_patch_failure:N \selectfont }
  }
\ExplSyntaxOn
%    \end{macrocode}
% \end{macro}
%
% \tn{CJK@plane} 有定义,说明处于 \pkg{CJK} 宏包的 \tn{CJKsymbol} 之内,不必使用钩子。
%    \begin{macrocode}
%<*pdftex>
\EverySelectfont { \cs_if_exist:NF \CJK@plane { \CTEX@selectfont@hook } }
%</pdftex>
%<*xetex|luatex>
\EverySelectfont { \CTEX@selectfont@hook }
%</xetex|luatex>
%    \end{macrocode}
%
% Attribute 寄存器 \tn{ltj@curjfnt} 的初始值是 $-1$,必须把它设置为一个有效的
% \texttt{font.id},否则编译时会直接退出。
%    \begin{macrocode}
%<*luatex>
\ctex_add_to_selectfont:n
  {
    \ctex_ltj_select_font:
    \ctex_ltj_select_alternate_font:
    \ctex_ltj_update_xkanjiskip:
  }
\tl_set:Nn \CJK@family { song } \selectfont
\tl_clear:N \CJK@family
%</luatex>
%    \end{macrocode}
%
% \begin{macro}{space}
% 在导言区或正文中设置忽略空格方式。
%    \begin{macrocode}
\keys_define:nn { ctex }
  {
%<*pdftex|xetex>
    space .choice: ,
    space .default:n = { true } ,
    space / true  .code:n =
%<pdftex>      { \cs_set_eq:NN \CJK@ignorespaces \prg_do_nothing: } ,
%<xetex>      { \xeCJKsetup { CJKspace = true } } ,
    space / auto  .code:n =
%<pdftex>      { \cs_set_eq:NN \CJK@ignorespaces \ctex_auto_ignorespaces: } ,
%<xetex>      { \xeCJKsetup { CJKspace = false } } ,
    space / false .code:n =
%<pdftex>      { \cs_set_eq:NN \CJK@ignorespaces \tex_ignorespaces:D }
%<xetex>      { \xeCJKsetup { CJKspace = false } }
%</pdftex|xetex>
%<*luatex>
    space .code:n =
      { \msg_warning:nn { ctex } { invalid-option } }
%</luatex>
  }
%    \end{macrocode}
% \end{macro}
%
% \begin{macro}{punct}
% 在导言区或正文中设置标点符号输出格式。\pkg{luatexja} 设置的是字体的默认 \texttt{JFM},
% 只会影响到之后设置的字体。
%    \begin{macrocode}
\keys_define:nn { ctex }
  {
    punct .code:n =
      {
        \tl_set:Nx \l_@@_punct_tl { #1 }
%<pdftex>        \punctstyle { \l_@@_punct_tl }
%<xetex>        \xeCJKsetup { PunctStyle = \l_@@_punct_tl }
%<luatex>        \ctex_mono_jfm:o { \l_@@_punct_tl }
      } ,
    punct .default:n = { quanjiao } ,
  }
%    \end{macrocode}
% \end{macro}
%
%    \begin{macrocode}
%</pdftex|xetex|luatex>
%<*class|style|ctexsize>
%    \end{macrocode}
%
% \subsection{中文字号}
%
% \changes{v2.0}{2014/03/08}{将中文字号功能提取到可以独立使用的 \pkg{ctexsize}。}
%
% \begin{macro}{\zihao}
%    \begin{macrocode}
\NewDocumentCommand \zihao { m }
  { \exp_args:Nx \ctex_zihao:n {#1} \tex_ignorespaces:D }
%    \end{macrocode}
% \end{macro}
%
% \begin{macro}[internal]{\ctex_zihao:n}
%    \begin{macrocode}
\cs_new_protected_nopar:Npn \ctex_zihao:n #1
  {
    \prop_get:NnNTF \c_@@_font_size_prop {#1} \l_@@_font_size_tl
      { \exp_after:wN \fontsize \l_@@_font_size_tl \selectfont }
      { \msg_error:nnn { ctex } { fontsize } {#1} }
  }
\msg_new:nnnn { ctex } { fontsize }
  { Undefined~Chinese~font~size~`#1'~in~command~\token_to_str:N \zihao.}
  {
    The~old~font~size~is~used~if~you~continue.\\
    The~available~font~sizes~are~listed~as~follow.\\
    \seq_use:Nnnn \c_@@_font_size_seq { ~and~ } { ,~ } { ,~and~ }.
  }
%    \end{macrocode}
% \end{macro}
%
% \subsubsection{定义中文字号}
%
% \changes{v2.0}{2014/03/08}{中文字号不再采用近似值。}
%
% \begin{variable}[internal]{\c_@@_font_size_prop}
% \begin{macro}[aux]{\@@_save_font_size:nn}
% 基础行距是字号的 $1.2$ 倍,采用 \hologo{eTeX} 的 scaling 运算得到的结果
% 要比简单的 |1.2\dimexpr| 精确^^A
% \footnote{\url{http://thread.gmane.org/gmane.comp.tex.latex.latex3/3190}}。
%    \begin{macrocode}
\prop_new:N \c_@@_font_size_prop
\seq_new:N \c_@@_font_size_seq
\cs_new_protected_nopar:Npn \@@_save_font_size:nn #1#2
  {
    \use:x
      {
        \prop_gput:Nnn \exp_not:N \c_@@_font_size_prop {#1}
          {
            { \dim_to_decimal:n {#2} }
            { \dim_to_decimal:n { (#2) * \c_six / \c_five } }
          }
      }
    \seq_gput_right:Nn \c_@@_font_size_seq {#1}
  }
\clist_map_inline:nn
  {
    {  8 } { 5    bp } ,
    {  7 } { 5.5  bp } ,
    { -6 } { 6.5  bp } ,
    {  6 } { 7.5  bp } ,
    { -5 } { 9    bp } ,
    {  5 } { 10.5 bp } ,
    { -4 } { 12   bp } ,
    {  4 } { 14   bp } ,
    { -3 } { 15   bp } ,
    {  3 } { 16   bp } ,
    { -2 } { 18   bp } ,
    {  2 } { 22   bp } ,
    { -1 } { 24   bp } ,
    {  1 } { 26   bp } ,
    { -0 } { 36   bp } ,
    {  0 } { 42   bp }
  }
  { \@@_save_font_size:nn #1 }
%    \end{macrocode}
% \end{macro}
% \end{variable}
%
% \subsubsection{修改默认字号大小}
%
% \begin{macro}[internal]{\ctex_set_font_size:Nnn}
%    \begin{macrocode}
\cs_new_protected:Npn \ctex_set_font_size:Nnn #1#2#3
  {
    \prop_get:NnNTF \c_@@_font_size_prop {#2} \l_@@_font_size_tl
      { \exp_after:wN \@@_set_font_size:nnNn \l_@@_font_size_tl #1 {#3} }
      { \msg_error:nnn { ctex } { fontsize } {#2} }
  }
\cs_new_protected:Npn \@@_set_font_size:nnNn #1#2#3#4
  { \cs_set_protected_nopar:Npn #3 { \@setfontsize #3 {#1} {#2} #4 } }
%    \end{macrocode}
% \end{macro}
%
%    \begin{macrocode}
\if_case:w \g_@@_font_size_flag
  \ctex_set_font_size:Nnn \normalsize { 5 }
    {
      \abovedisplayskip 10\p@ \@plus2\p@ \@minus5\p@
      \abovedisplayshortskip \z@ \@plus3\p@
      \belowdisplayshortskip 6\p@ \@plus3\p@ \@minus3\p@
      \belowdisplayskip \abovedisplayskip
      \let\@listi\@listI
    }
  \ctex_set_font_size:Nnn \small { -5 }
    {
      \abovedisplayskip 8.5\p@ \@plus3\p@ \@minus4\p@
      \abovedisplayshortskip \z@ \@plus2\p@
      \belowdisplayshortskip 4\p@ \@plus2\p@ \@minus2\p@
      \def\@listi{\leftmargin\leftmargini
                  \topsep 4\p@ \@plus2\p@ \@minus2\p@
                  \parsep 2\p@ \@plus\p@ \@minus\p@
                  \itemsep \parsep}
      \belowdisplayskip \abovedisplayskip
    }
  \ctex_set_font_size:Nnn \footnotesize { 6 }
    {
      \abovedisplayskip 6\p@ \@plus2\p@ \@minus4\p@
      \abovedisplayshortskip \z@ \@plus\p@
      \belowdisplayshortskip 3\p@ \@plus\p@ \@minus2\p@
      \def\@listi{\leftmargin\leftmargini
                  \topsep 3\p@ \@plus\p@ \@minus\p@
                  \parsep 2\p@ \@plus\p@ \@minus\p@
                  \itemsep \parsep}
      \belowdisplayskip \abovedisplayskip
    }
  \ctex_set_font_size:Nnn \scriptsize { -6 } { }
  \ctex_set_font_size:Nnn \tiny  {  7 } { }
  \ctex_set_font_size:Nnn \large { -4 } { }
  \ctex_set_font_size:Nnn \Large { -3 } { }
  \ctex_set_font_size:Nnn \LARGE { -2 } { }
  \ctex_set_font_size:Nnn \huge  {  2 } { }
  \ctex_set_font_size:Nnn \Huge  {  1 } { }
\or:
  \ctex_set_font_size:Nnn \normalsize { -4 }
    {
      \abovedisplayskip 12\p@ \@plus3\p@ \@minus7\p@
      \abovedisplayshortskip \z@ \@plus3\p@
      \belowdisplayshortskip 6.5\p@ \@plus3.5\p@ \@minus3\p@
      \belowdisplayskip \abovedisplayskip
      \let\@listi\@listI
    }
  \ctex_set_font_size:Nnn \small { 5 }
    {
      \abovedisplayskip 11\p@ \@plus3\p@ \@minus6\p@
      \abovedisplayshortskip \z@ \@plus3\p@
      \belowdisplayshortskip 6.5\p@ \@plus3.5\p@ \@minus3\p@
      \def\@listi{\leftmargin\leftmargini
                  \topsep 9\p@ \@plus3\p@ \@minus5\p@
                  \parsep 4.5\p@ \@plus2\p@ \@minus\p@
                  \itemsep \parsep}
      \belowdisplayskip \abovedisplayskip
    }
  \ctex_set_font_size:Nnn \footnotesize { -5 }
    {
      \abovedisplayskip 10\p@ \@plus2\p@ \@minus5\p@
      \abovedisplayshortskip \z@ \@plus3\p@
      \belowdisplayshortskip 6\p@ \@plus3\p@ \@minus3\p@
      \def\@listi{\leftmargin\leftmargini
                  \topsep 6\p@ \@plus2\p@ \@minus2\p@
                  \parsep 3\p@ \@plus2\p@ \@minus\p@
                  \itemsep \parsep}
      \belowdisplayskip \abovedisplayskip
    }
  \ctex_set_font_size:Nnn \scriptsize { 6 } { }
  \ctex_set_font_size:Nnn \tiny  { -6 } { }
  \ctex_set_font_size:Nnn \large { -3 } { }
  \ctex_set_font_size:Nnn \Large { -2 } { }
  \ctex_set_font_size:Nnn \LARGE {  2 } { }
  \ctex_set_font_size:Nnn \huge  { -1 } { }
  \ctex_set_font_size:Nnn \Huge  {  1 } { }
\fi:
%    \end{macrocode}
%
% \begin{macro}[internal]{\ctex_declare_math_sizes:nnnn}
%    \begin{macrocode}
\cs_new_protected_nopar:Npn \ctex_declare_math_sizes:nnnn #1#2#3#4
  {
    \@@_get_font_sizes:Nn \l_@@_font_size_tl { {#1} {#2} {#3} {#4} }
    \exp_after:wN \DeclareMathSizes \l_@@_font_size_tl
  }
%    \end{macrocode}
% \end{macro}
%
% \begin{macro}[aux]{\@@_get_font_sizes:Nn}
%    \begin{macrocode}
\cs_new_protected_nopar:Npn \@@_get_font_sizes:Nn #1#2
  {
    \tl_clear:N #1
    \tl_map_inline:nn {#2}
      {
        \prop_get:NnNTF \c_@@_font_size_prop {##1} \l_@@_tmp_tl
          { \tl_put_right:Nx #1 { { \tl_head:N \l_@@_tmp_tl } } }
          { \tl_put_right:Nx #1 { { \dim_to_decimal:n { ##1 } } } }
      }
  }
%    \end{macrocode}
% \end{macro}
%
%    \begin{macrocode}
\clist_map_inline:nn
  {
    {  8 }{  8 }{ 5pt }{ 5pt } ,
    {  7 }{  7 }{ 5pt }{ 5pt } ,
    { -6 }{ -6 }{ 5pt }{ 5pt } ,
    {  6 }{  6 }{ 5pt }{ 5pt } ,
    { -5 }{ -5 }{ 6pt }{ 5pt } ,
    {  5 }{  5 }{ 7pt }{ 5pt } ,
    { -4 }{ -4 }{ 8pt }{ 6pt } ,
    {  4 }{  4 }{  5 }{  6 } ,
    { -3 }{ -3 }{ -4 }{ -5 } ,
    {  3 }{  3 }{  4 }{  5 } ,
    { -2 }{ -2 }{ -3 }{ -4 } ,
    {  2 }{  2 }{  3 }{  4 } ,
    { -1 }{ -1 }{ -2 }{ -3 } ,
    {  1 }{  1 }{  2 }{  3 } ,
    { -0 }{ -0 }{ -1 }{ -2 } ,
    {  0 }{  0 }{  1 }{  2 }
  }
  { \ctex_declare_math_sizes:nnnn #1 }
%    \end{macrocode}
%
%    \begin{macrocode}
%<ctexsize>\normalsize
%    \end{macrocode}
%
%    \begin{macrocode}
%</class|style|ctexsize>
%<*class|style>
%    \end{macrocode}
%
% \subsubsection{字距与缩进}
%
% \begin{macro}{autoindent}
% |autoindent| 也是可以用在正文中的选项,意义与宏包选项 |option/autoindent| 相同。
%    \begin{macrocode}
\keys_define:nn { ctex }
  {
    autoindent .choice: ,
    autoindent .default:n = { true } ,
    autoindent / true  .code:n =
      {
        \bool_set_true:N \l_@@_autoindent_bool
        \ctex_update_parindent:
      } ,
    autoindent / false .code:n =
      { \bool_set_false:N \l_@@_autoindent_bool }
  }
%    \end{macrocode}
% \end{macro}
%
% \begin{macro}{\CTEXsetfont}
% 无论字体大小是否变化都更新相关信息。
%    \begin{macrocode}
\NewDocumentCommand \CTEXsetfont { }
  { \cs_if_free:NTF \size@update { \ctex_update_size: } { \selectfont } }
%    \end{macrocode}
% \end{macro}
%
% \begin{macro}[internal]{\ctex_update_size:}
% 在字号变化时更新 \tn{ccwd}、\tn{parindent} 和汉字间距。字距为零则恢复正常设置。
%    \begin{macrocode}
\cs_new_protected_nopar:Npn \ctex_update_size:
  {
    \tl_if_eq:NNTF \l_@@_ziju_tl \c_@@_zero_tl
      {
        \ctex_update_stretch:
        \ctex_update_parindent:
      }
      { \ctex_update_ziju: }
  }
\tl_const:Nx \c_@@_zero_tl { \fp_use:N \c_zero_fp }
\tl_new:N \l_@@_ziju_tl
\tl_set_eq:NN \l_@@_ziju_tl \c_@@_zero_tl
%    \end{macrocode}
% 在 \tn{selectfont} 中,若 \tn{size@update} 为 \tn{relax},说明字体大小没有变化,
% 我们也就不用更新相关参数。
%    \begin{macrocode}
\ctex_add_to_selectfont:n
  { \cs_if_free:NF \size@update { \ctex_update_size: } }
%    \end{macrocode}
% \end{macro}
%
% \begin{macro}{linestretch}
% 若行宽不是汉字宽度的整数倍,自然要求伸展它们之间的差。这里设置的是在此基础上的
% 额外伸展量。初始化为一个汉字的宽度。若设置为 \tn{maxdimen},则禁用此功能。
% 参数的默认单位是汉字的宽度 \tn{ccwd}。
%    \begin{macrocode}
\keys_define:nn { ctex }
  {
    linestretch .code:n =
      {
        \dim_compare:nNnTF
          { \ctex_default_pt:n {#1} } = { \ctex_default_pt:n { #1 in } }
          { \tl_set:Nn \l_@@_line_stretch_tl {#1} }
          { \tl_set:Nn \l_@@_line_stretch_tl { #1 \ccwd } }
        \CTEXsetfont
      } ,
    linestretch .value_required:
  }
\tl_new:N \l_@@_line_stretch_tl
\tl_set:Nn \l_@@_line_stretch_tl { \ccwd }
%    \end{macrocode}
% \end{macro}
%
% \begin{macro}[internal]{\ctex_update_stretch:}
% 首先计算一行上汉字的字数,\tn{CJKglue} 相当于将 \tn{linewidth} 与汉字总宽度之差
% 均匀地填充到汉字之间。\hologo{eTeX} 的除法是四舍五入,而我们这里应该用截断。由于
% 没有可展性的要求,直接用原语 \cs{tex_divide:D} 要比 \cs{int_div_truncate:nn}
% 快一些。下面的算法还兼顾到了 \tn{linewidth} 不为汉字字宽的整数倍的情况。
% 若用户禁用 \opt{linestretch} 并且修改过 \tn{CJKglue},则只更新
% \tn{ccwd},否则设置伸展量为 $0.08$ 倍 \tn{baselineskip}。注意 \pkg{everysel} 的
% 钩子位于 \tn{size@update} 之前,\tn{baselineskip} 还未更新,不能直接使用它。
%    \begin{macrocode}
\cs_new_protected_nopar:Npn \ctex_update_stretch:
  {
    \ctex_update_em_unit:
    \dim_set:Nn \l_@@_tmp_dim { \l_@@_line_stretch_tl }
    \dim_compare:nNnTF \l_@@_tmp_dim = \c_max_dim
      {
        \ctex_if_ccglue_touched:TF
          { \ctex_update_ccwd: }
          {
            \dim_set:Nn \l_@@_tmp_dim
              { \baselinestretch \etex_glueexpr:D \f@baselineskip \scan_stop: }
            \skip_set:Nn \l_@@_ccglue_skip
              { \c_zero_dim plus .08 \l_@@_tmp_dim }
            \ctex_update_ccglue:
          }
      }
      {
        \int_set:Nn \l_@@_tmp_int
          { \etex_dimexpr:D \linewidth - \ccwd - \l_@@_tmp_dim \scan_stop: }
        \tex_divide:D \l_@@_tmp_int \ccwd
        \int_compare:nNnTF \l_@@_tmp_int > \c_zero
          {
            \skip_set:Nn \l_@@_ccglue_skip
              {
                \c_zero_dim plus \dim_eval:n
                  {
                    ( \linewidth - \ccwd - \l_@@_tmp_int \ccwd ) /
                    \l_@@_tmp_int
                  }
              }
          }
          { \skip_zero:N \l_@@_ccglue_skip }
        \ctex_update_ccglue:
      }
  }
%    \end{macrocode}
% \end{macro}
%
% \begin{macro}[internal]{\ctex_update_parindent:}
% 更新段落首行缩进。此函数在字号变化时调用。
%    \begin{macrocode}
\cs_new_protected_nopar:Npn \ctex_update_parindent:
  {
    \bool_if:NT \l_@@_autoindent_bool
      {
        \dim_compare:nNnF \parindent = \c_zero_dim
          { \dim_set:Nn \parindent { 2 \ccwd } }
      }
  }
%    \end{macrocode}
% \end{macro}
%
% \begin{macro}{\ziju}
% 若参数为 $0$,则恢复正常间距。
%    \begin{macrocode}
\NewDocumentCommand \ziju { m }
  { \exp_args:Nx \ctex_ziju:n {#1} \tex_ignorespaces:D }
\cs_new_protected_nopar:Npn \ctex_ziju:n #1
  {
    \tl_set:Nx \l_@@_ziju_tl { \fp_eval:n {#1} }
    \CTEXsetfont
  }
%    \end{macrocode}
% \end{macro}
%
% \begin{macro}[internal]{\ctex_update_ziju:}
% 更新字距。若字距不大于 $-1$,即 \tn{ccwd} 为非正值,则不计算伸缩值。
% 否则,首先假定汉字的宽度为正常宽度加上字距,看一行上能正常放下多少个汉字。
%    \begin{macrocode}
\cs_new_protected_nopar:Npn \ctex_update_ziju:
  {
    \ctex_update_em_unit:
    \dim_set:Nn \l_@@_ziju_dim { \l_@@_ziju_tl \ccwd }
    \dim_add:Nn \ccwd { \l_@@_ziju_dim }
    \dim_compare:nNnTF \ccwd > \c_zero_dim
%    \end{macrocode}
% 伸展量保证行内的剩余空白能够被均匀地填充到汉字之间,收缩的最大限度是让当前行
% 还能够再挤下一个汉字并且不会出现负间距。由 \TeX{} 决定伸展还是收缩。
%    \begin{macrocode}
      {
        \dim_set:Nn \l_@@_tmp_dim
          { \linewidth - \ccwd + \l_@@_ziju_dim }
        \int_set:Nn \l_@@_tmp_int { \l_@@_tmp_dim }
        \tex_divide:D \l_@@_tmp_int \ccwd
        \dim_sub:Nn \l_@@_tmp_dim { \l_@@_tmp_int \ccwd }
%    \end{macrocode}
% 由于 \tn{parindent} 是一个固定值,并不参与伸缩,容易导致第一行出现坏盒子。
% 我们在这里将字数减去 $2$,以此放大伸缩值。
%    \begin{macrocode}
        \dim_compare:nNnF \parindent = \c_zero_dim
          {
            \int_compare:nNnF \l_@@_tmp_int < \c_three
              { \int_sub:Nn \l_@@_tmp_int { \c_two } }
          }
        \skip_set:Nn \l_@@_ccglue_skip
          {
            \l_@@_ziju_dim
            plus  \dim_eval:n { \l_@@_tmp_dim / \l_@@_tmp_int }
            minus \dim_min:nn { \dim_abs:n { \l_@@_ziju_dim } }
              { ( \ccwd - \l_@@_tmp_dim ) / ( \l_@@_tmp_int + \c_one ) }
          }
      }
      { \skip_set:Nn \l_@@_ccglue_skip { \l_@@_ziju_dim } }
    \ctex_update_ccglue:
%    \end{macrocode}
% 字距设置得比较大时,为了尽量保证段首缩进能够与下一行对齐,应该需要相应地加上
% 或者减去伸缩值。但是这里并不清楚 \TeX{} 是伸展还是收缩,之前以“当前行是否还
% 放得下一个汉字”为标准加上或减去伸缩值的做法也未必与实际结果一致,所以只好还
% 是设置为 |2\ccwd|。
%    \begin{macrocode}
    \ctex_update_parindent:
  }
\dim_new:N \l_@@_ziju_dim
%    \end{macrocode}
% \end{macro}
%
% 更新行距。
%    \begin{macrocode}
\linespread { \fp_use:N \l_@@_line_spread_fp }
%    \end{macrocode}
%
% 激活默认字体大小,更新 \tn{parindent} 和 \tn{CJKglue}。
%    \begin{macrocode}
\normalsize
%    \end{macrocode}
%
% \changes{v2.0}{2014/04/23}{调整 \tn{footnotesep} 的大小,以适合行距的变化。}
%
% \begin{variable}[internal]{\footnotesep}
% 由于我们加宽了行距,导致脚注的间距与行距不协调,需要调整 \tn{footnotesep}。标准
% 文档类对 \tn{footnotesep} 的设置是,字体大小为 \tn{footnotesize} 时 \tn{strutbox}
% 的高度(默认值是 |.7\baselineskip|)。我们沿用这个设置方法,只需要更新具体的大小。
%    \begin{macrocode}
\group_begin: \footnotesize \exp_args:NNNo \group_end:
\dim_set:Nn \footnotesep { \dim_use:N \box_ht:N \strutbox }
%    \end{macrocode}
% \end{variable}
%
% \changes{v2.0}{2015/03/21}{\tn{CTEXindent}, \tn{CTEXnoindent} 是过时命令。}
% \begin{macro}{\CTEXindent,\CTEXnoindent}
% 过时命令。
%    \begin{macrocode}
\NewDocumentCommand \CTEXindent { }
  {
    \msg_warning:nnnn { ctex } { deprecated-command } { \CTEXindent }
      { \parindent is~ set~ to~ 2\ccwd. }
    \ctex_update_ccwd: \dim_set:Nn \parindent { 2 \ccwd }
  }
\NewDocumentCommand \CTEXnoindent { }
  {
    \msg_warning:nnnn { ctex } { deprecated-command } { \CTEXnoindent }
      { \parindent is~ set~ to~ 0pt. }
    \dim_zero:N \parindent
  }
%    \end{macrocode}
% \end{macro}
%
%    \begin{macrocode}
\bool_if:NT \l_@@_indent_bool
  { \RequirePackage { indentfirst } }
%    \end{macrocode}
%
% \subsection{中文数字与日期}
%
%    \begin{macrocode}
\PassOptionsToPackage { encoding = \l_@@_encoding_tl } { zhnumber }
\RequirePackage { zhnumber }
%    \end{macrocode}
%
% \begin{macro}{\chinese}
%    \begin{macrocode}
\cs_new_eq:NN \chinese \zhnum
\cs_new_eq:NN \Chinese \chinese
\cs_new_eq:NN \CTEXcounter \use_none:n
%    \end{macrocode}
% \end{macro}
%
% \begin{macro}{\CTEXnumber,\CTEXdigits}
%    \begin{macrocode}
\NewDocumentCommand \CTEXnumber { m m }
  { \protected@edef #1 { \zhnumber {#2} } }
\NewDocumentCommand \CTEXdigits { m m }
  { \protected@edef #1 { \zhdigits {#2} } }
%    \end{macrocode}
% \end{macro}
%
% \begin{macro}{today}
%    \begin{macrocode}
\cs_set_eq:NN \CTEX@todayold \today
\keys_define:nn { ctex }
  {
    today .choice: ,
    today / old     .code:n =
      { \cs_set_eq:NN \today \CTEX@todayold } ,
    today / small   .code:n =
      {
        \cs_set_eq:NN \today \zhtoday
        \zhnumsetup { time = Arabic }
      } ,
    today / big     .code:n =
      {
        \cs_set_eq:NN \today \zhtoday
        \zhnumsetup { time = Chinese }
      } ,
    today / unknown .code:n =
      { \msg_error:nnx { ctex } { today-undef } {#1} }
  }
\msg_new:nnnn { ctex } { today-undef }
  { Today~format~`#1'~is~undefined. }
  { Available~today~formats~are~`old',~`small',~and~`big'. }
\bool_if:NT \l_@@_caption_bool
  { \keys_set:nn { ctex } { today = small } }
%    \end{macrocode}
% \end{macro}
%
% \subsection{其它中文标题定义}
%
% \changes{v2.0}{2014/03/08}{将标题汉化功能加入 \pkg{ctex.sty}。}
%
%    \begin{macrocode}
\keys_define:nn { ctex }
  {
    contentsname   .tl_set:N = \contentsname ,
    listfigurename .tl_set:N = \listfigurename ,
    listtablename  .tl_set:N = \listtablename ,
    figurename     .tl_set:N = \figurename ,
    tablename      .tl_set:N = \tablename ,
    abstractname   .tl_set:N = \abstractname ,
    indexname      .tl_set:N = \indexname ,
    appendixname   .tl_set:N = \appendixname ,
%<article>    bibname        .tl_set:N = \refname
%<book|report>    bibname        .tl_set:N = \bibname
  }
%    \end{macrocode}
%
%    \begin{macrocode}
%<*style>
\msg_new:nnn { ctex } { ctexbibname }
  {
    Neither~`\token_to_str:N \bibname'~nor~`\token_to_str:N \refname'~can~be~found.\\
    The~key~`bibname'~will~set~`\token_to_str:N \ctexbibname'~to~the~given~value.
  }
\tl_if_exist:NTF \bibname
  { \keys_define:nn { ctex } { bibname .tl_set:N = \bibname } }
  {
    \tl_if_exist:NTF \refname
      { \keys_define:nn { ctex } { bibname .tl_set:N = \refname } }
      {
        \msg_warning:nn { ctex } { ctexbibname }
        \keys_define:nn { ctex } { bibname .tl_set:N = \ctexbibname }
      }
  }
%</style>
%    \end{macrocode}
%
%    \begin{macrocode}
%</class|style>
%<*class|heading>
%    \end{macrocode}
%
% \subsection{中文化的标题结构}
%
% \subsubsection{定义标题格式选项}
%
% \begin{variable}[internal]{\c_@@_headings_seq}
%    \begin{macrocode}
\seq_new:N \c_@@_headings_seq
\seq_gset_from_clist:Nn \c_@@_headings_seq
  {
%<article>    part , section , subsection , subsubsection ,
%<book|report>    part , chapter , section , subsection , subsubsection ,
    paragraph , subparagraph
  }
%    \end{macrocode}
% \end{variable}
%
% \begin{macro}[internal]{\@@_initial_heading:n}
%    \begin{macrocode}
\cs_new_protected_nopar:Npn \@@_initial_heading:n #1
  {
    \tl_new:c { CTEX@pre#1 }
    \tl_new:c { CTEX@post#1 }
    \tl_const:cx { CTEXthe#1 }
      {
        \exp_not:c { CTEX@pre#1 }
        \exp_not:c { CTEX@the#1 }
        \exp_not:c { CTEX@post#1 }
      }
    \tl_const:cx { CTEX@#1name }
      {
        \exp_not:c { CTEX@#1@nameformat }
        \exp_not:c { CTEX@pre#1 }
        \exp_not:N \tl_if_empty:NTF \exp_not:c { CTEX@#1@numberformat }
          { \exp_not:c { CTEX@the#1 } }
          {
            \group_begin:
              \exp_not:c { CTEX@#1@numberformat }
              \exp_not:c { CTEX@the#1 }
            \group_end:
          }
        \exp_not:c { CTEX@post#1 }
        \exp_not:c { CTEX@#1@aftername }
      }
  }
%    \end{macrocode}
% \end{macro}
%
% \begin{macro}[internal]{\@@_def_heading_keys:n}
%    \begin{macrocode}
\cs_new_protected_nopar:Npn \@@_def_heading_keys:n #1
  {
    \tl_put_right:Nx \l_@@_tmp_tl
      {
        #1                 .meta:nn = { ctex / #1 } { ####1 } ,
        #1 / name           .code:n =
          { \ctex_assign_heading_name:nn {#1} { ####1 } } ,
        #1 / number       .tl_set:N = \exp_not:c { CTEX@the#1 } ,
        #1 / format       .tl_set:N = \exp_not:c { CTEX@#1@format } ,
        #1 / nameformat   .tl_set:N = \exp_not:c { CTEX@#1@nameformat } ,
        #1 / numberformat .tl_set:N = \exp_not:c { CTEX@#1@numberformat } ,
        #1 / aftername    .tl_set:N = \exp_not:c { CTEX@#1@aftername } ,
        #1 / titleformat  .tl_set:N = \exp_not:c { CTEX@#1@titleformat } ,
        #1 / beforeskip   .tl_set:N = \exp_not:c { CTEX@#1@beforeskip } ,
        #1 / afterskip    .tl_set:N = \exp_not:c { CTEX@#1@afterskip} ,
        #1 / indent       .tl_set:N = \exp_not:c { CTEX@#1@indent } ,
        #1 / format+        .code:n =
          { \tl_put_right:Nn \exp_not:c { CTEX@#1@format } { ####1 } } ,
        #1 / nameformat+    .code:n =
          { \tl_put_right:Nn \exp_not:c { CTEX@#1@nameformat } { ####1 } } ,
        #1 / numberformat+  .code:n =
          { \tl_put_right:Nn \exp_not:c { CTEX@#1@numberformat } { ####1 } } ,
        #1 / aftername+     .code:n =
          { \tl_put_right:Nn \exp_not:c { CTEX@#1@aftername } { ####1 } } ,
        #1 / titleformat+   .code:n =
          { \tl_put_right:Nn \exp_not:c { CTEX@#1@titleformat } { ####1 } } ,
        #1 / beforeskip  .initial:n = \c_zero_skip ,
        #1 / afterskip   .initial:n = \c_zero_skip ,
        #1 / indent      .initial:n = \c_zero_dim ,
        #1 / beforeskip  .value_required: ,
        #1 / afterskip   .value_required: ,
        #1 / indent      .value_required: ,
      }
  }
%    \end{macrocode}
% \end{macro}
%
% \begin{macro}[internal]{\ctex_assign_heading_name:nn,\@@_assign_heading_name:nnn}
% \opt{name} 的值是一个至多两个元素的逗号分隔列表。由于 \LTXIII{} 的
% \texttt{clist} 总是会自动忽略空元素,所以设置 |name={,章}| 后,第一个元素将会
% 是“章”,必须用空的分组保护空元素:|name={{},章}|,这在使用中有些许不便。我们
% 可以改用 \texttt{seq} 或者手写函数解析参数来加以改进。为实现的简单起见,这里用
% 了 \pkg{xparse} 的 \tn{SplitArgument},它带有参数的长度检查。
%    \begin{macrocode}
\NewDocumentCommand \ctex_assign_heading_name:nn
  { m > { \SplitArgument { \c_one } { , } } +m }
  { \@@_assign_heading_name:nnn {#1} #2 }
\cs_new_protected:Npn \@@_assign_heading_name:nnn #1#2#3
  {
    \tl_set:cn { CTEX@pre#1 } {#2}
    \IfNoValueTF {#3}
      { \tl_clear:c { CTEX@post#1 } }
      { \tl_set:cn { CTEX@post#1 } {#3} }
  }
%    \end{macrocode}
% \end{macro}
%
%    \begin{macrocode}
\tl_clear:N \l_@@_tmp_tl
\seq_map_inline:Nn \c_@@_headings_seq
  {
    \@@_initial_heading:n {#1}
    \@@_def_heading_keys:n {#1}
  }
\use:x { \keys_define:nn { ctex } { \exp_not:o { \l_@@_tmp_tl } } }
\tl_clear:N \l_@@_tmp_tl
%    \end{macrocode}
%
% \changes{v2.0}{2014/03/21}{标题设置新增 \opt{pagestyle} 选项。}
%
% \begin{macro}{pagestyle}
% 只在 \cls{ctexbook} 和 \cls{ctexrep} 下有定义。
%    \begin{macrocode}
%<*book|report>
\keys_define:nn { ctex }
  {
    part    / pagestyle .tl_set:N = \CTEX@part@pagestyle ,
    chapter / pagestyle .tl_set:N = \CTEX@chapter@pagestyle
  }
%</book|report>
%    \end{macrocode}
% \end{macro}
%
%
% \subsubsection{标准标题命令的修改}
%
% \paragraph{part 的标题}
%
%    \begin{macrocode}
%<@@=>
%    \end{macrocode}
%
% \begin{macro}[internal]{\part}
%    \begin{macrocode}
%<*article>
\renewcommand\part{%
   \if@noskipsec \leavevmode \fi
   \par
%  \addvspace{4ex}%
   \@tempskipa \CTEX@part@beforeskip \relax
   \ifdim \@tempskipa <\z@
     \@tempskipa -\@tempskipa \@afterindentfalse
   \else
     \@afterindenttrue
   \fi
   \addvspace{\@tempskipa}%
   \secdef\@part\@spart}
%</article>
%    \end{macrocode}
% \end{macro}
%
% \begin{macro}[internal]{\@part}
%    \begin{macrocode}
%<*article>
\def\@part[#1]#2{%
  \ifnum \c@secnumdepth >\m@ne
    \refstepcounter{part}%
%   \addcontentsline{toc}{part}{\thepart\hspace{1em}#1}%
    \addcontentsline{toc}{part}{\CTEXthepart\hspace{1em}#1}%
  \else
    \addcontentsline{toc}{part}{#1}%
  \fi
  {\interlinepenalty \@M
%  \normalfont \parindent \z@ \raggedright
   \normalfont \parindent \CTEX@part@indent \CTEX@part@format
   \ifnum \c@secnumdepth >\m@ne
%    \Large\bfseries\partname\nobreakspace\thepart\par\nobreak
     \CTEX@partname
   \fi
%  \huge\bfseries #2%
   \CTEX@part@titleformat{#2}%
   \markboth{}{}\par}%
  \nobreak
% \vskip 3ex
  \vskip \CTEX@part@afterskip
  \@afterheading}
%</article>
%    \end{macrocode}
%
% 标准文档类是在 \tn{part} 和 \tn{chapter} 定义的最开始设置 \tn{thispagestyle},
% 我们这里的修改出现在它之后,可以覆盖之前的设置。
%    \begin{macrocode}
%<*book|report>
\def\@part[#1]#2{%
  \thispagestyle{\CTEX@part@pagestyle}%
  \ifnum \c@secnumdepth >-2\relax
    \refstepcounter{part}%
%   \addcontentsline{toc}{part}{\thepart\hspace{1em}#1}%
    \addcontentsline{toc}{part}{\CTEXthepart\hspace{1em}#1}%
  \else
    \addcontentsline{toc}{part}{#1}%
  \fi
  \markboth{}{}%
  {\interlinepenalty \@M
%  \normalfont \centering
   \normalfont \CTEX@part@format
   \ifnum \c@secnumdepth >-2\relax
%    \huge\bfseries\partname\nobreakspace\thepart\par\vskip 20\p@
     \CTEX@partname
   \fi
%  \Huge\bfseries #2\par}%
   \CTEX@part@titleformat{#2}\par}%
  \@endpart}
%</book|report>
%    \end{macrocode}
% \end{macro}
%
% \begin{macro}[internal]{\@spart}
%    \begin{macrocode}
%<*article>
\def\@spart#1{%
    {\interlinepenalty \@M
%    \normalfont \parindent \z@ \raggedright
     \normalfont \parindent \CTEX@part@indent \CTEX@part@format
%    \huge \bfseries #1\par}%
     \CTEX@part@titleformat{#1}\par}%
     \nobreak
%    \vskip 3ex
     \vskip \CTEX@part@afterskip
     \@afterheading}
%</article>
%<*book|report>
\def\@spart#1{%
    {\interlinepenalty \@M
%    \normalfont \centering
     \normalfont \CTEX@part@format
%    \Huge \bfseries #1\par}%
     \CTEX@part@titleformat{#1}\par}%
    \@endpart}
%</book|report>
%    \end{macrocode}
% \end{macro}
%
% \paragraph{chapter 的标题}
%
%    \begin{macrocode}
%<*book|report>
%    \end{macrocode}
%
% \begin{macro}[internal]{\@chapter}
%    \begin{macrocode}
\def\@chapter[#1]#2{%
  \ifnum \c@secnumdepth >\m@ne
%<book>    \if@mainmatter
      \refstepcounter{chapter}%
%     \typeout{\@chapapp\space\thechapter.}%
      \typeout{\CTEXthechapter}%
      \addcontentsline{toc}{chapter}
%       {\protect\numberline{\thechapter}#1}%
        {\protect\numberline{\CTEXthechapter\hspace{0.3em}}#1}%
%<book>    \else
%<book>      \addcontentsline{toc}{chapter}{#1}%
%<book>    \fi
  \else
    \addcontentsline{toc}{chapter}{#1}%
  \fi
  \chaptermark{#1}%
  \addtocontents{lof}{\protect\addvspace{10\p@}}%
  \addtocontents{lot}{\protect\addvspace{10\p@}}%
  \if@twocolumn
    \@topnewpage[\@makechapterhead{#2}]%
  \else
    \@makechapterhead{#2}%
  \@afterheading
  \fi}
%    \end{macrocode}
% \end{macro}
%
% \begin{macro}[internal]{\@makechapterhead}
%    \begin{macrocode}
\def\@makechapterhead#1{%
  \thispagestyle{\CTEX@chapter@pagestyle}%
% \vspace*{50\p@}%
  \@tempskipa \CTEX@chapter@beforeskip \relax
  \ifdim \@tempskipa <\z@
    \@tempskipa -\@tempskipa \@afterindentfalse
  \else
    \@afterindenttrue
  \fi
  \vspace*{\@tempskipa}%
% {\normalfont \parindent \z@ \raggedright
  {\normalfont \parindent \CTEX@chapter@indent \CTEX@chapter@format
   \ifnum \c@secnumdepth >\m@ne
%<book>     \if@mainmatter
%      \huge\bfseries\@chapapp\space\thechapter\par\nobreak\vskip 20\p@
       \CTEX@chaptername
%<book>     \fi
   \fi
   \interlinepenalty\@M
%  \Huge \bfseries #1\par\nobreak
   \CTEX@chapter@titleformat{#1}\par\nobreak
%  \vskip 40\p@
   \vskip \CTEX@chapter@afterskip
  }}
%    \end{macrocode}
% \end{macro}
%
% \begin{macro}[internal]{\@makeschapterhead}
%    \begin{macrocode}
\def\@makeschapterhead#1{%
  \thispagestyle{\CTEX@chapter@pagestyle}%
% \vspace*{50\p@}%
  \@tempskipa \CTEX@chapter@beforeskip \relax
  \ifdim \@tempskipa <\z@
    \@tempskipa -\@tempskipa \@afterindentfalse
  \else
    \@afterindenttrue
  \fi
  \vspace*{\@tempskipa}%
% {\normalfont \parindent \z@ \raggedright
  {\normalfont \parindent \CTEX@chapter@indent \CTEX@chapter@format
   \interlinepenalty\@M
%  \Huge \bfseries  #1\par\nobreak
   \CTEX@chapter@titleformat{#1}\par\nobreak
%  \vskip 40\p@
   \vskip \CTEX@chapter@afterskip
  }}
%    \end{macrocode}
% \end{macro}
%
%    \begin{macrocode}
%</book|report>
%    \end{macrocode}
%
% \paragraph{section 类的标题}
%
% \begin{macro}[internal]{\@seccntformat}
%    \begin{macrocode}
\def\@seccntformat#1{%
  \@ifundefined{CTEX@#1name}%
    {\csname the#1\endcsname\quad}%
    {\csname CTEX@#1name\endcsname}}
%    \end{macrocode}
% \end{macro}
%
% \begin{macro}[internal]{\@sect}
%    \begin{macrocode}
\def\@sect#1#2#3#4#5#6[#7]#8{%
  \ifnum #2>\c@secnumdepth
    \let\@svsec\@empty
  \else
    \refstepcounter{#1}%
    \protected@edef\@svsec{\@seccntformat{#1}\relax}%
  \fi
  \@tempskipa #5\relax
  \ifdim \@tempskipa>\z@
    \begingroup
      #6{%
        \@hangfrom{\hskip #3\relax\@svsec}%
%       \interlinepenalty \@M #8\@@par}%
        \interlinepenalty \@M
        \csname CTEX@#1@titleformat\endcsname{#8}\@@par}%
    \endgroup
    \csname #1mark\endcsname{#7}%
    \addcontentsline{toc}{#1}{%
      \ifnum #2>\c@secnumdepth \else
%       \protect\numberline{\csname the#1\endcsname}%
        \protect\numberline{\@ifundefined{CTEXthe#1}%
                              {\csname the#1\endcsname}%
                              {\csname CTEXthe#1\endcsname}}%
      \fi
      #7}%
  \else
    \def\@svsechd{%
    #6{\hskip #3\relax
%     \@svsec #8}%
      \@svsec \csname CTEX@#1@titleformat\endcsname{#8}}%
    \csname #1mark\endcsname{#7}%
    \addcontentsline{toc}{#1}{%
      \ifnum #2>\c@secnumdepth \else
%       \protect\numberline{\csname the#1\endcsname}%
        \protect\numberline{\@ifundefined{CTEXthe#1}%
                              {\csname the#1\endcsname}%
                              {\csname CTEXthe#1\endcsname}}%
      \fi
      #7}}%
  \fi
  \@xsect{#5}}
%    \end{macrocode}
% \end{macro}
%
% \begin{macro}[internal]{\@ssect}
% \tn{@ssect} 并没有参数给出当前标题的名字,扩展它的参数会与 \pkg{hyperref} 冲突。
% 它的第二个参数是 BEFORESKIP(\tn{@startsection} 的第四个参数),在定义中并没有
% 被用到,应该可以用它来传递名字。我们这里通过函数 \tn{CTEX@titleformat@n} 来传递,
% 它将在 |#4| 中被重定义为相应的 \opt{titleformat}。
%    \begin{macrocode}
\def\@ssect#1#2#3#4#5{%
  \@tempskipa #3\relax
  \ifdim \@tempskipa>\z@
    \begingroup
      #4{%
        \@hangfrom{\hskip #1}%
%         \interlinepenalty \@M #5\@@par}%
          \interlinepenalty \@M
          \CTEX@titleformat@n{#5}\@@par}%
    \endgroup
  \else
%   \def\@svsechd{#4{\hskip #1\relax #5}}%
    \def\@svsechd{#4{\hskip #1\relax \CTEX@titleformat@n{#5}}}%
  \fi
  \@xsect{#3}}
%    \end{macrocode}
% \end{macro}
%
%    \begin{macrocode}
%<@@=ctex>
%    \end{macrocode}
%
% \begin{macro}[internal]{\CTEX@set@titleformat@n, \CTEX@titleformat@n}
% 在 \tn{@startsection} 中设置 \tn{CTEX@titleformat@n} 为相应函数。
%    \begin{macrocode}
\cs_new_protected_nopar:Npn \CTEX@set@titleformat@n #1
  { \cs_set_eq:Nc \CTEX@titleformat@n { CTEX@#1@titleformat } }
\cs_new_eq:NN \CTEX@titleformat@n \use:n
%    \end{macrocode}
% \end{macro}
%
%    \begin{macrocode}
\int_zero:N \l_@@_tmp_int
\clist_map_inline:nn
  { section , subsection , subsubsection , paragraph , subparagraph }
  {
    \int_incr:N \l_@@_tmp_int
    \cs_gset_protected_nopar:cpx  {#1}
      {
        \exp_not:N \@startsection {#1}
          { \int_use:N \l_@@_tmp_int }
          { \exp_not:c { CTEX@#1@indent } }
          { \exp_not:c { CTEX@#1@beforeskip } }
          { \exp_not:c { CTEX@#1@afterskip } }
          {
            \CTEX@set@titleformat@n {#1}
            \exp_not:N \normalfont \exp_not:c { CTEX@#1@format }
          }
      }
  }
%    \end{macrocode}
%
%
% \paragraph{附录标题}
%
%    \begin{macrocode}
\tl_new:N \CTEX@preappendix
\tl_new:N \CTEX@postappendix
\keys_define:nn { ctex }
  { appendix .meta:nn = { ctex / appendix } {#1} }
\keys_define:nn { ctex / appendix }
  {
    name      .code:n = { \ctex_assign_heading_name:nn { appendix } {#1} } ,
    name   .initial:n = { \appendixname \space } ,
    number  .tl_set:N = \CTEX@appendixnumber ,
%<article>    number .initial:n = { \@Alph \c@section }
%<book|report>    number .initial:n = { \@Alph \c@chapter }
  }
%    \end{macrocode}
%
% \begin{macro}[internal]{\appendix}
%    \begin{macrocode}
\cs_new_eq:NN \CTEX@save@appendix \appendix
\cs_gset_protected_nopar:Npn \appendix
  {
    \CTEX@save@appendix
%<*article>
    \gdef \CTEX@presection { \CTEX@preappendix }
    \gdef \CTEX@thesection { \CTEX@appendixnumber }
    \gdef \CTEX@postsection { \CTEX@postappendix }
%</article>
%<*book|report>
    \gdef \CTEX@prechapter { \CTEX@preappendix }
    \gdef \CTEX@thechapter { \CTEX@appendixnumber }
    \gdef \CTEX@postchapter { \CTEX@postappendix }
%</book|report>
  }
%    \end{macrocode}
% \end{macro}
%
% \subsubsection{目录标签的宽度}
%
% \begin{macro}[internal]{\numberline}
%   \begin{macrocode}
\cs_new_protected:Npn \CTEX@toc@width@n #1
  {
    \hbox_set:Nn \l_@@_tmp_box {#1}
    \dim_set:Nn \@tempdima
      {
        \dim_max:nn { \@tempdima }
          { \box_wd:N \l_@@_tmp_box + \f@size \p@ / \c_two }
      }
  }
\group_begin:
\char_set_catcode_other:N \#
\use:n
  {
    \group_end:
    \ExplSyntaxOff
    \ctex_preto_cmd:NnTF \numberline { \CTEX@toc@width@n {#1} } { }
      { \ctex_patch_failure:N \numberline }
    \ExplSyntaxOn
    \AtBeginDocument
      {
        \@ifpackageloaded { tocloft }
          {
            \ctex_preto_cmd:NnTF \numberline { \CTEX@toc@width@n {#1} } { }
              { \ctex_patch_failure:N \numberline }
          } { }
      }
  }
%    \end{macrocode}
% \end{macro}
%
% \subsubsection{页眉信息的修改}
%
% \begin{macro}[internal]{\ps@headings}
%    \begin{macrocode}
%<*article>
\ctex_patch_cmd:Nnn \ps@headings { \thesection } { \CTEXthesection }
\if@twoside
  \ctex_patch_cmd:Nnn \ps@headings { \thesubsection } { \CTEXthesubsection }
\fi:
%</article>
%<*book|report>
\ctex_patch_cmd:Nnn \ps@headings
  { \@chapapp\ \thechapter.~\ } { \CTEXthechapter \quad }
\if@twoside
  \ctex_patch_cmd:Nnn \ps@headings { \thesection.~\ } { \CTEXthesection \quad }
\fi:
%</book|report>
%    \end{macrocode}
% \end{macro}
%
%    \begin{macrocode}
\pagestyle { headings }
%    \end{macrocode}
%
%    \begin{macrocode}
\bool_if:NT \l_@@_fancyhdr_bool
  { \RequirePackage { fancyhdr } }
%    \end{macrocode}
%
% \begin{macro}[internal]{\ps@fancy}
%    \begin{macrocode}
\if_cs_exist:N \ps@fancy
%<*article>
  \ctex_patch_cmd:Nnn \ps@fancy
    { \thesection \hskip 1em \relax } { \CTEXthesection \quad }
  \ctex_patch_cmd:Nnn \ps@fancy
    { \thesubsection \hskip 1em \relax } { \CTEXthesubsection \quad }
%</article>
%<*book|report>
  \ctex_patch_cmd:Nnn \ps@fancy { \@chapapp\ \thechapter.~\ }
%<book>    { \if@mainmatter \CTEXthechapter \quad \fi }
%<report>    { \CTEXthechapter \quad }
  \ctex_patch_cmd:Nnn \ps@fancy { \thesection.~\ } { \CTEXthesection \quad }
%</book|report>
\fi:
%    \end{macrocode}
% \end{macro}
%
% \subsubsection{标签引用数字的汉化}
%
% \begin{macro}[internal]{\refstepcounter}
% 对标题进行引用时,设置标签为通过 \opt{number} 选项设置的形式。
%    \begin{macrocode}
\cs_new_protected_nopar:Npn \CTEX@setcurrentlabel@n #1
  {
    \protected@edef \@currentlabel
      {
        \cs_if_exist:cTF { CTEX@the#1 }
          { \exp_args:cc { p@#1 } { CTEX@the#1 } }
          { \exp_not:o { \@currentlabel } }
      }
  }
%    \end{macrocode}
% \end{macro}
%
% \begin{macro}[internal]{\ctex_varioref_hook:}
% 关于标签引用的宏包可能会修改 \tn{refstepcounter}。其中 \pkg{cleveref} 和
% \pkg{hyperref} 宏包都会保存之前的定义,并且它们都要求尽可能晚的被载入,所以
% 对我们上述的修改影响不大。需要注意的是 \pkg{varioref} 宏包,如果它在
% \CTeX{} 之后被载入,我们之前的修改将会被覆盖。
%    \begin{macrocode}
\cs_new_protected_nopar:Npn \ctex_varioref_hook:
  {
    \seq_map_inline:Nn \c_@@_headings_seq
      { \ctex_fix_varioref_label:n { ##1 } }
  }
%    \end{macrocode}
% \end{macro}
%
% \begin{macro}[internal]{\@@_fix_varioref_label:n}
% \pkg{varioref} 宏包的 \tn{labelformat} 实际上是定义一个以 |\the<#1>| 为参数的宏
% |\p@<#1>|。\LaTeX{} 在定义计数器 |<#1>| 时,都会将 |\p@<#1>| 初始化为 \tn{@empty}。
% 如果这个宏非空,说明用户自定义了标签格式,我们就不再修改。这里不能使用
% \cs{exp_args:Nnc},因为 \texttt{c} 这种展开格式不会将参数放在花括号内。而
% \tn{labelformat} 的定义是
% \begin{verbatim}
%   \def\labelformat#1{\expandafter\def\csname p@#1\endcsname##1}
% \end{verbatim}
% 它的第二个参数必须放在花括号内,否则将会被作为宏的定界符号。
%    \begin{macrocode}
\cs_new_protected_nopar:Npn \ctex_fix_varioref_label:n #1
  {
    \tl_if_empty:cT { p@#1 }
      { \exp_args:Nno \labelformat {#1} { \cs:w CTEX@the#1 \cs_end: } }
  }
%    \end{macrocode}
% \end{macro}
%
% 如果 \pkg{varioref} 已经被载入,则使用它来设置。
%    \begin{macrocode}
\if_bool:N \l_@@_caption_bool
  \@ifpackageloaded { varioref }
    { \ctex_varioref_hook: }
    {
      \cs_new_eq:NN \CTEX@save@refstepcounter \refstepcounter
      \RenewDocumentCommand \refstepcounter { m }
        {
          \CTEX@save@refstepcounter {#1}
          \CTEX@setcurrentlabel@n {#1}
        }
      \AtBeginDocument
        { \@ifpackageloaded { varioref } { \ctex_varioref_hook: } { } }
    }
\fi:
%    \end{macrocode}
%
% \subsubsection{模拟标准文档类格式}
%
% 下面使用 \CTeX 文档类的设置方式,重新模拟标准文档类直接定义或以
% \tn{@startsection} 设定的章节标题格式。
%
%    \begin{macrocode}
\keys_set:nn { ctex / part }
  {
    name        = \partname \space ,
    number      = \thepart ,
%<*article>
    format      = \raggedright ,
    nameformat  = \Large \bfseries ,
    aftername   = \par \nobreak ,
    titleformat = \huge \bfseries ,
    beforeskip  = -4ex ,
    afterskip   = 3ex
%</article>
%<*book|report>
    format      = \centering ,
    nameformat  = \huge \bfseries ,
    aftername   = \par \vskip 20 \p@ ,
    titleformat = \Huge \bfseries ,
    pagestyle   = plain
%</book|report>
  }
%    \end{macrocode}
%
%    \begin{macrocode}
%<*book|report>
\keys_set:nn { ctex / chapter }
  {
    name        = \chaptername \space ,
    number      = \thechapter ,
    format      = \raggedright ,
    nameformat  = \huge \bfseries ,
    aftername   = \par \nobreak \vskip 20 \p@ ,
    titleformat = \Huge \bfseries ,
    beforeskip  = -50 \p@ ,
    afterskip   = 40 \p@ ,
    pagestyle   = plain
  }
%</book|report>
%    \end{macrocode}
%
%    \begin{macrocode}
\keys_set:nn { ctex / section }
  {
    number      = \thesection ,
    format      = \Large \bfseries ,
    aftername   = \quad ,
    beforeskip  = -3.5ex \@plus -1ex \@minus -.2ex ,
    afterskip   = 2.3ex \@plus .2ex
  }
%    \end{macrocode}
%
%    \begin{macrocode}
\keys_set:nn { ctex / subsection }
  {
    number      = \thesubsection ,
    format      = \large \bfseries ,
    aftername   = \quad ,
    beforeskip  = -3.25ex \@plus -1ex \@minus -.2ex ,
    afterskip   = 1.5ex  \@plus .2ex
  }
%    \end{macrocode}
%
%    \begin{macrocode}
\keys_set:nn { ctex / subsubsection }
  {
    number      = \thesubsubsection ,
    format      = \normalsize \bfseries ,
    aftername   = \quad ,
    beforeskip  = -3.25ex \@plus -1ex \@minus -.2ex ,
    afterskip   = 1.5ex \@plus .2ex
  }
%    \end{macrocode}
%
%    \begin{macrocode}
\keys_set:nn { ctex / paragraph }
  {
    number      = \theparagraph ,
    format      = \normalsize \bfseries ,
    aftername   = \quad ,
    beforeskip  = 3.25ex \@plus 1ex \@minus .2ex ,
    afterskip   = -1em
  }
%    \end{macrocode}
%
%    \begin{macrocode}
\keys_set:nn { ctex / subparagraph }
  {
    number      = \thesubparagraph ,
    format      = \normalsize \bfseries ,
    aftername   = \quad ,
    beforeskip  = 3.25ex \@plus 1ex \@minus .2ex ,
    afterskip   = -1em ,
    indent      = \parindent
  }
%    \end{macrocode}
%
% 处理 \opt{sub3section} 与 \opt{sub4section} 的格式。
%    \begin{macrocode}
\int_compare:nNnT \g_@@_section_depth_flag > \c_two
  {
    \keys_set:nn { ctex / paragraph }
      {
        beforeskip = -3.25ex \@plus 1ex \@minus .2ex,
        afterskip = 1ex \@plus .2ex
      }
  }
\int_compare:nNnT \g_@@_section_depth_flag > \c_three
  {
    \keys_set:nn { ctex / subparagraph }
      {
        beforeskip = -3.25ex \@plus 1ex \@minus .2ex,
        afterskip = 1ex \@plus .2ex
      }
  }
\int_compare:nNnT \g_@@_section_depth_flag > \c_two
  {
    \keys_set:nn { ctex / subparagraph }
      { indent = \z@ }
  }
%    \end{macrocode}
%
% 处理附录的格式。
%    \begin{macrocode}
%<article>\keys_set:nn { ctex / appendix } { name = { } }
%    \end{macrocode}
%
% \subsubsection{汉化默认标题格式}
%
% 在 \opt{cap} 选项为 |true| 时,设置中文化的标题格式。
%    \begin{macrocode}
\if_bool:N \l_@@_caption_bool
  \keys_set:nn { ctex / part }
    {
      number      = \chinese { part } ,
%<*article>
      format      = \centering ,
      aftername   = \quad ,
      titleformat = \Large\bfseries ,
      beforeskip  = 4ex
%</article>
%<*book|report>
      titleformat = \huge \bfseries
%</book|report>
    }
%<*book|report>
  \keys_set:nn { ctex / chapter }
    {
      number      = \chinese { chapter } ,
      format      = \centering ,
      aftername   = \quad ,
      titleformat = \huge \bfseries ,
      beforeskip  = 50\p@
    }
%</book|report>
  \keys_set:nn { ctex / section }
    {
      format     = \Large \bfseries \centering ,
      beforeskip = 3.5ex \@plus 1ex \@minus .2ex
    }
  \keys_set:nn { ctex / subsection }
    { beforeskip = 3.25ex \@plus 1ex \@minus .2ex }
  \keys_set:nn { ctex / subsubsection }
    { beforeskip = 3.25ex \@plus 1ex \@minus .2ex }
  \keys_set:nn { ctex / paragraph }
    { beforeskip = 3.25ex \@plus 1ex \@minus .2ex }
  \keys_set:nn { ctex / subparagraph }
    { beforeskip = 3.25ex \@plus 1ex \@minus .2ex }
  \str_if_eq:onTF { \l_@@_encoding_tl } { GBK }
    { \ctex_file_input:n { ctexcap-gbk.cfg } }
    { \ctex_file_input:n { ctexcap-utf8.cfg } }
\fi:
%    \end{macrocode}
%
%    \begin{macrocode}
%</class|heading>
%<*style>
%    \end{macrocode}
%
% \subsubsection{\pkg{ctex.sty} 的 \opt{heading} 选项}
%
%    \begin{macrocode}
\msg_new:nnn { ctex } { not-standard-class }
  {
    None~of~the~standard~document~classes~was~loaded.\\
    ctex~may~not~work~as~expected.
  }
\bool_if:NTF \l_@@_heading_bool
  {
    \clist_map_inline:nn { article , book , report }
      {
        \@ifclassloaded {#1}
          { \clist_map_break:n { \tl_const:Nn \c_@@_class_tl {#1} } } { }
      }
    \tl_if_exist:NTF \c_@@_class_tl
      { \ctex_file_input:n { ctex- \c_@@_class_tl .def } }
      {
        \msg_warning:nn { ctex } { not-standard-class }
        \cs_if_exist:NTF \chapter
          {
            \cs_if_exist:NF \if@mainmatter
              { \cs_new_eq:NN \if@mainmatter \tex_iftrue:D }
            \ctex_file_input:n { ctex-book.def }
          }
          { \ctex_file_input:n { ctex-article.def } }
      }
  }
  {
    \bool_if:NT \l_@@_caption_bool
      {
        \str_if_eq:onTF { \l_@@_encoding_tl } { GBK }
          { \ctex_file_input:n { ctexcap-gbk.cfg } }
          { \ctex_file_input:n { ctexcap-utf8.cfg } }
      }
  }
%    \end{macrocode}
%
%    \begin{macrocode}
%</style>
%<*UTF8|GBK>
%    \end{macrocode}
%
% \subsubsection{标题配置文件}
%
%    \begin{macrocode}
\keys_set:nn { ctex }
  {
    contentsname   = 目录 ,
    listfigurename = 插图 ,
    listtablename  = 表格 ,
    figurename     = 图 ,
    tablename      = 表 ,
    abstractname   = 摘要 ,
    indexname      = 索引 ,
    bibname        = 参考文献 ,
    appendixname   = 附录
  }
%    \end{macrocode}
%
%    \begin{macrocode}
\keys_if_exist:nnT { ctex / part } { name }
  {
    \keys_set:nn { ctex / part } { name = { 第 , 部分 } }
    \keys_if_exist:nnT { ctex / chapter } { name }
      { \keys_set:nn { ctex / chapter } { name = { 第 , 章 } } }
  }
%    \end{macrocode}
%
%    \begin{macrocode}
%</UTF8|GBK>
%    \end{macrocode}
%
% \subsection{其它功能}
%
% \begin{macro}{\CTeX}
% \file{ctex-faq.sty} 中的定义是
% \begin{verbatim}
%   \DeclareRobustCommand\CTeX{$\mathbb{C}$\kern-.05em\TeX}
% \end{verbatim}
% 然而 \tn{mathbb} 未必有定义,这里就不采用它了,只定义最简单的形式。
% 同 \pkg{hologo} 宏包的设置类似,\CTeX{} 可以用在 \tn{csname} 和 PDF 书签中。
%    \begin{macrocode}
%<*class|style>
\NewDocumentCommand \CTeX { }
  { \ifincsname CTeX \else: C \TeX \fi: }
\AtBeginDocument
  {
    \cs_if_exist_use:NT \pdfstringdefDisableCommands
      { { \tl_set:Nn \CTeX { CTeX } } }
  }
%</class|style>
%    \end{macrocode}
% \end{macro}
%
% \changes{v2.0}{2014/03/28}{\opt{captiondelimiter} 是过时选项。}
% \begin{macro}[internal]{captiondelimiter}
% 过时选项。
%    \begin{macrocode}
%<*class|style>
\keys_define:nn { ctex }
  {
    captiondelimiter .code:n =
      {
        \msg_warning:nnn { ctex } { deprecated-option }
          { You~can~load~the~package~`caption'~to~get~its~functionality. }
      }
  }
%</class|style>
%    \end{macrocode}
% \end{macro}
%
% \subsubsection{列表环境的缩进}
%
% \begin{macro}[internal]{\verse,\quotation}
% 只在使用文档类的时候修改诗歌和引用环境的缩进。
%    \begin{macrocode}
%<*class|heading>
\ctex_patch_cmd:Nnn \verse { -1.5em } { -2 \ccwd }
\ctex_patch_cmd:Nnn \verse {  1.5em } {  2 \ccwd }
\ctex_patch_cmd:Nnn \quotation { 1.5em } { 2 \ccwd }
%</class|heading>
%    \end{macrocode}
% \end{macro}
%
%    \begin{macrocode}
%<*class|style>
%    \end{macrocode}
%
% \subsubsection{其它兼容性修改}
%
% \begin{macro}[internal]{\end}
% \changes{v2.0}{2014/03/09}
% {解决 \pkg{etoolbox} 与 \pkg{breqn} 关于 \tn{end} 的冲突。}
% \pkg{breqn} 宏包对 \tn{end} 作了如下处理,然而这个处理并不保险。
% \begin{verbatim}
%   \def\@tempa#1\endcsname#2\@nil{\def\latex@end##1{#2}}
%   \expandafter\@tempa\end{#1}\@nil
%   \def\end#1{\csname end#1\endcsname \latex@end{#1}}%
% \end{verbatim}
% \pkg{etoolbox} 在 \tn{end} 定义中的 \tn{csname} 前加入
% 钩子 |\csuse{@end@#1@hook}|。如果 \pkg{etoolbox} 先于 \pkg{breqn} 被载入(这
% 在使用 \cls{ctexart} 等文档类时几乎是必然的),|\csuse{@end@#1@hook}| 将会被
% 忽略,即 \tn{AtEndEnvironment} 失效。如果交换两个宏包的载入顺序,则
% \pkg{etoolbox} 会给出警告:\tn{AfterEndEnvironment} 失效,我们不打算处理这种
% 情况。我们通过一个特殊的环境来完成检查。
%    \begin{macrocode}
\newenvironment { @@_test_env }
  { \bool_gset_false:N \g_@@_tmp_bool } { }
\AtEndEnvironment { @@_test_env }
  { \bool_gset_true:N \g_@@_tmp_bool }
\group_begin:
\char_set_catcode_other:N \#
\cs_new_protected_nopar:Npn \ctex_fix_end_env_hook:
  {
    \begin { @@_test_env } \end { @@_test_env }
    \bool_if:NF \g_@@_tmp_bool
      {
        \ctex_patch_cmd:NnnTF \end { \csname end#1 \endcsname }
          {
            \csuse { @end@#1@hook }
            \csname end#1 \endcsname
          } { }
          {
            \ctex_preto_cmd:NnTF \end { \csuse { @end@#1@hook } }
              { } { \ctex_patch_failure:N \end }
          }
      }
  }
\group_end:
\AtBeginDocument { \ctex_fix_end_env_hook: }
%    \end{macrocode}
% \end{macro}
%
% \subsection{载入中文字体}
%
% \begin{macro}[internal]{\ctex_fontset_error:n}
% 字库不可用时给出紧急错误信息,停止读取定义文件。
%    \begin{macrocode}
\cs_new_protected_nopar:Npn \ctex_fontset_error:n #1
  { \msg_critical:nnn { ctex } { fontset-unavailable } {#1} }
\msg_new:nnn { ctex } { fontset-unavailable }
  { CTeX~fontset~`#1'~is~unavailable~in~current~mode. }
%    \end{macrocode}
% \end{macro}
%
% \begin{macro}[internal]{\ctex_load_fontset:}
% 如果用户没有指定字体,则探测操作系统,载入相应的字体配置。
%    \begin{macrocode}
\cs_new_protected_nopar:Npn \ctex_load_fontset:
  {
    \tl_if_empty:NTF \g_@@_fontset_tl
      {
        \ctex_detected_platform:
        \ctex_file_input:n { ctex-fontset- \g_@@_fontset_tl .def }
      }
      {
        \file_if_exist:nTF { ctex-fontset- \g_@@_fontset_tl .def }
          { \ctex_file_input:n { ctex-fontset- \g_@@_fontset_tl .def } }
          {
            \use:x
              {
                \ctex_detected_platform:
                \msg_error:nnxx { ctex } { fontset-not-found }
                  { \g_@@_fontset_tl } { \exp_not:N \g_@@_fontset_tl }
              }
            \ctex_file_input:n { ctex-fontset- \g_@@_fontset_tl .def }
          }
      }
  }
\@onlypreamble \ctex_load_fontset:
\msg_new:nnnn { ctex } { fontset-not-found }
  {
    CTeX~fontset~`#1'~could~not~be~found.\\
    Fontset~`#2'~will~be~used~instead.
  }
  { You~may~run~`mktexlsr'~firstly. }
%    \end{macrocode}
% \end{macro}
%
% \begin{macro}{fontset}
% 在导言区通过 \tn{ctexset} 载入中文字库的选项。
%    \begin{macrocode}
\keys_define:nn { ctex }
  {
    fontset .code:n =
      {
        \ctex_if_preamble:TF
          {
            \str_if_eq_x:nnTF {#1} { none }
              { \msg_warning:nnn { ctex } { invalid-value } {#1} }
              {
                \str_if_eq:onTF { \g_@@_fontset_tl } { none }
                  {
                    \tl_gset:Nx \g_@@_fontset_tl {#1}
                    \ctex_load_fontset:
                  }
                  {
                    \msg_error:nnxx { ctex } { fontset-loaded }
                      { \g_@@_fontset_tl } {#1}
                  }
              }
          }
          { \msg_error:nn { ctex } { fontset-only-preamble } }
      }
  }
\msg_new:nnnn { ctex } { fontset-loaded }
  {
    CTeX~fontset~`#1'~has~been~loaded.
    \str_if_eq:nnF {#1} {#2} { \\ Fontset~`#2'~will~be~ignored. }
  }
  { Only~one~fontset~can~be~loaded~in~the~preamble. }
\msg_new:nnn { ctex } { fontset-only-preamble }
  {
    The~ `fontset'~ option~ can~ be~ used~ only~ in~ preamble.
  }
%    \end{macrocode}
% \end{macro}
%
% 载入中文字库。
%    \begin{macrocode}
\str_if_eq:onF { \g_@@_fontset_tl } { none }
  { \ctex_load_fontset: }
%    \end{macrocode}
%
% \subsection{宏包配置文件}
%
% \subsubsection{\pkg{ctex.cfg}}
%
%    \begin{macrocode}
\ctex_at_end:n { \ctex_file_input:n { ctex.cfg } }
%    \end{macrocode}
%
%    \begin{macrocode}
%</class|style>
%    \end{macrocode}
%
%    \begin{macrocode}
%<*config>
%%
%</config>
%    \end{macrocode}
%
% \subsubsection{\pkg{ctexopts.cfg}}
%
% 这里仅为配置文件示例:使用 Windows Vista 或以后版本的字体设置。
%    \begin{macrocode}
%<*ctexopts>
%%
%% \keys_set:nn { ctex / option } { fontset = windowsnew }
%</ctexopts>
%    \end{macrocode}
%
% \subsection{字体定义文件}
%
% \subsubsection{传统定义方式}
%
%    \begin{macrocode}
%<*c19|c70>
%%
%% Chinese characters
%%
%<c19>%% character set: GBK (extension of GB 2312)
%<c70>%% character set: Unicode
%% font encoding: Unicode
%%
%</c19|c70>
%    \end{macrocode}
%
%    \begin{macrocode}
%<rm&c19>\DeclareFontFamily{C19}{rm}{\hyphenchar\font\m@ne}
%<rm&c70>\DeclareFontFamily{C70}{rm}{\hyphenchar\font\m@ne}
%<sf&c19>\DeclareFontFamily{C19}{sf}{\hyphenchar\font\m@ne}
%<sf&c70>\DeclareFontFamily{C70}{sf}{\hyphenchar\font\m@ne}
%<tt&c19>\DeclareFontFamily{C19}{tt}{\hyphenchar\font\m@ne}
%<tt&c70>\DeclareFontFamily{C70}{tt}{\hyphenchar\font\m@ne}
%    \end{macrocode}
%
%    \begin{macrocode}
%<*rm>
%<*c19>
\DeclareFontShape{C19}{rm}{m}{n}{<-> CJK * gbksong}{\CJKnormal}
\DeclareFontShape{C19}{rm}{b}{n}{<-> CJK * gbkhei}{\CJKnormal}
\DeclareFontShape{C19}{rm}{bx}{n}{<-> CJK * gbkhei}{\CJKnormal}
\DeclareFontShape{C19}{rm}{m}{sl}{<-> CJK * gbksongsl}{\CJKnormal}
\DeclareFontShape{C19}{rm}{b}{sl}{<-> CJK * gbkheisl}{\CJKnormal}
\DeclareFontShape{C19}{rm}{bx}{sl}{<-> CJK * gbkheisl}{\CJKnormal}
\DeclareFontShape{C19}{rm}{m}{it}{<-> CJK * gbkkai}{\CJKnormal}
\DeclareFontShape{C19}{rm}{b}{it}{<-> CJKb * gbkkai}{\CJKbold}
\DeclareFontShape{C19}{rm}{bx}{it}{<-> CJKb * gbkkai}{\CJKbold}
%</c19>
%<*c70>
\DeclareFontShape{C70}{rm}{m}{n}{<-> CJK * unisong}{\CJKnormal}
\DeclareFontShape{C70}{rm}{b}{n}{<-> CJK * unihei}{\CJKnormal}
\DeclareFontShape{C70}{rm}{bx}{n}{<-> CJK * unihei}{\CJKnormal}
\DeclareFontShape{C70}{rm}{m}{sl}{<-> CJK * unisongsl}{\CJKnormal}
\DeclareFontShape{C70}{rm}{b}{sl}{<-> CJK * uniheisl}{\CJKnormal}
\DeclareFontShape{C70}{rm}{bx}{sl}{<-> CJK * uniheisl}{\CJKnormal}
\DeclareFontShape{C70}{rm}{m}{it}{<-> CJK * unikai}{\CJKnormal}
\DeclareFontShape{C70}{rm}{b}{it}{<-> CJKb * unikai}{\CJKbold}
\DeclareFontShape{C70}{rm}{bx}{it}{<-> CJKb * unikai}{\CJKbold}
%</c70>
%</rm>
%    \end{macrocode}
%
%    \begin{macrocode}
%<*sf>
%<*c19>
\DeclareFontShape{C19}{sf}{m}{n}{<-> CJK * gbkyou}{\CJKnormal}
\DeclareFontShape{C19}{sf}{b}{n}{<-> CJKb * gbkyou}{\CJKbold}
\DeclareFontShape{C19}{sf}{bx}{n}{<-> CJKb * gbkyou}{\CJKbold}
\DeclareFontShape{C19}{sf}{m}{sl}{<-> CJK * gbkyousl}{\CJKnormal}
\DeclareFontShape{C19}{sf}{b}{sl}{<-> CJKb * gbkyousl}{\CJKbold}
\DeclareFontShape{C19}{sf}{bx}{sl}{<-> CJKb * gbkyousl}{\CJKbold}
\DeclareFontShape{C19}{sf}{m}{it}{<-> CJK * gbkyou}{\CJKnormal}
\DeclareFontShape{C19}{sf}{b}{it}{<-> CJKb * gbkyou}{\CJKbold}
\DeclareFontShape{C19}{sf}{bx}{it}{<-> CJKb * gbkyou}{\CJKbold}
%</c19>
%<*c70>
\DeclareFontShape{C70}{sf}{m}{n}{<-> CJK * uniyou}{\CJKnormal}
\DeclareFontShape{C70}{sf}{b}{n}{<-> CJKb * uniyou}{\CJKbold}
\DeclareFontShape{C70}{sf}{bx}{n}{<-> CJKb * uniyou}{\CJKbold}
\DeclareFontShape{C70}{sf}{m}{sl}{<-> CJK * uniyousl}{\CJKnormal}
\DeclareFontShape{C70}{sf}{b}{sl}{<-> CJKb * uniyousl}{\CJKbold}
\DeclareFontShape{C70}{sf}{bx}{sl}{<-> CJKb * uniyousl}{\CJKbold}
\DeclareFontShape{C70}{sf}{m}{it}{<-> CJK * uniyou}{\CJKnormal}
\DeclareFontShape{C70}{sf}{b}{it}{<-> CJKb * uniyou}{\CJKbold}
\DeclareFontShape{C70}{sf}{bx}{it}{<-> CJKb * uniyou}{\CJKbold}
%</c70>
%</sf>
%    \end{macrocode}
%
%    \begin{macrocode}
%<*tt>
%<*c19>
\DeclareFontShape{C19}{tt}{m}{n}{<-> CJK * gbkfs}{\CJKnormal}
\DeclareFontShape{C19}{tt}{b}{n}{<-> CJKb * gbkfs}{\CJKbold}
\DeclareFontShape{C19}{tt}{bx}{n}{<-> CJKb * gbkfs}{\CJKbold}
\DeclareFontShape{C19}{tt}{m}{sl}{<-> CJK * gbkfssl}{\CJKnormal}
\DeclareFontShape{C19}{tt}{b}{sl}{<-> CJKb * gbkfssl}{\CJKbold}
\DeclareFontShape{C19}{tt}{bx}{sl}{<-> CJKb * gbkfssl}{\CJKbold}
\DeclareFontShape{C19}{tt}{m}{it}{<-> CJK * gbkfs}{\CJKnormal}
\DeclareFontShape{C19}{tt}{b}{it}{<-> CJKb * gbkfs}{\CJKbold}
\DeclareFontShape{C19}{tt}{bx}{it}{<-> CJKb * gbkfs}{\CJKbold}
%</c19>
%<*c70>
\DeclareFontShape{C70}{tt}{m}{n}{<-> CJK * unifs}{\CJKnormal}
\DeclareFontShape{C70}{tt}{b}{n}{<-> CJKb * unifs}{\CJKbold}
\DeclareFontShape{C70}{tt}{bx}{n}{<-> CJKb * unifs}{\CJKbold}
\DeclareFontShape{C70}{tt}{m}{sl}{<-> CJK * unifssl}{\CJKnormal}
\DeclareFontShape{C70}{tt}{b}{sl}{<-> CJKb * unifssl}{\CJKbold}
\DeclareFontShape{C70}{tt}{bx}{sl}{<-> CJKb * unifssl}{\CJKbold}
\DeclareFontShape{C70}{tt}{m}{it}{<-> CJK * unifs}{\CJKnormal}
\DeclareFontShape{C70}{tt}{b}{it}{<-> CJKb * unifs}{\CJKbold}
\DeclareFontShape{C70}{tt}{bx}{it}{<-> CJKb * unifs}{\CJKbold}
%</c70>
%</tt>
%    \end{macrocode}
%
%    \begin{macrocode}
%<*fontset>
%    \end{macrocode}
%
% \subsubsection{\pkg{ctex-fontset-windows.def},
% \pkg{ctex-fontset-windowsnew.def}, \pkg{ctex-fontset-windowsold.def}}
%
% \pkg{ctex} 包利用 \path{C:\boot.ini} 文件的存在性来判断是否使用 Windows XP
% 及以前的版本,分别载入新旧字体设置。
%    \begin{macrocode}
%<*windows>
\file_if_exist:nTF { C:/boot.ini }
  { \ctex_file_input:n { ctex-fontset-windowsold.def } }
  { \ctex_file_input:n { ctex-fontset-windowsnew.def } }
%</windows>
%    \end{macrocode}
%
% 旧的 Windows 字体设置使用黑体作为无衬线体,楷体和仿宋是 GB2312 编码;新的
% Windows 字体设置使用微软雅黑作为无衬线体,楷体和仿宋是大字库。
%    \begin{macrocode}
%<*windowsnew,windowsold>
\pdftex_if_engine:TF
  {
    \ctex_zhmap_case:nnn
      {
        \ctex_punct_set:n { windows }
        \setCJKmainfont
          [ BoldFont = simhei.ttf , ItalicFont = simkai.ttf ] { simsun.ttc }
%<*windowsold>
        \setCJKsansfont { simhei.ttf }
        \ctex_punct_map_family:nn { \CJKsfdefault } { zhhei }
%</windowsold>
%    \end{macrocode}
% Windows 8 以后,微软雅黑由原来的 \file{.ttf} 后缀改为 \file{.ttc} 后缀,需要
% 加以区分。
%    \begin{macrocode}
%<*windowsnew>
        \file_if_exist:nTF { C:/Windows/Fonts/msyh.ttc }
          {
            \setCJKsansfont [ BoldFont = msyhbd.ttc ] { msyh.ttc }
            \setCJKfamilyfont { zhyahei }
              [ BoldFont = msyhbd.ttc ] { msyh.ttc }
          }
          {
            \setCJKsansfont [ BoldFont = msyhbd.ttf ] { msyh.ttf }
            \setCJKfamilyfont { zhyahei }
              [ BoldFont = msyhbd.ttf ] { msyh.ttf }
          }
        \ctex_punct_map_family:nn { \CJKsfdefault } { zhyahei }
        \ctex_punct_map_bfseries:nn { \CJKsfdefault , zhyahei } { zhyaheib }
%</windowsnew>
        \setCJKmonofont { simfang.ttf }
        \setCJKfamilyfont { zhkai }  { simkai.ttf }
        \setCJKfamilyfont { zhfs }   { simfang.ttf }
        \setCJKfamilyfont { zhsong } { simsun.ttc }
        \setCJKfamilyfont { zhhei }  { simhei.ttf }
        \setCJKfamilyfont { zhli }   { simli.ttf }
        \setCJKfamilyfont { zhyou }  { simyou.ttf }
        \ctex_punct_map_family:nn { \CJKrmdefault } { zhsong }
        \ctex_punct_map_family:nn { \CJKttdefault } { zhfs }
        \ctex_punct_map_itshape:nn { \CJKrmdefault } { zhkai }
        \ctex_punct_map_bfseries:nn { \CJKrmdefault } { zhhei }
      }
      {
        \ctex_load_zhmap:nnnn { rm } { zhhei } { zhfs } { zhwinfonts }
        \ctex_punct_set:n { windows }
        \ctex_punct_map_family:nn { \CJKrmdefault } { zhsong }
        \ctex_punct_map_bfseries:nn { \CJKrmdefault } { zhhei }
        \ctex_punct_map_itshape:nn { \CJKrmdefault } { zhkai }
      }
      {
        \tl_set:Nn \CJKrmdefault { rm }
        \tl_set:Nn \CJKsfdefault { sf }
        \tl_set:Nn \CJKttdefault { tt }
      }
  }
  {
%<*windowsold>
    \setCJKmainfont
      [ BoldFont = SimHei , ItalicFont = KaiTi_GB2312 ] { SimSun }
    \setCJKsansfont { SimHei }
    \setCJKmonofont { FangSong_GB2312 }
    \setCJKfamilyfont { zhkai } { KaiTi_GB2312 }
    \setCJKfamilyfont { zhfs }  { FangSong_GB2312 }
%</windowsold>
%<*windowsnew>
    \setCJKmainfont
      [ BoldFont = SimHei , ItalicFont = KaiTi ] { SimSun }
    \setCJKsansfont
      [ BoldFont = { *~Bold } ] { Microsoft~YaHei }
    \setCJKmonofont { FangSong }
    \setCJKfamilyfont { zhkai } { KaiTi }
    \setCJKfamilyfont { zhfs }  { FangSong }
%</windowsnew>
    \setCJKfamilyfont { zhsong }  { SimSun }
    \setCJKfamilyfont { zhhei }   { SimHei }
    \setCJKfamilyfont { zhli }    { LiSu }
    \setCJKfamilyfont { zhyou }   { YouYuan }
    \setCJKfamilyfont { zhyahei }
      [ BoldFont = { *~Bold } ] { Microsoft~YaHei }
  }
%</windowsnew,windowsold>
%    \end{macrocode}
%
% \subsubsection{\pkg{ctex-fontset-adobe.def}}
%
%    \begin{macrocode}
%<*adobe>
%    \end{macrocode}
%
% \tn{pdfmapline} 不支持 OpenType 字体,因而 \opt{adobe} 字体集在 pdf 模式下
% 就没有定义。\opt{fandol} 的情况类似。
%    \begin{macrocode}
\pdftex_if_engine:TF
  {
    \ctex_if_pdfmode:TF
      { \ctex_fontset_error:n { adobe } }
      {
        \ctex_zhmap_case:nnn
          {
            \setCJKmainfont
              [
                      cmap = UniGB-UTF16-H ,
                  BoldFont = AdobeHeitiStd-Regular.otf ,
                ItalicFont = AdobeKaitiStd-Regular.otf
              ] { AdobeSongStd-Light.otf }
            \setCJKsansfont [ cmap = UniGB-UTF16-H ] { AdobeHeitiStd-Regular.otf }
            \setCJKmonofont [ cmap = UniGB-UTF16-H ] { AdobeFangsongStd-Regular.otf }
            \setCJKfamilyfont { zhsong }
              [ cmap = UniGB-UTF16-H ] { AdobeSongStd-Light.otf }
            \setCJKfamilyfont { zhhei }
              [ cmap = UniGB-UTF16-H ] { AdobeHeitiStd-Regular.otf }
            \setCJKfamilyfont { zhkai }
              [ cmap = UniGB-UTF16-H ] { AdobeKaitiStd-Regular.otf }
            \setCJKfamilyfont { zhfs }
              [ cmap = UniGB-UTF16-H ] { AdobeFangsongStd-Regular.otf }
            \ctex_punct_set:n { adobe }
            \ctex_punct_map_family:nn { \CJKrmdefault } { zhsong }
            \ctex_punct_map_family:nn { \CJKsfdefault } { zhhei }
            \ctex_punct_map_family:nn { \CJKttdefault } { zhfs }
            \ctex_punct_map_itshape:nn { \CJKrmdefault } { zhkai }
            \ctex_punct_map_bfseries:nn { \CJKrmdefault } { zhhei }
          }
          {
            \ctex_load_zhmap:nnnn { rm } { zhhei } { zhfs } { zhadobefonts }
            \ctex_punct_set:n { adobe }
            \ctex_punct_map_family:nn { \CJKrmdefault } { zhsong }
            \ctex_punct_map_bfseries:nn { \CJKrmdefault } { zhhei }
            \ctex_punct_map_itshape:nn { \CJKrmdefault } { zhkai }
          }
          { \ctex_fontset_error:n { adobe } }
      }
  }
  {
    \setCJKmainfont
      [
          BoldFont = AdobeHeitiStd-Regular ,
        ItalicFont = AdobeKaitiStd-Regular
      ] { AdobeSongStd-Light }
    \setCJKsansfont { AdobeHeitiStd-Regular}
    \setCJKmonofont { AdobeFangsongStd-Regular}
    \setCJKfamilyfont { zhsong } { AdobeSongStd-Light }
    \setCJKfamilyfont { zhhei }  { AdobeHeitiStd-Regular }
    \setCJKfamilyfont { zhfs }   { AdobeFangsongStd-Regular }
    \setCJKfamilyfont { zhkai }  { AdobeKaitiStd-Regular }
  }
%    \end{macrocode}
%
%    \begin{macrocode}
%</adobe>
%<*fandol>
%    \end{macrocode}
%
% \subsubsection{\pkg{ctex-fontset-fandol.def}}
%
%    \begin{macrocode}
\pdftex_if_engine:TF
  {
    \ctex_if_pdfmode:TF
      { \ctex_fontset_error:n { fandol } }
      {
        \ctex_zhmap_case:nnn
          {
            \setCJKmainfont
              [
                      cmap = UniGB-UTF16-H ,
                  BoldFont = FandolSong-Bold.otf ,
                ItalicFont = FandolKai-Regular.otf
              ] { FandolSong-Regular.otf }
            \setCJKsansfont
              [
                    cmap = UniGB-UTF16-H ,
                BoldFont = FandolHei-Bold.otf
              ] { FandolHei-Regular.otf }
            \setCJKmonofont [ cmap = UniGB-UTF16-H ] { FandolFang-Regular.otf }
            \setCJKfamilyfont { zhsong }
              [
                    cmap = UniGB-UTF16-H ,
                BoldFont = FandolSong-Bold.otf
              ] { FandolSong-Regular.otf }
            \setCJKfamilyfont { zhhei }
              [
                    cmap = UniGB-UTF16-H ,
                BoldFont = FandolHei-Bold.otf
              ] { FandolHei-Regular.otf }
            \setCJKfamilyfont { zhfs }
              [ cmap = UniGB-UTF16-H ] { FandolFang-Regular.otf }
            \setCJKfamilyfont { zhkai }
              [ cmap = UniGB-UTF16-H ] { FandolKai-Regular.otf }
            \ctex_punct_set:n { fandol }
            \ctex_punct_map_family:nn { \CJKrmdefault } { zhsong }
            \ctex_punct_map_family:nn { \CJKsfdefault } { zhhei }
            \ctex_punct_map_family:nn { \CJKttdefault } { zhfs }
            \ctex_punct_map_itshape:nn { \CJKrmdefault } { zhkai }
            \ctex_punct_map_bfseries:nn { \CJKrmdefault , zhsong } { zhsongb }
            \ctex_punct_map_bfseries:nn { \CJKsfdefault , zhhei } { zhheib }
          }
          {
            \ctex_load_zhmap:nnnn { rm } { zhhei } { zhfs } { zhfandolfonts }
            \ctex_punct_set:n { fandol }
            \ctex_punct_map_family:nn { \CJKrmdefault } { zhsong }
            \ctex_punct_map_bfseries:nn { \CJKrmdefault } { zhhei }
            \ctex_punct_map_itshape:nn { \CJKrmdefault } { zhkai }
          }
          { \ctex_fontset_error:n { fandol } }
      }
  }
  {
    \setCJKmainfont
      [ Extension = .otf , BoldFont = FandolSong-Bold , ItalicFont = FandolKai-Regular ]
      { FandolSong-Regular }
    \setCJKsansfont [ Extension = .otf , BoldFont = FandolHei-Bold ] { FandolHei-Regular }
    \setCJKmonofont [ Extension = .otf ] { FandolFang-Regular }
    \setCJKfamilyfont { zhsong }
      [ Extension = .otf , BoldFont = FandolSong-Bold ] { FandolSong-Regular }
    \setCJKfamilyfont { zhhei }
      [ Extension = .otf , BoldFont = FandolHei-Bold ] { FandolHei-Regular }
    \setCJKfamilyfont { zhfs }  [ Extension = .otf ] { FandolFang-Regular }
    \setCJKfamilyfont { zhkai } [ Extension = .otf ] { FandolKai-Regular }
  }
%    \end{macrocode}
%
%    \begin{macrocode}
%</fandol>
%<*mac>
%    \end{macrocode}
%
% \subsubsection{\pkg{ctex-fontset-mac.def}}
%
% OS X Mavericks (10.9) 预装的主要简体中文字体如下^^A
% \footnote{\url{http://support.apple.com/kb/HT5944}}:
% \begin{verbatim}
%   /Library/Fonts/Baoli.ttc: 报隶-简,Baoli SC:style=Regular
%   /Library/Fonts/Hannotate.ttc: 手札体-简,Hannotate SC:style=Bold
%   /Library/Fonts/Hannotate.ttc: 手札体-简,Hannotate SC:style=Regular
%   /Library/Fonts/Hanzipen.ttc: 翩翩体-简,HanziPen SC:style=Bold
%   /Library/Fonts/Hanzipen.ttc: 翩翩体-简,HanziPen SC:style=Regular
%   /Library/Fonts/Hei.ttf: Hei
%   /Library/Fonts/Hiragino Sans GB W3.otf: 冬青黑体简体中文 W3,Hiragino Sans GB W3
%   /Library/Fonts/Hiragino Sans GB W6.otf: 冬青黑体简体中文 W6,Hiragino Sans GB W6
%   /Library/Fonts/Kai.ttf: Kai
%   /Library/Fonts/Kaiti.ttc: STKaiti
%   /Library/Fonts/Kaiti.ttc: 楷体-简,Kaiti SC:style=Black
%   /Library/Fonts/Kaiti.ttc: 楷体-简,Kaiti SC:style=Bold
%   /Library/Fonts/Kaiti.ttc: 楷体-简,Kaiti SC:style=Regular
%   /Library/Fonts/Lantinghei.ttc: 兰亭黑-简,Lantinghei SC:style=Demibold
%   /Library/Fonts/Lantinghei.ttc: 兰亭黑-简,Lantinghei SC:style=Heavy
%   /Library/Fonts/Lantinghei.ttc: 兰亭黑-简,Lantinghei SC:style=Extralight
%   /Library/Fonts/Libian.ttc: 隶变-简,Libian SC
%   /Library/Fonts/Songti.ttc: STSong
%   /Library/Fonts/Songti.ttc: 宋体-简,Songti SC:style=Black
%   /Library/Fonts/Songti.ttc: 宋体-简,Songti SC:style=Bold
%   /Library/Fonts/Songti.ttc: 宋体-简,Songti SC:style=Light
%   /Library/Fonts/Songti.ttc: 宋体-简,Songti SC:style=Regular
%   /Library/Fonts/WawaSC-Regular.otf: 娃娃体-简,Wawati SC
%   /Library/Fonts/WeibeiSC-Bold.otf: 魏碑-简,Weibei SC
%   /Library/Fonts/Xingkai.ttc: 行楷-简,Xingkai SC:style=Bold
%   /Library/Fonts/Xingkai.ttc: 行楷-简,Xingkai SC:style=Light
%   /Library/Fonts/Yuanti.ttc: 圆体-简,Yuanti SC:style=Bold
%   /Library/Fonts/Yuanti.ttc: 圆体-简,Yuanti SC:style=Light
%   /Library/Fonts/Yuanti.ttc: 圆体-简,Yuanti SC:style=Regular
%   /Library/Fonts/YuppySC-Regular.otf: 雅痞-简,Yuppy SC
%   /Library/Fonts/华文仿宋.ttf: STFangsong
%   /Library/Fonts/华文细黑.ttf: STHeiti:style=Light
%   /Library/Fonts/华文黑体.ttf: STHeiti:style=Regular
%   /System/Library/Fonts/STHeiti Light.ttc: 黑体-简,Heiti SC:style=Light
%   /System/Library/Fonts/STHeiti Medium.ttc: 黑体-简,Heiti SC:style=Medium
% \end{verbatim}
% 在 \dvipdfmx{} 下,可以通过下述方式使用华文宋体和华文楷体:
% \begin{verbatim}
%   \special{pdf:mapline unisong@Unicode@ unicode :4:Songti.ttc}
%   \special{pdf:mapline unikai@Unicode@  unicode :4:Kaiti.ttc}
% \end{verbatim}
% 而 \tn{pdfmapline} 似乎不支持带索引的 \texttt{ttc} 字体,\file{Songti.ttc} 默认
% 使用的是 Songti SC Black,\file{Kaiti.ttc} 默认使用的是 Kaiti SC Black。
% 华文黑体不能通过这种方式使用:
% \begin{verbatim}
%   \special{pdf:mapline unihei@Unicode@ unicode \detokenize{华文黑体}.ttf}
% \end{verbatim}
% \dvipdfmx{} 不能生成 PDF,报下述错误:
% \begin{verbatim}
%   ** WARNING ** UCS-4 TrueType cmap table...
%   ** ERROR ** Unable to read OpenType/TrueType Unicode cmap table.
% \end{verbatim}
% 如果将 CMap 改为 UniGB-UTF16-H,错误信息是
% \begin{verbatim}
%   ** WARNING ** No usable TrueType cmap table found for font "华文黑体.ttf".
%   ** WARNING ** CID character collection for this font is set to "Adobe-GB1"
%   ** ERROR ** Cannot continue without this...
% \end{verbatim}
% 在 \pdfTeX{} 下生成的 PDF 只有方框^^A
% \footnote{\url{http://www.newsmth.net/bbscon.php?bid=460&id=312640}}。
% 华文细黑和华文仿宋的情况类似。
%    \begin{macrocode}
\pdftex_if_engine:TF
  { \ctex_fontset_error:n { mac } }
  {
    \setCJKmainfont [ BoldFont = STHeiti , ItalicFont = STKaiti ]  { STSong }
    \setCJKsansfont [ BoldFont = STHeiti ] { STXihei }
    \setCJKmonofont { STFangsong }
    \setCJKfamilyfont { zhsong } { STSong }
    \setCJKfamilyfont { zhhei }  { STHeiti }
    \setCJKfamilyfont { zhfs }   { STFangsong }
    \setCJKfamilyfont { zhkai }  { STKaiti }
  }
%    \end{macrocode}
%
%    \begin{macrocode}
%</mac>
%<*founder>
%    \end{macrocode}
%
% \subsubsection{\pkg{ctex-fontset-founder.def}}
%
%    \begin{macrocode}
\pdftex_if_engine:TF
  {
    \ctex_zhmap_case:nnn
      {
        \setCJKmainfont
          [ BoldFont = FZXBSK.TTF , ItalicFont = FZKTK.TTF ] { FZSSK.TTF }
        \setCJKsansfont [ BoldFont = FZHTK.TTF ] { FZXH1K.TTF }
        \setCJKmonofont { FZFSK.TTF }
        \setCJKfamilyfont { zhsong } [ BoldFont = FZXBSK.TTF ] { FZSSK.TTF }
        \setCJKfamilyfont { zhhei }  { FZHTK.TTF }
        \setCJKfamilyfont { zhkai }  { FZKTK.TTF }
        \setCJKfamilyfont { zhfs }   { FZFSK.TTF }
        \setCJKfamilyfont { zhli }   { FZLSK.TTF }
        \setCJKfamilyfont { zhyou } [ BoldFont = FZY3K.TTF ] { FZY1K.TTF }
        \ctex_punct_set:n { founder }
        \ctex_punct_map_family:nn { \CJKrmdefault } { zhsong }
        \ctex_punct_map_family:nn { \CJKsfdefault } { zhheil }
        \ctex_punct_map_family:nn { \CJKttdefault } { zhfs }
        \ctex_punct_map_itshape:nn { \CJKrmdefault } { zhkai }
        \ctex_punct_map_bfseries:nn { \CJKrmdefault , zhsong } { zhsongb }
        \ctex_punct_map_bfseries:nn { \CJKsfdefault } { zhhei }
        \ctex_punct_map_bfseries:nn { zhyou } { zhyoub }
      }
      {
        \ctex_load_zhmap:nnnn { rm } { zhhei } { zhfs } { zhfounderfonts }
        \ctex_punct_set:n { founder }
        \ctex_punct_map_family:nn { \CJKrmdefault } { zhsong }
        \ctex_punct_map_bfseries:nn { \CJKrmdefault } { zhhei }
        \ctex_punct_map_itshape:nn { \CJKrmdefault } { zhkai }
      }
      { \ctex_fontset_error:n { founder } }
  }
  {
    \setCJKmainfont
      [ BoldFont = FZXiaoBiaoSong-B05 , ItalicFont = FZKai-Z03 ] { FZShuSong-Z01 }
%    \end{macrocode}
%
% 在 WPS For Linux 附带的 5.00 版和家庭版 5.20 版的“方正细黑一\_GBK”的字体名称
% 是 |FZXiHeiI-Z08|。但在网上发现不少文档和资料都是 \verb*|FZXiHei I-Z08|,而在
% 官方资料《2013 方正字库字体样张》中对应的英文名字是 \verb*|FZXiHei I|。可以用
% Postscript 名字 |FZXH1K--GBK1-0| 来统一。经测试时发现(WPS 中的字体),\XeTeX{}
% 找该字体时会出现明显的延迟,而用字体文件名 |FZXH1K.TTF| 又可能会出现大小写问题,
% 遂采用汉字名称。由于 \pkg{luaotfload} 不支持汉字名称,故使用 Postscript 名字,
% 速度不受影响。
%    \begin{macrocode}
    \setCJKsansfont [ BoldFont = FZHei-B01 ]
      { \xetex_if_engine:TF { 方正细黑一_GBK } { FZXH1K--GBK1-0 } }
    \setCJKmonofont { FZFangSong-Z02 }
    \setCJKfamilyfont { zhsong } [ BoldFont = FZXiaoBiaoSong-B05 ] { FZShuSong-Z01 }
    \setCJKfamilyfont { zhhei }  { FZHei-B01 }
    \setCJKfamilyfont { zhkai }  { FZKai-Z03 }
    \setCJKfamilyfont { zhfs }   { FZFangSong-Z02 }
    \setCJKfamilyfont { zhli }   { FZLiShu-S01 }
    \setCJKfamilyfont { zhyou } [ BoldFont = FZZhunYuan-M02 ] { FZXiYuan-M01 }
  }
%    \end{macrocode}
%
%    \begin{macrocode}
%</founder>
%<*ubuntu>
%    \end{macrocode}
%
% \subsubsection{\pkg{ctex-fontset-ubuntu.def}}
%
% 以下根据 Ubuntu 12.04 的中文字体情况设置。CMap 不清楚应该是什么,指定为
% UniGB-UTF16-H 还是有警告:
% \begin{verbatim}
%   ** WARNING ** UCS-4 TrueType cmap table...
% \end{verbatim}
% 需要注意的是 \file{uming.ttc} 和 \file{ukai.ttc} 看起来像有四种字形的样子,但
% 其实只有“令”和“骨”这区区两个字有新字形,其余都取旧字形^^A
% \footnote{\url{http://www.freedesktop.org/wiki/Software/CJKUnifonts/Download/}}。
%    \begin{macrocode}
\pdftex_if_engine:TF
  {
    \ctex_zhmap_case:nnn
      {
        \setCJKmainfont
          [ BoldFont = wqy-zenhei.ttc , ItalicFont = ukai.ttc ] { uming.ttc }
        \setCJKsansfont { wqy-zenhei.ttc }
        \setCJKmonofont { uming.ttc }
        \setCJKfamilyfont { zhsong } { uming.ttc }
        \setCJKfamilyfont { zhhei }  { wqy-zenhei.ttc }
        \setCJKfamilyfont { zhkai }  { ukai.ttc }
        \ctex_punct_set:n { ubuntu }
        \ctex_punct_map_family:nn { \CJKrmdefault } { zhsong }
        \ctex_punct_map_family:nn { \CJKsfdefault } { zhhei }
        \ctex_punct_map_family:nn { \CJKttdefault } { zhsong }
        \ctex_punct_map_itshape:nn { \CJKrmdefault } { zhkai }
        \ctex_punct_map_bfseries:nn { \CJKrmdefault } { zhhei }
      }
      {
        \ctex_load_zhmap:nnnn { rm } { zhhei } { zhsong } { zhubuntufonts }
        \ctex_punct_set:n { ubuntu }
        \ctex_punct_map_family:nn { \CJKrmdefault } { zhsong }
        \ctex_punct_map_bfseries:nn { \CJKrmdefault } { zhhei }
        \ctex_punct_map_itshape:nn { \CJKrmdefault } { zhkai }
      }
      { \ctex_fontset_error:n { ubuntu } }
  }
  {
    \setCJKmainfont [ ItalicFont = AR~PL~UKai~CN ] { AR~PL~UMing~CN }
    \setCJKsansfont { WenQuanYi~Zen~Hei }
    \setCJKmonofont { AR~PL~UMing~CN }
    \setCJKfamilyfont { zhsong } { AR~PL~UMing~CN }
    \setCJKfamilyfont { zhhei }  { WenQuanYi~Zen~Hei }
    \setCJKfamilyfont { zhkai }  { AR~PL~UKai~CN }
  }
%    \end{macrocode}
%
%    \begin{macrocode}
%</ubuntu>
%    \end{macrocode}
%
% \subsubsection{中文字体命令}
%
%    \begin{macrocode}
%<*!windows>
%    \end{macrocode}
%
%    \begin{macrocode}
\NewDocumentCommand \songti   { } { \CJKfamily { zhsong } }
\NewDocumentCommand \heiti    { } { \CJKfamily { zhhei } }
%<!ubuntu>\NewDocumentCommand \fangsong { } { \CJKfamily { zhfs } }
\NewDocumentCommand \kaishu   { } { \CJKfamily { zhkai } }
%<*windowsnew|windowsold|founder>
\NewDocumentCommand \lishu    { } { \CJKfamily { zhli } }
\NewDocumentCommand \youyuan  { } { \CJKfamily { zhyou } }
%</windowsnew|windowsold|founder>
%<windowsnew>\NewDocumentCommand \yahei    { } { \CJKfamily { zhyahei } }
%    \end{macrocode}
%
%    \begin{macrocode}
%</!windows>
%</fontset>
%<*zhmap>
%    \end{macrocode}
%
% \subsubsection{\pkg{zhmetrics} 的字体映射}
%
% 确认 \tn{catcode},没有重复载入检查。
%    \begin{macrocode}
\begingroup\catcode61\catcode48\catcode32=10\relax%
  \catcode 35=6 % #
  \catcode123=1 % {
  \catcode125=2 % }
  \toks0{\endlinechar=\the\endlinechar\relax}%
  \toks2{\endlinechar=13 }%
  \def\x#1 #2 {%
    \toks0\expandafter{\the\toks0 \catcode#1=\the\catcode#1\relax}%
    \toks2\expandafter{\the\toks2 \catcode#1=#2 }}%
  \x  13  5 % carriage return
  \x  32 10 % space
  \x  35  6 % #
  \x  40 12 % (
  \x  41 12 % )
  \x  45 12 % -
  \x  46 12 % .
  \x  47 12 % /
  \x  58 12 % :
  \x  60 12 % <
  \x  61 12 % =
  \x  64 11 % @
  \x  91 12 % [
  \x  93 12 % ]
  \x 123  1 % {
  \x 125  2 % }
  \edef\x#1{\endgroup%
    \edef\noexpand#1{%
      \the\toks0 %
      \let\noexpand\noexpand\noexpand#1%
          \noexpand\noexpand\noexpand\undefined%
      \noexpand\noexpand\noexpand\endinput}%
    \the\toks2}%
\expandafter\x\csname ctex@zhmap@endinput\endcsname
%    \end{macrocode}
%
%    \begin{macrocode}
\input ifpdf.sty\relax
%    \end{macrocode}
%
% 提供非 \LaTeX{} 格式下的 \tn{ProvidesFile}。
%    \begin{macrocode}
\begingroup
\expandafter\ifx\csname ProvidesFile\endcsname\relax
  \long\def\x#1\ProvidesFile#2[#3]{%
    #1%
    \immediate\write-1{File: #2 #3}%
    \expandafter\xdef\csname ver@#2\endcsname{#3}}
  \expandafter\x%
\fi
\endgroup
%    \end{macrocode}
%
% \paragraph{\pkg{zhwinfonts.tex}}
%
%    \begin{macrocode}
%<*windows>
\ProvidesFile{zhwinfonts.tex}%
  [2014/06/03 v2.0 Windows font map loader for pdfTeX and DVIPDFMx (CTEX)]

\ifpdf
  \pdfmapline{=gbk@UGBK@     <simsun.ttc}
  \pdfmapline{=gbksong@UGBK@ <simsun.ttc}
  \pdfmapline{=gbkkai@UGBK@  <simkai.ttf}
  \pdfmapline{=gbkhei@UGBK@  <simhei.ttf}
  \pdfmapline{=gbkfs@UGBK@   <simfang.ttf}
  \pdfmapline{=gbkli@UGBK@   <simli.ttf}
  \pdfmapline{=gbkyou@UGBK@  <simyou.ttf}

  \pdfmapline{=cyberb@Unicode@  <simsun.ttc}
  \pdfmapline{=unisong@Unicode@ <simsun.ttc}
  \pdfmapline{=unikai@Unicode@  <simkai.ttf}
  \pdfmapline{=unihei@Unicode@  <simhei.ttf}
  \pdfmapline{=unifs@Unicode@   <simfang.ttf}
  \pdfmapline{=unili@Unicode@   <simli.ttf}
  \pdfmapline{=uniyou@Unicode@  <simyou.ttf}

  \pdfmapline{=gbksongsl@UGBK@ <simsun.ttc}
  \pdfmapline{=gbkkaisl@UGBK@  <simkai.ttf}
  \pdfmapline{=gbkheisl@UGBK@  <simhei.ttf}
  \pdfmapline{=gbkfssl@UGBK@   <simfang.ttf}
  \pdfmapline{=gbklisl@UGBK@   <simli.ttf}
  \pdfmapline{=gbkyousl@UGBK@  <simyou.ttf}

  \pdfmapline{=unisongsl@Unicode@ <simsun.ttc}
  \pdfmapline{=unikaisl@Unicode@  <simkai.ttf}
  \pdfmapline{=uniheisl@Unicode@  <simhei.ttf}
  \pdfmapline{=unifssl@Unicode@   <simfang.ttf}
  \pdfmapline{=unilisl@Unicode@   <simli.ttf}
  \pdfmapline{=uniyousl@Unicode@  <simyou.ttf}

\else
  \special{pdf:mapline gbk@UGBK@     unicode :0:simsun.ttc -v 50}
  \special{pdf:mapline gbksong@UGBK@ unicode :0:simsun.ttc -v 50}
  \special{pdf:mapline gbkkai@UGBK@  unicode simkai.ttf -v 70}
  \special{pdf:mapline gbkhei@UGBK@  unicode simhei.ttf -v 150}
  \special{pdf:mapline gbkfs@UGBK@   unicode simfang.ttf -v 50}
  \special{pdf:mapline gbkli@UGBK@   unicode simli.ttf -v 150}
  \special{pdf:mapline gbkyou@UGBK@  unicode simyou.ttf -v 60}

  \special{pdf:mapline cyberb@Unicode@  unicode :0:simsun.ttc -v 50}
  \special{pdf:mapline unisong@Unicode@ unicode :0:simsun.ttc -v 50}
  \special{pdf:mapline unikai@Unicode@  unicode simkai.ttf -v 70}
  \special{pdf:mapline unihei@Unicode@  unicode simhei.ttf -v 150}
  \special{pdf:mapline unifs@Unicode@   unicode simfang.ttf -v 50}
  \special{pdf:mapline unili@Unicode@   unicode simli.ttf -v 150}
  \special{pdf:mapline uniyou@Unicode@  unicode simyou.ttf -v 60}

  \special{pdf:mapline gbksongsl@UGBK@ unicode :0:simsun.ttc -s .167 -v 50}
  \special{pdf:mapline gbkkaisl@UGBK@  unicode simkai.ttf -s .167 -v 70}
  \special{pdf:mapline gbkheisl@UGBK@  unicode simhei.ttf -s .167 -v 150}
  \special{pdf:mapline gbkfssl@UGBK@   unicode simfang.ttf -s .167 -v 50}
  \special{pdf:mapline gbklisl@UGBK@   unicode simli.ttf -s .167 -v 150}
  \special{pdf:mapline gbkyousl@UGBK@  unicode simyou.ttf -s .167 -v 60}

  \special{pdf:mapline unisongsl@Unicode@ unicode :0:simsun.ttc -s .167 -v 50}
  \special{pdf:mapline unikaisl@Unicode@  unicode simkai.ttf -s .167 -v 70}
  \special{pdf:mapline uniheisl@Unicode@  unicode simhei.ttf -s .167 -v 150}
  \special{pdf:mapline unifssl@Unicode@   unicode simfang.ttf -s .167 -v 50}
  \special{pdf:mapline unilisl@Unicode@   unicode simli.ttf -s .167 -v 150}
  \special{pdf:mapline uniyousl@Unicode@  unicode simyou.ttf -s .167 -v 60}

%</windows>
%    \end{macrocode}
%
% \paragraph{\pkg{zhadobefonts.tex}}
%
%    \begin{macrocode}
%<*adobe>
\ProvidesFile{zhadobefonts.tex}%
  [2014/06/03 v2.0 Adobe font map loader for DVIPDFMx (CTEX)]

\ifpdf
%% pdfTeX does not support OTF fonts

\else
  \special{pdf:mapline gbk@UGBK@     UniGB-UTF16-H AdobeSongStd-Light.otf}
  \special{pdf:mapline gbksong@UGBK@ UniGB-UTF16-H AdobeSongStd-Light.otf}
  \special{pdf:mapline gbkkai@UGBK@  UniGB-UTF16-H AdobeKaitiStd-Regular.otf}
  \special{pdf:mapline gbkhei@UGBK@  UniGB-UTF16-H AdobeHeitiStd-Regular.otf}
  \special{pdf:mapline gbkfs@UGBK@   UniGB-UTF16-H AdobeFangsongStd-Regular.otf}

  \special{pdf:mapline cyberb@Unicode@  UniGB-UTF16-H AdobeSongStd-Light.otf}
  \special{pdf:mapline unisong@Unicode@ UniGB-UTF16-H AdobeSongStd-Light.otf}
  \special{pdf:mapline unikai@Unicode@  UniGB-UTF16-H AdobeKaitiStd-Regular.otf}
  \special{pdf:mapline unihei@Unicode@  UniGB-UTF16-H AdobeHeitiStd-Regular.otf}
  \special{pdf:mapline unifs@Unicode@   UniGB-UTF16-H AdobeFangsongStd-Regular.otf}

  \special{pdf:mapline gbksongsl@UGBK@ UniGB-UTF16-H AdobeSongStd-Light.otf -s .167}
  \special{pdf:mapline gbkkaisl@UGBK@  UniGB-UTF16-H AdobeKaitiStd-Regular.otf -s .167}
  \special{pdf:mapline gbkheisl@UGBK@  UniGB-UTF16-H AdobeHeitiStd-Regular.otf -s .167}
  \special{pdf:mapline gbkfssl@UGBK@   UniGB-UTF16-H AdobeFangsongStd-Regular.otf -s .167}

  \special{pdf:mapline unisongsl@Unicode@ UniGB-UTF16-H AdobeSongStd-Light.otf -s .167}
  \special{pdf:mapline unikaisl@Unicode@  UniGB-UTF16-H AdobeKaitiStd-Regular.otf -s .167}
  \special{pdf:mapline uniheisl@Unicode@  UniGB-UTF16-H AdobeHeitiStd-Regular.otf -s .167}
  \special{pdf:mapline unifssl@Unicode@   UniGB-UTF16-H AdobeFangsongStd-Regular.otf -s .167}

%</adobe>
%    \end{macrocode}
%
% \paragraph{\pkg{zhfandolfonts.tex}}
%
%    \begin{macrocode}
%<*fandol>
\ProvidesFile{zhfandolfonts.tex}%
  [2014/06/03 v2.0 Fandol font map loader for DVIPDFMx (CTEX)]

\ifpdf
%% pdfTeX does not support OTF fonts

\else
  \special{pdf:mapline gbk@UGBK@     UniGB-UTF16-H FandolSong-Regular.otf}
  \special{pdf:mapline gbksong@UGBK@ UniGB-UTF16-H FandolSong-Regular.otf}
  \special{pdf:mapline gbkkai@UGBK@  UniGB-UTF16-H FandolKai-Regular.otf}
  \special{pdf:mapline gbkhei@UGBK@  UniGB-UTF16-H FandolHei-Regular.otf}
  \special{pdf:mapline gbkfs@UGBK@   UniGB-UTF16-H FandolFang-Regular.otf}

  \special{pdf:mapline cyberb@Unicode@  UniGB-UTF16-H FandolSong-Regular.otf}
  \special{pdf:mapline unisong@Unicode@ UniGB-UTF16-H FandolSong-Regular.otf}
  \special{pdf:mapline unikai@Unicode@  UniGB-UTF16-H FandolKai-Regular.otf}
  \special{pdf:mapline unihei@Unicode@  UniGB-UTF16-H FandolHei-Regular.otf}
  \special{pdf:mapline unifs@Unicode@   UniGB-UTF16-H FandolFang-Regular.otf}

  \special{pdf:mapline gbksongsl@UGBK@ UniGB-UTF16-H FandolSong-Regular.otf -s .167}
  \special{pdf:mapline gbkkaisl@UGBK@  UniGB-UTF16-H FandolKai-Regular.otf -s .167}
  \special{pdf:mapline gbkheisl@UGBK@  UniGB-UTF16-H FandolHei-Regular.otf -s .167}
  \special{pdf:mapline gbkfssl@UGBK@   UniGB-UTF16-H FandolFang-Regular.otf -s .167}

  \special{pdf:mapline unisongsl@Unicode@ UniGB-UTF16-H FandolSong-Regular.otf -s .167}
  \special{pdf:mapline unikaisl@Unicode@  UniGB-UTF16-H FandolKai-Regular.otf -s .167}
  \special{pdf:mapline uniheisl@Unicode@  UniGB-UTF16-H FandolHei-Regular.otf -s .167}
  \special{pdf:mapline unifssl@Unicode@   UniGB-UTF16-H FandolFang-Regular.otf -s .167}

%</fandol>
%    \end{macrocode}
%
% \paragraph{\pkg{zhfounderfonts.tex}}
%
%    \begin{macrocode}
%<*founder>
\ProvidesFile{zhfounderfonts.tex}%
  [2014/06/03 v2.0 Founder font map loader for pdfTeX and DVIPDFMx (CTEX)]

\ifpdf
  \pdfmapline{=gbk@UGBK@     <FZSSK.TTF}
  \pdfmapline{=gbksong@UGBK@ <FZSSK.TTF}
  \pdfmapline{=gbkkai@UGBK@  <FZKTK.TTF}
  \pdfmapline{=gbkhei@UGBK@  <FZHTK.TTF}
  \pdfmapline{=gbkfs@UGBK@   <FZFSK.TTF}
  \pdfmapline{=gbkli@UGBK@   <FZLSK.TTF}
  \pdfmapline{=gbkyou@UGBK@  <FZY1K.TTF}

  \pdfmapline{=cyberb@Unicode@  <FZSSK.TTF}
  \pdfmapline{=unisong@Unicode@ <FZSSK.TTF}
  \pdfmapline{=unikai@Unicode@  <FZKTK.TTF}
  \pdfmapline{=unihei@Unicode@  <FZHTK.TTF}
  \pdfmapline{=unifs@Unicode@   <FZFSK.TTF}
  \pdfmapline{=unili@Unicode@   <FZLSK.TTF}
  \pdfmapline{=uniyou@Unicode@  <FZY1K.TTF}

  \pdfmapline{=gbksongsl@UGBK@ <FZSSK.TTF}
  \pdfmapline{=gbkkaisl@UGBK@  <FZKTK.TTF}
  \pdfmapline{=gbkheisl@UGBK@  <FZHTK.TTF}
  \pdfmapline{=gbkfssl@UGBK@   <FZFSK.TTF}
  \pdfmapline{=gbklisl@UGBK@   <FZLSK.TTF}
  \pdfmapline{=gbkyousl@UGBK@  <FZY1K.TTF}

  \pdfmapline{=unisongsl@Unicode@ <FZSSK.TTF}
  \pdfmapline{=unikaisl@Unicode@  <FZKTK.TTF}
  \pdfmapline{=uniheisl@Unicode@  <FZHTK.TTF}
  \pdfmapline{=unifssl@Unicode@   <FZFSK.TTF}
  \pdfmapline{=unilisl@Unicode@   <FZLSK.TTF}
  \pdfmapline{=uniyousl@Unicode@  <FZY1K.TTF}

\else
  \special{pdf:mapline gbk@UGBK@     unicode FZSSK.TTF}
  \special{pdf:mapline gbksong@UGBK@ unicode FZSSK.TTF}
  \special{pdf:mapline gbkkai@UGBK@  unicode FZKTK.TTF}
  \special{pdf:mapline gbkhei@UGBK@  unicode FZHTK.TTF}
  \special{pdf:mapline gbkfs@UGBK@   unicode FZFSK.TTF}
  \special{pdf:mapline gbkli@UGBK@   unicode FZLSK.TTF}
  \special{pdf:mapline gbkyou@UGBK@  unicode FZY1K.TTF}

  \special{pdf:mapline cyberb@Unicode@  unicode FZSSK.TTF}
  \special{pdf:mapline unisong@Unicode@ unicode FZSSK.TTF}
  \special{pdf:mapline unikai@Unicode@  unicode FZKTK.TTF}
  \special{pdf:mapline unihei@Unicode@  unicode FZHTK.TTF}
  \special{pdf:mapline unifs@Unicode@   unicode FZFSK.TTF}
  \special{pdf:mapline unili@Unicode@   unicode FZLSK.TTF}
  \special{pdf:mapline uniyou@Unicode@  unicode FZY1K.TTF}

  \special{pdf:mapline gbksongsl@UGBK@ unicode FZSSK.TTF -s .167}
  \special{pdf:mapline gbkkaisl@UGBK@  unicode FZKTK.TTF -s .167}
  \special{pdf:mapline gbkheisl@UGBK@  unicode FZHTK.TTF -s .167}
  \special{pdf:mapline gbkfssl@UGBK@   unicode FZFSK.TTF -s .167}
  \special{pdf:mapline gbklisl@UGBK@   unicode FZLSK.TTF -s .167}
  \special{pdf:mapline gbkyousl@UGBK@  unicode FZY1K.TTF -s .167}

  \special{pdf:mapline unisongsl@Unicode@ unicode FZSSK.TTF -s .167}
  \special{pdf:mapline unikaisl@Unicode@  unicode FZKTK.TTF -s .167}
  \special{pdf:mapline uniheisl@Unicode@  unicode FZHTK.TTF -s .167}
  \special{pdf:mapline unifssl@Unicode@   unicode FZFSK.TTF -s .167}
  \special{pdf:mapline unilisl@Unicode@   unicode FZLSK.TTF -s .167}
  \special{pdf:mapline uniyousl@Unicode@  unicode FZY1K.TTF -s .167}

%</founder>
%    \end{macrocode}
%
% \paragraph{\pkg{zhubuntufonts.tex}}
%
%    \begin{macrocode}
%<*ubuntu>
\ProvidesFile{zhubuntufonts.tex}%
  [2014/06/03 v2.0 Ubuntu font map loader for pdfTeX and DVIPDFMx (CTEX)]

\ifpdf
  \pdfmapline{=gbk@UGBK@     <uming.ttc}
  \pdfmapline{=gbksong@UGBK@ <uming.ttc}
  \pdfmapline{=gbkkai@UGBK@  <ukai.ttc}
  \pdfmapline{=gbkhei@UGBK@  <wqy-zenhei.ttc}
  \pdfmapline{=gbkfs@UGBK@   <uming.ttc}
  \pdfmapline{=gbkyou@UGBK@  <wqy-zenhei.ttc}

  \pdfmapline{=cyberb@Unicode@  <uming.ttc}
  \pdfmapline{=unisong@Unicode@ <uming.ttc}
  \pdfmapline{=unikai@Unicode@  <ukai.ttc}
  \pdfmapline{=unihei@Unicode@  <wqy-zenhei.ttc}
  \pdfmapline{=unifs@Unicode@   <uming.ttc}
  \pdfmapline{=uniyou@Unicode@  <wqy-zenhei.ttc}

  \pdfmapline{=gbksongsl@UGBK@ <uming.ttc}
  \pdfmapline{=gbkkaisl@UGBK@  <ukai.ttc}
  \pdfmapline{=gbkheisl@UGBK@  <wqy-zenhei.ttc}
  \pdfmapline{=gbkfssl@UGBK@   <uming.ttc}
  \pdfmapline{=gbkyousl@UGBK@  <wqy-zenhei.ttc}

  \pdfmapline{=unisongsl@Unicode@ <uming.ttc}
  \pdfmapline{=unikaisl@Unicode@  <ukai.ttc}
  \pdfmapline{=uniheisl@Unicode@  <wqy-zenhei.ttc}
  \pdfmapline{=unifssl@Unicode@   <uming.ttc}
  \pdfmapline{=uniyousl@Unicode@  <wqy-zenhei.ttc}

\else
  \special{pdf:mapline gbk@UGBK@     unicode :0:uming.ttc}
  \special{pdf:mapline gbksong@UGBK@ unicode :0:uming.ttc}
  \special{pdf:mapline gbkkai@UGBK@  unicode :0:ukai.ttc}
  \special{pdf:mapline gbkhei@UGBK@  unicode :0:wqy-zenhei.ttc}
  \special{pdf:mapline gbkfs@UGBK@   unicode :0:uming.ttc}

  \special{pdf:mapline cyberb@Unicode@  unicode :0:uming.ttc}
  \special{pdf:mapline unisong@Unicode@ unicode :0:uming.ttc}
  \special{pdf:mapline unikai@Unicode@  unicode :0:ukai.ttc}
  \special{pdf:mapline unihei@Unicode@  unicode :0:wqy-zenhei.ttc}
  \special{pdf:mapline unifs@Unicode@   unicode :0:uming.ttc}

  \special{pdf:mapline gbksongsl@UGBK@ unicode :0:uming.ttc -s .167}
  \special{pdf:mapline gbkkaisl@UGBK@  unicode :0:ukai.ttc -s .167}
  \special{pdf:mapline gbkheisl@UGBK@  unicode :0:wqy-zenhei.ttc -s .167}
  \special{pdf:mapline gbkfssl@UGBK@   unicode :0:uming.ttc -s .167}

  \special{pdf:mapline unisongsl@Unicode@ unicode :0:uming.ttc -s .167}
  \special{pdf:mapline unikaisl@Unicode@  unicode :0:ukai.ttc -s .167}
  \special{pdf:mapline uniheisl@Unicode@  unicode :0:wqy-zenhei.ttc -s .167}
  \special{pdf:mapline unifssl@Unicode@   unicode :0:uming.ttc -s .167}

%</ubuntu>
%    \end{macrocode}
%
%    \begin{macrocode}
\fi

\ctex@zhmap@endinput
%</zhmap>
%    \end{macrocode}
%
% \subsubsection{制作 \texttt{spa} 文件}
%
% 我们通过 \XeTeX{} 的 \tn{XeTeXglyphbounds} 取得字体中标点符号的边界信息,为
% \pkg{CJKpunct} 宏包制作 \file{spa}。
%
%    \begin{macrocode}
%<*spa>
%<*macro>
\input expl3-generic %
\ExplSyntaxOn
\xetex_if_engine:F
  {
    \msg_new:nn { ctex } { xetex }
      { XeTeX~is~required~to~compile~this~document! }
    \msg_fatal:nn { ctex } { xetex }
  }
%    \end{macrocode}
%
% \pkg{CJKpunct} 定义的标点符号是:
% \begin{verbatim}
%   ‘“「『〔([{〈《〖【
%   —…、。,.:;!?%〕)]}〉》〗】’”」』
% \end{verbatim}
% 注意顺序不能改变。
%    \begin{macrocode}
\seq_new:N \c_@@_punct_seq
\seq_gset_from_clist:Nn \c_@@_punct_seq
  {
    "2018 , "201C , "300C , "300E , "3014 , "FF08 , "FF3B , "FF5B ,
    "3008 , "300A , "3016 , "3010 ,
    "2014 , "2026 , "3001 , "3002 , "FF0C , "FF0E , "FF1A , "FF1B ,
    "FF01 , "FF1F , "FF05 , "3015 , "FF09 , "FF3D , "FF5D , "3009 ,
    "300B , "3017 , "3011 , "2019 , "201D , "300D , "300F
  }
%    \end{macrocode}
%
% \begin{macro}[internal]{\ctex_make_spa:nn}
% |#1| 是 \file{spa} 文件名,|#2| 是由 CJK 族名与字体构成的逗号列表。
%    \begin{macrocode}
\cs_new_protected_nopar:Npn \ctex_make_spa:nn #1#2
  {
    \iow_open:Nn \g_@@_spa_iow {#1}
    \clist_map_inline:nn {#2}
      { \@@_write_family:nn ##1 }
    \iow_close:N \g_@@_spa_iow
  }
\iow_new:N \g_@@_spa_iow
\cs_new_eq:NN \MAKESPA \ctex_make_spa:nn
%    \end{macrocode}
% \end{macro}
%
%    \begin{macrocode}
\cs_new_protected:Npn \@@_write_family:nn #1#2
  {
    \group_begin:
      \tex_font:D \l_@@_punct_font = "#2" ~ at ~ 100 pt \scan_stop:
      \l_@@_punct_font
      \clist_clear:N \l_@@_punct_bounds_clist
      \seq_map_inline:Nn \c_@@_punct_seq
        { \exp_args:No \@@_save_bounds:n { \int_use:N \XeTeXcharglyph ##1 } }
      \iow_now:Nx \g_@@_spa_iow
        {
          \token_to_str:N \@namedef { CJKpunct@#1@spaces }
%    \end{macrocode}
% 最后这三个逗号对 \pkg{CJKpunct} 来说是必要的。
%    \begin{macrocode}
            { \l_@@_punct_bounds_clist , , , }
        }
    \group_end:
  }
\cs_new_protected_nopar:Npn \@@_save_bounds:n #1
  {
    \clist_put_right:Nx \l_@@_punct_bounds_clist
      {
        \@@_calc_bounds:nn { \c_one }   {#1} ,
        \@@_calc_bounds:nn { \c_three } {#1}
      }
  }
\clist_new:N \l_@@_punct_bounds_clist
%    \end{macrocode}
%
% \pkg{CJKpunct} 要求的格式是边界空白宽度与 1\,em 的比值的一百倍。
%    \begin{macrocode}
\cs_new_nopar:Npn \@@_calc_bounds:nn #1#2
  {
    \fp_eval:n
      {
        round
          (
            \dim_to_decimal_in_unit:nn
              { 100 \XeTeXglyphbounds #1 ~ #2 }
              { 1 em }
          )
      }
  }
\ExplSyntaxOff
%</macro>
%    \end{macrocode}
%
% 下面是 \CTeX{} 定义的一些字体。
%    \begin{macrocode}
%<*make>
\input ctexspamacro %

\MAKESPA {ctexpunct.spa}
  {
    {adobezhsong}     {AdobeSongStd-Light} ,
    {adobezhhei}      {AdobeHeitiStd-Regular} ,
    {adobezhkai}      {AdobeKaitiStd-Regular} ,
    {adobezhfs}       {AdobeFangsongStd-Regular} ,
    {fandolzhsong}    {FandolSong} ,
    {fandolzhsongb}   {FandolSong-Bold} ,
    {fandolzhhei}     {FandolHei} ,
    {fandolzhheib}    {FandolHei-Bold} ,
    {fandolzhkai}     {FandolKai} ,
    {fandolzhfs}      {FandolFang} ,
    {founderzhsong}   {FZShuSong-Z01} ,
    {founderzhsongb}  {FZXiaoBiaoSong-B05} ,
    {founderzhhei}    {FZHei-B01} ,
    {founderzhheil}   {FZXiHeiI-Z08} ,
    {founderzhkai}    {FZKai-Z03} ,
    {founderzhfs}     {FZFangSong-Z02} ,
    {founderzhli}     {FZLiShu-S01} ,
    {founderzhyou}    {FZXiYuan-M01} ,
    {founderzhyoub}   {FZZhunYuan-M02} ,
    {ubuntuzhsong}    {AR PL UMing CN} ,
    {ubuntuzhhei}     {WenQuanYi Zen Hei} ,
    {ubuntuzhkai}     {AR PL UKai CN} ,
    {windowszhsong}   {SimSun} ,
    {windowszhhei}    {SimHei} ,
    {windowszhkai}    {KaiTi} ,
    {windowszhfs}     {FangSong} ,
    {windowszhli}     {LiSu} ,
    {windowszhyou}    {YouYuan} ,
    {windowszhyahei}  {Microsoft YaHei} ,
    {windowszhyaheib} {Microsoft YaHei Bold}
  }

\primitive\end
%</make>
%</spa>
%    \end{macrocode}
%
% \end{implementation}
%
% \Finale
%
\endinput
