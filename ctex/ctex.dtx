% \iffalse meta-comment
% !TeX program  = XeLaTeX
% !TeX encoding = UTF-8
%
% Copyright (C) 2003--2017
% CTEX.ORG and any individual authors listed elsewhere in this file.
% --------------------------------------------------------------------------
%
% This work may be distributed and/or modified under the
% conditions of the LaTeX Project Public License, either
% version 1.3c of this license or (at your option) any later
% version. This version of this license is in
%    http://www.latex-project.org/lppl/lppl-1-3c.txt
% and the latest version of this license is in
%    http://www.latex-project.org/lppl.txt
% and version 1.3 or later is part of all distributions of
% LaTeX version 2005/12/01 or later.
%
% This work has the LPPL maintenance status `maintained'.
%
% The Current Maintainers of this work are Leo Liu, Qing Lee and Liam Huang.
%
% --------------------------------------------------------------------------
%
%<*internal>
\iffalse
%</internal>
%<*readme>
ctex
====

`ctex` is a collection of macro packages and document classes
for LaTeX Chinese typesetting.

Authors and Contributors
------------------------

* Wu Lingyun <aloft@ctex.org>
* Jiang Jiang <gzjjgod@gmail.com>
* Wang Yue <yuleopen@gmail.com>
* Liu Haiyang <leoliu.pku@gmail.com>
* Li Yanrui <liyanrui.m2@gmail.com>
* Chen Zhichu <zhichu.chen@gmail.com>
* Li Qing <sobenlee@gmail.com>
* Liam Huang <liamhuang0205@gmail.com>

Contributing
------------

This package is a part of the [ctex-kit](https://github.com/CTeX-org/ctex-kit) project.

Issues and pull requests are welcome.

Copyright and Licence
---------------------

    Copyright (C) 2003--2017
    CTEX.ORG and any individual authors listed elsewhere in this file.
    ----------------------------------------------------------------------

    This work may be distributed and/or modified under the
    conditions of the LaTeX Project Public License, either
    version 1.3c of this license or (at your option) any later
    version. This version of this license is in
       http://www.latex-project.org/lppl/lppl-1-3c.txt
    and the latest version of this license is in
       http://www.latex-project.org/lppl.txt
    and version 1.3 or later is part of all distributions of
    LaTeX version 2005/12/01 or later.

    This work has the LPPL maintenance status `maintained'.

    The Current Maintainers of this work are Leo Liu, Qing Lee and Liam Huang.

    This package consists of the file  ctex.dtx,
                 and the derived files ctex.pdf,
                                       ctex.ins,
                                       ctex.sty,
                                       ctexsize.sty,
                                       ctexheading.sty,
                                       ctexart.cls,
                                       ctexbook.cls,
                                       ctexrep.cls,
                                       ctexbeamer.cls,
                                       ctxdoc.cls,
                                       ctexcap.sty,
                                       ctexhook.sty,
                                       ctexpatch.sty,
                                       ctex-c5size.clo,
                                       ctex-cs4size.clo,
                                       ctex-article.def,
                                       ctex-book.def,
                                       ctex-report.def,
                                       ctex-beamer.def,
                                       ctex-scheme-plain.def,
                                       ctex-scheme-plain-article.def,
                                       ctex-scheme-plain-book.def,
                                       ctex-scheme-plain-report.def,
                                       ctex-scheme-plain-beamer.def,
                                       ctex-scheme-chinese.def,
                                       ctex-scheme-chinese-article.def,
                                       ctex-scheme-chinese-book.def,
                                       ctex-scheme-chinese-report.def,
                                       ctex-scheme-chinese-beamer.def,
                                       ctex-name-gbk.cfg,
                                       ctex-name-utf8.cfg,
                                       ctex.cfg,
                                       ctexopts.cfg,
                                       ctex-engine-pdftex.def,
                                       ctex-engine-xetex.def,
                                       ctex-engine-luatex.def,
                                       ctex-engine-aptex.def,
                                       ctex-engine-uptex.def,
                                       c19rm.fd,
                                       c19sf.fd,
                                       c19tt.fd,
                                       c70rm.fd,
                                       c70sf.fd,
                                       c70tt.fd,
                                       ctex-fontset-windows.def,
                                       ctex-fontset-windowsnew.def,
                                       ctex-fontset-windowsold.def,
                                       ctex-fontset-adobe.def,
                                       ctex-fontset-fandol.def,
                                       ctex-fontset-mac.def,
                                       ctex-fontset-founder.def,
                                       ctex-fontset-ubuntu.def,
                                       ctexspa.def,
                                       ctexpunct.spa,
                                       ctexmakespa.tex,
                                       ctexspamacro.tex,
                                       ctxdocstrip.tex,
                                       zhadobefonts.tex,
                                       zhfandolfonts.tex,
                                       zhfounderfonts.tex,
                                       zhubuntufonts.tex,
                                       zhwindowsfonts.tex,
                                       translator-theorem-dictionary-ChineseGBK.dict,
                                       translator-theorem-dictionary-ChineseUTF8.dict, and
                                       README.md (this file).
%</readme>
%<*internal>
\fi
\begingroup
  \def\temp{LaTeX2e}
\expandafter\endgroup\ifx\temp\fmtname\else
\csname fi\endcsname
%</internal>
%<*install>

\input ctxdocstrip %

\preamble

    Copyright (C) 2003--2017
    CTEX.ORG and any individual authors listed in the documentation.
------------------------------------------------------------------------------

    This work may be distributed and/or modified under the
    conditions of the LaTeX Project Public License, either
    version 1.3c of this license or (at your option) any later
    version. This version of this license is in
       http://www.latex-project.org/lppl/lppl-1-3c.txt
    and the latest version of this license is in
       http://www.latex-project.org/lppl.txt
    and version 1.3 or later is part of all distributions of
    LaTeX version 2005/12/01 or later.

    This work has the LPPL maintenance status `maintained'.

    The Current Maintainers of this work are Leo Liu, Qing Lee and Liam Huang.

------------------------------------------------------------------------------

\endpreamble
\postamble

    This package consists of the file  ctex.dtx,
                 and the derived files ctex.pdf,
                                       ctex.ins,
                                       ctex.sty,
                                       ctexsize.sty,
                                       ctexheading.sty,
                                       ctexart.cls,
                                       ctexbook.cls,
                                       ctexrep.cls,
                                       ctexbeamer.cls,
                                       ctxdoc.cls,
                                       ctexcap.sty,
                                       ctexhook.sty,
                                       ctexpatch.sty,
                                       ctex-c5size.clo,
                                       ctex-cs4size.clo,
                                       ctex-article.def,
                                       ctex-book.def,
                                       ctex-report.def,
                                       ctex-beamer.def,
                                       ctex-scheme-plain.def,
                                       ctex-scheme-plain-article.def,
                                       ctex-scheme-plain-book.def,
                                       ctex-scheme-plain-report.def,
                                       ctex-scheme-plain-beamer.def,
                                       ctex-scheme-chinese.def,
                                       ctex-scheme-chinese-article.def,
                                       ctex-scheme-chinese-book.def,
                                       ctex-scheme-chinese-report.def,
                                       ctex-scheme-chinese-beamer.def,
                                       ctex-name-gbk.cfg,
                                       ctex-name-utf8.cfg,
                                       ctex.cfg,
                                       ctexopts.cfg,
                                       ctex-engine-pdftex.def,
                                       ctex-engine-xetex.def,
                                       ctex-engine-luatex.def,
                                       ctex-engine-aptex.def,
                                       ctex-engine-uptex.def,
                                       c19rm.fd,
                                       c19sf.fd,
                                       c19tt.fd,
                                       c70rm.fd,
                                       c70sf.fd,
                                       c70tt.fd,
                                       ctex-fontset-windows.def,
                                       ctex-fontset-windowsnew.def,
                                       ctex-fontset-windowsold.def,
                                       ctex-fontset-adobe.def,
                                       ctex-fontset-fandol.def,
                                       ctex-fontset-mac.def,
                                       ctex-fontset-founder.def,
                                       ctex-fontset-ubuntu.def,
                                       ctexspa.def,
                                       ctexpunct.spa,
                                       ctexmakespa.tex,
                                       ctexspamacro.tex,
                                       ctxdocstrip.tex,
                                       zhadobefonts.tex,
                                       zhfandolfonts.tex,
                                       zhfounderfonts.tex,
                                       zhubuntufonts.tex,
                                       zhwindowsfonts.tex,
                                       translator-theorem-dictionary-ChineseGBK.dict,
                                       translator-theorem-dictionary-ChineseUTF8.dict, and
                                       README.md.
\endpostamble

\generate
  {
    \usedir{tex/latex/ctex}
    \file{ctex.sty}                        {\from{\jobname.dtx}{style,ctex}}
    \file{ctexsize.sty}                    {\from{\jobname.dtx}{style,ctexsize}}
    \file{ctexheading.sty}                 {\from{\jobname.dtx}{style,ctexheading}}
    \file{ctexart.cls}                     {\from{\jobname.dtx}{class,article}}
    \file{ctexbook.cls}                    {\from{\jobname.dtx}{class,book}}
    \file{ctexrep.cls}                     {\from{\jobname.dtx}{class,report}}
    \file{ctexbeamer.cls}                  {\from{\jobname.dtx}{class,beamer}}
    \usepostamble\emptypostamble
    \file{ctxdoc.cls}                      {\from{\jobname.dtx}{ctxdoc}}
    \file{ctexcap.sty}                     {\from{\jobname.dtx}{ctexcap}}
    \file{ctexhook.sty}                    {\from{\jobname.dtx}{ctexhook}}
    \file{ctexpatch.sty}                   {\from{\jobname.dtx}{ctexpatch}}
    \file{ctex-c5size.clo}                 {\from{\jobname.dtx}{c5size}}
    \file{ctex-cs4size.clo}                {\from{\jobname.dtx}{cs4size}}
    \file{ctex-article.def}                {\from{\jobname.dtx}{heading,article}}
    \file{ctex-book.def}                   {\from{\jobname.dtx}{heading,book}}
    \file{ctex-report.def}                 {\from{\jobname.dtx}{heading,report}}
    \file{ctex-beamer.def}                 {\from{\jobname.dtx}{heading,beamer}}
    \file{ctex-scheme-plain.def}           {\from{\jobname.dtx}{scheme,generic,plain}}
    \file{ctex-scheme-plain-article.def}   {\from{\jobname.dtx}{scheme,article,plain}}
    \file{ctex-scheme-plain-book.def}      {\from{\jobname.dtx}{scheme,book,plain}}
    \file{ctex-scheme-plain-report.def}    {\from{\jobname.dtx}{scheme,report,plain}}
    \file{ctex-scheme-plain-beamer.def}    {\from{\jobname.dtx}{scheme,beamer,plain}}
    \file{ctex-scheme-chinese.def}         {\from{\jobname.dtx}{scheme,generic,chinese}}
    \file{ctex-scheme-chinese-article.def} {\from{\jobname.dtx}{scheme,article,chinese}}
    \file{ctex-scheme-chinese-book.def}    {\from{\jobname.dtx}{scheme,book,chinese}}
    \file{ctex-scheme-chinese-report.def}  {\from{\jobname.dtx}{scheme,report,chinese}}
    \file{ctex-scheme-chinese-beamer.def}  {\from{\jobname.dtx}{scheme,beamer,chinese}}
    \file{ctex-name-gbk.cfg}               {\from{\jobname.dtx}{name,GBK}}
    \file{ctex-name-utf8.cfg}              {\from{\jobname.dtx}{name,UTF8}}
    \file{ctex.cfg}                        {\from{\jobname.dtx}{config}}
    \file{ctexopts.cfg}                    {\from{\jobname.dtx}{ctexopts}}
    \file{ctex-engine-pdftex.def}          {\from{\jobname.dtx}{pdftex}}
    \file{ctex-engine-xetex.def}           {\from{\jobname.dtx}{xetex}}
    \file{ctex-engine-luatex.def}          {\from{\jobname.dtx}{luatex}}
    \file{ctex-engine-aptex.def}           {\from{\jobname.dtx}{aptex}}
    \file{ctex-engine-uptex.def}           {\from{\jobname.dtx}{uptex}}
    \file{c19rm.fd}                        {\from{\jobname.dtx}{rm,c19}}
    \file{c19sf.fd}                        {\from{\jobname.dtx}{sf,c19}}
    \file{c19tt.fd}                        {\from{\jobname.dtx}{tt,c19}}
    \file{c70rm.fd}                        {\from{\jobname.dtx}{rm,c70}}
    \file{c70sf.fd}                        {\from{\jobname.dtx}{sf,c70}}
    \file{c70tt.fd}                        {\from{\jobname.dtx}{tt,c70}}
    \file{ctex-fontset-windows.def}        {\from{\jobname.dtx}{fontset,windows}}
    \file{ctex-fontset-windowsnew.def}     {\from{\jobname.dtx}{fontset,windowsnew}}
    \file{ctex-fontset-windowsold.def}     {\from{\jobname.dtx}{fontset,windowsold}}
    \file{ctex-fontset-adobe.def}          {\from{\jobname.dtx}{fontset,adobe}}
    \file{ctex-fontset-fandol.def}         {\from{\jobname.dtx}{fontset,fandol}}
    \file{ctex-fontset-mac.def}            {\from{\jobname.dtx}{fontset,mac}}
    \file{ctex-fontset-founder.def}        {\from{\jobname.dtx}{fontset,founder}}
    \file{ctex-fontset-ubuntu.def}         {\from{\jobname.dtx}{fontset,ubuntu}}
    \file{translator-theorem-dictionary-ChineseGBK.dict}
                                           {\from{\jobname.dtx}{dict,theorem,GBK}}
    \file{translator-theorem-dictionary-ChineseUTF8.dict}
                                           {\from{\jobname.dtx}{dict,theorem,UTF8}}
    \file{ctexspa.def}
      {
        \from{\jobname.dtx}  {ctexspa}
        \from{ctexpunct.spa} {}
      }
    \usedir{tex/generic/ctex}
    \file{ctexmakespa.tex}                 {\from{\jobname.dtx}{spa,make}}
    \file{ctexspamacro.tex}                {\from{\jobname.dtx}{spa,macro}}
    \file{zhadobefonts.tex}                {\from{\jobname.dtx}{zhmap,adobe}}
    \file{zhfandolfonts.tex}               {\from{\jobname.dtx}{zhmap,fandol}}
    \file{zhfounderfonts.tex}              {\from{\jobname.dtx}{zhmap,founder}}
    \file{zhubuntufonts.tex}               {\from{\jobname.dtx}{zhmap,ubuntu}}
    \file{zhwindowsfonts.tex}              {\from{\jobname.dtx}{zhmap,windows}}
    \file{ctxdocstrip.tex}                 {\from{\jobname.dtx}{docstrip}}
%</install>
%<*internal>
    \usedir{source/latex/ctex}
    \file{\jobname.ins}                    {\from{\jobname.dtx}{install}}
%</internal>
%<*install>
    \nopreamble\nopostamble
    \usedir{doc/latex/ctex}
    \file{README.md}                       {\from{\jobname.dtx}{readme}}
  }

\catcode32=12\space

\Msg{*************************************************************}
\Msg{*                                                           *}
\Msg{* To finish the installation you have to move the following *}
\Msg{* file into proper directories searched by TeX:             *}
\Msg{*                                                           *}
\Msg{* The recommended directory is TDS:tex/latex/ctex           *}
\Msg{*                                                           *}
\Msg{*     ctex.sty                                              *}
\Msg{*     ctexsize.sty                                          *}
\Msg{*     ctexheading.sty                                       *}
\Msg{*     ctexart.cls                                           *}
\Msg{*     ctexbook.cls                                          *}
\Msg{*     ctexrep.cls                                           *}
\Msg{*     ctexbeamer.cls                                        *}
\Msg{*     ctxdoc.cls                                            *}
\Msg{*     ctexcap.sty                                           *}
\Msg{*     ctexhook.sty                                          *}
\Msg{*     ctexpatch.sty                                         *}
\Msg{*     ctex-c5size.clo                                       *}
\Msg{*     ctex-cs4size.clo                                      *}
\Msg{*     ctex-article.def                                      *}
\Msg{*     ctex-book.def                                         *}
\Msg{*     ctex-report.def                                       *}
\Msg{*     ctex-beamer.def                                       *}
\Msg{*     ctex-scheme-plain.def                                 *}
\Msg{*     ctex-scheme-plain-article.def                         *}
\Msg{*     ctex-scheme-plain-book.def                            *}
\Msg{*     ctex-scheme-plain-report.def                          *}
\Msg{*     ctex-scheme-plain-beamer.def                          *}
\Msg{*     ctex-scheme-chinese.def                               *}
\Msg{*     ctex-scheme-chinese-article.def                       *}
\Msg{*     ctex-scheme-chinese-book.def                          *}
\Msg{*     ctex-scheme-chinese-report.def                        *}
\Msg{*     ctex-scheme-chinese-beamer.def                        *}
\Msg{*     ctex-name-gbk.cfg                                     *}
\Msg{*     ctex-name-utf8.cfg                                    *}
\Msg{*     ctex.cfg                                              *}
\Msg{*     ctexopts.cfg                                          *}
\Msg{*     ctex-engine-pdftex.def                                *}
\Msg{*     ctex-engine-xetex.def                                 *}
\Msg{*     ctex-engine-luatex.def                                *}
\Msg{*     ctex-engine-aptex.def                                 *}
\Msg{*     ctex-engine-uptex.def                                 *}
\Msg{*     c19rm.fd                                              *}
\Msg{*     c19sf.fd                                              *}
\Msg{*     c19tt.fd                                              *}
\Msg{*     c70rm.fd                                              *}
\Msg{*     c70sf.fd                                              *}
\Msg{*     c70tt.fd                                              *}
\Msg{*     ctex-fontset-windows.def                              *}
\Msg{*     ctex-fontset-windowsnew.def                           *}
\Msg{*     ctex-fontset-windowsold.def                           *}
\Msg{*     ctex-fontset-adobe.def                                *}
\Msg{*     ctex-fontset-fandol.def                               *}
\Msg{*     ctex-fontset-mac.def                                  *}
\Msg{*     ctex-fontset-founder.def                              *}
\Msg{*     ctex-fontset-ubuntu.def                               *}
\Msg{*     ctexspa.def                                           *}
\Msg{*     ctexmakespa.tex                                       *}
\Msg{*     ctexspamacro.tex                                      *}
\Msg{*     ctxdocstrip.tex                                       *}
\Msg{*     zhadobefonts.tex                                      *}
\Msg{*     zhfandolfonts.tex                                     *}
\Msg{*     zhfounderfonts.tex                                    *}
\Msg{*     zhubuntufonts.tex                                     *}
\Msg{*     zhwindowsfonts.tex                                    *}
\Msg{*     translator-theorem-dictionary-ChineseGBK.dict         *}
\Msg{*     translator-theorem-dictionary-ChineseUTF8.dict        *}
\Msg{*                                                           *}
\Msg{* To produce the documentation run the file ctex.dtx        *}
\Msg{* through XeLaTeX.                                          *}
\Msg{*                                                           *}
\Msg{* Happy TeXing!                                             *}
\Msg{*                                                           *}
\Msg{*************************************************************}

\endbatchfile
%</install>
%<*internal>
\fi
%</internal>
%<*!(driver|readme|install|zhmap|spa|docstrip)>
%<*!(c19|c70|ctexspa|dict)>
%<class|style|ctexcap|ctexhook|ctexpatch|ctxdoc>\NeedsTeXFormat{LaTeX2e}
%<class|style|ctexcap|ctexhook|ctexpatch|ctxdoc>\RequirePackage{expl3}
%<+!driver>\GetIdInfo$Id$
%<ctxdoc>  {ctex documentation (CTEX)}
%<ctxdoc>\ProvidesExplClass{ctxdoc}
%<ctex>  {Chinese adapter in LaTeX (CTEX)}
%<ctex>\ProvidesExplPackage{\ExplFileName}
%<ctexsize>  {Chinese font size definition (CTEX)}
%<ctexsize>\ProvidesExplPackage{ctexsize}
%<ctexheading>  {Heading style modification (CTEX)}
%<ctexheading>\ProvidesExplPackage{ctexheading}
%<class&article>  {Chinese adapter for class article (CTEX)}
%<class&article>\ProvidesExplClass{ctexart}
%<class&book>  {Chinese adapter for class book (CTEX)}
%<class&book>\ProvidesExplClass{ctexbook}
%<class&report>  {Chinese adapter for class report (CTEX)}
%<class&report>\ProvidesExplClass{ctexrep}
%<class&beamer>  {Chinese adapter for class beamer (CTEX)}
%<class&beamer>\ProvidesExplClass{ctexbeamer}
%<ctexcap>  {Chinese adapter in LaTeX (CTEX)}
%<ctexcap>\ProvidesExplPackage{ctexcap}
%<ctexhook>  {Document and package hooks (CTEX)}
%<ctexhook>\ProvidesExplPackage{ctexhook}
%<ctexpatch>  {Patching commands (CTEX)}
%<ctexpatch>\ProvidesExplPackage{ctexpatch}
%<c5size>  {c5size option (CTEX)}
%<c5size>\ProvidesExplFile{ctex-c5size.clo}
%<cs4size>  {cs4size option (CTEX)}
%<cs4size>\ProvidesExplFile{ctex-cs4size.clo}
%<heading&article>  {Heading modification for article (CTEX)}
%<heading&article>\ProvidesExplFile{ctex-article.def}
%<heading&book>  {Heading modification for book (CTEX)}
%<heading&book>\ProvidesExplFile{ctex-book.def}
%<heading&report>  {Heading modification for report (CTEX)}
%<heading&report>\ProvidesExplFile{ctex-report.def}
%<heading&beamer>  {Heading modification for beamer (CTEX)}
%<heading&beamer>\ProvidesExplFile{ctex-beamer.def}
%<scheme&plain&generic>  {Plain scheme for generic (CTEX)}
%<scheme&plain&generic>\ProvidesExplFile{ctex-scheme-plain.def}
%<scheme&plain&article>  {Plain scheme for article (CTEX)}
%<scheme&plain&article>\ProvidesExplFile{ctex-scheme-plain-article.def}
%<scheme&plain&book>  {Plain scheme for book (CTEX)}
%<scheme&plain&book>\ProvidesExplFile{ctex-scheme-plain-book.def}
%<scheme&plain&report>  {Plain scheme for report (CTEX)}
%<scheme&plain&report>\ProvidesExplFile{ctex-scheme-plain-report.def}
%<scheme&plain&beamer>  {Plain scheme for beamer (CTEX)}
%<scheme&plain&beamer>\ProvidesExplFile{ctex-scheme-plain-beamer.def}
%<scheme&chinese&generic>  {Chinese scheme for generic (CTEX)}
%<scheme&chinese&generic>\ProvidesExplFile{ctex-scheme-chinese.def}
%<scheme&chinese&article>  {Chinese scheme for article (CTEX)}
%<scheme&chinese&article>\ProvidesExplFile{ctex-scheme-chinese-article.def}
%<scheme&chinese&book>  {Chinese scheme for book (CTEX)}
%<scheme&chinese&book>\ProvidesExplFile{ctex-scheme-chinese-book.def}
%<scheme&chinese&report>  {Chinese scheme for report (CTEX)}
%<scheme&chinese&report>\ProvidesExplFile{ctex-scheme-chinese-report.def}
%<scheme&chinese&beamer>  {Chinese scheme for beamer (CTEX)}
%<scheme&chinese&beamer>\ProvidesExplFile{ctex-scheme-chinese-beamer.def}
%<name&GBK>  {Caption with encoding GBK (CTEX)}
%<name&GBK>\ProvidesExplFile{ctex-name-gbk.cfg}
%<name&UTF8>  {Caption with encoding UTF8 (CTEX)}
%<name&UTF8>\ProvidesExplFile{ctex-name-utf8.cfg}
%<config>  {Configuration file (CTEX)}
%<config>\ProvidesExplFile{\ExplFileName.cfg}
%<ctexopts>  {Option configuration file (CTEX)}
%<ctexopts>\ProvidesExplFile{ctexopts.cfg}
%<pdftex>  {(pdf)LaTeX adapter (CTEX)}
%<pdftex>\ProvidesExplFile{ctex-engine-pdftex.def}
%<xetex>  {XeLaTeX adapter (CTEX)}
%<xetex>\ProvidesExplFile{ctex-engine-xetex.def}
%<luatex>  {LuaLaTeX adapter (CTEX)}
%<luatex>\ProvidesExplFile{ctex-engine-luatex.def}
%<aptex>  {Asian pTeX adapter (CTEX)}
%<aptex>\ProvidesExplFile{ctex-engine-aptex.def}
%<uptex>  {upTeX adapter (CTEX)}
%<uptex>\ProvidesExplFile{ctex-engine-uptex.def}
%<windows>  {Windows fonts definition (CTEX)}
%<windows>\ProvidesExplFile{ctex-fontset-windows.def}
%<windowsnew>  {Windows fonts definition for Vista or later version (CTEX)}
%<windowsnew>\ProvidesExplFile{ctex-fontset-windowsnew.def}
%<windowsold>  {Windows fonts definition for XP or earlier version (CTEX)}
%<windowsold>\ProvidesExplFile{ctex-fontset-windowsold.def}
%<adobe>  {Adobe fonts definition (CTEX)}
%<adobe>\ProvidesExplFile{ctex-fontset-adobe.def}
%<fandol>  {Fandol fonts definition (CTEX)}
%<fandol>\ProvidesExplFile{ctex-fontset-fandol.def}
%<mac>  {Mac OS X fonts definition (CTEX)}
%<mac>\ProvidesExplFile{ctex-fontset-mac.def}
%<founder>  {Founder fonts definition (CTEX)}
%<founder>\ProvidesExplFile{ctex-fontset-founder.def}
%<ubuntu>  {Ubuntu fonts definition (CTEX)}
%<ubuntu>\ProvidesExplFile{ctex-fontset-ubuntu.def}
%<!driver>  {\ExplFileDate}{2.4.9}{\ExplFileDescription}
%</!(c19|c70|ctexspa|dict)>
%<c19&rm>\ProvidesFile{c19rm.fd}%
%<c19&sf>\ProvidesFile{c19sf.fd}%
%<c19&tt>\ProvidesFile{c19tt.fd}%
%<c70&rm>\ProvidesFile{c70rm.fd}%
%<c70&sf>\ProvidesFile{c70sf.fd}%
%<c70&tt>\ProvidesFile{c70tt.fd}%
%<ctexspa>\ProvidesFile{ctexspa.def}%
%<dict&theorem&GBK>\ProvidesDictionary{translator-theorem-dictionary}{ChineseGBK}%
%<dict&theorem&UTF8>\ProvidesDictionary{translator-theorem-dictionary}{ChineseUTF8}%
%<c19|c70>  [2017/04/01 v2.4.9 Chinese font definition (CTEX)]
%<ctexspa>  [2017/04/01 v2.4.9 Space info for CJKpunct (CTEX)]
%<dict&theorem>  [2017/04/01 v2.4.9 Chinese translation for theorem name (CTEX)]
%</!(driver|readme|install|zhmap|spa|docstrip)>
%<*driver>
\documentclass{ctxdoc}
\begin{document}
  \DocInput{\jobname.dtx}
  \IndexLayout
  \PrintChanges
  \PrintIndex
\end{document}
%</driver>
%
% \fi
%
% \changes{v2.0}{2014/03/06}{应用 \LaTeXiii{} 重新整理代码。}
% \changes{v2.0}{2014/03/12}{删除 \file{c19gbsn.fd} 和 \file{c19gkai.fd}。}
% \changes{v2.1}{2015/05/18}{将章节标题设置功能提取到可以独立使用的宏包
%   \pkg{ctexheading} 中。}
% \changes{v2.2}{2015/06/24}{不再依赖 \pkg{etoolbox} 宏包。}
% \changes{v2.4}{2015/02/19}{加强 \pkg{beamer} 宏包支持。}
%
%
% \CheckSum{5896}
%
% \CharacterTable
%  {Upper-case    \A\B\C\D\E\F\G\H\I\J\K\L\M\N\O\P\Q\R\S\T\U\V\W\X\Y\Z
%   Lower-case    \a\b\c\d\e\f\g\h\i\j\k\l\m\n\o\p\q\r\s\t\u\v\w\x\y\z
%   Digits        \0\1\2\3\4\5\6\7\8\9
%   Exclamation   \!     Double quote  \"     Hash (number) \#
%   Dollar        \$     Percent       \%     Ampersand     \&
%   Acute accent  \'     Left paren    \(     Right paren   \)
%   Asterisk      \*     Plus          \+     Comma         \,
%   Minus         \-     Point         \.     Solidus       \/
%   Colon         \:     Semicolon     \;     Less than     \<
%   Equals        \=     Greater than  \>     Question mark \?
%   Commercial at \@     Left bracket  \[     Backslash     \\
%   Right bracket \]     Circumflex    \^     Underscore    \_
%   Grave accent  \`     Left brace    \{     Vertical bar  \|
%   Right brace   \}     Tilde         \~}
%
% \GetFileId{ctex.sty}%
%
% \title{\bfseries \CTeX{} 宏集手册}
% \author{\href{http://www.ctex.org}{CTEX.ORG}}
% \date{\filedate\qquad\fileversion\thanks{\ctexkitrev{\ExplFileVersion}.}}
% \maketitle
%
% \begin{abstract}
% \CTeX{} 宏集是面向中文排版的通用 \LaTeX{} 排版框架,为中文 \LaTeX{} 文档
% 提供了汉字输出支持、标点压缩、字体字号命令、标题文字汉化、中文版式调整、数字
% 日期转换等支持功能,可适应论文、报告、书籍、幻灯片等不同类型的中文文档。
%
% \CTeX{} 宏集支持 \LaTeX、\pdfLaTeX、\XeLaTeX、\LuaLaTeX、\upLaTeX{} 等多种不同
% 的编译方式,并为它们提供了统一的界面。主要功能由宏包 \pkg{ctex} 和中文文档类
% \cls{ctexart}、\cls{ctexrep}、\cls{ctexbook} 和 \cls{ctexbeamer} 实现。
% \end{abstract}
%
% \tableofcontents
%
% \clearpage
% \setlength{\parskip}{0.8ex}
%
% \begin{documentation}
%
% \section{介绍}
%
% \subsection*{历史}
%
% \CTeX{} 宏集的源头有两个:一是王磊编写的 \cls{cjkbook} 文档类,二是吴凌云编写的
% \file{GB.cap}。
% 这些工作没有经过认真、系统的设计,也没有用户文档,不利于维护和改进。
%
% 2003 年,吴凌云使用 \pkg{doc} 和 \pkg{DocStrip} 重构了整个工程,并增加了许多新的功能,
% 称为 \pkg{ctex} 宏包。2007 年,oseen 和王越在 \pkg{ctex} 宏包的基础上,
% 增加了对 UTF-8 编码的支持,开发出了 \pkg{ctexutf8} 宏包。
%
% 2009 年 5 月,我们在 Google Code 建立了 ctex-kit 项目^^A
% \footnote{\nolinkurl{http://code.google.com/p/ctex-kit/}},
% 对 \pkg{ctex} 宏包及相关脚本进行了整合,并加入了对 \XeTeX{} 引擎的支持。
% 在开发新版本时,考虑到合作开发和调试的方便,我们放弃了 \pkg{doc} 和 \pkg{DocStrip},
% 采取了直接编写宏包代码的方式。
%
% 2014 年 3 月,为了适应 \LaTeX{} 的最新发展,特别是 \LaTeXiii{} 的逐渐成熟,李清用
% \LaTeXiii{} 重构了整个宏包的代码,并重新使用 \pkg{doc} 和 \pkg{DocStrip} 工具进行代码
% 的管理,升级版本号为 2.0,并改称 \CTeX{} 宏集。
%
% 2015 年 3 月,由于 Google Code 即将停止服务,ctex-kit 项目迁移至
% \href{https://github.com/CTeX-org/ctex-kit}{GitHub}^^A
% \footnote{\url{https://github.com/CTeX-org/ctex-kit}}。
%
% 最初,Knuth 在设计开发 \TeX{} 的时候没有考虑到多国语言支持,特别是对多字节的中日韩
% 语言的支持。这使得 \TeX{} 以至后来的 \LaTeX{} 对中文的支持一直不是很好。即使在
% \pkg{CJK} 宏包解决了中文字符处理的问题以后,中文用户使用 \LaTeX{} 仍然要面对许
% 多困难。
% 这些困难里,以章节标题的中文化为最。由于中文和西文语言习惯的差异,用户很难使用标准
% 文档类中的代码结构来表达中文标题。于是,用户不得不对标准文档类做较大的修改。
% 除此之外,日期格式、首行缩进、中文字号和字距等细节问题,也需要精细的调校。
% 我们设计 \CTeX{} 宏集的目的之一就是解决这些 \LaTeX{} 文档的汉化难题。
%
% 另一方面,随着 \TeX{} 引擎和 \LaTeX{} 宏包的不断发展,\LaTeX{} 的中文支持方式从早期的
% 专用系统(如 \pkg{CCT})发展为适用于不同引擎的多种方式^^A
% \footnote{比如:\pdfTeX{} 引擎下的 \pkg{CJK}、\pkg{zhmCJK}宏包,
% \XeTeX{} 引擎下的 \pkg{xeCJK} 宏包和 \LuaTeX{} 引擎下的 \pkg{LuaTeX-ja} 宏
% 包}。这些方式的适用情况和使用方式有不少细节上的差异,同时操作系统的不同、语言环境的不同等
% 客观情况又进一步带来了更多的细节差异。我们设计 \CTeX{} 宏集的另一个主要目的就是
% 尽可能消除这些差异带来的影响,使用户能够以一个统一的接口来使用不同的中文支持方式,
% 使得同一份文档能够在不同环境下交换使用。
%
% \CTeX{} 宏集的许多实现细节离不开热心朋友们在 \url{bbs.ctex.org} 论坛上的讨论,
% 在此对参与讨论的朋友们表示感谢。
%
% \subsection*{关于宏集名字的说明}
%
% \CTeX{} 之名是英文单词 China(中国)或 Chinese(中文)的首字母“C”与 “\TeX{}”
% 结合而成的。在纯文本环境下,该名字应写作“CTeX”。
%
% \CTeX{} 宏集是由 \href{http://bbs.ctex.org}{\CTeX{} 社区} 发起并维护的
% \LaTeX{} \emph{宏包和文档类的集合}。
% 社区另有发布名为 \href{http://www.ctex.org/CTeX}{\CTeX{} 套装}
% 的 \TeX{} 发行版,与本文档所述的 \CTeX{} 宏集并非是同一事物。
%
% \pkg{ctex} 则是本宏集中的 \pkg{ctex.sty} 的名字。这一完全小写的名称,在过去
% 也被用来指代整个 \CTeX{} 宏集,不过现在则特指 \pkg{ctex.sty} 这一宏包。
% 在不引起歧义的情况下,它也可以沿用过去的习惯,代指整个宏集。
%
% \section{简明教程}
%
% \subsection{\CTeX{} 宏集的组成}
%
% 为了适应用户不同的需求,我们将 \CTeX{} 宏集的主要功能分散在四个中文文档类和
% 三个宏包当中,具体的组成见表~\ref{tab:ctex}。
%
% \begin{table}[htbp]
% \centering
% \caption{\CTeX{} 宏集的组成}\label{tab:ctex}
% \begin{tabularx}{\linewidth}{llX}
% \toprule
%   类别   & 文件 & 说明 \\
% \midrule
%   文档类 & \file{ctexart.cls}  & 标准文档类 \cls{article} 的汉化版本,一般适用于
%                                  短篇幅的文章 \\
%          & \file{ctexrep.cls}  & 标准文档类 \cls{report} 的汉化版本,一般适用于
%                                  中篇幅的报告 \\
%          & \file{ctexbook.cls} & 标准文档类 \cls{book} 的汉化版本,一般适用于
%                                  长篇幅的书籍 \\
%          & \file{ctexbeamer.cls} & 文档类 \cls{beamer} 的汉化版本,适用于
%                                    幻灯片演示 \\
% \midrule
%   宏包   & \file{ctex.sty}     & 提供全部功能,但\emph{默认不开启章节标题设置功能},
%                                  需要使用 \opt{heading} 选项来开启 \\
%          & \file{ctexsize.sty} & 定义和调整中文字号,在 \pkg{ctex} 宏包
%                                  或 \CTeX{} 中文文档类之外单独调用 \\
%          & \file{ctexheading.sty} & 提供章节标题设置功能(见 \ref{sec:secstyle}
%                                     节),在 \pkg{ctex} 宏包或 \CTeX{} 中文
%                                     文档类之外单独调用 \\
% \bottomrule
% \end{tabularx}
% \end{table}
%
% \subsection{\CTeX{} 宏集的安装和更新}
% \label{subsec:easy-ins}
%
% \CTeX{} 宏集依赖的宏包和宏集已被最常见的 \TeX{} 发行版 \TeXLive{} 和 \MiKTeX{}
% 所收录。如果本地安装 \TeXLive{} 或 \MiKTeX{} 不是完全版本,就可能需要通过这
% 两个发行版提供的\emph{宏包管理器}来安装宏包。
%
% \TeXLive{} 的宏包管理器是 tlmgr。用户可以在系统命令行中^^A
% \footnote{Windows 系统的命令行是 CMD 命令提示符,你可以使用 Win + R 组合键^^A
% 打开“运行”对话框,然后输入 cmd 确认打开命令提示符窗口。}^^A
% 执行
% \begin{frameverb}
%   tlmgr gui
% \end{frameverb}
% 启动管理器的图形界面(Windows 用户也可以通过开始菜单的
% TeX Live 2015 \ding{212} TeX Live Manager 打开)。
% 连接上远程仓库之后,搜索 ctex 安装即可。
% tlmgr 的图形界面使用 Perl 编写,容易造成系统假死。遇到这种问题的用户,
% 也可以直接在系统命令行执行
% \begin{frameverb}
%   tlmgr install ctex
% \end{frameverb}
% 来安装 \CTeX{} 宏集\footnote{*nix 用户可能需要超级用户权限才能正确安装宏集。}。
%
% \MiKTeX{} 的宏包管理器是 mpm (\MiKTeX{} Package Manager)。用户可以在开始菜单
% 找到 MiKTeX \ding{212} Maintenance (Admin) \ding{212} Package Manager (Admin),
% 打开管理器,连接上远程仓库之后,搜索 ctex 安装即可。
%
% 若希望了解 \CTeX{} 宏集具体的依赖情况和手工安装宏集的方法,
% 请参阅第 \ref{sec:dep-ins}~节。
%
% 当宏包发布新版本,并被发行版在远程仓库安装之后,在本地就可以通过宏包管理器来
% 取得新版本。
%
% 对于 \TeXLive{},可以在 tlmgr 的图形界面点击“更新全部已安装的”按钮或者在
% 命令行执行
% \begin{frameverb}
%   tlmgr update --all
% \end{frameverb}
% 来完整更新已安装的宏包。
%
% 对于 \MiKTeX{},在开始菜单找到
% MiKTeX \ding{212} Maintenance (Admin) \ding{212} Update (Admin),
% 按照界面说明更新即可。
%
% \subsection{使用 \CTeX{} 文档类}
%
% \emph{如果用户需要在标准文档类的基础上添加中文支持和中文版式支持,我们建议用户使用 \CTeX{}
% 宏集提供的四个中文文档类。}
%
% \CTeX{} 宏集提供了四个中文文档类:\cls{ctexart}、\cls{ctexrep}、\cls{ctexbook} 和
% \cls{ctexbeamer},分别对应 \LaTeX{} 的标准文档类 \cls{article}、\cls{report}、
% \cls{book} 和 \cls{beamer}。使用它们的时候,需要将涉及到的所有源文件使用 UTF-8
% 编码保存\footnote{使用 (pdf)\LaTeX{} 时也能够使用 GBK 编码,但不推荐。(见
% \ref{subs:encoding}~节)}。
%
% \begin{ctexexam}
%   \documentclass[UTF8]{ctexart}
%   \begin{document}
%   中文文档类测试。你需要将所有源文件保存为 UTF-8 编码。
%
%   你可以使用 XeLaTeX、LuaLaTeX 或 upLaTeX 编译,也可以使用 (pdf)LaTeX 编译。
%   推荐使用 XeLaTeX 或 LuaLaTeX 编译。
%   \end{document}
% \end{ctexexam}
%
% 以下是使用 \cls{ctexbeamer} 文档类编写中文演示文稿的一个示例。
% \begin{ctexexam}
%   \documentclass[UTF8]{ctexbeamer}
%   \begin{document}
%   \begin{frame}{中文演示文档}
%   \begin{itemize}
%     \item 你需要将所有源文件保存为 UTF-8 编码
%     \item 你可以使用 XeLaTeX、LuaLaTeX 或 upLaTeX 编译
%     \item 也可以使用 (pdf)LaTeX 编译
%     \item 推荐使用 XeLaTeX 或 LuaLaTeX 编译
%   \end{itemize}
%   \end{frame}
%   \end{document}
% \end{ctexexam}
%
% \subsection{使用 \pkg{ctex} 宏包}
%
% \emph{用户在使用非标准文档类时,如果需要添加中文支持或中文版式支持,则可以使用 \pkg{ctex}
% 宏包。}
%
% 有些文档类是建立在 \LaTeX{} 标准文档类之上开发的。这时,给 \pkg{ctex} 宏包
% 加上 \opt{heading} 选项,可以将章节标题设置为中文风格。
% \begin{ctexexam}
%   \documentclass{ltxdoc}
%   \usepackage[UTF8, heading = true]{ctex}
%   \begin{document}
%   \section{简介}
%   章节标题中文化的 \LaTeX{} 手册。
%   \end{document}
% \end{ctexexam}
%
% \section{宏包选项与 \tn{ctexset} 命令}
% \label{sec:options}
%
% \CTeX{} 宏集已经尽可能就中文的行文和版式习惯做了调整和配置,通常而言,这些配置
% 已经够用。因此,除非必要,我们不建议普通用户修改这些默认配置。如果你认为 \CTeX{} 宏集
% 的默认配置还可以完善,可以在项目主页上%
% \href{https://github.com/CTeX-org/ctex-kit/issues}{提交 issue},
% 向我们反映,我们会酌情在后续版本中予以改进。
%
% 不过,\CTeX{} 宏集也提供了一系列选项。用户可以使用这些选项来控制 \CTeX{} 宏集的行为。
% 具体来说,这些选项里,有的以传统的方式提供,
% 也有的以 \meta{key}|=|\meta{value} 的形式提供。对于以键值对形式提供的选项,
% 在下面的说明中使用\textbf{粗体}来表示 \CTeX{} 的默认设置。
%
% 另一方面,这些选项可以分为以下三类:
% \begin{itemize}
% \item 名字后带有 \rexptarget\rexpstar{} 号的选项,只能作为宏包/文档类选项,需要
%   在引入宏包/文档类的时候指定;
% \item 名字后带有 \exptarget\expstar{} 号的选项,只能通过 \CTeX{} 宏集提供的
%   用户接口 \tn{ctexset} 来设定;
% \item 名字后不带有特殊符号的选项,既可以作为宏包/文档类选项,也可以通过
%   \tn{ctexset} 来设定。
% \end{itemize}
% 后续文档将在使用说明中对某些特殊的选项加以说明。
%
% \begin{function}[added=2014-03-18]{\ctexset}
%   \begin{syntax}
%     \tn{ctexset} \Arg{键值列表}
%   \end{syntax}
% 是 \CTeX{} 宏集的通用控制命令,用来在宏包载入后控制宏包的各项功能。
% \tn{ctexset} 的参数是一个键值列表,以通用的接口完成各项设置。
% \end{function}
%
% \tn{ctexset} 的参数是一组由逗号分隔的选项列表,列表中的选项通常是一个
% \meta{key}|=|\meta{value} 格式的定义。例如设置摘要与参考文献标题名称
% (\ref{subs:capname}~节)就可以使用:
% \begin{ctexexam}[labelref=exam:capname]
%   \ctexset{
%     abstractname = {本文概要},
%     bibname      = {文\quad 献}
%   }
% \end{ctexexam}
%
% \tn{ctexset} 采用 \LaTeXiii{} 风格的键值设置,支持不同类型的选项与层次化的选
% 项设置,相关示例见 \ref{sec:secstyle}~节。
%
% \section{编译方式、编码与中文字库}
% \label{sec:chinese}
%
% \subsection{编译方式}
% \label{subs:compile}
%
% \CTeX{} 宏集会根据用户使用的编译方式\footnote{\LaTeX、\pdfLaTeX、\XeLaTeX、
% \LuaLaTeX{} 及 \upLaTeX。},在底层选择不同的中文支持方式(见
% 表~\ref{tab:chinese-support})。
%
% \begin{table}[htbp]
% \centering
% \begin{threeparttable}
% \caption{\CTeX{} 宏集的中文支持方式}
% \label{tab:chinese-support}
% \begin{tabular}{*6c}
%   \toprule
%   编译方式 & (pdf)\LaTeX & \XeLaTeX & \LuaLaTeX & \upLaTeX\tnote{*} \\
%   \midrule
%   支持宏包 & \pkg{CJK} & \pkg{xeCJK} & \pkg{LuaTeX-ja} & 原生 \\
%   \bottomrule
% \end{tabular}
% \begin{tablenotes}
% \item[*] p\LaTeX-ng(或称 \ApLaTeX)与 \upLaTeX{} 兼容。使用 p\LaTeX-ng 编译
% 时,\pkg{ctex} 采用与 \upLaTeX{} 相同的设置。
% \end{tablenotes}
% \end{threeparttable}
% \end{table}
%
% 不同的编译方式和中文支持方式会在一定程度上影响 \CTeX{} 宏集的行为,比如宏包对
% 编码的处理。
% 在用户使用 \XeLaTeX、\LuaLaTeX{} 及 \upLaTeX{} 编译时,\CTeX{} 宏集使用(且仅
% 能使用)UTF-8 编码;而因为历史原因,在用户使用 \LaTeX{} 及 \pdfLaTeX{} 编译
% 时,宏包默认使用 GBK 编码。用户需要\emph{保证编译方式、
% 源文件编码、宏包编码选项三者一致}。
% 关于宏包编码选项,可以参考 \ref{subs:encoding}~节。
%
% 除去文档编码之外,选择不同的编译方式还可能影响 \CTeX{} 宏集对字体选择、空格处理、
% 标点处理的处理。具体的影响将在本文档后续内容中进行阐述。
%
% \subsection{中文编码}
% \label{subs:encoding}
%
% \begin{function}[rEXP]{GBK, UTF8}
%   指明编写文档时使用的编码格式。\CTeX{} 宏集无法检测用户编写文档时使用的编码格式,因此
%   需要用户自行指定编码。\emph{我们建议用户总是使用 UTF-8 编码,并显式指定 \opt{UTF8}
%   选项,并使用 \XeLaTeX、\LuaLaTeX{} 或 \upLaTeX{} 编译。}
%
%   使用 \XeLaTeX、\LuaLaTeX{} 或 \upLaTeX{} 编译时,\CTeX{} 宏集强制使用 UTF-8
%   编码,此时 \opt{GBK} 选项无效。
%   使用 (pdf)\LaTeX{} 编译时,\CTeX{} 宏集默认启用 \opt{GBK} 选项;不过,用户也可以
%   显式声明 \opt{UTF8} 选项,使 \CTeX{} 宏集工作在 UTF-8 编码下。
% \end{function}
%
% \subsection{中文字库}
% \label{subs:options-CJK-font}
%
% 以往,为 \LaTeX{} 文档配置中文支持是一件相当繁琐的事情。默认情况下,
% \CTeX{} 宏集能自动检测用户使用的编译方式(参见 \ref{subs:compile}~节)和
% 操作系统\footnote{\CTeX{} 宏集现在能够识别 Mac~OS~X 系统以及 Windows 系统。},
% 选择合适的底层支持和字库,从而简化配置过程。自动配置的情况参见
% 表~\ref{tab:default-font-select}。
%
% \begin{table}[htbp]
% \centering
% \begin{threeparttable}
% \caption{\CTeX{} 宏集自动配置字体策略}
% \label{tab:default-font-select}
% \begin{tabular}{*{5}{c}}
%   \toprule
%              & Mac OS X & Windows New\tnote{1} & Windows Old\tnote{2} &
%               其他 \\
%   \midrule
%   \XeLaTeX{} & \makecell{\pkg{xeCJK}\\华文字库} &
%     \makecell{\pkg{xeCJK}\\中易字库 + 微软雅黑} & \makecell{\pkg{xeCJK}\\中易字库} &
%     \makecell{\pkg{xeCJK}\\Fandol 字库\tnote{3}} \\
%   \cmidrule(lr){1-5}
%   \LuaLaTeX{}\tnote{4} & \makecell{\pkg{LuaTeX-ja}\\华文字库} &
%     \makecell{\pkg{LuaTeX-ja}\\中易字库 + 微软雅黑} &
%     \makecell{\pkg{LuaTeX-ja}\\中易字库} &
%     \makecell{\pkg{LuaTeX-ja}\\Fandol 字库} \\
%   \cmidrule(lr){1-5}
%   \pdfLaTeX{} & 不可用\tnote{5} &
%     \makecell{\pkg{CJK} + \pkg{zhmetrics}\\中易字库 + 微软雅黑} &
%     \makecell{\pkg{CJK} + \pkg{zhmetrics}\\中易字库} & 不可用\tnote{5} \\
%   \cmidrule(lr){1-5}
%   \makecell{\LaTeX{} + \\\dvipdfmx{}} & 不可用\tnote{6} &
%     \makecell{\pkg{CJK} + \pkg{zhmetrics}\\中易字库 + 微软雅黑} &
%     \makecell{\pkg{CJK} + \pkg{zhmetrics}\\中易字库} &
%     \makecell{\pkg{CJK} + \pkg{zhmetrics}\\Fandol 字库} \\
%   \cmidrule(lr){1-5}
%   \makecell{\upLaTeX{} + \\\dvipdfmx{}} & 不可用\tnote{6} &
%     \makecell{\pkg{zhmetrics-uptex}\\中易字库 + 微软雅黑} &
%     \makecell{\pkg{zhmetrics-uptex}\\中易字库} &
%     \makecell{\pkg{zhmetrics-uptex}\\Fandol 字库} \\
%   \bottomrule
% \end{tabular}
% \begin{tablenotes}
%   \item [1] Windows Vista 及以后的 Windows 操作系统。
%   \item [2] Windows XP 及以前的 Windows 操作系统。
%   \item [3] 由马起园、苏杰、黄晨成等人开发的开源中文字体,
%     参见:\url{https://github.com/clerkma/fandol-fonts}。
%   \item [4] \LuaLaTeX{} 编译时使用 \pkg{LuaTeX-ja} 宏包。对此,
%     第 \ref{sec:lualatex-chinese}~节有特别说明。
%   \item [5] 受 \pdfTeX{} 的限制,无法嵌入 OpenType 字体。
%   \item [6] 目前受 \dvipdfmx{} 的限制,Mac OS X 系统上的黑体和仿宋无法读取。
% \end{tablenotes}
% \end{threeparttable}
% \end{table}
%
% 通常,由 \CTeX{} 宏集进行的自动配置已经足够使用,无需用户手工干预;但
% 是 \CTeX{} 仍然提供了一系列选项,供在 \CTeX{} 的自动选择机制因为
% 意外情况失效,或者在用户有特殊需求的情况下使用。\emph{除非必要,用户不
% 应使用这些选项。}
%
% \begin{function}[rEXP,updated=2014-03-08]{zhmap}
%   \begin{syntax}
%     zhmap = <\TTF|zhmCJK>
%   \end{syntax}
%   指定字体映射机制。本选项只在使用 \pdfLaTeX/\LaTeX{} 编译时有意义。
% \end{function}
% \begin{optdesc}
%   \item[true] 使用 \pkg{zhmetrics} 宏包,将 CJK 字库通过 \tn{special}
%   命令映射到 \file{.ttf} 文件。
%
%   \item[false] 使用传统的 CJK 字库(Type 1)^^A
%   \footnote{如果需要使用自定义的字体映射文件,或者希望使用 Type1 字库,请禁用本选项。}。
%
%   \item[zhmCJK] 载入 \pkg{zhmCJK} 宏包^^A
%   \footnote{\pkg{zhmCJK} 宏包基于 \pkg{zhmetrics} 和 \pkg{CJK} 宏包,提供与
%   \pkg{xeCJK} 宏包类似的用户接口。}^^A
%   ,由 \pkg{zhmCJK} 宏包提供从 CJK 字库到 \file{.ttf} 的映射。
% \end{optdesc}
%
% \begin{function}[added=2014-03-08]{fontset}
%   \begin{syntax}
%     fontset =^^A
% <none|adobe|fandol|founder|mac|ubuntu|windows|windowsnew|windowsold|...>
%   \end{syntax}
%   指定 \CTeX{} 宏集加载的字库。
%
%   如果没有指定 \opt{fontset} 的值,\CTeX{} 宏集将自动检测用户使用的操作系统,配置
%   相应的字体(参见表~\ref{tab:default-font-select})。
% \end{function}
%
% \CTeX{} 预定义了以下六种中文字库。
%
% \begin{optdesc}
%   \item[adobe] 使用 Adobe 公司的四款中文字体,\emph{不支持 \pdfLaTeX}。
%   \item[fandol] 使用 Fandol 中文字体,\emph{不支持 \pdfLaTeX}。
%   \item[founder] 使用方正公司的中文字体。
%   \item[mac] 使用 Mac OS X 系统下的华文字体,\emph{不支持 (pdf)\LaTeX}。
%   \item[ubuntu] 使用 Ubuntu 系统下的文泉驿和文鼎字体。
%   \item[windows] 使用简体中文 Windows 系统下的中文字体,自动判断 Windows 系
%   统版本,采用 |windowsnew| 或 |windowsold| 的设置。
%   \item[windowsnew] 使用简体中文 Windows Vista 或之后系统下的中易字体和微软
%   雅黑字体。
%   \item[windowsold] 使用简体中文 Windows XP 或之前系统下的中易字体。
% \end{optdesc}
%
% 注意:使用 (pdf)\LaTeX{} 编译的时候,若设置 \opt{zhmap = false}(比如需要
% 使用 \LaTeX{} + Dvips 编译),则需要按照传统方式^^A
% \footnote{可以使用 \pkg{zhmetrics} 宏包提供的脚本
% \href{https://github.com/CTeX-org/ctex-kit/blob/master/zhmetrics/CTeXFonts.lua}
% {\file{CTeXFonts.lua}}。}^^A
% 在本地安装好 CJK 字体。
%
% 如果不想使用 \CTeX{} 预定义的中文字库,可以设置 \opt{fontset} 为下述值之一。
%
% \begin{optdesc}
%   \item[none] 不配置中文字体,需要用户自己配置。
%   \item[\meta{name}] 这里 \meta{name} 为自定义的名字。
%   \CTeX{} 宏集将载入名为 |ctex-fontset-|\meta{name}|.def| 的文件作为字体配置
%   文件。因此,请先保证文件的存在。可以在当前工作目录或者本地 \texttt{TDS} 目录
%   树下合适位置建立一个名为 |ctex-fontset-|\meta{name}|.def| 的文件,在这个文件
%   里面自定义中文字体。然后通过使用 |fontset=|\meta{name} 选项来调用它。字体配置
%   文件的具体写法可以参考 \CTeX{} 宏集 \texttt{fontset} 目录下的字体配置文件。
% \end{optdesc}
%
% 注意:\emph{如果希望使用 \tn{ctexset} 在导言区指定字库,则需要先在宏包/文档类选项中指定
% \opt{fontset = none}}。例如:
% \begin{ctexexam}
%   \documentclass[fontset = none]{ctexart}
%   \ctexset{fontset = founder}
%   \begin{document}
%   在文档类选项中声明 \verb|fontset = none|,随后在导言区用 \verb|\ctexset|
%   指定字体。
%   \end{document}
% \end{ctexexam}
%
% \CTeX{} 宏集预定义的中文字库还定义了一些字体命令。
%
% \begin{optdesc}
%   \item[\tn{songti}] 宋体,CJK 等价命令 |\CJKfamily{zhsong}|。
%   \item[\tn{heiti}] 黑体,CJK 等价命令 |\CJKfamily{zhhei}|。
%   \item[\tn{fangsong}] 仿宋,CJK 等价命令 |\CJKfamily{zhfs}|。
%   \item[\tn{kaishu}] 楷书,CJK 等价命令 |\CJKfamily{zhkai}|。
% \end{optdesc}
% 其中 \tn{fangsong} 在 \opt{ubuntu} 字库中没有定义。在
% \opt{windows} 和 \opt{founder} 字库中,还有 \tn{lishu} 和 \tn{youyuan}。
% \begin{optdesc}
%   \item[\tn{lishu}] 隶书,CJK 等价命令 |\CJKfamily{zhli}|。
%   \item[\tn{youyuan}] 圆体,CJK 等价命令 |\CJKfamily{zhyou}|。
% \end{optdesc}
% 在 \opt{windowsnew} 字库中,还有 \tn{yahei}。
% \begin{optdesc}
%   \item[\tn{yahei}] 微软雅黑,CJK 等价命令 |\CJKfamily{zhyahei}|。
% \end{optdesc}
%
% \section{排版格式设定}
% \label{sec:format-settings}
% \subsection{文档默认字号}
% \label{subs:options-class}
%
% \begin{function}[rEXP,added=2015-05-06]{zihao}
%   \begin{syntax}
%     zihao = <-4|5|false>
%   \end{syntax}
%   将文章默认字号(\tn{normalsize})设置为小四号字或五号字,
%   具体情况见表 \ref{tab:fontsize}。\opt{false} 禁用本功能。
%   本选项可以用于四个 \CTeX{} 文档类和 \pkg{ctex} 宏包,
%   也可以用于 \pkg{ctexsize} 宏包。
%
%   \opt{scheme = chinese} 时,对标准文档类默认值为 |5|,即设置
%   \tn{normalsize} 为五号字;对 \cls{beamer} 则为 |false|,使用文档类原有的设
%   置。
% \end{function}
%
% \begin{table}[htbp]
% \centering
% \setlength\tabcolsep{1em}
% \caption{标准字体命令与字号的对应}\label{tab:fontsize}
% \begin{tabular}{l*2{c>{\ttfamily}r}*3{>{\ttfamily}c}}
% \toprule
% & \multicolumn2c{|zihao = 5|} & \multicolumn2c{|zihao = -4|} &
% \multicolumn1c{|10pt|} & \multicolumn1c{|11pt|} & \multicolumn1c{|12pt|} \\
% \cmidrule(lr){2-3} \cmidrule(lr){4-5}
% \cmidrule(lr){6-6} \cmidrule(lr){7-7} \cmidrule(lr){8-8}
% 字体命令        & 字号 & \si{bp} & 字号 & \si{bp}
%                        & \si{pt} & \si{pt} & \si{pt} \\
% \midrule
% |\tiny|         & 七号 & 5.5  & 小六 & 6.5  & ~5 & ~6 & ~6 \\
% |\scriptsize|   & 小六 & 6.5  & 六号 & 7.5  & ~7 & ~8 & ~8 \\
% |\footnotesize| & 六号 & 7.5  & 小五 & 9~~  & ~8 & ~9 & 10 \\
% |\small|        & 小五 & 9~~  & 五号 & 10.5 & ~9 & 10 & 11 \\
% |\normalsize|   & 五号 & 10.5 & 小四 & 12~~ & 10 & 11 & 12 \\
% |\large|        & 小四 & 12~~ & 小三 & 15~~ & 12 & 12 & 14 \\
% |\Large|        & 小三 & 15~~ & 小二 & 18~~ & 14 & 14 & 17 \\
% |\LARGE|        & 小二 & 18~~ & 二号 & 22~~ & 17 & 17 & 20 \\
% |\huge|         & 二号 & 22~~ & 小一 & 24~~ & 20 & 20 & 25 \\
% |\Huge|         & 一号 & 26~~ & 一号 & 26~~ & 25 & 25 & 25 \\
% \bottomrule
% \end{tabular}
%\end{table}
%
% \begin{function}[rEXP]{10pt, 11pt, 12pt}
%   \CTeX{} 文档类是在 \LaTeX{} 标准文档类之上开发的。因此,除了可以使用 \CTeX{}
%   宏包定义的字号选项之外,还可以使用标准文档类的同类选项(\opt{10pt}、\opt{11pt}
%   和 \opt{12pt})。在使用这些来自标准文档类的选项的时候,\CTeX{} 文档类的字号
%   选项会被抑制。亦即,在 \opt{zihao} 选项之后设置 \opt{10pt} 选项,
%   \opt{zihao} 选项将不再起作用。
% \end{function}
%
% 标准文档类的其他选项在 \CTeX{} 文档类中依旧有效。例如,设置纸张大小和方向的
% \opt{a4paper} 和 \opt{landscape},设置单双面的 \opt{oneside} 和
% \opt{twoside} 等。\CTeX{} 会将这些选项传给标准文档类^^A
% \footnote{事实上,\LaTeX{} 在文档类中的选项是全局设定的,除了对使用的文档类有
% 影响外,也可能会影响到随后使用的宏包。如果这些宏包中有某些选项出现在文档类的
% 选项列表中,那么该选项将会被自动激活。}。
%
% \subsection{章节标题风格}
% \label{subs:options-heading}
%
% \begin{function}[rEXP,added=2014-03-08]{heading}
%   \begin{syntax}
%     heading = <\TFF>
%   \end{syntax}
%   本选项只能在调用 \pkg{ctex.sty} 时作为宏包选项使用。
%
%   \CTeX{} 宏集提供了一套用于修改文档章节标题格式的接口。该选项用于选择是否
%   启用该功能。详细的设置方法请参见
%   \ref{sec:secstyle}~节和 \ref{subs:pagestyle}~节。
% \end{function}
%
% \CTeX{} 宏集提供的四个文档类总是启用该功能。如果在 \pkg{ctex.sty} 下启用该选项,
% 将会检查当前是否使用 \LaTeX{} 标准文档类。
% 若然,则该选项将会使得 \pkg{ctex.sty} 宏包的行为和 \CTeX{} 宏集提供的
% 四个中文文档类\emph{完全}一致;若不然,则会根据 \tn{chapter}
% 是否有定义来使用 \cls{ctexbook} 或者 \cls{ctexart} 的标题设置。
%
% \begin{function}[rEXP]{sub3section, sub4section}
%   修改 \tn{paragraph} 和 \tn{subparagraph} 的格式。
%
%   默认情况下,\tn{paragraph} 和 \tn{subparagraph} 会将标题与随后的正文
%   排版在同一个段落。启用 \opt{sub3section} 会将 \tn{paragraph} 的格式修改为
%   类似 \tn{section} 的格式,并将 \tn{subparagraph} 的格式修改为原本 \tn{paragraph}
%   的格式。启用 \opt{sub4section} 会将 \tn{paragraph} 和 \tn{subparagraph} 的格式
%   都修改为类似 \tn{section} 的格式。
%
%   启用该选项通常需要将计数器 |secnumdepth| 的值为设置为 4 或 5。
%
%   具体格式可参考 \ref{sec:secstyle}~节中的 \opt{runin} 和 \opt{afterskip} 选项。
%
%   注意,上述两个选项只有在非 \cls{beamer} 文档类下 \opt{heading} 选项启用的时候
%   才有意义。亦即,只有在使用除了 \cls{ctexbeamer} 的三个 \CTeX{} 文档类或启用了
%   \opt{heading} 的 \pkg{ctex.sty} 的时候才有意义。
% \end{function}
%
% \subsection{排版方案选项}
% \label{subs:options-type-style}
%
% \begin{function}[rEXP,added=2015-04-15]{scheme}
%   \begin{syntax}
%     scheme = <(chinese)|plain>
%   \end{syntax}
%   选择文章的排版方案,预设有 \opt{chinese} 和 \opt{plain} 两种方案。%
% \end{function}
%   \begin{optdesc}[itemsep=\medskipamount]
%     \item[chinese] 对 \cls{beamer} 以外的文档类,调整默认字号为五号字,并调
%       整行距为 |1.3|;汉化文档中的标题名字(如“图”、“表”、“目录”和“参
%       考文献”等,见 \ref{subs:capname}~节);
%       在 \opt{heading = true} 的情况下^^A
%       \footnote{使用 \CTeX{} 文档类,或者使用 \pkg{ctex} 宏包且开启该选项时。}^^A
%       (\ref{subs:options-heading}~节),还会将章节标题的风格修改为
%       中文样式(见 \ref{sec:secstyle}~节)。
%
%       当关闭 \opt{heading} 选项的 \pkg{ctex} 宏包与标准文档类或其衍生文档类
%       联用时,会载入 \pkg{indentfirst} 宏包,以实现章节标题后的段首缩进。
%     \item[plain] 不调整默认字号和行距,不会汉化文档中的标题名字,也不会将章节
%       标题风格修改为中文样式,同时不会调整 \tn{pagestyle},并禁用 \opt{autoindent}
%       选项。事实上,此时的 \CTeX{} 宏集只提供了中文支持功能,而不对文章版式进行任何修改。
%   \end{optdesc}
%
% \begin{function}[updated=2014-04-11]{punct}
%   \begin{syntax}
%     punct = <(quanjiao)|banjiao|kaiming|CCT|plain>
%   \end{syntax}
%   设置标点处理格式。预定义好的格式有:
% \end{function}
%   \begin{optdesc}
%     \item[quanjiao] 全角式:所有标点占一个汉字宽度,相邻两个标点占 1.5 汉字宽度;
%     \item[banjiao]  半角式:所有标点占半个汉字宽度;
%     \item[kaiming]  开明式:句末点号^^A
%       \footnote{标点符号分为标号与点号。点号分为两类,一共七种:句末点号有^^A
%       句号、问号和叹号;句内点号有逗号、顿号、冒号和分号。}用占一个汉字宽度,标^^A
%       号和句内点号占半个汉字宽度;
%     \item[CCT]   CCT 式:所有标点符号的宽度略小于一个汉字宽度;
%     \item[plain] 原样(不调整标点间距)。
%   \end{optdesc}
%
% \begin{function}[updated=2014-03-08]{space}
%   \begin{syntax}
%     space = <\TF|(auto)>
%   \end{syntax}
%   是否在生成的 PDF 中保留汉字后面的空格。
% \end{function}
%
% \begin{optdesc}
%   \item[true] 总是保留汉字后的空格。此时,用户需要自行在行尾加上~|%|~处理换行产生
%     的空格\footnote{\LaTeX{} 将单个换行视作一个空格。}。
%   \item[false] 总是忽略掉汉字后面的空格,不论汉字后是什么(使用 (pdf)\LaTeX{}
%     编译时);等同于 \opt{auto} 的效果(使用 \XeLaTeX{} 编译时)。不建议使用该选项。
%   \item[auto] 根据空格后面的情况决定是否保留:如果空格后面是汉字,则忽略该
%   空格,否则保留。
% \end{optdesc}
%
%   例如,使用
%   \begin{ctexexam}
%   \ctexset{space=true}
%   汉字 分词
%   技术 English
%   \end{ctexexam}
%   将得到“{\ctexset{space=true}汉字 分词 技术 English}”;使用
%   \begin{ctexexam}
%   \ctexset{space=auto}
%   汉字 分词
%   技术 English
%   \end{ctexexam}
%   则会得到“{\ctexset{space=auto}汉字 分词 技术 English}”。
%
% \emph{使用 \textup{\LuaLaTeX{} 及 \upLaTeX} 编译的时候,该选项无效:汉字间的
% 空格以及汉字与西文字符之间的空格总是有效,不会被忽略,但可以自动忽略掉由换行
% 产生的空格。}
%
% \begin{function}[rEXP,added=2014-04-23]{linespread}
%   \begin{syntax}
%     linespread = <数值>
%   \end{syntax}
%   接受一个浮点数值,设置行距倍数。本选项的初始值与 \opt{scheme} 有关。
% \end{function}
% \begin{optdesc}
%   \item[scheme = chinese] 对标准文档类初始值为 $1.3$,即 $1.3$ 倍行距。此
%   时,相邻两行的基线(\tn{baselineskip})距离为 $1.3\times 1.2=1.56$ 倍字体
%   高度。对 \cls{beamer} 不改变行距,即使用默认的单倍行距。
%
%   \item[scheme = plain] \CTeX{} 宏集默认不调整行距倍数,文档中的行距由所选文档类和
%   其他宏包或用户设置决定。
% \end{optdesc}
%
% \begin{function}[added=2014-03-13]{autoindent}
%   \begin{syntax}
%     autoindent = <\TTF|数值|带单位的数值>
%   \end{syntax}
%   在字体大小发生变化时,是否自动调整段首缩进(\tn{parindent})的大小。
% \end{function}
% \begin{optdesc}
%   \item[\meta{数值或带单位的数值}] 用于设置段首缩进的长度。如果不带单位,则默认
%   单位是单个汉字字宽 \tn{ccwd};如果带单位,则使用该单位。
%
%   \item[true] 等价于设置 \opt{autoindent = 2}。
%
%   \item[false] 禁用自动调整功能,可以设置固定长度的段首缩进。如设置每段缩进 40 点:
%   \begin{ctexexam}
%   \ctexset{autoindent=false}
%   \setlength\parindent{40pt}
%   \end{ctexexam}
% \end{optdesc}
%
% \begin{function}[EXP,added=2014-03-26]{linestretch}
%   \begin{syntax}
%     linestretch = <数值或长度>
%   \end{syntax}
%   \opt{linestretch} 是一个比较特殊的选项,它用来设置汉字之间弹性间距的弹性程
%   度。如果有单位,则可以在选项中直接写;如果是数字,单位则是汉字宽度
%   \tn{ccwd} 的倍数。
% \end{function}
%
%   如果行宽不是汉字宽度的整数倍,为了让段落左右两端对齐,自然就要求伸展汉字之
%   间的间距,而 \opt{linestretch} 选项就是设置每行总的允许伸行量。初始值是允
%   许每行伸行一个汉字的宽度 \tn{ccwd},并且此宽度能根据字号变化动态调整。
%
%   过小的 \opt{linestretch} 可能导致段落文字右侧可能参差不齐;较大的
%   \opt{linestretch} 选项则可以帮助拥有较长不可断行内容的复杂段落方便地断行,
%   而不会产生大量编译警告;但很大的 \opt{linestretch} 则会掩盖段落不良断行产
%   生的坏盒子警告。
%
%   如果将 \opt{linestretch} 选项的值设置为 \tn{maxdimen},则可以禁止按字号自
%   动修改每行的允许伸长量。此时汉字间的弹性间距则固定为 \tn{baselineskip} 的
%   $0.08$ 倍。
%
% \section{文档汉化}
% \label{sec:chinese-localization}
% \subsection{日期汉化}
%
% \CTeX 宏包对显示当前日期的 \tn{today} 命令进行了汉化,使之以中文的方式显示今
% 天的日期。如编译本文档的日期就是“\today”。
%
% \begin{function}[EXP]{today}
%   \begin{syntax}
%     today = <(small)|big|old>
%   \end{syntax}
% 该选项用来控制 \tn{today} 命令的输出格式:
% \begin{optdesc}
%   \item[small] \ctexset{today=small}
%     效果为“\today”。使用阿拉伯数字和汉字的日期格式。
%   \item[big] \ctexset{today=big}
%     效果为“\today”。使用全汉字的日期格式。
%   \item[old] \ctexset{today=old}
%     效果为“\today”。使用文档原来的(英文)日期格式。
% \end{optdesc}
% \end{function}
%
% 设置日期格式使用 \tn{ctexset} 命令完成,例如设置全汉字的日期格式:
% \begin{ctexexam}
%   \ctexset{today=big}
% \end{ctexexam}
%
% \CTeX 宏包的中文日期功能实际上是调用 \pkg{zhnumber} 宏包完成的。如果需要更
% 多有关日期、时间的命令和更复杂的设置,可以查阅 \pkg{zhnumber} 宏包的文档。
%
% \subsection{文档标题汉化}
% \label{subs:capname}
%
% 这里主要介绍由宏包 \opt{scheme} 选项(\ref{subs:options-type-style}~节)控制
% 的文档标题汉化功能。
%
% 设置文档标题名的示例可见例~\ref{exam:capname}。下面的选项(如
% \opt{contentsname})主要用来重新定义与选项同名的宏(如 \tn{contentsname})的
% 定义。
%
% \begin{defaultcapconfig}
%
% \begin{function}[EXP]{contentsname}
%   \begin{syntax}
%     contentsname = <名字>
%   \end{syntax}
% 设置目录标题名 \tn{contentsname}。中文默认为“\contentsname”。
% \end{function}
%
% \begin{function}[EXP]{listfigurename}
%   \begin{syntax}
%     listfigurename = <名字>
%   \end{syntax}
% 设置插图目录标题名 \tn{listfigurename}。中文默认为“\listfigurename”。
% \end{function}
%
% \begin{function}[EXP]{listtablename}
%   \begin{syntax}
%     listtablename = <名字>
%   \end{syntax}
% 设置表格目录标题名 \tn{listtablename}。中文默认为“\listtablename”。
% \end{function}
%
% \begin{function}[EXP]{figurename}
%   \begin{syntax}
%     figurename = <名字>
%   \end{syntax}
% 设置图片环境标题名 \tn{figurename}。中文默认为“\figurename”。
% \end{function}
%
% \begin{function}[EXP]{tablename}
%   \begin{syntax}
%     tablename = <名字>
%   \end{syntax}
% 设置表格环境标题名 \tn{tablename}。中文默认为“\tablename”。
% \end{function}
%
% \begin{function}[EXP]{abstractname}
%   \begin{syntax}
%     abstractname = <名字>
%   \end{syntax}
% 设置摘要 \env{abstract} 环境标题名 \tn{abstractname}。中文默认
% 为“\abstractname”。注意 \cls{book} 类没有摘要,该选项无效。
% \end{function}
%
% \begin{function}[EXP]{indexname}
%   \begin{syntax}
%     indexname = <名字>
%   \end{syntax}
% 设置索引标题名 \tn{indexname}。中文默认为“\indexname”。
% \end{function}
%
% \begin{function}[EXP]{appendixname}
%   \begin{syntax}
%     appendixname = <名字>
%   \end{syntax}
% 设置附录标题名 \tn{appendixname}。中文默认为“\appendixname”。
% \end{function}
%
% \begin{function}[EXP]{bibname}
%   \begin{syntax}
%     bibname = <名字>
%   \end{syntax}
% 设置参考文献标题名 \tn{refname}(对 \cls{article})或 \tn{bibname}(对
% \cls{report}、\cls{book} 和 \cls{beamer})。中文默认为“\refname”。
% \end{function}
%
% \begin{function}[EXP]{proofname}
%   \begin{syntax}
%     proofname = <名字>
%   \end{syntax}
% 设置证明环境的名称 \tn{proofname}。中文默认为“\proofname”。
% \end{function}
%
% 如果使用 \cls{ctexbeamer} 文档类或者在 \cls{beamer} 文档类下使用 \pkg{ctex} 包,
% 还会汉化常用定理类环境的诸如“定义”、“定理”和“引理”等名称。此时,还有下列
% 三个选项。
%
% \begin{function}[EXP]{refname}
%   \begin{syntax}
%     refname = <名字>
%   \end{syntax}
% 设置参考文献标题名 \tn{refname}。中文默认为“\refname”。
% \end{function}
%
% \begin{function}[EXP]{algorithmname}
%   \begin{syntax}
%     algorithmname = <名字>
%   \end{syntax}
% 设置算法环境标题名 \tn{algorithmname}。中文默认为“算法”。
% \end{function}
%
% \begin{function}[EXP]{continuation}
%   \begin{syntax}
%     continuation = <名字>
%   \end{syntax}
% 设置 \cls{beamer} 可断页的帧在续页标题中的延续标识 \tn{insertcontinuationtext}。
% 中文默认为“(续)”。
% \end{function}
%
% \end{defaultcapconfig}
%
% \subsection{页面格式设置与汉化}
% \label{subs:pagestyle}
%
% 当使用了 \CTeX 的文档类或是用 \pkg{ctex} 宏包加载了 \opt{heading} 选项时,会
% 设置整个文档的页面格式(page style)为 |headings|,即相当于设置了
% \begin{frameverb}
%   \pagestyle{headings}
% \end{frameverb}
% 在页眉中显示当前章节的编号与标题。
%
% 同时,\CTeX 宏包也会对默认的 |headings| 页面格式进行修改,使之调用
% \tn{CTEXthechapter}、\tn{CTEXthesection} 等宏来正确显示中文的章节编号。
%
% \CTeX 宏包的默认页面格式设置是经过汉化的 |headings|,其基本效果如本文档所
% 示,只在页眉一侧显示章节编号和标题,另一侧显示页码。
%
% 更复杂的页面格式可以通过调用 \pkg{fancyhdr}、\pkg{titleps} 等宏包来设
% 置。\CTeX 宏包同时也为这些自定义页面格式
% 的包提供了以下宏供使用:
% \begin{itemize}
% \item \tn{CTEXthechapter}、\tn{CTEXthesection} 等章节编号(见
% \ref{sec:secstyle} 节)。它们用来代替英文文档类中的
% \tn{thechapter}、\tn{thesection} 等宏。
%
% \item \tn{leftmark}、\tn{rightmark},它们是在使用章节标题命令后,自动设置的
% 宏。它们实际是在与章节标题命令对应的标记命令
% \tn{chaptermark}、\tn{sectionmark} 中调用 \tn{markright} 或 \tn{markboth} 生
% 成的。
% \end{itemize}
% 有关 \LaTeX 页面标记的涵义与使用细节,已经超出了本文档讨论的范围。可以参考
% \cite[Chapter~23]{knuthtex1986}、\cite[\S4.3, \S4.4]{mittelbach2004} 等书籍。
%
% 这里举一个例子,说明通过重定义 \tn{sectionmark},在 \cls{ctexart} 文档类中的
% 标准 |headings| 页面格式下控制页眉的方式:
% \begin{ctexexam}
%   \documentclass{ctexart}
%   \pagestyle{headings}
%   \ctexset{section={
%       name={第,节},
%       number=\arabic{section},
%     }
%   }
%   \renewcommand\sectionmark[1]{%
%     \markright{\CTEXifname{\CTEXthesection——}{}#1}}
%
%   \begin{document}
%
%   \section{天地玄黄}
%   \newpage
%
%   \section{宇宙洪荒}
%
%   \end{document}
% \end{ctexexam}
% 在上例中,我们设置了页眉的形式是用破折号分开的节编号与节标题,即“第 1 节
% ——天地玄黄”、“第 2 节——宇宙洪荒”。
%
% \CTeX 宏包已经对 \pkg{fancyhdr} 宏包进行了补丁,载入 \pkg{fancyhdr} 后,其
% |fancy| 页面格式将使用 \tn{CTEXthechapter} 等宏显示中文章节编号。
%
% 关于 \pkg{fancyhdr} 的具体用法可以参见其宏包手册。通常也只要像在标准的英文文
% 档类中使用 \pkg{fancyhdr} 一样定义页眉页脚格式即可,并不需要额外的定义。
%
% 下面我则给出一个与前例类似而稍复杂的例子,展示如何在文档中设置页眉内容与页眉
% 的格式。
% \begin{ctexexam}
%   \documentclass{ctexart}
%   \ctexset{section={
%       name={第,节},
%       number=\arabic{section},
%     }
%   }
%   \usepackage{fancyhdr}
%   \fancyhf{}
%   \lhead{\textnormal{\kaishu\rightmark}}
%   \rhead{--\ \thepage\ --}
%   \pagestyle{fancy}
%   % \sectionmark 的重定义需要在 \pagestyle 之后生效
%   \renewcommand\sectionmark[1]{%
%     \markright{\CTEXifname{\CTEXthesection——}{}#1}}
%
%   \begin{document}
%
%   \section{天地玄黄}
%   \newpage
%
%   \section{宇宙洪荒}
%
%   \end{document}
% \end{ctexexam}
% 本例的页眉效果大致如下(有页眉线):
% \begin{trivlist}\item
% \textnormal{\kaishu 第 1 节——天地玄黄}\hfill -- 1 --\par
% \smallskip\hrule
% \end{trivlist}
%
% \section{章节标题格式设置}
% \label{sec:secstyle}
%
% \CTeX 宏包对 \LaTeX 的标准文档类(\cls{article}、\cls{report} 和
% \cls{book})和 \cls{beamer} 进行了扩充。当以 \opt{heading} 选项调用 \CTeX
% 宏包时(\ref{subs:options-heading}~节),则会启用章
% 节标题的格式设置功能。本节就来介绍有关章节标题的格式选项,所有选项使用
% \tn{ctexset} 命令设置。
%
% 第 \ref{subs:pagestyle} 节和本节介绍的功能已经被提取到 \pkg{ctexheading}
% 宏包之中,可以在 \pkg{ctex} 宏包和 \cls{ctexart} 等文档类之外独立使用。
% 各选项的默认值与 \opt{scheme = plain} 时的情形相同。
%
% \changes{v2.0}{2015/03/21}{\tn{CTEXsetup}, \tn{CTEXoptions} 是过时命令。}
% 章节标题的格式选项是分层设置的。顶层的选项是章节标题名称,次一级的选项是章节
% 标题的格式。章节标题名包括 |part|, |chapter|, |section|, |subsection|,
% |subsubsection|, |paragraph|, |subparagraph|;而可用的格式包括 \opt{numbering},
% \opt{name}, \opt{number}, \opt{format}, \opt{nameformat}, \opt{numberformat},
% \opt{aftername}, \opt{titleformat}, \opt{aftertitle}, \opt{runin},
% \opt{afterindent}, \opt{beforeskip}, \opt{afterskip}, \opt{fixskip},
% \opt{lotskip}, \opt{lofskip}, \opt{indent}, \opt{hang},
% \opt{pagestyle}, \opt{break}, \opt{tocline} 等。
%
% 注意,对 \cls{article} 及其衍生的 \cls{ctexart} 等文档类,没有 |chapter|
% 级别的标题。而对于 \cls{beamer} 文档类,这些选项控制的是由 \tn{partpage},
% \tn{sectionpage} 和 \tn{subsectionpage} 产生的标题格式,此时只有 |part|,
% |section| 和 |subsection| 这三层级别,并且 \opt{runin}, \opt{afterindent},
% \opt{fixskip}, \opt{hang}, \opt{break} 和 \opt{tocline} 这六个格式无效。
%
% 多级选项之间用斜线分开,例如,\opt{part/name} 选项设置 \tn{part} 标题的在数
% 字前后的名称,而 \opt{section/number} 选项设置 \tn{section} 标题的数字类型。
%
% 使用 \tn{ctexset} 设置多级选项时,可以在同一个上级选项下设置多个下级选项。
%
% 例如,同时设置 |part| 一级标题的 \opt{pagestyle} 选项,|chapter| 一级标题的
% \opt{format} 与 \opt{pagestyle} 选项和 |section| 一级标题的 \opt{name} 与
% \opt{number} 选项:
% \begin{ctexexam}
%   \ctexset {
%     part/pagestyle = empty,
%     chapter = {
%       format    = \raggedright,
%       pagestyle = empty,
%     },
%     section = {
%       name   = {第,节},
%       number = \chinese{section},
%     }
%   }
% \end{ctexexam}
%
% \begin{function}[EXP,added=2015-06-21]{part/numbering, chapter/numbering,
%   section/numbering, subsection/numbering, subsubsection/numbering,
%   paragraph/numbering, subparagraph/numbering}
%   \begin{syntax}
%     numbering = \TTF
%   \end{syntax}
%   控制是否对章节标题编号。对各级标题的默认值均为 \opt{true}。
%
%   我们知道,\LaTeX{} 带星号的章节标题命令(如 \tn{section*})不会对标题编号,
%   但也不会将该没编号的标题编入目录中。本选项控制的是不带星号的标题命令是否编号。
%   设置本选项为 \opt{false},除了不对标题编号以外,功能与正常标题一致,
%   比如可以编目录,正确的 \pkg{hyperref} 目录超链接位置和页眉标记。
%
%   注意,章节标题的编号深度受 \LaTeX{} 计数器 |secnumdepth| 的控制。
%   \opt{numbering} 选项在 |secnumdepth| 的控制下起作用。
% \end{function}
%
% 如果没有特别说明,以下将用 “|...|” 代表各级章节标题名。
%
% \begin{function}[EXP,updated=2014-03-08]{.../name}
%   \begin{syntax}
%     name = \{<前名字>,<后名字>\}
%     name = \Arg{前名字}
%   \end{syntax}
%   设置章节的名字。名字可以分为前后两部分,即章节编号前后的词语,两个词之间用
%   一个半角逗号分开;也可以只有一部分,表示只有章节编号之前的名字。例如:
%   \begin{ctexexam}
%   \ctexset{
%     chapter/name = {第,章},
%     section/name = {\S},
%   }
%   \end{ctexexam}
%   会使得 \tn{chapter} 标题使用形如“第一章”的名字,而 \tn{section} 标题则使
%   用形如“\S1”的名字。
% \end{function}
%
% \begin{table}[htbp]
% \small\centering
% \caption{\opt{name} 选项的默认设置}
% \begin{tabular}{llll}
% \toprule
% 标题名 & \opt{scheme = chinese} & \opt{scheme = plain} & 注 \\
% \midrule
% part & |{第,部分}| & |{\partname\space}| & 原 \tn{partname} 为 Part \\
% chapter & |{第,章}| & |{\chaptername\space}|
%   & 原 \tn{chaptername} 为 Chapter \\
% section (beamer) & |{}| & |{\sectionname\space}|
%   & \BSTACK 原 \tn{sectionname} 为\\ |\translate{Section}| \ESTACK \\
% section & 同右 & |{}| & \\
% subsection (beamer) & |{}| & |{\subsectionname\space}|
%   & \BSTACK 原 \tn{subsectionname} 为\\ |\translate{Subsection}| \ESTACK \\
% subsection & 同右 & |{}| & \\
% subsubsection & 同右 & |{}| & \\
% paragraph & 同右 & |{}| & \\
% subparagraph & 同右 & |{}| & \\
% \bottomrule
% \end{tabular}
% \end{table}
%
% \begin{function}[EXP]{.../number}
%   \begin{syntax}
%     number = \Arg{数字输出命令}
%   \end{syntax}
%   设置章节编号的数字输出格式。\meta{数字输出命令} 通常是对应章节编号计数器的
%   输出命令,如 \tn{thesection} 或 |\chinese{chapter}| 之类。
%   \begin{ctexexam}
%   \ctexset{
%     section/number = \Roman{section}
%   }
%   \end{ctexexam}
% \end{function}
%
%   \opt{number} 选项的定义同时将控制对章节计数器的交叉引用。在引用计数器时,
%   记录在 \LaTeX{} 辅助文件中的是 \opt{number} 选项的定义。
%
%   但是,\opt{number} 选项不会影响计数器本身的输出。即设置 |section/number|
%   不会影响 \tn{thesection} 的定义。(但该选项会影响 \tn{CTEXthesection} 的定
%   义,见后。)
%
% \begin{table}[htbp]
% \small\centering
% \caption{\opt{number} 选项的默认设置}
% \setlength\leftskip{0pt plus 1 fil minus \marginparwidth}
% \begin{tabular}{llll}
% \toprule
% 标题名 & \opt{scheme = chinese} & \opt{scheme = plain}
%   & 原 |\the|\meta{标题} 等价定义 \\
% \midrule
% part (beamer) & |\chinese{part}| & |\insertromanpartnumber| & 意义为 |\Roman{part}| \\
% part & |\chinese{part}| & |\thepart| & |\Roman{part}| \\
% chapter & |\chinese{chapter}| & |\thechapter| & |\arabic{chapter}| \\
% section (beamer) & 同右 & |\insertsectionnumber| & 意义为 |\arabic{section}| \\
% section & 同右 & |\thesection| & |\arabic{section}| \\
% subsection (beamer)
%   & \BSTACK |\arabic{section}.|\\ |\arabic{subsection}| \ESTACK
%   & |\insertsubsectionnumber| & 意义为 |\arabic{subsection}| \\
% subsection & 同右 & |\thesubsection| & |\thesection.\arabic{subsection}| \\
% subsubsection & 同右 & |\thesubsubsection|
%   & |\thesubsection.\arabic{subsubsection}| \\
% paragraph & 同右 & |\theparagraph|
%   & |\thesubsubsection.\arabic{paragraph}| \\
% subparagraph & 同右 & |\thesubparagraph|
%   & |\theparagraph.\arabic{subparagraph}| \\
% \bottomrule
% \end{tabular}
% \end{table}
%
% \begin{function}{\CTEXthepart, \CTEXthechapter, \CTEXthesection,
%   \CTEXthesubsection, \CTEXthesubsubsection, \CTEXtheparagraph,
%   \CTEXthesubparagraph}
%   以 |\CTEXthe| 开头的这组宏给出结合了 \opt{name} 与 \opt{number} 选项的章节
%   编号输出格式。例如在 \opt{scheme = chinese} 时,默认章编号输出格式就是
%   \tn{CTEXthechapter},形如“第一章”。
%
%   这组宏在 \CTeX 文档类中将代替 \tn{thechapter} 等宏的作用,在章节中引用本章
%   节的完整编号。例如用于帮助定义自定义的目录格式、页眉格式等。
% \end{function}
%
% \begin{function}[added=2016-09-18]{\CTEXifname}
%   \begin{syntax}
%     \tn{CTEXifname} \Arg{有名字时的格式命令} \Arg{无名字时的格式命令}
%   \end{syntax}
%   \tn{CTEXifname} 用于根据当前章节的名字的有无设置不同的格式。
%
%   它可用于 \opt{format}, \opt{titleformat}, \opt{aftertitle}, \opt{afterskip},
%   \opt{indent} 这五个选项和 \tn{chapter} 标题 \opt{beforeskip} 选项的格式设置之中。
%   也可用于帮助定义自定义的目录格式、页眉格式等。
%
%   例如,设置章的标题有名字时左对齐,无名字时居中对齐,并且在标题后画一条横线。
%   \begin{ctexexam}
%   \ctexset{
%     chapter/format = \CTEXifname{\raggedright}{\centering},
%     chapter/aftertitle = \par\CTEXifname{}{\hrule},
%   }
%   \end{ctexexam}
% \end{function}
%
% \begin{function}[EXP,updated=2015-06-30]{.../format, .../format+}
%   \begin{syntax}
%     format = \Arg{格式命令}
%     format+= \Arg{格式命令}
%   \end{syntax}
%   \opt{format} 选项用于控制章节标题的全局格式,作用域为章节名字和随后的标题
%   内容。可以用于控制章节标题的对齐方式、整体字体字号等格式。
%
%   带加号的 \opt{format+} 选项则用于在已有格式之后追加新的格式命令。
%
%   例如,设置章格式为无衬线字体左对齐,为节格式增加无衬线字体设置:
%   \begin{ctexexam}
%   \ctexset{
%     chapter/format  = \sffamily\raggedright,
%     section/format += \sffamily
%   }
%   \end{ctexexam}
% \end{function}
%
% \begin{table}[htbp]
% \small\centering
% \caption{\opt{format} 选项的默认设置}
% \begin{tabular}{lll}
% \toprule
% 标题名 & \opt{scheme = chinese} & \opt{scheme = plain} \\
% \midrule
% part (article) & |\Large\bfseries\centering| & |\raggedright| \\
% part (beamer) & 同右 & |\centering| \\
% part & |\huge\bfseries\centering| & |\centering| \\
% chapter & |\huge\bfseries\centering| & |\raggedright| \\
% section (beamer) & 同右 & |\centering| \\
% section & |\Large\bfseries\centering| & |\Large\bfseries| \\
% subsection (beamer) & 同右 & |\centering| \\
% subsection & 同右 & |\large\bfseries| \\
% subsubsection & 同右 & |\normalsize\bfseries| \\
% paragraph & 同右 & |\normalsize\bfseries| \\
% subparagraph & 同右 & |\normalsize\bfseries| \\
% \bottomrule
% \end{tabular}
% \end{table}
%
% \begin{function}[EXP,updated=2015-06-30]{.../nameformat, .../nameformat+}
%   \begin{syntax}
%     nameformat = \Arg{格式命令}
%     nameformat+= \Arg{格式命令}
%   \end{syntax}
%   \opt{nameformat} 用于控制章节名字的格式,作用域为章节名字,包括编号。它一
%   般用于章节名(包括编号)与章节标题的字体、字号等设置不一致的情形。参见
%   \opt{titleformat} 选项。
%
%   \opt{nameformat+} 用于在已有的章节名字格式后附加内容。
%
%   \opt{nameformat} 选项的最后一个格式命令可以是一个有一个参数的命令。
%   这个命令的参数用于接受章节名字和编号,实现特殊效果(见例~\ref{exam:miscopt})。
%
%   \opt{nameformat} 选项的默认值,在 \opt{scheme} 选项的不同取值下相同。
% \end{function}
%
% \begin{table}[htbp]
% \small\centering
% \caption{\opt{nameformat} 选项的默认设置}
% \begin{tabular}{lll}
% \toprule
% 标题名 & \opt{scheme = chinese} & \opt{scheme = plain} \\
% \midrule
% part (article) & |{}| & |\Large\bfseries| \\
% part (beamer) & 同右
%   & \BSTACK |\usebeamerfont{part name}| \\
%             |\usebeamercolor[fg]{part name}| \ESTACK \\
% part & |{}| & |\huge\bfseries| \\
% chapter & |{}| & |\huge\bfseries| \\
% section (beamer) & 同右
%   & \BSTACK |\usebeamerfont{section name}| \\
%             |\usebeamercolor[fg]{section name}| \ESTACK \\
% section & 同右 & |{}| \\
% subsection (beamer) & 同右
%   & \BSTACK |\usebeamerfont{subsection name}| \\
%             |\usebeamercolor[fg]{subsection name}| \ESTACK \\
% subsection & 同右 & |{}| \\
% subsubsection & 同右 & |{}| \\
% paragraph & 同右 & |{}| \\
% subparagraph & 同右 & |{}| \\
% \bottomrule
% \end{tabular}
% \end{table}
%
% \begin{function}[EXP,updated=2015-06-19]{.../numberformat, .../numberformat+}
%   \begin{syntax}
%     numberformat = \Arg{格式命令}
%     numberformat+= \Arg{格式命令}
%   \end{syntax}
%   \opt{numberformat} 选项用于控制章节编号的格式,作用域仅为编号数字本身。对
%   各级标题默认均为空,当你需要编号的格式和前后的章节名字不一样时可以使用。
%
%   \opt{numberformat+} 选项用于在原有编号格式后面附加格式命令。
%
%   \opt{numberformat} 选项的最后一个格式命令可以是一个有一个参数的命令。
%   这个命令的参数用于接受编号数字。
% \end{function}
%
%   例如,我们可以使用 \opt{numberformat} 特别强调章标题中的数字:
%   \begin{ctexexam}
%   \ctexset{
%     chapter/number = \arabic{chapter},
%     chapter/numberformat = \color{blue}\zihao{0}\itshape,
%   }
%   \end{ctexexam}
%   上面的代码在 \opt{scheme = chinese} 时可以做出类似这样的章标题效果:
%   \begin{center}
%   \huge\bfseries 第 \textit{\color{blue}\zihao{0}4} 章
%   \end{center}
%
% \begin{function}[EXP,updated=2014-03-08]{.../aftername, .../aftername+}
%   \begin{syntax}
%   aftername = \Arg{代码}
%   aftername+= \Arg{代码}
%   \end{syntax}
%   \opt{aftername} 选项的参数 \meta{代码} 将被插入到章节编号与其后的标题内容之
%   间,用于控制格式变换。常用于控制章节编号与标题内容之间的距离,或者控制标题
%   是否另起一行。
%
%   \opt{aftername+} 选项用于在原有代码后面附加代码。
% \end{function}
%
% \begin{table}[htbp]
% \small\centering
% \caption{\opt{aftername} 选项的默认设置}
% \begin{tabular}{lll}
% \toprule
% 标题名 & \opt{scheme = chinese} & \opt{scheme = plain} \\
% \midrule
% part (article) & |\quad| & |\par\nobreak| \\
% part (beamer) & 同右 & |\vskip 1em \par| \\
% part & 同右 & |\par\vskip 20pt| \\
% chapter & |\quad| & |\par\nobreak\vskip 20pt| \\
% section (beamer) & 同右 & |\vskip 1em \par| \\
% section & 同右 & |\quad| \\
% subsection (beamer) & 同右 & |\vskip 1em \par| \\
% subsection & 同右 & |\quad| \\
% subsubsection & 同右 & |\quad| \\
% paragraph & 同右 & |\quad| \\
% subparagraph & 同右 & |\quad| \\
% \bottomrule
% \end{tabular}
% \end{table}
%
% \begin{function}[EXP,updated=2015-06-30]{.../titleformat, .../titleformat+}
%   \begin{syntax}
%     titleformat = \Arg{格式命令}
%     titleformat+= \Arg{格式命令}
%   \end{syntax}
%   \opt{titleformat} 选项用于控制标题内容的格式,作用域为章节标题内容。
%
%   \opt{titleformat+} 选项用于在原有标题格式后面附加格式命令。
%
%   需要注意的是,\opt{titleformat} 选项的最后一个格式命令可以是一个有一个
%   参数的命令。这个命令的参数接受标题内容,用于实现特殊效果。
%   例如,实现多行标题的居中悬挂对齐:
% \end{function}
% \begin{ctexexam}
%   \usepackage{varwidth} %% 提供 varwidth 环境
%   \ctexset{
%     chapter/name = {第,回},
%     chapter/titleformat = \chaptertitleformat
%   }
%   \newcommand\chaptertitleformat[1]{%% 以标题内容为参数
%     \begin{varwidth}[t]{.7\linewidth}#1\end{varwidth}}
%   ......
%   \chapter{情中情因情感妹妹\\错里错以错劝哥哥}
% \end{ctexexam}
% 上面的代码可以做出类似这样的章标题效果:
% \begin{center}\Large\bfseries
% 第三十四回\quad
% \begin{tabular}[t]{l}
%   情中情因情感妹妹\\
%   错里错以错劝哥哥
% \end{tabular}
% \end{center}
%
% \begin{table}[htbp]
% \small\centering
% \caption{\opt{titleformat} 选项的默认设置}
% \begin{tabular}{lll}
% \toprule
% 标题名 & \opt{scheme = chinese} & \opt{scheme = plain} \\
% \midrule
% part (article) & |{}| & |\huge\bfseries| \\
% part (beamer) & 同右 & |\usebeamerfont{part title}| \\
% part & |{}| & |\Huge\bfseries| \\
% chapter & |{}| & |\Huge\bfseries| \\
% section (beamer) & 同右 & |\usebeamerfont{section title}| \\
% section & 同右 & |{}| \\
% subsection (beamer) & 同右 & |\usebeamerfont{subsection title}| \\
% subsection & 同右 & |{}| \\
% subsubsection & 同右 & |{}| \\
% paragraph & 同右 & |{}| \\
% subparagraph & 同右 & |{}| \\
% \bottomrule
% \end{tabular}
% \end{table}
%
% \begin{function}[EXP,added=2015-06-19]{.../aftertitle, .../aftertitle+}
%   \begin{syntax}
%   aftertitle = \Arg{代码}
%   aftertitle+= \Arg{代码}
%   \end{syntax}
%   \opt{aftertitle} 选项的参数 \meta{代码} 将被插入到章节标题内容之后。
%
%   \opt{aftertitle+} 选项用于在原有代码后面附加代码。
%
%   \opt{aftertitle} 选项的默认值,在 \opt{scheme} 选项的不同取值下相同。
%
%   \opt{sub3section} 或 \opt{sub4section} 宏包选项(见
%   \ref{subs:options-heading}~节)会影响 \opt{aftertitle} 选项的默认值。
% \end{function}
%
% \begin{table}[htbp]
% \begin{minipage}[t]{.5\linewidth}
% \small\centering
% \caption{\opt{aftertitle} 选项的默认设置}
% \begin{tabular}{ll}
% \toprule
% 标题名 & 默认值 \\
% \midrule
% part & |\par| \\
% chapter & |\par| \\
% section & |\@@par| \\
% subsection & |\@@par| \\
% subsubsection & |\@@par| \\
% paragraph & |{}| \\
% \qquad(sub3section) & |\@@par| \\
% \qquad(sub4section) & 同上 \\
% subparagraph & |{}| \\
% \qquad(sub4section) & |\@@par| \\
% \bottomrule
% \end{tabular}
% \end{minipage}%
% \begin{minipage}[t]{.5\linewidth}
% \small\centering
% \caption{\opt{runin} 选项的默认设置}
% \begin{tabular}{ll}
% \toprule
% 标题名 & 默认值 \\
% \midrule
% part & 无效 \\
% chapter & 无效 \\
% section & |false| \\
% subsection & |false| \\
% subsubsection & |false| \\
% paragraph & |true| \\
% \qquad(sub3section) & |false| \\
% \qquad(sub4section) & 同上 \\
% subparagraph & |true| \\
% \qquad(sub4section) & |false| \\
% \bottomrule
% \end{tabular}
% \end{minipage}
% \end{table}
%
% \begin{function}[EXP,added=2015-06-27]{section/runin, subsection/runin,
%   subsubsection/runin, paragraph/runin, subparagraph/runin}
%   \begin{syntax}
%   runin = \TF
%   \end{syntax}
%   \opt{runin} 选项只对 \tn{section} 级以下标题有意义。
%   用于确定标题与随后的正文是否排在同一段之上。
%
%   \opt{runin} 选项的默认值,在 \opt{scheme} 选项的不同取值下相同。
%
%   默认情况下,\tn{paragraph}、\tn{subparagraph} 两级标题是与后面正文排在同一
%   段的,\opt{runin} 选项为 \opt{true};但使用 \opt{sub3section} 或
%   \opt{sub4section} 宏包选项(见 \ref{subs:options-heading}~节)后,
%   将对这两级标题设 \opt{runin} 选项为 \opt{false},这两级标题会改为排在不同段。
% \end{function}
%
% \begin{function}[EXP,added=2015-06-27]{.../afterindent}
%   \begin{syntax}
%   afterindent = \TF
%   \end{syntax}
%   \opt{afterindent} 选项用于设置章节标题后首段的缩进。
%
%   \cls{book} 和 \cls{report} 类的 \tn{part} 标题被单独排在一页之上,
%   \opt{afterindent} 选项没有意义。
%
%   对于 \tn{section} 级以下标题,若设置了 \opt{runin} 选项为 \opt{true},
%   即标题与随后正文排在同一段,\opt{afterindent} 选项也就没有了意义。
% \end{function}
%
% \begin{table}[htbp]
% \small\centering
% \caption{\opt{afterindent} 选项的默认设置}
% \begin{tabular}{lll}
% \toprule
% 标题名 & \opt{scheme = chinese} & \opt{scheme = plain} \\
% \midrule
% part (article) & |true| & |false| \\
% part & 无效 & 无效 \\
% chapter & |true| & |false| \\
% section & |true| & |false| \\
% subsection & |true| & |false| \\
% subsubsection & |true| & |false| \\
% paragraph & |true| & |false| \\
% subparagraph & |true| & |false| \\
% \bottomrule
% \end{tabular}
% \end{table}
%
% \begin{function}[EXP,updated=2016-05-10]{.../beforeskip}
%   \begin{syntax}
%   beforeskip = \Arg{弹性间距}
%   \end{syntax}
%   \opt{beforeskip} 选项用于设置章节标题前的垂直间距。
%
%   \opt{beforeskip} 选项的默认值,在 \opt{scheme} 选项的不同取值下相同。
% \end{function}
%
% \begin{table}[htbp]
% \small\centering
% \caption{\opt{beforeskip} 选项的默认设置}
% \begin{tabular}{ll}
% \toprule
% 标题名 & 默认值 \\
% \midrule
% part (article) & |4ex| \\
% part (beamer) & |0pt| \\
% part & |0pt plus 1fil| \\
% chapter & |50pt| \\
% section (beamer) & |0pt| \\
% section & |3.5ex plus 1ex minus .2ex| \\
% subsection (beamer) & |0pt| \\
% subsection & |3.25ex plus 1ex minus .2ex| \\
% subsubsection & |3.25ex plus 1ex minus .2ex| \\
% paragraph & |3.25ex plus 1ex minus .2ex| \\
% subparagraph & |3.25ex plus 1ex minus .2ex| \\
% \bottomrule
% \end{tabular}
% \end{table}
%
% \begin{function}[EXP,updated=2015-06-27]{.../afterskip}
%   \begin{syntax}
%   afterskip = \Arg{弹性间距}
%   \end{syntax}
%   \opt{afterskip} 选项控制章节标题与后面下方之间的距离。
%
%   对于 \tn{section} 级以下标题,\opt{runin} 选项会影响 \opt{afterskip} 选项的意义:
%   若 \opt{runin} 为 \opt{true},标题与随后正文排在同一段,\meta{弹性间距} 给出水平间距。
%   否则,正文另起一段,\meta{弹性间距} 给出的是垂直间距。
%
%   \opt{afterskip} 选项的默认值,在 \opt{scheme} 选项的不同取值下相同。
%
%   \opt{sub3section} 或 \opt{sub4section} 宏包选项(见
%   \ref{subs:options-heading}~节)会影响 \opt{aftertitle} 选项的默认值。
% \end{function}
%
% \begin{table}[htbp]
% \begin{minipage}[t]{.5\linewidth}
% \small\centering
% \caption{\opt{afterskip} 选项的默认设置}
% \begin{tabular}{ll}
% \toprule
% 标题名 & 默认值 \\
% \midrule
% part (article) & |3ex| \\
% part (beamer) & |0pt| \\
% part & |0pt plus 1fil| \\
% chapter & |40pt| \\
% section (beamer) & |0pt| \\
% section & |2.3ex plus .2ex| \\
% subsection (beamer) & |0pt| \\
% subsection & |1.5ex plus .2ex| \\
% subsubsection & |1.5ex plus .2ex| \\
% paragraph & |1em| \\
% \qquad(sub3section) & |1ex plus .2ex| \\
% \qquad(sub4section) & 同上 \\
% subparagraph & |1em| \\
% \qquad(sub4section) & |1ex plus .2ex| \\
% \bottomrule
% \end{tabular}
% \end{minipage}%
% \begin{minipage}[t]{.5\linewidth}
% \small\centering
% \caption{\opt{indent} 选项的默认设置}
% \begin{tabular}{ll}
% \toprule
% 标题名 & 默认值 \\
% \midrule
% part (article) & |0pt| \\
% part (beamer) & |0pt| \\
% part & 无效 \\
% chapter & |0pt| \\
% section & |0pt| \\
% subsection & |0pt| \\
% subsubsection & |0pt| \\
% paragraph & |0pt| \\
% subparagraph & |\parindent| \\
% \qquad(sub3section) & |0pt| \\
% \qquad(sub4section) & 同上 \\
% \bottomrule
% \end{tabular}
% \end{minipage}
% \end{table}
%
% \begin{function}[EXP,added=2016-06-03]{.../fixskip}
%   \begin{syntax}
%   fixskip = \TFF
%   \end{syntax}
%   默认情况下,\cls{article}、\cls{book} 和 \cls{report} 类的标题与正文的距离除了由
%   \opt{beforeskip} 和 \opt{afterskip} 选项设置的垂直间距外,还会有一些多余的间距。
%   \opt{fixskip} 选项用于抑制这些多余间距。
% \end{function}
%
% \begin{function}[EXP,added=2016-10-01]{chapter/lofskip, chapter/lotskip}
%   \begin{syntax}
%   lofskip = \Arg{弹性间距}
%   lotskip = \Arg{弹性间距}
%   \end{syntax}
%   \opt{lofskip} 选项控制插图目录(\file{.lof})中,章之间的插图标题的距离。
%
%   同样,\opt{lotskip} 选项控制表格目录(\file{.lot})中,章之间的表格标题的距离。
%
%   目前,这两个选项只在 \opt{chapter} 标题下有定义。
%   他们的默认值,在 \opt{scheme} 选项的不同取值下都为 \SI{10}{pt}。
% \end{function}
%
% \begin{function}[EXP,updated=2015-06-27]{.../indent}
%   \begin{syntax}
%   indent = \Arg{缩进间距}
%   \end{syntax}
%   \opt{indent} 选项用于设置章节标题本身的首行缩进。
%
%   \opt{indent} 选项的默认值,在 \opt{scheme} 选项的不同取值下相同。
%
%   例如,设置 \tn{section} 标题缩进 \SI{20}{pt}:
% \end{function}
% \begin{ctexexam}
%   \ctexset{section={
%       format=\Large\bfseries,
%       indent=20pt,
%     }
%   }
%   \section{首行缩进的标题}
%   \noindent 无缩进的正文。
% \end{ctexexam}
%
% \begin{function}[EXP,added=2016-10-22]{section/hang, subsection/hang,
%   subsubsection/hang, paragraph/hang, subparagraph/hang}
%   \begin{syntax}
%   hang = \TTF
%   \end{syntax}
%   默认情况下,\tn{section} 级以下标题具有悬挂缩进的效果(缩进的宽度为名字宽度和 \opt{indent} 选项
%   设置的宽度之和)。设置 \opt{hang} 选项为 \opt{false} 可以取消这一效果。
% \end{function}
%
% \begin{function}[EXP,added=2014-03-21]{part/pagestyle, chapter/pagestyle}
%   \begin{syntax}
%     pagestyle = \Arg{页面格式}
%   \end{syntax}
%   设置 \cls{book}/\cls{ctexbook} 或 \cls{report}/\cls{ctexrep} 文档类
%   中,\tn{part} 与 \tn{chapter} 标题所在页的页面格式(page style)。
% \end{function}
%
% \begin{table}[htbp]
% \small\centering
% \caption{\opt{pagestyle} 选项的默认设置}
% \begin{tabular}{ll}
% \toprule
% 标题名 & 默认值 \\
% \midrule
% part (article) & 无效 \\
% part & |plain| \\
% chapter & |plain| \\
% \bottomrule
% \end{tabular}
% \end{table}
%
% \begin{function}[EXP,added=2016-09-19]{.../break, .../break+}
%   \begin{syntax}
%     break = \Arg{格式命令}
%     break+= \Arg{格式命令}
%   \end{syntax}
%   \opt{break} 选项用于控制章节标题与之前正文的分隔关系。一般用于设置是否在标题之前分页或者设置行间罚点。
%
%   带加号的 \opt{break+} 选项则用于在已有格式之后追加新的格式命令。
%
%   \opt{break} 选项的默认值,在 \opt{scheme} 选项的不同取值下相同。
%
%   例如,若当前页剩余高度小于正文高度的一半时,则另起一页输出 \tn{section} 标题:
%   \begin{ctexexam}
%   \usepackage{needspace}
%   \ctexset{section/break = \Needspace{.5\textheight}}
%   \end{ctexexam}
% \end{function}
%
% \begin{table}[htbp]
% \small\centering
% \caption{\opt{break} 选项的默认设置}
% \begin{tabular}{ll}
% \toprule
% 标题名 & 默认值 \\
% \midrule
% part (article) & |{}| \\
% part & |\if@openright\cleardoublepage\else\clearpage\fi| \\
% chapter & 同上 \\
% section & |\addpenalty{\@secpenalty}| \\
% subsection & 同上 \\
% subsubsection & 同上 \\
% paragraph & 同上 \\
% subparagraph & 同上 \\
% \bottomrule
% \end{tabular}
% \end{table}
%
% \begin{function}[EXP,added=2016-10-25]{.../tocline}
%   \begin{syntax}
%     tocline = \Arg{格式定义}
%   \end{syntax}
%   \opt{tocline} 选项用于定义章节标题在目录文件(\file{.toc})中的格式。\meta{格式定义}有两个参数:
%   参数 |#1| 是 |part|、|chapter| 等名字,参数 |#2| 是标题内容。
% \end{function}
%
% \begin{table}[htbp]
% \small\centering
% \caption{\opt{tocline} 选项的默认设置}
% \begin{tabular}{ll}
% \toprule
% 标题名 & 默认值 \\
% \midrule
% part & |\CTEXifname{\CTEXthepart\hspace{1em}}{}#2| \\
% chapter (\opt{chinese})
%   & |\CTEXifname{\protect\numberline{\CTEXthechapter\hspace{.3em}}}{}#2| \\
% chapter (\opt{plain})
%   & |\CTEXnumberline{#1}#2| \\
% section & |\CTEXnumberline{#1}#2| \\
% subsection & 同上 \\
% subsubsection & 同上 \\
% paragraph & 同上 \\
% subparagraph & 同上 \\
% \bottomrule
% \end{tabular}
%
% \medskip
% 其中 \tn{CTEXnumberline} 的意义是若标题 |#1| 没有名字,则不输出 |\numberline{\CTEXthe#1}|
% 等编号:
% \begin{verbatim}
%   \CTEXifname{\protect\numberline{\csname CTEXthe#1\endcsname}}{}
% \end{verbatim}
% \end{table}
%
% \begin{function}[EXP,added=2015-06-21]{appendix/numbering}
%   \begin{syntax}
%     numbering = \TTF
%   \end{syntax}
%   控制是否对附录章(对 \cls{book} 与 \cls{report})或附录节(对 \cls{article})
%   进行编号。
%
%   用法与普通章节 \opt{numbering} 选项类似。
% \end{function}
%
% \begin{function}[EXP,updated=2014-03-08]{appendix/name}
%   \begin{syntax}
%     name = \{<前名字>,<后名字>\}
%     name = \Arg{前名字}
%   \end{syntax}
%   设置附录章(对 \cls{book} 与 \cls{report})或附录节(对 \cls{article})的
%   名字。
%
%   用法与普通章节 \opt{name} 选项类似。
%
%   注意该选项与 \opt{appendixname} 选项(\ref{subs:capname}~节)在意义上有
%   些重叠,但意义不同。\opt{appendixname} 选项只用来重定义
%   \tn{appendixname},而不管 \tn{appendixname} 如何使用;该选项则决定在章节标
%   题中输出的名字,可以调用 \tn{appendixname} 设置。
% \end{function}
%
% \begin{table}[htbp]
% \small\centering
% \caption{\opt{appendix/name} 选项的默认设置}
% \begin{tabular}{llllll}
% \toprule
% 文档类 & 影响命令 & \opt{scheme = chinese} & 实际定义
%   & \opt{scheme = plain} & 实际定义 \\
% \midrule
% article & \tn{section} & |{}| & & |{}| & \\
% book, report & \tn{chapter} & |\appendixname\space| & \verb*|附录 |
%                             & |\appendixname\space| & \verb*|Appendix | \\
% \bottomrule
% \end{tabular}
% \end{table}
%
% \begin{function}[EXP]{appendix/number}
%   \begin{syntax}
%     number = \Arg{数字输出命令}
%   \end{syntax}
%   设置附录章(对 \cls{book} 与 \cls{report})或附录节(对 \cls{article})编
%   号的数字输出格式。
%
%   用法与普通章节的 \opt{number} 选项类似。
%
%   该选项也同时控制附录章节计数器的交叉引用。
%
%   与普通章节的 \opt{number} 选项类似,同样需要注意,该选项不会影响计数器本身
%   的输出,即不影响 \tn{thesection} 或 \tn{thechapter} 的定义。
% \end{function}
%
% \begin{table}[htbp]
% \small\centering
% \caption{\opt{appendix/number} 选项的默认设置}
% \begin{tabular}{llllll}
% \toprule
% 文档类 & 影响命令 & 默认值 \\
% \midrule
% article & \tn{section} & |\Alph{section}| \\
% book, report & \tn{chapter} & |\Alph{chapter}| \\
% \bottomrule
% \end{tabular}
% \end{table}
%
% 我们最后举一个稍微复杂的例子,来看看上述选项的综合应用。
%
% \begin{ctexexam}[labelref=exam:miscopt]
%   \ctexset {
%     chapter = {
%       beforeskip = 0pt,
%       fixskip    = true,
%       format     = \Huge\bfseries,
%       nameformat = \rule{\linewidth}{1bp}\par\bigskip\hfill\chapternamebox,
%       number     = \arabic{chapter},
%       aftername  = \par\medskip,
%       aftertitle = \par\bigskip\nointerlineskip\rule{\linewidth}{2bp}\par
%     }
%   }
%   \newcommand\chapternamebox[1]{%
%     \parbox{\ccwd}{\linespread{1}\selectfont\centering #1}}
%   ......
%   \chapter{熟悉 \LaTeX}
% \end{ctexexam}
% 本例的设置效果大致如下:
% \begin{center}
% \begin{minipage}{.75\linewidth}\Large\bfseries
% \hrule height .5bp \relax
% \medskip
% \hfill\parbox{\ccwd}{\linespread{1}\selectfont\centering 第 1 章}
% \par\smallskip
% \noindent 熟悉 \LaTeX
% \medskip\hrule height 1bp \relax
% \end{minipage}
% \end{center}
%
% \section{实用命令}
% \label{sec:useful-commands}
% \subsection{字号与间距}
%
% \begin{function}[updated = 2014-03-08, label = ]{\zihao}
%   \begin{syntax}
%     \tn{zihao} \Arg{字号}
%   \end{syntax}
%   用于调整字号大小。其中 \meta{字号} 的有效值共有 16 个,如表 \ref{tab:zihao}
%   所示。使用 \tn{zihao} 命令调整字体大小时,西文字号大小会始终和中文字号保持一致。
% \end{function}
%
% \begin{table}[!htbp]
% \centering
% \def\ZH#1{\zihaopt{#1} & \zihao{#1}}
% \tabcolsep=1em
% \caption{中文字号}\label{tab:zihao}
% \begin{tabular}{>{\ttfamily}S[mode=text,detect-family,table-format=-1]
%                             S[table-format=2.1]S[table-format=2.5]l}
% \toprule
% {\meta{字号}} & {大小(bp)} & {大小(pt)} & 意义 \\
% \midrule
%  0   & 42   & \ZH{0}    初号    \\
% {−0} & 36   & \ZH{-0}   小初号  \\
%  1   & 26   & \ZH{1}    一号    \\
% -1   & 24   & \ZH{-1}   小一号  \\
%  2   & 22   & \ZH{2}    二号    \\
% -2   & 18   & \ZH{-2}   小二号  \\
%  3   & 16   & \ZH{3}    三号    \\
% -3   & 15   & \ZH{-3}   小三号  \\
%  4   & 14   & \ZH{4}    四号    \\
% -4   & 12   & \ZH{-4}   小四号  \\
%  5   & 10.5 & \ZH{5}    五号    \\
% -5   &  9   & \ZH{-5}   小五号  \\
%  6   &  7.5 & \ZH{6}    六号    \\
% -6   &  6.5 & \ZH{-6}   小六号  \\
%  7   &  5.5 & \ZH{7}    七号    \\
%  8   &  5   & \ZH{8}    八号    \\
% \bottomrule
% \end{tabular}
% \end{table}
%
% \begin{function}[updated=2014-03-28]{\ziju}
%   \begin{syntax}
%     \tn{ziju} \Arg{中文字符宽度的倍数}
%   \end{syntax}
%   用于调整相邻汉字之间的间距,即(在正常中文行文中)前一个汉字的右边缘与后一个汉字
%   的左边缘之间的距离。其中参数可以是任意浮点数值;而中文字符宽度指的是实际汉字的
%   宽度,不包含当前字距。
%
%   这个命令会影响 \tn{ccwd} 的值,但不会影响英文字距。
% \end{function}
%
% \begin{function}[updated=2014-03-27]{\ccwd}
%   当前汉字的字宽保存在长度寄存器 \tn{ccwd} 之中。汉字字宽是相邻两个汉字中心
%   之间的距离,包含字距在内。因此修改字距会间接修改字宽。
% \end{function}
%
% \subsection{中文数字转换}
%
% \CTeX{} 宏集的中文数字转换功能实际上是调用 \pkg{zhnumber} 宏包来完成。下面只
% 介绍一些基本的用法,更高级的用法可以查阅 \pkg{zhnumber} 宏包的文档。
%
% \begin{function}[updated=2016-05-01]{\chinese}
%   \begin{syntax}
%     \tn{chinese} \Arg{counter}
%     \tn{pagenumbering} \{chinese\}
%   \end{syntax}
%   \tn{chinese} 命令与 \tn{roman} 等命令的用法类似,作用在一个 \LaTeX{}
%   计数器上,将计数器的值以中文数字的形式输出。
% \end{function}
%
% \begin{function}[added=2014-03-08]{\zhnumber}
%   \begin{syntax}
%     \tn{zhnumber} \Arg{number}
%   \end{syntax}
%   以中文格式输出数字。这里的数字可以是整数、小数和分数。
% \end{function}
%
% \begin{function}[added=2014-03-08]{\zhdigits}
%   \begin{syntax}
%     \tn{zhdigits} \Arg{number}
%   \end{syntax}
%   将阿拉伯数字转换为中文数字串。
% \end{function}
%
% \begin{function}{\CTEXnumber}
%   \begin{syntax}
%     \tn{CTEXnumber} "\"<macro> \Arg{number}
%   \end{syntax}
%   |\|<macro> 必须是一个 \TeX{} 宏,不需预先定义。\tn{CTEXnumber} 通过
%   \tn{zhnumber} 将 \meta{number} 转为中文数字,最后将结果存储在 |\|<macro>
%   里。对 |\|<macro> 的定义是局部的,将它展开一次就可以得到转换结果。
% \end{function}
%
% 一般来说,并不需要使用 \tn{CTEXnumber},直接使用 \tn{zhnumber} 即可。但是,如果
% 在文档中需要多次使用同一个数字 \meta{number} 的中文形式,就可以先用
% \tn{CTEXnumber} 将结果保存起来备用,而不是每次使用时都用 \tn{zhnumber} 现场
% 转换一次。
%
% \begin{function}{\CTEXdigits}
%   \begin{syntax}
%     \tn{CTEXdigits} "\"<macro> \Arg{number}
%   \end{syntax}
%   \tn{CTEXdigits} 与 \tn{CTEXnumber} 类似,但其转换的结果是中文数字串,而不是
%   中文数字。
% \end{function}
%
% \subsection{杂项}
%
% \begin{function}{\CTeX}
% 用于显示 \CTeX 标志。
% \end{function}
%
% \section{\LuaLaTeX{} 下的中文支持方式}
% \label{sec:lualatex-chinese}
%
% 在 \LuaLaTeX{} 下,\CTeX{} 宏集依赖 \pkg{LuaTeX-ja} 宏包来完成中文支持。
% 该宏包是日本 \TeX{} 社区的北川弘典、前田一贵、八登崇之等人开发的,设计目的主要
% 是在 \LuaTeX{} 引擎下实现日本 p\TeX{} 引擎的(大部分)功能。它为了兼容 p\LaTeX
% 的使用习惯,对 \LaTeXe 的 \pkg{NFSS} 作了不少修改和扩充。这对于简体中文用户来说
% 不是必要的,因而 \CTeX{} 禁用了它在 \LaTeX{} 格式下的大部分设置,只保留了必要的
% 部分。同时修改了它的字体设置方式,使得相关命令与 \pkg{xeCJK} 宏包大致相同。
%
% 20150420 版以后的 \pkg{LuaTeX-ja} 宏包开始支持竖排,但 \CTeX{} 暂不支持竖排。
%
% \subsection{\LuaLaTeX{} 下替代字体的设置}
%
% \begin{function}[added=2014-04-14]{AlternateFont}
%   \begin{syntax}
%     \tn{setCJKfamilyfont} \Arg{family}
%     \  [
%     \    AlternateFont =
%     \      \{
%     \        \Arg{character range_1} \oarg{alternate font features_1} \Arg{alternate font name_1} ||
%     \        \Arg{character range_2} \oarg{alternate font features_2} \Arg{alternate font name_2} ||
%     \        ......
%     \      \} ,
%     \    <base font features>
%     \  ] \Arg{base font name}
%   \end{syntax}
%   在设置字体族 \meta{family} 的时候,同时设置该字体族在字符范围
%   \meta{character range_n} 内,对应字形的替代字体。
% \end{function}
%
% \begin{function}[added=2014-04-14]{CharRange}
%   \begin{syntax}
%     \tn{setCJKfamilyfont} \Arg{family}
%     \  [
%     \    CharRange = \Arg{character range} ,
%     \    <alternate font features>
%     \  ] \Arg{alternate font name}
%   \end{syntax}
%   只设置字体族 \meta{family} 在字符范围 \meta{character range} 内,对应字形的
%   替代字体。
% \end{function}
%
% 一个 \tn{setCJKfamilyfont} 里只能使用一次 \opt{CharRange} 或者
% \opt{AlternateFont},但可以将它们分开重叠使用。例如下面的方式是有效的。
%
% \begin{ctexexam}
%   \setCJKmainfont[AlternateFont={...}{...}, ...]{...}
%   \setCJKmainfont[CharRange={"4E00->"67FF,-2}, ...]{...}
%   \setCJKmainfont[CharRange={"6800->"9FFF}, ...]{...}
% \end{ctexexam}
%
% \begin{function}[EXP,added=2014-04-14]{declarecharrange}
%   \begin{syntax}
%     \tn{ctexset}
%     \  \{
%     \    declarecharrange =
%     \      \{
%     \        \Arg{name_1} \Arg{character range_1} ,
%     \        \Arg{name_2} \Arg{character range_2} ,
%     \        ...
%     \      \}
%     \  \}
%   \end{syntax}
%   预先声明字符范围。声明字符范围 \meta{name} 之后,它的名字 \meta{name} 可以
%   用在 \opt{AlternateFont} 和 \opt{CharRange} 选项的 \meta{character range}
%   之中,表示对应的字符范围。
% \end{function}
%
% 在声明字符范围 \meta{name} 的同时,还为 \tn{setCJKmainfont} 等字体设置命令定义
% 了选项 \meta{name},用于设置对应字符的替代字体:
% \begin{quote}\linespread{1}\small\ttfamily
%   \meta{name} = \oarg{alternate font features} \Arg{alternate font name}
% \end{quote}
% \meta{name} 选项可以与 \opt{AlternateFont} 共同使用,但不能与 \opt{CharRange}
% 一起使用。如果没有给 \meta{name} 设置值,则等价于设置 \opt{CharRange=\meta{name}},
% 即只设置 \meta{name} 对应的字符范围的替代字体。
%
% \begin{function}[EXP,added=2014-04-15]{clearalternatefont,resetalternatefont}
%   \begin{syntax}
%     \tn{ctexset}
%     \  \{
%     \    clearalternatefont = \Arg{family_1, family_2, ...} ,
%     \    resetalternatefont = \Arg{family_1, family_2, ...} ,
%     \    clearalternatefont ,
%     \    resetalternatefont
%     \  \}
%   \end{syntax}
%   清除与重置 CJK 字体族 \meta{family} 的替换字体设置。如果没有给定值,则作用于
%   当前 CJK 字体族。清除与重置操作总是全局的。
% \end{function}
%
% \section{\CTeX{} 宏集的配置文件}
%
% \CTeX{} 宏集提供了不同的配置文件,可以通过修改配置文件来改变 \CTeX{} 宏集的
% 默认行为。
%
% 在多数情况下,并不需要修改配置文件,\CTeX{} 宏集的默认设置已经能满足大多数用
% 户的需要。不恰当地修改 \CTeX{} 宏集的默认行为也可能导致同一文件在别处无法正
% 常编译或排版效果完全不同,因此修改应该慎重。
%
% 但在一些情况下,直接修改配置文件仍是必要的,例如:
% \begin{itemize}
% \item 系统没有安装默认设置的字体文件,无法编译。
% \item 需要经常编译来自其他系统的中文 \TeX{} 文件,但对方的操作系统或默认设置
% 与本机不同。
% \end{itemize}
%
% 与 \CTeX{} 宏集的源代码一样,配置文件采用 \LaTeXiii{} 的语法编写。
%
% \CTeX{} 宏集的配置文件随宏包其他文件一起安装在 \TeX{} 系统 TDS 目录树中,文
% 件后缀是 \file{.cfg}。为了避免本地配置文件内容因 \CTeX{} 宏集的更新而丢失,
% 不要直接修改系统 TDS 目录树中的配置文件,而应该将系统自带的配置文件复制到本
% 地的或用户私有的 TDS 目录树中修改,并运行 \bashcmd{texhash} 命令刷新文件名数据库。
%
% 例如对于 \TeX{} Live,系统自带的配置文件就在 \TeX{} Live 安装目录下的
% \path{texmf-dist/tex/latex/ctex/config/} 子目录下,可以修改它的副本,保存在
% 本地 TDS 树的 \path{texmf-local/tex/latex/ctex/} 目录下,或者用户 TDS 树的
% \path{~/.texlive2015/texmf-var/tex/latex/ctex/} 目录下,作为本地/用户专有的
% 配置文件。复制配置文件后需要运行 |texhash| 命令使本地配置文件生效。
%
% \MiKTeX{} 的配置文件也保存在类似的目录结构中,\MiKTeX{} 管理的
% 几个 TDS 根目录可以在 \MiKTeX{} Options 设置项中查看到,这里不再赘述。
%
% 除了修改本地 \TeX{} 系统中的配置文件,对于特定文档,也可以将修改过的配置文件
% 保存在文档的工作目录下。此时配置文件就只对工作目录下的所有文档生效。
%
% \subsection{修改宏包默认选项}
%
% 配置文件 \file{ctexopts.cfg} 可以用来修改宏包的默认选项。随系统安装的配置文
% 件除了文件信息声明外没有实际的内容,但在注释中给出了一个简单的示例,只要取消
% 注释就可以生效。
%
% \begin{ctexexam}
%   % 系统自带 ctexopts.cfg 注释中的示例语句,固定默认字体集为 windowsnew。
%   % 该设置可以用在安装了 Windows 字体的非 Windows 系统中。
%   \keys_set:nn { ctex / option } { fontset = windowsnew }
% \end{ctexexam}
% 如上例所示,宏包选项通常使用 \LaTeXiii{} 的 \cs{keys_set:nn} 命令完成键值设置,
% 第一个参数是固定的子模块 |ctex/option|,第二个参数中是用户定义的新的默认宏包
% 选项。
%
% \file{ctexopts.cfg} 中的设置将在 \CTeX{} 宏集的开始处,定义过宏包选项之后,
% \tn{ProcessKeysOptions} 命令之前生效。最好只使用此配置文件修改宏包默认选项。
%
% \subsection{宏包载入后的配置}
%
% 配置文件 \file{ctex.cfg} 将在宏包的末尾被载入生效。可以用它完成任意的设置,
% 或是覆盖已有的定义。随系统安装的配置文件除版本信息外没有实际内容,注意配置文
% 件中也使用 \LaTeXiii{} 语法。
%
% \begin{ctexexam}
%   % 简单的 ctex.cfg 内容示例。
%   % 修改默认的页面格式设置。
%   \pagestyle{plain}
% \end{ctexexam}
%
% \begin{ctexexam}
%   % 略复杂的 ctex.cfg 内容示例:禁止段末孤字成行。
%   % 在使用 XeTeX 编译时,打开 xeCJK 的 CheckSingle 选项。
%   \sys_if_engine_xetex:T
%     {
%       \xeCJKsetup { CheckSingle }
%     }
%   % 在使用 LuaTeX 编译时,设置 LuaTeX-ja 的 jcharwidowpenalty 参数。
%   \sys_if_engine_luatex:T
%     {
%       \ltjsetparameter { jcharwidowpenalty = 10000 }
%     }
% \end{ctexexam}
%
% \subsection{配置标题中文翻译}
%
% 由于 \CTeX{} 宏集需要同时支持 GBK 和 UTF-8 两种编码,因此对标题的中文翻译写
% 在两个配置文件当中:\file{ctex-name-gbk.cfg} 和 \file{ctex-name-utf8.cfg}。
% 两个文件的设置相同,只是编码不同。
%
% 为了同一文档在不同电脑上编译效果的一致性,通常不建议修改默认的中文翻译。
%
% \subsection{自定义字体集}
%
% \ref{subs:options-CJK-font}~节介绍的用于 |fontset| 选项的自定义字库文件,
% 类似于 \CTeX{} 宏集的配置文件,也应该与其他本地配置文件一起保存在本地
% \texttt{TDS} 目录树下,并可以配合 \file{ctexopts.cfg} 等配置文件使用。
%
% \section{对旧版本的兼容性}
%
% \subsection{\CTeX\ 0.8a 及以前的版本}
%
% 在 ctex-kit 项目成立之前,\CTeX 宏包的最后一个版本是 \CTeX\
% 0.8a(2007/05/06)。
%
% 第 2 版未考虑对这些很早版本的兼容性。
%
% \subsection{\CTeX\ 0.9--\CTeX\ 1.0d}
%
% 在 2009 年在 ctex-kit 项目成立后,新增了 \XeTeX{} 引擎的支持,并增加了不少控
% 制字体的命令和选项。
%
% 这里主要介绍新版本 \CTeX 宏包相对 1.02d 版本(2014/06/09)的兼容性。
%
% 第 2 版的 \CTeX 宏包已尽力保证对 1.0x 版本的兼容性,原有为 1.0x 编写的代码,
% 在第 2 版的 \CTeX 宏包下保证仍能编译,并且在大多数情况下保持编译效果不变。
%
% \CTeX 宏包在 0.8a 以前的版本支持以 \pkg{CCT} 作为底层中文支持方式,从 0.9 版
% 之后即不再推荐使用,只保留向后兼容。在 \CTeX 宏包第 2 版中则完全不再支持
% \pkg{CCT}。
%
% 下面这些是在旧版本 \CTeX 宏包中存在,而在新版本中已不建议使用的选项和命令,
% 在未来版本中可能会删去它们的支持。
%
% 在多数情况下它们的功能仍将保留,但也有部分选项命令功能已失效。
%
% \begin{function}{cs4size, c5size}
%   分别相当于 |zihao=-4| 和 |zihao=5|,过时选项。
% \end{function}
%
% \begin{function}{CCT, CCTfont}
% 相关选项已删除。
% \end{function}
%
% \begin{function}{indent, noindent}
% \opt{indent} 和 \opt{noindent} 什么也不做,过时选项。
%
% 在中文版式下,\pkg{ctex} 宏包的相关功能在与标准文档类及其衍生文档类联用时
% 默认打开。\CTeX{} 文档类的相关功能由章节标题的 \opt{afterindent} 选项的值
% 来确定。
% \end{function}
%
% \begin{function}[label = ]{zhmap, nozhmap}
% \opt{zhmap} 宏包选项增加了参数,扩充了功能,除了支持真假值参数外,还支持选择
% \pkg{zhmCJK} 作为底层中文处理宏包。(\ref{subs:options-CJK-font}~节)
%
% \opt{nozhmap} 选项相当于 |zhmap=false|。过时选项。
% \end{function}
%
% \begin{function}{winfonts, adobefonts, nofonts}
% 宏包选项 \opt{winfonts} 相当于 |fontset=windows|,\opt{adobefonts} 相当于
% |fontset=adobe|,\opt{nofonts} 相当于 |fontset=none|。这几个选项是过时选项,
% 对于新文档,应使用 \opt{fontset} 选项设置不同字体集。
%
% 另外,第 2 版 \CTeX 宏包的默认字体不再是 Windows 系统字体,而是根据检测到的
% 操作系统选择使用 Windows、Mac 的系统字体还是 Fandol 字体
% (\ref{subs:options-CJK-font}~节)。
% \end{function}
%
% \begin{function}[label = ]{punct, nopunct}
% 旧版本中宏包 \opt{punct} 选项没有参数,现在可以用参数设定标点风格
% (\ref{subs:options-type-style}~节)。原有无参形式的 \opt{punct} 选项相当
% 于 |punct=quanjiao|。
%
% 旧版宏包中 \opt{nopunct} 选项的效果大致相当于 |punct=plain|。过时选项,不推荐使用。
% \end{function}
%
% \begin{function}{cap, nocap}
% 原有的 \opt{cap} 和 \opt{nocap} 选项由新的 \opt{scheme} 选项代替。
% (\ref{subs:options-type-style}~节)
%
% \opt{cap} 选项相当于 |scheme = chinese|,\opt{nocap} 选项相当
% 于 |scheme = plain|。它们均已过时,仅因兼容性而保留。
% \end{function}
%
% \begin{function}[label = ]{space, nospace}
% 新版本宏包 \opt{space} 选项增加真假值参数。
% (\ref{subs:options-type-style}~节)
%
% \opt{nospace} 选项相当于 |space=false|,成为过时选项。
% \end{function}
%
% \begin{function}{fancyhdr}
% 新版本宏包中总是自动处理对 \pkg{fancyhdr} 宏包的兼容性,而由用户自己使用
% \tn{usepackage} 载入 \pkg{fancyhdr} 宏包。
%
% \opt{fancyhdr} 选项过时,因兼容性保留,功能是载入 \pkg{fancyhdr} 宏包。
% \end{function}
%
% \begin{function}{hyperref}
% 新版本宏包中总是自动处理对 \pkg{hyperref} 宏包的兼容性,而由用户自己使用
% \tn{usepackage} 载入 \pkg{hyperref} 宏包。
%
% \opt{hyperref} 选项过时,因兼容性保留,功能是在导言区末尾载入 \pkg{hyperref}
% 宏包。
% \end{function}
%
% \begin{function}{fntef}
% 旧版本的 \opt{fntef} 选项用于统一 \pkg{CCTfntef} 与 \pkg{CJKfntef} 的界面,
% 新版本 \CTeX{} 宏集不再支持 \pkg{CCT},而是直接载入 \pkg{CJKfntef} 或
% \pkg{xeCJKfntef} 宏包并禁用其彩色设置。该选项是过时选项。
% \end{function}
%
% \begin{function}{\CTEXunderdot, \CTEXunderline, \CTEXunderdblline,
%   \CTEXunderwave, \CTEXsout, \CTEXxout, \CTEXfilltwosides}
% 在调用 \opt{fntef} 宏包选项的同时,旧版本 \CTeX{} 宏包由于需要支持 \pkg{CCT}
% 系统,会将以 |\CJK| 开头的 \tn{CJKunderline} 等宏换名为以 |\CTEX| 开头的
% \tn{CTEXunderline} 等宏。此功能在新版本的 \CTeX{} 宏集中已失去意义。
%
% 此外,在 \pdfTeX{} 引擎下,用于设置格式的 \tn{CJKunderdotbasesep} 等宏也被换
% 名为 \tn{CTEXunderdotbasesep} 等宏。
%
% 在新版本中,上述由 \opt{fntef} 衍生的相关宏都成为过时命令。
% \end{function}
%
% \begin{function}{\CTEXsetfont}
% 更新当前的中文字体信息,包括当前字距(\tn{ccwd})和段首缩进(\tn{parindent})。
% 一般来说,用户无需使用这个命令。
% \end{function}
%
% \begin{function}{\CTEXindent}
% 更新 \tn{ccwd} 宽度后设置 |\parindent=2\ccwd|。过时命令。
% \end{function}
%
% \begin{function}{\CTEXnoindent}
% 设置 |\parindent=0pt|。过时命令。
% \end{function}
%
% \begin{function}{\CTEXsetup}
% \begin{syntax}
%   \tn{CTEXsetup}\oarg{选项}\Arg{标题}
% \end{syntax}
% 相当于设置了
% \texttt{\tn{ctexset}\{ \meta{标题} = \Arg{选项} \}}。
% 过时命令。
% \end{function}
%
% \begin{function}{\CTEXoptions}
% \begin{syntax}
%   \tn{CTEXoptions}\oarg{选项}
% \end{syntax}
% 相当于设置了
% \texttt{\tn{ctexset}\Arg{选项}}。
% 过时命令。
% \end{function}
%
% \begin{function}{\Chinese}
% \begin{syntax}
%   \tn{Chinese}\Arg{counter}
% \end{syntax}
% 新版宏集中 \tn{chinese} 统一了旧版本中 \tn{chinese} 和 \tn{Chinese} 的功能。因此,
% 该命令已过时。
% \end{function}
%
% \begin{function}{captiondelimiter}
% 原为 \tn{CTEXoptions} 命令的选项,用于控制 \tn{caption} 编号后面的标点。此选
% 项已过时,并在新版本的 \CTeX 宏包中失效。
%
% 可以使用 \pkg{caption} 宏包的 \opt{labelsep} 选项来完成同样的功能。
% \begin{ctexexam}
%   % 代替 \CTEXoptions[captiondelimiter={:}]
%   \usepackage{caption}
%   \captionsetup{labelsep=colon}
% \end{ctexexam}
% \end{function}
%
% \subsection{\CTeX\ 1.02c 以后的 SVN 开发版}
%
% \CTeX 宏包在 1.02c 版本(2011/03/11)之后在 Google code 上的 SVN 开发版本,
% 内部版本号一直升到 1.11 版,但从未正式发布。SVN 开发版在 1.02c 版本的基础上
% 新增的功能在第 2 版中大多继承了过来,但新增的命令与选项都不再保持兼容。
%
% \CTeX 宏包第 2 版不保证对未发布的 SVN 开发版兼容。
%
% \subsection{\CTeX\ 2.2 之前的版本}
%
% \begin{function}{part/beforeskip, chapter/beforeskip, section/beforeskip,
%   subsection/beforeskip, subsubsection/beforeskip, paragraph/beforeskip,
%   subparagraph/beforeskip}
%   在 \CTeX\ 2.2 之前的版本中,\opt{beforeskip} 选项的符号还用于确定章节标题后
%   首段的缩进。当 \opt{beforeskip} 是负值时,章节标题后的第一段按英文文档的排版
%   习惯,没有首行缩进,否则保留首行缩进。
%
%   这一特性在 2.2 版和后续版本中不再保留,相应的功能通过新的 \opt{afterindent}
%   选项来设置。如果原先设置 \opt{beforeskip} 为负值,在新版本中需要改为正值,
%   并设置相应的 \opt{afterindent} 选项为 \opt{false}。
% \end{function}
%
% \begin{function}{section/afterskip, subsection/afterskip,
%   subsubsection/afterskip, paragraph/afterskip, subparagraph/afterskip}
%   在 \CTeX\ 2.2 之前的版本中,对于 \tn{section} 级以下标题,\opt{afterskip}
%   选项的符号用于确定标题与随后正文是否排在同一段。
%   如果是正值,则正文另起一段,否则标题与随后正文排在同一段,
%   \opt{afterskip} 的绝对值给出水平间距。
%
%   这一特性在 2.2 版和后续版本中不再保留,相应的功能通过新的 \opt{runin}
%   选项来设置。如果原先设置 \opt{afterskip} 为负值,在新版本中需要改为正值,
%   并设置相应的 \opt{runin} 选项为 \opt{true}。
% \end{function}
%
% \subsection{\CTeX\ 2.4.1 和 2.4.2}
%
% \begin{function}{part/fixbeforeskip, chapter/fixbeforeskip}
%   这两个选项已经被删除,相应功能由新的选项 \opt{fixskip} 提供。
% \end{function}
%
% \section{宏集依赖情况与手工安装方法}
% \label{sec:dep-ins}
%
% 本节介绍 \CTeX{} 宏集的依赖情况,并介绍手工编译安装的具体方法。
% 通常用户只需参照第 \ref{subsec:easy-ins}~节介绍的方法,使用发行版自带的宏包管理器安装
% 本宏集。
%
% \CTeX{} 宏集有两个源文件:\file{ctex.dtx}、\file{ctexpunct.spa}。
% 使用不同的编译方式时,\CTeX{} 依赖的宏包略有不同。在手工安装 \CTeX{} 宏集之前,请确保
% 你的 \TeX{} 发行版中已经正确安装了这些宏包。\CTeX{} 依赖宏包的详情叙述如下:
%
% \begin{itemize}
%   \item \pkg{expl3}、\pkg{xparse} 和 \pkg{l3keys2e} 宏包。它们属于 \pkg{l3kernel}
%   和 \pkg{l3packages} 宏集。
%   \item \pkg{indentfirst} 宏包,属于 \pkg{tools} 宏集。
%   \item \pkg{everysel} 宏包,属于 \pkg{ms} 宏集。
%   \item \pkg{ulem} 宏包。
%   \item \pkg{zhnumber} 宏包。
%   \item[\ding{229}] 以上是各种编译方式都必需的依赖项。
%   \item \pkg{CJK} 宏集。
%   \item \pkg{CJKpunct} 宏包。
%   \item \pkg{xCJK2uni} 宏包。
%   \item \pkg{zhmetrics} 宏包。
%   \item \pkg{zhmCJK} 宏包,它还依赖 \pkg{oberdiek} 宏集。
%   \item[\ding{229}] 以上是使用 \pdfLaTeX{} 或 \LaTeX{} + \dvipdfmx{} 的编译方式所需要
%   的依赖项,其中 \pkg{zhmCJK} 是可选的。
%   \item \pkg{xeCJK} 宏包,它还依赖
%   \begin{itemize}
%     \item \pkg{fontspec} 宏包,它还依赖
%     \begin{itemize}
%       \item \pkg{euenc} 宏包。
%       \item \pkg{xunicode} 宏包,它还依赖
%       \begin{itemize}
%         \item \pkg{graphics} 宏集。
%         \item \pkg{graphics-cfg} 宏包。
%         \item \pkg{graphics-def} 宏包。
%       \end{itemize}
%     \end{itemize}
%   \end{itemize}
%   \item \pkg{environ} 宏包,它还依赖 \pkg{trimspaces} 宏包。
%   \item[\ding{229}] 以上是使用 \XeLaTeX{} 编译时的依赖项。
%   \item \pkg{LuaTeX-ja} 宏集,它还依赖
%   \begin{itemize}
%     \item \pkg{adobemapping} 宏包。
%     \item \pkg{fontspec} 宏包。
%     \item \pkg{lualibs} 宏包。
%     \item \pkg{luaotfload} 宏包。
%     \item \pkg{luatexbase} 宏包,它还依赖 \pkg{ctablestack} 宏包。
%     \item \pkg{oberdiek} 宏集。
%     \item \pkg{xkeyval} 宏包。
%   \end{itemize}
%   \item[\ding{229}] 以上是使用 \LuaLaTeX{} 编译时的依赖项。
%   \item \pkg{pxeverysel} 宏包,属于 \pkg{platex-tools} 宏集。
%   \item \pkg{zhmetrics-uptex} 宏包。
%   \item[\ding{229}] 以上是使用 \upLaTeX{} 编译时的依赖项。
% \end{itemize}
%
% 出于一些原因,\pkg{zhmCJK} 尚未被收入 \TeXLive{} 和 \MiKTeX。因此,若
% 你希望使用 \pkg{zhmCJK} 作为 \CTeX{} 宏集的底层中文支持方式,那么你需要自行安装该宏包。
% \pkg{zhmCJK} 的安装较为复杂。我们建议你
% \begin{enumerate}
%   \item 从 CTAN 下载 \pkg{zhmCJK} 宏包的
%   \href{http://mirrors.ctan.org/install/language/chinese/zhmcjk.tds.zip}
%   {TDS 安装包},
%   \item 按目录结构将文件复制到 \TeX{} 发行版的本地 TDS 根目录,
%   \item 最后执行 \bashcmd{texhash} 刷新 \TeX{} 发行版的 ls-R 数据库以完成安装。
% \end{enumerate}
% 其他细节,可参照其
% \href{http://mirrors.ctan.org/language/chinese/zhmcjk/zhmCJK.pdf}{宏包手册}
% 中第 3 节的指导。
%
% \emph{\CTeX{} 宏集已被 \TeXLive{} 和 \MiKTeX{} 收录,若无特别理由,
% 我们强烈建议用户使用宏包管理器安装本宏集。}
%
% 若要手工安装,请遵循如下步骤:
% \begin{enumerate}
%   \item 从 CTAN 下载 \CTeX{} 宏集的
%   \href{http://mirrors.ctan.org/install/language/chinese/ctex.tds.zip}
%   {TDS 安装包},
%   \item 按目录结构将文件复制到 \TeX{} 发行版的本地 TDS 根目录,
%   \item 最后执行 \bashcmd{texhash} 刷新 \TeX{} 发行版的 ls-R 数据库以完成安装。
% \end{enumerate}
%
% \section{开发人员}
%
% \begin{itemize}
% \item 吴凌云 (\email{aloft@ctex.org})
% \item 江疆 (\email{gzjjgod@gmail.com})
% \item 王越 (\email{yuleopen@gmail.com})
% \item 刘海洋 (\email{LeoLiu.PKU@gmail.com})
% \item 李延瑞 (\email{LiYanrui.m2@gmail.com})
% \item 陈之初 (\email{zhichu.chen@gmail.com})
% \item 李清 (\email{sobenlee@gmail.com})
% \item 黄晨成 (\email{liamhuang0205@gmail.com})
% \end{itemize}
%
% 目前比较活跃的开发维护人员是刘海洋、李清和黄晨成。
%
%
% \begin{thebibliography}{9}
% \bibitem{knuthtex1986}
% \textsc{Donald~Ervin Knuth}.
% \newblock \textit{The {{\TeX{}book}}}, \textit{Computers \& Typesetting},
%   volume~A.
% \newblock Addison-Wesley, 1986
%
% \bibitem{mittelbach2004}
% \textsc{Frank Mittelbach} and \textsc{Michel Goossens}.
% \newblock \textit{The {{\LaTeX}} Companion}.
% \newblock Tools and Techniques for Computer Typesetting. Boston:
%   Addison-Wesley, second edition, 2004
%
% \end{thebibliography}
%
% \end{documentation}
%
%
% \StopEventually{}
%
%
%\begin{implementation}
% \clearpage
% \section{代码实现}
%
%    \begin{macrocode}
%<@@=ctex>
%    \end{macrocode}
%
% 宏包载入检查。
%    \begin{macrocode}
%<*class|ctex>
\tl_const:Nx \c_@@_version_tl
  { \cs_if_exist_use:cF { ver@ \@currname . \@currext } { 9999/99/99 } }
%<*class>
\cs_new_eq:cN { ver@ctex.        \@pkgextension } \c_@@_version_tl
\cs_new_eq:cN { ver@ctexcap.     \@pkgextension } \c_@@_version_tl
\cs_new_eq:cN { ver@ctexsize.    \@pkgextension } \c_@@_version_tl
\cs_new_eq:cN { ver@ctexheading. \@pkgextension } \c_@@_version_tl
%</class>
%<*ctex>
\msg_new:nnnn { ctex } { subpackage-loaded }
  { Package~`#1'~can~not~be~loaded~with~`ctex'. }
  {
    `#1'~is~actually~a~part~of~`ctex'.\\
    It~is~not~necessary~to~load~it~separately.
  }
\@ifpackageloaded { ctexsize }
  { \msg_error:nnn { ctex } { subpackage-loaded } { ctexsize } }
  { \cs_new_eq:cN { ver@ctexsize. \@pkgextension } \c_@@_version_tl }
\@ifpackageloaded { ctexheading }
  { \msg_error:nnn { ctex } { subpackage-loaded } { ctexheading } }
  { \cs_new_eq:cN { ver@ctexheading. \@pkgextension } \c_@@_version_tl }
%</ctex>
%</class|ctex>
%    \end{macrocode}
%
%    \begin{macrocode}
%<*class|style>
\RequirePackage { xparse , l3keys2e }
%</class|style>
%    \end{macrocode}
%
%    \begin{macrocode}
%<*class|ctex>
%    \end{macrocode}
%
% \changes{v2.3}{2015/12/20}{与 \LaTeXiii{} (2015/12/20) 同步。}
% \changes{v2.4.10}{2017/07/19}{常数 \cs{c_minus_one} 已过时。}
% \changes{v2.4.10}{2017/07/22}{使用 \texttt{lazy} 函数对 Boolean 表达式
% 进行最小化运算(\hologo{LaTeX3} 2017/07/19)。}
%
% 检查 \pkg{expl3} 和 \pkg{l3keys2e} 的版本。
%    \begin{macrocode}
\msg_new:nnnn { ctex } { l3-too-old }
  { Support~package~`#1'~too~old. }
  {
    Please~update~an~up~to~date~version~of~the~bundles\\\\
    `l3kernel'~and~`l3packages'\\\\
    using~your~TeX~package~manager~or~from~CTAN.
  }
\@ifpackagelater { expl3 } { 2017/07/19 } { }
  { \msg_error:nnn { ctex } { l3-too-old } { expl3 } }
%<*class>
\@ifpackagelater { l3keys2e } { 2015/12/20 } { }
  { \msg_error:nnn { ctex } { l3-too-old } { l3keys2e } }
%</class>
%    \end{macrocode}
%
% \begin{variable}[int]{\c_@@_engine_str,\c_@@_engine_file_str}
% 引擎检查。目前 \LaTeXiii{} 将 \ApTeX{} 识别为 \upTeX。
%    \begin{macrocode}
\str_const:Nx \c_@@_engine_str
  { \cs_if_exist:NTF \ngostype { aptex } { \c_sys_engine_str } }
\msg_new:nnnn { ctex } { engine-not-supported }
  { Engine~`#1'~is~not~yet~supported,~ctex~will~abort! }
  { You~can~switch~to~xelatex,~lualatex,~pdflatex,~uplatex,~or~aplatex. }
\file_if_exist:nTF { ctex-engine- \c_@@_engine_str .def }
  {
    \str_const:Nx \c_@@_engine_file_str
      { ctex-engine- \c_@@_engine_str .def }
  }
  { \msg_critical:nnx { ctex } { engine-not-supported } { \c_@@_engine_str } }
%    \end{macrocode}
% \end{variable}
%
%    \begin{macrocode}
%</class|ctex>
%    \end{macrocode}
%
%    \begin{macrocode}
%<*class|ctex|ctexheading>
%    \end{macrocode}
%
%    \begin{macrocode}
\RequirePackage { ctexhook , ctexpatch }
%<!ctexheading>\RequirePackage { fix-cm , everysel }
%    \end{macrocode}
%
% \subsection{内部函数与变量}
%
% \begin{variable}[int]{\l_@@_tmp_tl,\l_@@_tmp_int,\l_@@_tmp_box,\l_@@_tmp_dim}
% 临时变量。
%    \begin{macrocode}
\tl_clear_new:N \l_@@_tmp_tl
\int_new:N \l_@@_tmp_int
\box_new:N \l_@@_tmp_box
%<!ctexheading>\dim_new:N \l_@@_tmp_dim
%    \end{macrocode}
% \end{variable}
%
% \begin{macro}[int]{\ctex_file_wrapper:nnn}
% 设置文件操作的 \tn{catcode} 环境,参数 |#1| 是设置,|#2| 是文件操作,|#3| 是恢复。
% 默认关闭 \LaTeXiii{} 语法环境,并设置 |@| 的 \tn{catcode} 为 $11$。
%    \begin{macrocode}
\cs_new_protected:Npn \ctex_file_wrapper:nnn #1#2#3
  {
    \use:x
      {
        \ExplSyntaxOff
        \char_set_catcode_letter:n { 64 }
        #1
        \exp_not:n {#2}
        \bool_if:NTF \l__kernel_expl_bool
          { \ExplSyntaxOn }
          { \ExplSyntaxOff }
        \char_set_catcode:nn { 64 } { \char_value_catcode:n { 64 } }
        #3
      }
  }
%    \end{macrocode}
% \end{macro}
%
% \begin{macro}[int]{\ctex_file_input:n}
% 输入文件。
%    \begin{macrocode}
\cs_new_protected_nopar:Npn \ctex_file_input:n #1
  { \ctex_file_wrapper:nnn { } { \file_input:n {#1} } { } }
%    \end{macrocode}
% \end{macro}
%
% \begin{macro}[int]{\ctex_scheme_input:n}
% 输入 \opt{scheme} 文件。先查找当前文档类下的 \meta{scheme},找不到再查找一般的文件。
%    \begin{macrocode}
\cs_new_protected_nopar:Npn \ctex_scheme_input:n #1
  {
    \ctex_file_wrapper:nnn
      { }
      {
        \tl_if_exist:NTF \c_@@_class_tl
          {
            \file_if_exist_input:nF { ctex-scheme- #1 - \c_@@_class_tl .def }
              { \file_input:n  { ctex-scheme- #1 .def } }
          }
          { \file_input:n  { ctex-scheme- #1 .def } }
      }
      { }
  }
\cs_generate_variant:Nn \ctex_scheme_input:n { o }
%    \end{macrocode}
% \end{macro}
%
% \begin{variable}[int]{\g_@@_section_depth_flag}
% 若大于 |3|,则 \tn{paragraph} 和 \tn{subparagraph} 标题单独占一行;若为 |3|,则
% \tn{paragraph} 单独占一行。
%    \begin{macrocode}
%<*!beamer>
\cs_new_eq:NN \g_@@_section_depth_flag \c_two
%</!beamer>
%    \end{macrocode}
% \end{variable}
%
%    \begin{macrocode}
%</class|ctex|ctexheading>
%<*class|ctex>
%    \end{macrocode}
%
% 对旧版本的宏包给出错误信息。
%    \begin{macrocode}
\msg_new:nnnn { ctex } { package-too-old }
  { Support~package~`#1'~too~old. }
  {
    Please~update~an~up~to~date~version~of~the~package~`#1'\\
    using~your~TeX~package~manager~or~from~CTAN.
  }
%    \end{macrocode}
%
% \changes{v2.1}{2015/05/25}{不依赖 \pkg{ifpdf} 宏包。}
%
% \begin{macro}[int]{\ifctexpdf}
% 在 \pkg{zhmetrics} 映射文件中使用。
%    \begin{macrocode}
\sys_if_output_pdf:TF
  { \cs_new_eq:NN \ifctexpdf \if_true: }
  { \cs_new_eq:NN \ifctexpdf \if_false: }
%    \end{macrocode}
% \end{macro}
%
% \begin{macro}[int]{\ctex_if_preamble:TF}
% 测试是否在 \LaTeXe{} 的导言区。在宏包内部初始为真,文档最开始位置再设置为假。
% 注意,钩子 \cs{ctex_after_end_preamble:n} 在 \tn{AtBeginDocument} 之后执行,
% 可以与 \tn{@onlypreamble} 的行为一致。
%    \begin{macrocode}
\cs_new_eq:NN \ctex_if_preamble:TF \use_i:nn
\ctex_after_end_preamble:n { \cs_set_eq:NN \ctex_if_preamble:TF \use_ii:nn }
%    \end{macrocode}
% \end{macro}
%
% \begin{macro}[int]{\ctex_set_default_ccwd:Nn}
% 若参数 |#2| 带长度单位,则设置它为 |tl| 变量 |#1| 的值,否则以 \tn{ccwd} 为单位。
%    \begin{macrocode}
\cs_new:Npn \ctex_set_default_ccwd:Nn #1#2
  {
    \dim_compare:nNnTF
      { \ctex_default_pt:n {#2} } = { \ctex_default_pt:n { #2 ~ mm } }
      { \tl_set:Nn #1 {#2} }
      { \tl_set:Nn #1 { #2 \ccwd } }
  }
%    \end{macrocode}
% \end{macro}
%
% \changes{v2.3}{2015/09/17}{代码实现避免使用 \tn{lowercase} 技巧(Joseph Wright)。}
%
% \begin{macro}[int]{\ctex_default_pt:n}
% 最新版本的 \pkg{expl3} 已经不允许 \cs{dim_to_decimal:n} 的参数带额外的单位。
% 然而我们需要这个特性实现可展的 \tn{@defaultunits}。
%    \begin{macrocode}
\cs_new:Npn \ctex_default_pt:n #1
  {
    \exp_after:wN \@@_default_pt:w
      \dim_use:N \etex_dimexpr:D #1 pt \scan_stop: \q_stop
  }
\use:x
  {
    \cs_new:Npn \exp_not:N \@@_default_pt:w
      ##1 \tl_to_str:n { pt } ##2 \exp_not:N \q_stop
      { ##1 \tl_to_str:n { pt } }
  }
%    \end{macrocode}
% \end{macro}
%
% \begin{variable}[int]{\l_@@_encoding_tl}
% (pdf)\LaTeX{} 初始化编码为 GBK,其它则是 UTF8。
%    \begin{macrocode}
\tl_new:N \l_@@_encoding_tl
\tl_set:Nx \l_@@_encoding_tl
  { \sys_if_engine_pdftex:TF { GBK } { UTF8 } }
%    \end{macrocode}
% \end{variable}
%
% \begin{variable}[int]{\g_@@_zhmCJK_bool}
% 是否使用 \pkg{zhmCJK} 宏包。
%    \begin{macrocode}
\bool_new:N \g_@@_zhmCJK_bool
%    \end{macrocode}
% \end{variable}
%
% \begin{variable}[int]{\l_@@_autoindent_tl}
% 保存 \opt{autoindent} 选项的值,空值表示不自动调整首行缩进。
%    \begin{macrocode}
\tl_new:N \l_@@_autoindent_tl
%    \end{macrocode}
% \end{variable}
%
% \begin{macro}[int]{\ctex_if_autoindent_touched:F}
% 检查 \opt{autoindent} 选项是否被用户设置。
%    \begin{macrocode}
\cs_new_eq:NN \ctex_if_autoindent_touched:F \use:n
%    \end{macrocode}
% \end{macro}
%
% \begin{macro}[int]{\ctex_zhmap_case:nnn}
% 参数 |#1| 是 \pkg{zhmCJK} 的内容,|#2| 是 \pkg{zhmetrics}。
%    \begin{macrocode}
\cs_new_eq:NN \ctex_zhmap_case:nnn \use_ii:nnn
%    \end{macrocode}
% \end{macro}
%
% \begin{macro}[int]{\ctex_at_end:n}
% 区分 \tn{AtEndOfClass} 和 \tn{AtEndOfPackage},虽然它们的意思都是一样的。
%    \begin{macrocode}
%<class>\cs_new_protected_nopar:Npn \ctex_at_end:n { \AtEndOfClass }
%<ctex>\cs_new_protected_nopar:Npn \ctex_at_end:n { \AtEndOfPackage }
%    \end{macrocode}
% \end{macro}
%
% \begin{variable}[int]{\g_@@_std_options_clist}
% 保存传递给标准文档类的选项。
%    \begin{macrocode}
%<*class>
\clist_new:N \g_@@_std_options_clist
%</class>
%    \end{macrocode}
% \end{variable}
%
% 对无效选项给出警告。
%    \begin{macrocode}
\msg_new:nnn { ctex } { invalid-option }
  { Option~`\l_keys_key_tl'~is~invalid~in~current~mode. }
\msg_new:nnn { ctex } { invalid-value }
  { Value~`#1'~is~invalid~for~the~key~`\l_keys_key_tl'. }
%    \end{macrocode}
%
% 对过时选项或命令给出警告。
%    \begin{macrocode}
\msg_new:nnn { ctex } { deprecated-option }
  { Option~ `\l_keys_key_tl'~ is~ deprecated.\\ #1 }
\msg_new:nnn { ctex } { deprecated-command }
  { Command~ #1 is~ deprecated.\\ #2 }
\msg_new:nnn { ctex } { deprecated-environment }
  { Environment~ `#1'~ is~ deprecated.\\ #2 }
%    \end{macrocode}
%
%    \begin{macrocode}
%</class|ctex>
%    \end{macrocode}
%
% \begin{variable}[int]{\g_@@_font_size_flag}
% |0| 表示修改默认字体大小为五号,|1| 为小四号,大于 1 则不作修改。初始值 |-1|
% 表示 \opt{zihao} 选项未初始化,会在将来根据文档类决定初值。
%    \begin{macrocode}
%<*class|ctex|ctexsize>
\int_new:N \g_@@_font_size_flag
\int_set:Nn \g_@@_font_size_flag { -1 }
%</class|ctex|ctexsize>
%    \end{macrocode}
% \end{variable}
%
% \subsection{宏包选项}
%
% \changes{v2.3}{2015/09/25}
%   {\texttt{.value_required:} 和 \texttt{.value_forbidden:} 已过时。}
%
%   \begin{macrocode}
%<*class|style>
\keys_define:nn { ctex / option }
  {
%</class|style>
%    \end{macrocode}
%
% \changes{v2.0}{2015/05/06}{新增 \opt{zihao} 选项。}
% \changes{v2.0}{2015/05/06}{\opt{c5size}, \opt{cs4size} 是过时选项。}
%
% \begin{macro}{zihao}
% \changes{v2.4.1}{2016/05/13}{不允许无参 \opt{zihao} 选项。}
%    \begin{macrocode}
%<*class|ctex|ctexsize>
    zihao .choice: ,
    zihao .value_required:n = true ,
    zihao /     5  .code:n = { \cs_gset_eq:NN \g_@@_font_size_flag \c_zero } ,
    zihao /    -4  .code:n = { \cs_gset_eq:NN \g_@@_font_size_flag \c_one } ,
    zihao / false  .code:n = { \cs_gset_eq:NN \g_@@_font_size_flag \c_two } ,
%<ctexsize>  }
%</class|ctex|ctexsize>
%<*class|ctex>
    c5size  .code:n =
      {
        \msg_warning:nnn { ctex } { deprecated-option }
          { Option~ `zihao=5'~ is~ set. }
        \keys_set:nn { ctex / option } { zihao = 5 }
      } ,
    cs4size  .code:n =
      {
        \msg_warning:nnn { ctex } { deprecated-option }
          { Option~ `zihao=-4'~ is~ set. }
        \keys_set:nn { ctex / option } { zihao = -4 }
      } ,
    c5size  .value_forbidden:n = true ,
    cs4size .value_forbidden:n = true ,
%    \end{macrocode}
% \end{macro}
%
% \changes{v2.0}{2014/04/23}{新增 \opt{linespread} 选项。}
%
% \begin{macro}{linespread}
% 行距初始值为标志 \texttt{nan},用于检查用户是否设置了 \opt{linespread} 选项。
%    \begin{macrocode}
    linespread  .fp_set:N = \l_@@_line_spread_fp ,
    linespread .initial:n = { \c_nan_fp } ,
    linespread .value_required:n = true ,
%    \end{macrocode}
% \end{macro}
%
% \changes{v2.0}{2014/03/13}{新增 \opt{autoindent} 选项。}
%
% \begin{macro}{autoindent}
% 自动调整段落的首行缩进功能。
%    \begin{macrocode}
    autoindent .choice: ,
    autoindent .default:n = { true } ,
    autoindent / true    .code:n =
      {
        \tl_set:Nn \l_@@_autoindent_tl { 2 \ccwd }
        \cs_set_eq:NN \ctex_if_autoindent_touched:F \use_none:n
      } ,
    autoindent / false   .code:n =
      {
        \tl_clear:N \l_@@_autoindent_tl
        \cs_set_eq:NN \ctex_if_autoindent_touched:F \use_none:n
      } ,
    autoindent / unknown .code:n =
      {
        \ctex_set_default_ccwd:Nn \l_@@_autoindent_tl {#1}
        \cs_set_eq:NN \ctex_if_autoindent_touched:F \use_none:n
      } ,
%    \end{macrocode}
% \end{macro}
%
% \changes{v2.0}{2015/03/21}{\opt{indent}, \opt{noindent} 是过时选项。}
% \begin{macro}{indent}
% 仅为兼容性保留,已过时。
%    \begin{macrocode}
    indent .code:n =
      {
        \msg_warning:nnn { ctex } { deprecated-option }
          {
            The~ functionality~ has~ been~ removed.\\
            It's~ better~ to~ set~ the~ heading~ styles~ via~ `afterindent'~
            options.~
          }
      } ,
    indent .value_forbidden:n = true ,
    noindent .code:n =
      {
        \msg_warning:nnn { ctex } { deprecated-option }
          {
            The~ functionality~ has~ been~ removed.\\
            It's~ better~ to~ set~ the~ heading~ styles~ via~ `afterindent'~
            options.
          }
      } ,
    noindent .value_forbidden:n = true ,
%    \end{macrocode}
% \end{macro}
%
% \begin{macro}{GBK,UTF8}
%   \begin{macrocode}
    GBK  .code:n = { \tl_set:Nn \l_@@_encoding_tl { GBK } } ,
    UTF8 .code:n = { \tl_set:Nn \l_@@_encoding_tl { UTF8 } } ,
    GBK  .value_forbidden:n = true ,
    UTF8 .value_forbidden:n = true ,
%    \end{macrocode}
% \end{macro}
%
% \changes{v2.0}{2014/03/08}{新增 \opt{fontset} 选项。}
% \changes{v2.0}{2015/03/21}{\opt{nofonts}, \opt{adobefonts}, \opt{winfonts}
% 是过时选项。}
%
% \begin{macro}{fontset}
% 初始值为空。若用户未指定,则根据操作系统载入对应字体配置,可以区分 Windows、
% Mac~OS~X 和其它。
%   \begin{macrocode}
    fontset    .tl_gset:N = \g_@@_fontset_tl ,
    nofonts    .code:n =
      {
        \msg_warning:nnn { ctex } { deprecated-option }
          {
            Option~ `fontset=none'~ is~ set.~ It~ is~ better~ to~ use~
            fontset~ option.
          }
        \keys_set:nn { ctex / option } { fontset = none }
      } ,
    adobefonts .code:n =
      {
        \msg_warning:nnn { ctex } { deprecated-option }
          {
            Option~ `fontset=adobe'~ is~ set.~ It~ is~ better~ to~ use~
            fontset~ option.
          }
        \keys_set:nn { ctex / option } { fontset = adobe }
      } ,
    winfonts   .code:n =
      {
        \msg_warning:nnn { ctex } { deprecated-option }
          {
            Option~ `fontset=windows'~ is~ set.~ It~ is~ better~ to~ use~
            fontset~ option.
          }
        \keys_set:nn { ctex / option } { fontset = windows }
      } ,
    nofonts    .value_forbidden:n = true ,
    winfonts   .value_forbidden:n = true ,
    adobefonts .value_forbidden:n = true ,
%    \end{macrocode}
% \end{macro}
%
% \changes{v2.0}{2014/03/08}{新增 \opt{zhmCJK} 支持选项。}
% \changes{v2.0}{2015/03/22}{\opt{nozhmap} 是过时选项。}
%
% \begin{macro}{zhmap}
%   \begin{macrocode}
    zhmap .choice: ,
    zhmap .default:n = { true } ,
    zhmap / zhmCJK .code:n =
      {
        \bool_gset_true:N \g_@@_zhmCJK_bool
        \cs_gset_eq:NN \ctex_zhmap_case:nnn \use_i:nnn
      } ,
    zhmap / true   .code:n =
      {
        \bool_gset_false:N \g_@@_zhmCJK_bool
        \cs_gset_eq:NN \ctex_zhmap_case:nnn \use_ii:nnn
      } ,
    zhmap / false  .code:n =
      {
        \bool_gset_false:N \g_@@_zhmCJK_bool
        \cs_gset_eq:NN \ctex_zhmap_case:nnn \use_iii:nnn
      } ,
    nozhmap   .code:n =
      {
        \msg_warning:nnn { ctex } { deprecated-option }
          { Option~ `zhmap=false'~ is~ set. }
        \keys_set:nn { ctex / option } { zhmap = false }
      } ,
    nozhmap   .value_forbidden:n = true ,
%    \end{macrocode}
% \end{macro}
%
% \changes{v2.0}{2014/04/11}{\opt{punct} 选项可以设置标点格式。}
% \changes{v2.0}{2015/03/21}{\opt{nopunct} 是过时选项。}
%
% \begin{macro}{punct}
% 设置标点符号输出格式。
%   \begin{macrocode}
    punct   .tl_set:N = \l_@@_punct_tl ,
    punct  .default:n = { quanjiao } ,
    punct  .initial:n = { quanjiao } ,
    nopunct   .code:n =
      {
        \msg_warning:nnn { ctex } { deprecated-option }
          { Option~ `punct=plain'~ is~ set.  }
        \keys_set:nn { ctex / option } { punct = plain }
      } ,
    nopunct   .value_forbidden:n = true ,
%    \end{macrocode}
% \end{macro}
%
% \changes{v2.0}{2015/03/22}{\opt{nospace} 是过时选项。}
% \begin{macro}{space}
%   \begin{macrocode}
    space .choices:nn =
      { true , auto , false }
      {
        \exp_args:Nx \ctex_at_end:n
          { \keys_set:nn { ctex } { space = \l_keys_choice_tl } }
      } ,
    space  .default:n = { true } ,
    nospace   .code:n =
      {
        \msg_warning:nnn { ctex } { deprecated-option }
          { Option~ `space=false'~ is~ set. }
        \keys_set:nn { ctex / option } { space = false }
      } ,
    nospace   .value_forbidden:n = true ,
%    \end{macrocode}
% \end{macro}
%
% \changes{v2.0}{2014/03/08}{\pkg{ctex} 宏包新增 \opt{heading} 选项。}
%
% \begin{macro}{heading}
%   \begin{macrocode}
    heading .bool_set:N = \l_@@_heading_bool ,
%    \end{macrocode}
% \end{macro}
%
%    \begin{macrocode}
%</class|ctex>
%<*class|ctex|ctexheading>
%    \end{macrocode}
%
% \begin{macro}{sub3section,sub4section}
% \begin{macrocode}
%<*!beamer>
    sub3section .code:n =
      { \cs_gset_eq:NN \g_@@_section_depth_flag \c_three } ,
    sub4section .code:n =
      { \cs_gset_eq:NN \g_@@_section_depth_flag \c_four } ,
    sub3section .value_forbidden:n = true ,
    sub4section .value_forbidden:n = true ,
%</!beamer>
%    \end{macrocode}
% \end{macro}
%
% \changes{v2.0}{2015/04/15}{新增 \opt{scheme} 选项,并将 \opt{cap} 和 \opt{nocap}
% 列为过时选项。}
% \begin{macro}{scheme}
%    \begin{macrocode}
    scheme .tl_set:N  = \l_@@_scheme_tl ,
%<*ctexheading>
    scheme .default:n = { plain } ,
    scheme .initial:n = { plain }
  }
%</ctexheading>
%<*!ctexheading>
    scheme .default:n = { chinese } ,
    scheme .initial:n = { chinese } ,
%</!ctexheading>
%    \end{macrocode}
%
%    \begin{macrocode}
%</class|ctex|ctexheading>
%<*class|ctex>
%    \end{macrocode}
%
% \begin{macrocode}
    cap    .code:n    =
      {
        \msg_warning:nnn { ctex } { deprecated-option }
          { Option~ `scheme = chinese' ~ is~ set. }
        \keys_set:nn { ctex / option } { scheme = chinese }
      } ,
    nocap  .code:n    =
      {
        \msg_warning:nnn { ctex } { deprecated-option }
          { Option~ `scheme = plain' ~ is~ set. }
        \keys_set:nn { ctex / option } { scheme = plain }
      } ,
    cap    .value_forbidden:n = true ,
    nocap  .value_forbidden:n = true ,
%    \end{macrocode}
% \end{macro}
%
% \changes{v2.0}{2015/04/20}{\opt{hyperref} 成为过时选项,原选项功能总是打开。}
% \changes{v2.0}{2015/04/20}{\opt{fancyhdr} 成为过时选项,原选项功能总是打开。}
% \changes{v2.0}{2015/04/20}{\opt{fntef} 成为过时选项,原选项功能总是打开。}
% \begin{macro}{fntef,fancyhdr,hyperref}
% \changes{v2.1}{2015/06/03}{补充定义 \tn{hypersetup}。}
% 这些都是过时的宏包兼容选项,原选项功能总是打开的。
%   \begin{macrocode}
    fntef    .code:n =
      {
        \msg_warning:nnn { ctex } { deprecated-option }
          { `(xe)CJKfntef'~ package~ is~ always~ loaded. }
      } ,
    fancyhdr .code:n =
      {
        \msg_warning:nnn { ctex } { deprecated-option }
          { `fancyhdr'~ package~ is~ loaded. }
        \RequirePackage { fancyhdr }
      } ,
    hyperref .code:n =
      {
        \msg_warning:nnn { ctex } { deprecated-option }
          { `hyperref'~ package~ will~ be~ loaded. }
        \ctex_at_end:n
          {
            \cs_if_exist:NF \hypersetup
              { \cs_new_eq:NN \hypersetup \ctex_hypersetup:n }
          }
        \ctex_at_end_preamble:n { \RequirePackage { hyperref } }
      } ,
  }
%    \end{macrocode}
% \end{macro}
%
%    \begin{macrocode}
%</class|ctex>
%<*class|ctex|ctexsize>
%    \end{macrocode}
%
% \changes{v2.0}{2015/05/06}{兼容 \pkg{extsizes} 宏包、\cls{beamer}、\pkg{memoir}
% 等提供的更多字号选项。}
% \changes{v2.0.1}{2015/05/15}{修复 \opt{10pt}、\opt{11pt} 等选项无效的问题。}
% \begin{macro}{10pt,11pt,12pt}
% 使 \pkg{ctex} 和 \pkg{ctexsize} 可以接受文档类的全局选项,不修改默认字体大小。
% 在文档类下还将参数传给标准文档类。
%    \begin{macrocode}
\tl_clear_new:N \l_@@_tmp_tl
\clist_map_inline:nn
  {
    10pt , 11pt , 12pt ,
     8pt ,  9pt , 14pt , 17pt , 20pt , 25pt , 30pt , 36pt , 48pt , 60pt
  }
  {
    \tl_put_right:Nn \l_@@_tmp_tl
      {
        #1 .code:n =
%<*!class>
          { \cs_gset_eq:NN \g_@@_font_size_flag \c_two } ,
%</!class>
%<*class>
          {
            \cs_gset_eq:NN \g_@@_font_size_flag \c_two
            \clist_gput_right:Nn \g_@@_std_options_clist {#1}
          } ,
%</class>
        #1 .value_forbidden:n = true ,
      }
  }
\use:x { \keys_define:nn { ctex / option } { \exp_not:o { \l_@@_tmp_tl } } }
\tl_clear:N \l_@@_tmp_tl
%    \end{macrocode}
% \end{macro}
%
% 将未知选项传给标准文档类。
%    \begin{macrocode}
%<*class>
\keys_define:nn { ctex / option }
  {
    unknown .code:n =
      { \clist_gput_right:No \g_@@_std_options_clist { \CurrentOption } }
  }
%</class>
%    \end{macrocode}
%
%    \begin{macrocode}
%<!ctexsize>\ctex_file_input:n { ctexopts.cfg }
%</class|ctex|ctexsize>
%    \end{macrocode}
%
%    \begin{macrocode}
%<*class|style>
\ProcessKeysOptions { ctex / option }
%</class|style>
%    \end{macrocode}
%
%    \begin{macrocode}
%<*class>
%    \end{macrocode}
%
% 五号字使用标准文档类的 |10pt| 字体大小设置,小四号字则使用 |12pt|。
%    \begin{macrocode}
\if_case:w \g_@@_font_size_flag
  \clist_gput_right:Nn \g_@@_std_options_clist { 10pt }
\or:
  \clist_gput_right:Nn \g_@@_std_options_clist { 12pt }
\fi:
%    \end{macrocode}
%
% 使用 \tn{PassOptionsToClass} 是为了预防可能存在的选项冲突。
%    \begin{macrocode}
%<*article>
\tl_const:Nn \c_@@_class_tl { article }
\PassOptionsToClass { \g_@@_std_options_clist } { article }
\LoadClass { article }
%</article>
%<*book>
\tl_const:Nn \c_@@_class_tl { book }
\PassOptionsToClass { \g_@@_std_options_clist } { book }
\LoadClass { book }
%</book>
%<*report>
\tl_const:Nn \c_@@_class_tl { report }
\PassOptionsToClass { \g_@@_std_options_clist } { report }
\LoadClass { report }
%</report>
%<*beamer>
\tl_const:Nn \c_@@_class_tl { beamer }
\PassOptionsToClass { \g_@@_std_options_clist } { beamer }
\LoadClass { beamer }
%</beamer>
%    \end{macrocode}
%
%    \begin{macrocode}
%</class>
%    \end{macrocode}
%
% \subsection{特定引擎支持与设置}
%
% \subsubsection{\pkg{ctex-engine-pdftex.def}}
%
% \begin{macro}[int]{\ctex_set_zhmap:n}
% 设置 \upTeX{} 字体映射,同时作用于 \tn{AtBeginDvi} 与
% \tn{AtBeginShipoutFirst}。该宏对 \pdfTeX{} 和 \upTeX{} 均有用。
%     \begin{macrocode}
%<*pdftex|uptex|aptex>
\cs_new_protected_nopar:Npn \ctex_set_zhmap:n #1
  {
    \AtBeginDvi {#1}
    \ctex_at_end_package:nn { atbegshi }
      { \AtBeginShipoutFirst {#1} }
  }
\@onlypreamble \ctex_set_zhmap:n
%</pdftex|uptex|aptex>
%    \end{macrocode}
% \end{macro}
%
%    \begin{macrocode}
%<*pdftex>
%    \end{macrocode}
%
% \changes{v2.1}{2015/05/18}{给 \pdfLaTeX{} 下的非 UTF8 编码 CJK 字体族加上 CMap。}
%
% \begin{variable}[int]{\c_@@_cmap_encoding_seq}
% 需要加上 CMap 的 CJK 字体编码。
%    \begin{macrocode}
\seq_new:N \c_@@_cmap_encoding_seq
\seq_set_from_clist:Nn \c_@@_cmap_encoding_seq
  { C19 , C10 , C00 , C09 , C40 , C60 }
%    \end{macrocode}
% \end{variable}
%
% \begin{macro}[int]{\ctex_family_cmap:nn}
% 在 \tn{DeclareFontFamily} 的 \meta{loading-settings} 中给 CJK 字体族加上 CMap。
%    \begin{macrocode}
\cs_new_protected_nopar:Npn \ctex_family_cmap:nn #1#2
  {
    \cs_if_free:cF { #1 + #2 }
      {
        \seq_if_in:NnT \c_@@_cmap_encoding_seq {#1}
          { \tl_gput_right:cn { #1 + #2 } { \ctex_add_cmap:n {#1} } }
      }
  }
\cs_generate_variant:Nn \ctex_family_cmap:nn { x }
\cs_new_eq:NN \CTEX@Family@CMap \ctex_family_cmap:xn
%    \end{macrocode}
% \end{macro}
%
% \begin{macro}[int]{\ctex_add_cmap:n}
% 给 |#1| 编码的 CJK 字体加上 CMap。
%    \begin{macrocode}
\cs_new_protected_nopar:Npn \ctex_add_cmap:n #1
  {
    \cs_if_free:NF \CJK@plane
      { \ctex_add_cmap:cn { @@_add_cmap_ #1 \CJK@plane : } {#1} }
  }
\cs_new_protected_nopar:Npn \ctex_add_cmap:Nn #1#2
  {
    \cs_if_exist:NF #1 { \@@_save_cmap:Nn #1 {#2} }
    #1
  }
\cs_generate_variant:Nn \ctex_add_cmap:Nn { c }
\cs_new_protected_nopar:Npn \@@_save_cmap:Nn #1#2
  {
    \tl_set:Nx \l_@@_tmp_tl { \str_lower_case:n {#2} \CJK@plane }
    \tex_immediate:D \pdftex_pdfobj:D stream ~ file { \l_@@_tmp_tl .cmap }
    \cs_new_protected_nopar:Npx #1
      {
        \exp_not:N \pdftex_pdffontattr:D \exp_not:N \tex_font:D
          { /ToUnicode ~ \int_use:N \pdftex_pdflastobj:D \c_space_tl 0 ~ R }
      }
  }
%    \end{macrocode}
% \end{macro}
%
% \begin{macro}[int]{\DeclareFontFamily}
% 只在 \pdfLaTeX{} 下加 CMap。如 \pkg{cmap} 宏包被引入,则不重复设置。
%    \begin{macrocode}
\group_begin:
\char_set_catcode_other:N \#
\sys_if_output_pdf:TF
  {
    \group_end:
    \ctex_appto_cmd:NnnTF \DeclareFontFamily { \ExplSyntaxOff }
      { \CTEX@Family@CMap {#1} {#2} }
      {
        \ctex_at_end_package:nn { cmap }
          { \cs_gset_eq:NN \CTEX@Family@CMap \use_none:nn }
      }
      { \ctex_patch_failure:N \DeclareFontFamily }
  }
  { \group_end: }
%    \end{macrocode}
% \end{macro}
%
% 首先检查选项,决定是否载入 \pkg{zhmCJK} 宏包。
%    \begin{macrocode}
\if_bool:N \g_@@_zhmCJK_bool
  \PassOptionsToPackage { encoding = \l_@@_encoding_tl } { zhmCJK }
  \RequirePackage { zhmCJK }
%    \end{macrocode}
% 不载入 \pkg{zhmCJK} 宏包时直接调用 \pkg{CJK} 及相关宏包。
%    \begin{macrocode}
\else:
  \str_if_eq:onTF { \l_@@_encoding_tl } { GBK }
    { \RequirePackage { CJK } }
    { \RequirePackage { CJKutf8 } }
  \RequirePackage { CJKpunct , CJKspace }
%    \end{macrocode}
%
% \begin{macro}[int]{\ctex_load_zhmap:nnnn}
% 载入 \pkg{zhmetrics} 的字体映射文件,同时设置 \tn{CJKrmdefault} 等。
%    \begin{macrocode}
  \cs_new_protected_nopar:Npn \ctex_load_zhmap:nnnn #1#2#3#4
    {
      \tl_set:Nn \CJKrmdefault {#1}
      \tl_set:Nn \CJKsfdefault {#2}
      \tl_set:Nn \CJKttdefault {#3}
      \ctex_set_zhmap:n { \ctex_zhmap_input:n {#4} }
    }
  \@onlypreamble \ctex_load_zhmap:nnnn
%    \end{macrocode}
% \end{macro}
%
% \changes{v2.4.4}{2016/09/09}{解决 \opt{zhmap} 文件的 \tn{catcode} 问题。}
% \begin{macro}[int]{\ctex_zhmap_input:n}
% 载入字体映射文件时,确认 |%| 和 |\| 的 \tn{catcode}。
%    \begin{macrocode}
  \cs_new_protected_nopar:Npn \ctex_zhmap_input:n #1
    {
      \ctex_file_wrapper:nnn
        {
          \char_set_catcode_comment:n { 37 } % %
          \char_set_catcode_escape:n  { 92 } % \
        }
        { \file_input:n {#1} }
        {
          \char_set_catcode:nn { 37 } { \char_value_catcode:n { 37 } }
          \char_set_catcode:nn { 92 } { \char_value_catcode:n { 92 } }
        }
    }
%    \end{macrocode}
% \end{macro}
%
%    \begin{macrocode}
  \tl_if_exist:NF \CJKfamilydefault
    { \tl_const:Nn \CJKfamilydefault { \CJKrmdefault } }
  \tl_if_exist:NF \CJKrmdefault { \tl_new:N \CJKrmdefault }
  \tl_if_exist:NF \CJKsfdefault { \tl_new:N \CJKsfdefault }
  \tl_if_exist:NF \CJKttdefault { \tl_new:N \CJKttdefault }
  \ctex_preto_cmd:NnnTF \rmfamily { \ExplSyntaxOff }
    { \CJKfamily { \CJKrmdefault } }
    { }
    { \ctex_patch_failure:N \rmfamily }
  \ctex_preto_cmd:NnnTF \sffamily { \ExplSyntaxOff }
    { \CJKfamily { \CJKsfdefault } }
    { }
    { \ctex_patch_failure:N \sffamily }
  \ctex_preto_cmd:NnnTF \ttfamily { \ExplSyntaxOff }
    { \CJKfamily { \CJKttdefault } }
    { }
    { \ctex_patch_failure:N \ttfamily }
  \ctex_preto_cmd:NnnTF \normalfont { \ExplSyntaxOff }
    { \CJKfamily { \CJKfamilydefault } }
    { \cs_set_eq:NN \reset@font \normalfont }
    { \ctex_patch_failure:N \normalfont }
%    \end{macrocode}
%
% \pkg{zhmCJK} 判断结束。
%    \begin{macrocode}
\fi:
%    \end{macrocode}
%
% \begin{macro}[int]{\ctex_CJK_input:n,\CJK@input}
% \pkg{breqn} 包可能会在正文中将 |^| 的 \tn{catcode} 改为 $12$ 或 $13$,这将
% 破坏 \pkg{CJK} 对汉字的首字节的定义(\tn{CJK@loadBinding} 和
% \tn{CJK@loadEncoding})。因此需要确保载入 \file{.enc} 和 \file{.bdg} 文件时,
% |^| 的 \tn{catcode} 为 $7$。
%    \begin{macrocode}
\cs_new_protected_nopar:Npn \ctex_CJK_input:n #1
  {
    \ctex_file_wrapper:nnn
      {
        \char_set_catcode_other:n            { 60 } % <
        \char_set_catcode_math_superscript:n { 94 } % ^
        \int_set:Nn \tex_endlinechar:D { -1 }
      }
      { \file_input:n {#1} }
      {
        \char_set_catcode:nn { 60 } { \char_value_catcode:n { 60 } }
        \char_set_catcode:nn { 94 } { \char_value_catcode:n { 94 } }
        \int_set:Nn \tex_endlinechar:D { \int_use:N \tex_endlinechar:D }
      }
  }
\cs_set_eq:NN \CJK@input \ctex_CJK_input:n
%    \end{macrocode}
% \end{macro}
%
% \begin{macro}[int]{\ctex_plane_to_utfxvibe:Nn,\CJK@surr}
% \changes{v2.0}{2014/04/08}{解决与 \tn{nouppercase} 的冲突。}
% \pkg{fancyhdr} 宏包的 \tn{nouppercase} 会将 \tn{uppercase} 定义为 \tn{relax},而
% \tn{CJK@surr} 需要用它将 \tn{CJK@plane} 转化成大写字母,这就造成了冲突^^A
% \footnote{\url{https://github.com/CTeX-org/ctex-kit/issues/146}}。
% 我们在这里给出 \tn{CJK@surr} 的一个不依赖 \tn{uppercase} 的实现。
%    \begin{macrocode}
\if_cs_exist:N \CJK@surr
  \cs_new_protected_nopar:Npn \ctex_plane_to_utfxvibe:Nn #1#2
    {
      \tl_set:Nx \l_@@_tmp_tl {#2}
      \int_set:Nn \l_@@_tmp_int
        { \exp_args:No \int_from_hex:n { \l_@@_tmp_tl } }
      \int_compare:nNnTF \l_@@_tmp_int < \c_two_hundred_fifty_six
        { \tl_gset:Nx #1 { \int_to_Hex:n { \l_@@_tmp_int } } }
        {
          \int_sub:Nn \l_@@_tmp_int { \c_two_hundred_fifty_six }
          \tl_gset:Nx #1
            {
              \int_to_Hex:n
                { \int_div_truncate:nn { \l_@@_tmp_int } { \c_four } + "D800 }
              \int_to_Hex:n
                { \int_mod:nn { \l_@@_tmp_int } { \c_four } + "DC }
            }
        }
    }
  \cs_set_eq:NN \CJK@surr \ctex_plane_to_utfxvibe:Nn
\fi:
%    \end{macrocode}
% \end{macro}
%
% \pkg{CJKpunct} 宏包会在 \tn{AtBeginDocument} 的里设置标点格式为 \opt{quanjiao}。
%    \begin{macrocode}
\AtBeginDocument
  {
    \str_if_eq_x:nnF { \l_@@_punct_tl } { quanjiao }
      { \punctstyle { \l_@@_punct_tl } }
  }
%    \end{macrocode}
%
% 在导言区末尾更新 \tn{CJKfamilydefault},注意要在 \tn{CJK@envStart} 之前使用。
%    \begin{macrocode}
\ctex_at_end_preamble:n { \ctex_update_default_family: }
%    \end{macrocode}
%
% 启用中文字符功能。\texttt{GBK} 编码时,将汉字的首字节设置为活动字符,并对这些
% 字符初始化;\texttt{UTF8} 编码时,上游宏包已经处理好。
% \tn{CJK@makeActive} 应该先于 \file{ctex-name-gbk.cfg} 等文件的载入。
% 注意 \tn{CJK@loadBinding} 需要调用补丁后的 \tn{CJK@input}。使用
% \pkg{zhmCJK} 时,此功能已经被启用。
%    \begin{macrocode}
\reverse_if:N \if_bool:N \g_@@_zhmCJK_bool
  \str_if_eq:onF { \l_@@_encoding_tl } { UTF8 }
    {
      \CJK@makeActive
      \CJK@loadBinding { standard }
    }
%    \end{macrocode}
%
% 在导言区结束时调用 \tn{CJK@envStart} 启用完整的中文功能。
%
% \tn{CJK@envStart} 的定义是
% \begin{verbatim}
%   \def\CJK@envStart#1#2#3{
%     \CJK@upperReset
%     \ifCJK@lowercase@
%       \CJK@lowerReset
%     \fi%
%     \CJK@makeActive%
%     \CJK@global\let\CJK@selectFamily \CJK@selFam
%     \CJK@global\let\CJK@selectEnc \CJK@selEnc%
%     \def\CJK@@@enc{#2}
%     \ifx\CJK@@@enc \@empty
%       \PackageInfo{CJK}{
%         no encoding parameter given,\MessageBreak
%         waiting for \protect\CJKenc\space commands}
%     \else
%       \CJKenc{#2}
%     \fi
%     \CJKfontenc{#2}{#1}
%     \CJKfamily{#3}
%     \def\CJK@series{\f@series}
%     \def\CJK@shape{\f@shape}%
%     \csname CJKhook\endcsname}
% \end{verbatim}
% \tn{CJK@upperReset} 可能会有一定风险,因此我们直到导言区末尾才使用
% \tn{CJK@envStart}。这样可以避免将 \env{CJK} 环境内置入 \env{document} 环境的
% 最里层,最后也就不需要 \tn{clearpage}。\pkg{zhmCJK} 已经提供类似功能。
% 注意先使用 \cs{ctex_update_default_family:} 更新 \tn{CJKfamilydefault}。
%    \begin{macrocode}
  \exp_args:Nx \ctex_at_end_preamble:n
    {
      \exp_not:N \CJK@envStart
        { } { \l_@@_encoding_tl } { \exp_not:N \CJKfamilydefault }
      \exp_not:N \CJKtilde
    }
%    \end{macrocode}
%
% \pkg{zhmCJK} 判断结束。
%    \begin{macrocode}
\fi:
%    \end{macrocode}
%
% \begin{macro}[int]{\ctex_auto_ignorespaces:}
% 关闭名字空间,保存 \tn{CJK@@ignorespaces} 的定义,方便使用。
%    \begin{macrocode}
%<@@=>
\cs_new_eq:NN \ctex_auto_ignorespaces: \CJK@@ignorespaces
%    \end{macrocode}
% 恢复名字空间,要把它放在一个 \env{macrocode} 环境中,\cls{l3doc} 才能正确工作。
%    \begin{macrocode}
%<@@=ctex>
%    \end{macrocode}
% \end{macro}
%
% \begin{macro}[int]{\ctex_ignorespaces_case:N,\ctex_set_ignorespaces:}
% 设置忽略空格的的方式。根据 \opt{space} 选项的值重定义 \tn{CJK@ignorespaces},
% 并保存起来供 \tn{CJKhook} 备用。
%    \begin{macrocode}
\cs_new_protected_nopar:Npn \ctex_ignorespaces_case:N #1
  {
    \cs_set_protected_nopar:Npn \ctex_set_ignorespaces:
      { \cs_set_eq:NN \CJK@ignorespaces #1 }
    \ctex_set_ignorespaces:
  }
\cs_new_protected_nopar:Npn \ctex_set_ignorespaces:
  { \cs_set_eq:NN \CJK@ignorespaces \ctex_auto_ignorespaces: }
%    \end{macrocode}
% \end{macro}
%
% \begin{macro}[int]{\CJKhook}
% \env{CJK} 和 \env{CJK*} 环境都会重新定义 \tn{CJK@ignorespaces}。我们在 \pkg{CJK}
% 宏包提供的 \tn{CJKhook} 里重新设置它,让这两个环境忽略空格的方式都受 \opt{space}
% 选项的控制。这对 \pkg{zhmCJK} 是必要的。
%    \begin{macrocode}
\ctex_gadd_hook:Nn \CJKhook { \ctex_set_ignorespaces: }
%    \end{macrocode}
% \end{macro}
%
% \begin{macro}[int]{\ctex_punct_set:n}
% 设置 CJK 族对应到实际的字体。|#1| 是 \opt{fontset} 的名字。
%    \begin{macrocode}
\cs_new_protected_nopar:Npn \ctex_punct_set:n #1
  {
    \clist_map_inline:Nn \c_@@_punct_family_clist
      {
        \cs_if_free:cF { c_@@_ #1 ##1 _punct_spaces_tl }
          {
            \cs_set_eq:cc
              { CJKpunct@ ##1 @spaces }
              { c_@@_ #1 ##1 _punct_spaces_tl }
          }
      }
  }
\clist_const:Nn \c_@@_punct_family_clist
  {
    zhsong , zhhei , zhfs , zhkai , zhli , zhyou ,
    zhsongb , zhheil , zhheib , zhyoub , zhyahei , zhyaheib
  }
%    \end{macrocode}
% \end{macro}
%
% \begin{macro}[int]{\ctex_punct_map_family:nn}
% CJK 族 |#1| 使用族 |#2| 的边界信息。
%    \begin{macrocode}
\cs_new_protected_nopar:Npn \ctex_punct_map_family:nn #1#2
  {
    \cs_if_free:cF { CJKpunct@ #2 @spaces }
      { \cs_set_eq:cc { CJKpunct@ #1 @spaces } { CJKpunct@ #2 @spaces } }
  }
%    \end{macrocode}
% \end{macro}
%
% \begin{macro}[int]{\ctex_punct_map_bfseries:nn}
% CJK 族 |#1| 的 \tn{bfseries} 使用族 |#2| 的边界信息。
%    \begin{macrocode}
\cs_new_protected_nopar:Npn \ctex_punct_map_bfseries:nn #1#2
  {
    \clist_map_inline:nn {#1}
      {
        \ctex_punct_map_series:nnn { ##1 } { b } {#2}
        \ctex_punct_map_series:nnn { ##1 } { bx } {#2}
      }
  }
\cs_new_protected_nopar:Npn \ctex_punct_map_series:nnn #1#2#3
  {
    \CJKpunctmapfamily { C19 } {#1} {#2} { m }  {#3}
    \CJKpunctmapfamily { C19 } {#1} {#2} { it } {#3}
    \CJKpunctmapfamily { C19 } {#1} {#2} { sl } {#3}
    \CJKpunctmapfamily { C70 } {#1} {#2} { m }  {#3}
    \CJKpunctmapfamily { C70 } {#1} {#2} { it } {#3}
    \CJKpunctmapfamily { C70 } {#1} {#2} { sl } {#3}
  }
%    \end{macrocode}
% \end{macro}
%
% \begin{macro}[int]{\ctex_punct_map_itshape:nn}
% CJK 族 |#1| 的 \tn{itshape} 使用族 |#2| 的边界信息。
%    \begin{macrocode}
\cs_new_protected_nopar:Npn \ctex_punct_map_itshape:nn #1#2
  {
    \CJKpunctmapfamily { C19 } {#1} { m }  { it } {#2}
    \CJKpunctmapfamily { C19 } {#1} { b }  { it } {#2}
    \CJKpunctmapfamily { C19 } {#1} { bx } { it } {#2}
    \CJKpunctmapfamily { C70 } {#1} { m }  { it } {#2}
    \CJKpunctmapfamily { C70 } {#1} { b }  { it } {#2}
    \CJKpunctmapfamily { C70 } {#1} { bx } { it } {#2}
  }
%    \end{macrocode}
% \end{macro}
%
% \begin{macro}[int]{\ctex_punct_space:nn,\ctexspadef}
% 定义标点的边界信息。
%    \begin{macrocode}
\cs_new_protected_nopar:Npn \ctex_punct_space:nn #1#2
  { \tl_const:cn { c_@@_ #1 _punct_spaces_tl } {#2} }
\cs_new_eq:NN \ctexspadef \ctex_punct_space:nn
%    \end{macrocode}
% \end{macro}
%
% 载入边界信息文件。
%     \begin{macrocode}
\ctex_file_input:n { ctexspa.def }
%    \end{macrocode}
%
%    \begin{macrocode}
%</pdftex>
%    \end{macrocode}
%
% \subsubsection{\pkg{ctex-engine-xetex.def}}
%
%    \begin{macrocode}
%<*xetex>
%    \end{macrocode}
%
% \changes{v2.4.4}{2016/09/12}{不再默认设置 \pkg{xeCJK} 的伪粗体。}
%    \begin{macrocode}
\RequirePackage { xeCJK }
\exp_args:Nx \xeCJKsetup
  {
    LoadFandol   = false ,
    PunctStyle   = \l_@@_punct_tl
  }
%    \end{macrocode}
%
% 最新版本的 \pkg{fontspec} 默认对 \tn{rmfamily} 和 \tn{sffamily} 设置
% |Ligatures=TeX|,对 \tn{ttfamily} 设置 |WordSpace={1,0,0}| 和
% |PunctuationSpace=WordSpace|。
%    \begin{macrocode}
\@ifpackagelater { fontspec } { 2014/05/25 } { }
  { \msg_error:nnn { ctex } { package-too-old } { fontspec } }
%    \end{macrocode}
%
%    \begin{macrocode}
%</xetex>
%    \end{macrocode}
%
% \subsubsection{\pkg{ctex-engine-luatex.def}}
%
%    \begin{macrocode}
%<*luatex>
%    \end{macrocode}
%
% \changes{v2.0}{2014/03/08}{通过 \pkg{LuaTeX-ja} 宏包支持 \LuaLaTeX。}
% \changes{v2.3}{2015/09/25}{更新 \pkg{LuaTeX-ja} 支持(20150922.0)。}
%
% \pkg{LuaTeX-ja} 为了兼容 p\LaTeX 的使用习惯,对 \LaTeXe 的 \pkg{NFSS} 作了不少
% 修改和扩充,这对于简体中文用户来说不是必要的。我们在这里禁用它。
%    \begin{macrocode}
\msg_new:nnn { ctex } { luatexja-loaded }
  {
    Package~`luatexja'~can~not~be~loaded~before~`ctex'.\\
    Loading~file~`#1'~will~abort!
  }
\@ifpackageloaded { luatexja }
  { \msg_critical:nnx { ctex } { luatexja-loaded } { \g_file_current_name_tl } }
  { \tl_const:cn { ver@ltj-latex.\@pkgextension } { 9999/99/99 } }
%    \end{macrocode}
%
%    \begin{macrocode}
\RequirePackage { luatexja }
\@ifpackagelater { luatexja } { 2015/09/21 } { }
  { \msg_error:nnn { ctex } { package-too-old } { luatexja } }
%    \end{macrocode}
%
%    \begin{macrocode}
\RequirePackage { fontspec }
\@ifpackagelater { fontspec } { 2014/05/25 } { }
  { \msg_error:nnn { ctex } { package-too-old } { fontspec } }
%    \end{macrocode}
%
% \paragraph{\pkg{LuaTeX-ja} 的默认设置}
%
%    \begin{macrocode}
\ExplSyntaxOff
%    \end{macrocode}
%
% 以下设置抄录自 \file{lltjdefs.sty}。
%    \begin{macrocode}
\ltjdefcharrange{1}{"80-"36F, "1E00-"1EFF}
\ltjdefcharrange{2}{"370-"4FF, "1F00-"1FFF}
\ltjdefcharrange{3}{%
  "2000-"243F, "2500-"27BF, "2900-"29FF, "2B00-"2BFF}
\ltjdefcharrange{4}{%
   "500-"10FF, "1200-"1DFF, "2440-"245F, "27C0-"28FF, "2A00-"2AFF,
  "2C00-"2E7F, "4DC0-"4DFF, "A4D0-"A82F, "A840-"ABFF, "FB00-"FE0F,
  "FE20-"FE2F, "FE70-"FEFF, "10000-"1FFFF, "E000-"F8FF} % non-Japanese
\ltjdefcharrange{5}{"D800-"DFFF, "E0000-"E00FF, "E01F0-"10FFFF}
\ltjdefcharrange{6}{%
  "2460-"24FF, "2E80-"2EFF, "3000-"30FF, "3190-"319F, "31F0-"4DBF,
  "4E00-"9FFF, "F900-"FAFF, "FE10-"FE6F, "20000-"2FFFF, "E0100-"E01EF}
\ltjdefcharrange{7}{
  "1100-"11FF, "2F00-"2FFF, "3100-"31EF, "A000-"A4CF, "A830-"A83F,
  "AC00-"D7FF}
\ltjdefcharrange{8}{"A7, "A8, "B0, "B1, "B4, "B6, "D7, "F7}
\ltjsetparameter{jacharrange={-1, +2, +3, -4, -5, +6, +7, -8}}
\directlua{for x=128,255 do luatexja.math.is_math_letters[x] = true end}
%    \end{macrocode}
%
% 以下设置抄录自 \file{ltj-latex.sty}。
%    \begin{macrocode}
\directlua{
  local s = kpse.find_file('ltj-kinsoku.lua', 'tex')
  luatexja.stack.charprop_stack_table[0] = s and dofile(s) or {}
}
\ltjsetparameter{kanjiskip=\z@ plus .4pt minus .5pt,
  xkanjiskip=.25\zw plus 1pt minus 1pt,
  autospacing, autoxspacing, jacharrange={-1},
  yalbaselineshift=\z@, yjabaselineshift=\z@,
  jcharwidowpenalty=500, differentjfm=paverage
}
%    \end{macrocode}
%
%    \begin{macrocode}
\ExplSyntaxOn
%    \end{macrocode}
%
% \paragraph{\pkg{LuaTeX-ja} 的补丁}
%
%    \begin{macrocode}
%<@@=ctex_ltj>
%    \end{macrocode}
%
% \changes{v2.3}{2015/09/26}{更新 \pkg{unicode-math} 宏包补丁。}
%
% 在 \LaTeX{} 下,\pkg{LuaTeX-ja} 对 \pkg{fontspec}、\pkg{xunicode}、\pkg{unicode-math}
% 和 \pkg{listings} 打了补丁。其中前三个是把 \tn{char} 换成 \tn{ltjalchar},确保
% 字符是 ALchar 类。我们这里用 \pkg{xunicode-addon} 来处理 \pkg{xunicode}。
%    \begin{macrocode}
\RequirePackage { xunicode-addon }
\AtBeginUTFCommand
  {
    \group_begin:
    \lua_now_x:n { tex.globaldefs = 0 }
    \ltj@allalchar
  }
\AtEndUTFCommand { \group_end: }
%    \end{macrocode}
%
% \changes{v2.4.2}{2016/05/15}{恢复 \pkg{luatexja} 对 \tn{emshape} 和
% \tn{eminnershape} 的重定义。}
% \changes{v2.4.3}{2016/08/26}{简化 \pkg{fontspec} 补丁。}
% \begin{macro}[int]{\fontspec_visible_space:}
% 我们不使用 \pkg{luatexja} 对 \pkg{fontspec} 的补丁,直接处理。
%    \begin{macrocode}
\cs_set_protected:Npn \fontspec_visible_space:
  {
    \font_glyph_if_exist:NnTF \font { "2423 }
      { \ltjalchar "2423 \exp_stop_f: }
      { \fontspec_visible_space_fallback: }
  }
%    \end{macrocode}
% \end{macro}
%
% 对 \pkg{listings} 的补丁是让代码环境支持 JAchar 类。\pkg{LuaTeX-ja} 的补丁会将
% 代码目录标题改为日文,我们不需要。
%    \begin{macrocode}
\ctex_at_end_package:nn { listings }
  {
    \use:x
      {
        \exp_not:N \RequirePackage { lltjp-listings }
        \tl_set:Nn \exp_not:N \lstlistingname
          { \exp_not:o { \lstlistingname } }
        \tl_set:Nn \exp_not:N \lstlistlistingname
          { \exp_not:o { \lstlistlistingname } }
      }
  }
%    \end{macrocode}
%
% \paragraph{字体切换方式}
%
% \begin{macro}[int]{\ctex_ltj_select_font:,\CJK@family}
% \tn{CJK@family} 保存的是当前 CJK 实际的字体族名,如果为空表示没有设置过字体。
%    \begin{macrocode}
\cs_new_protected_nopar:Npn \ctex_ltj_select_font:
  {
    \cs_if_exist_use:cF { \l_@@_current_font_tl }
      { \tl_if_empty:NF \CJK@family { \@@_select_font_aux: } }
  }
\tl_new:N \CJK@family
\tl_new:N \l_@@_current_font_tl
\tl_set:Nn \l_@@_current_font_tl
  { \CJK@encoding / \CJK@family / \f@series / \f@shape / \f@size }
%    \end{macrocode}
% \end{macro}
%
% \begin{macro}[int]{\@@_select_font_aux:}
% 使用 \tn{pickup@font} 取得字体名称前,总需要先设置 \tn{font@name}。在这里将
% \tn{f@family} 换成 CJK 字体族,并确保编码正确。
%    \begin{macrocode}
\cs_new_protected_nopar:Npn \@@_select_font_aux:
  {
    \group_begin:
      \tl_set_eq:NN \f@encoding \CJK@encoding
      \tl_set_eq:NN \f@family \CJK@family
      \@@_push_fontname:n { \use:c { \curr@fontshape / \f@size } }
      \ctex_ltj_pickup_font:
    \group_end:
    \font@name
    \@@_pop_fontname:
%    \end{macrocode}
% 当字形未定义的时候,\textsf{NFSS} 就会启动替换机制(\tn{wrong@fontshape})。
% 第一次启动后,\cs{l_@@_current_font_tl} 还是没有定义。为此,我们再次选择字体,
% 确保它有定义和指向正确的 \texttt{font.id}。这对 \opt{AlternateFont} 的设置
% 特别重要。
%    \begin{macrocode}
    \cs_if_exist:cF { \l_@@_current_font_tl }
      { \@@_select_font_aux: }
  }
\cs_new_protected_nopar:Npn \@@_push_fontname:n #1
  {
    \cs_gset_eq:NN \@@_save_fontname:w \font@name
    \cs_gset_nopar:Npx \font@name {#1}
  }
\cs_new_protected_nopar:Npn \@@_pop_fontname:
  { \cs_gset_eq:NN \font@name \@@_save_fontname:w }
%    \end{macrocode}
% \end{macro}
%
% \begin{macro}[int]{\ctex_ltj_pickup_font:}
% 替换 \tn{define@newfont} 内部调用的 \tn{extract@font} 和 \tn{do@subst@correction}。
%    \begin{macrocode}
\cs_new_protected_nopar:Npn \ctex_ltj_pickup_font:
  {
    \exp_after:wN \cs_if_exist:NF \font@name
      {
        \group_begin:
          \cs_set_eq:NN \extract@font \ctex_ltj_extract_font:
          \cs_set_eq:NN \do@subst@correction \ctex_ltj_subst_font:
          \define@newfont
        \group_end:
      }
  }
\cs_new_eq:NN \pickup@jfont \ctex_ltj_pickup_font:
%    \end{macrocode}
% \end{macro}
%
% \begin{macro}[int]{\ctex_ltj_extract_font:}
% \pkg{LuaTeX-ja} 的 \tn{globaljfont} 在 \pkg{luatexja-core} 中定义:
% \begin{verbatim}
%   %%%%%%%% \jfont\CS={...:...;jfm=metric;...}, \globaljfont
%   \protected\def\jfont#1{%
%     \afterassignment\ltj@@jfont
%     \directlua{luatexja.jfont.jfontdefX(false, 'yoko','\luatexluaescapestring{\noexpand#1}')}}
%   \protected\def\globaljfont#1{%
%     \afterassignment\ltj@@jfont
%     \directlua{luatexja.jfont.jfontdefX(true,  'yoko','\luatexluaescapestring{\noexpand#1}')}}
%   \def\ltj@@jfont{\directlua{luatexja.jfont.jfontdefY()}}
% \end{verbatim}
% \texttt{jfontdefX} 函数的作用是把 \tn{CS} 定义为其后的字体,\texttt{jfontdefY}
% 的作用是更新 \texttt{JFM} 和记录相关字体信息。最后的工作是:
% \begin{verbatim}
%   tex.sprint(cat_lp, global_flag, '\\protected\\expandafter\\def\\csname ',
%     (cstemp==' ') and '\\space' or cstemp, '\\endcsname{\\ltj@cur'..
%     (jfm_dir == 'yoko' and 'j' or 't') .. 'fnt', fn, '\\relax}')
% \end{verbatim}
% \tn{CS} 的作用就是把 \tn{ltj@curjfnt} 设置为刚才定义的字体的 \texttt{font.id}。
%    \begin{macrocode}
\cs_new_protected_nopar:Npn \ctex_ltj_extract_font:
  {
    \get@external@font
    \ctex_ltj_if_alternate_shape_exist:nT { \curr@fontshape }
      {
        \tl_set:Nx \external@font
          { \exp_after:wN \@@_patch_external_font:w \external@font }
      }
    \exp_after:wN \globaljfont \font@name \external@font \scan_stop:
%    \end{macrocode}
% 这里 \tn{font@name} 不会直接改变当前字体,而 \tn{DeclareFontFamily} 和
% \tn{DeclareFontShape} 的最后一个参数通常要使用 \tn{font} 来引用当前字体。
% 为此,我们在分组内启用之前定义的字体,以便能得到正确的 \tn{font}。对字体参数的
% 赋值总是全局的,不会受到分组的影响。
%    \begin{macrocode}
    \font@name
    \lua_now_x:n { font.current(tex.getattribute('ltj@curjfnt')) }
    \use:c { \f@encoding + \f@family }
    \use:c { \curr@fontshape }
  }
%    \end{macrocode}
% \end{macro}
%
% \begin{macro}[int]{\ctex_ltj_subst_font:}
% \tn{do@subst@correction} 在设置通过 \texttt{sub} 或者 \texttt{ssub} 函数定义的
% 字体时会用到。如果没有设置 \opt{SlantedFont},\pkg{fontspec} 会设置
% \tn{itdefault} 作为 \tn{sldefault} 的替代字形,因而会用到这个函数。它的本来定义是:
% \begin{verbatim}
%   \def\do@subst@correction{%
%       \xdef\subst@correction{%
%          \font@name
%          \global\expandafter\font
%            \csname \curr@fontshape/\f@size\endcsname
%            \noexpand\fontname\font
%           \relax}%
%       \aftergroup\subst@correction
%   }
% \end{verbatim}
% 我们在这里不需要定义新字体,而是设置对应字体的命令。
%    \begin{macrocode}
\cs_new_protected_nopar:Npn \ctex_ltj_subst_font:
  {
    \ctex_ltj_if_alternate_shape_exist:nF { \curr@fontshape }
      {
        \group_begin:
        \tl_set_eq:NN \CJK@family \f@family
        \cs_if_exist:cF { \l_@@_current_font_tl  }
          {
            \cs_gset_protected_nopar:Npx \subst@correction
              {
                \cs_new_eq:NN
                  \exp_not:c { \l_@@_current_font_tl }
                  \font@name
              }
            \group_insert_after:N \group_insert_after:N
            \group_insert_after:N \subst@correction
          }
        \group_end:
      }
  }
%    \end{macrocode}
% \end{macro}
%
% \begin{macro}[int,TF]{\ctex_ltj_if_alternate_shape_exist:n}
% 即 \pkg{LuaTeX-ja} 中的 \tn{ltj@@does@alt@set},判断是否存在替代字体。
%    \begin{macrocode}
\prg_new_conditional:Npnn \ctex_ltj_if_alternate_shape_exist:n #1 { T , F , TF }
  {
    \lua_now_x:n { luatexja.jfont.does_alt_set ('\lua_escape_x:n {#1}') }
      \prg_return_true: \else: \prg_return_false: \fi:
  }
%    \end{macrocode}
% \end{macro}
%
% \begin{macro}[aux]{\@@_patch_external_font:w}
% 若对字体的定义完全相同,则它们有相同的 \texttt{font.id}。因此如果字形是由
% \textsf{NFSS} 的替换机制定义的,它们就有相同的 \texttt{font.id}。
% |print_aftl_address| 函数的定义是
% \begin{verbatim}
%   function print_aftl_address()
%     tex.sprint(cat_lp, ';ltjaltfont' .. tostring(aftl_base):sub(8))
%   end
% \end{verbatim}
% 主要目的是,如果当前字形有替代字体,则往字形的定义中加入一些标志,确保
% \texttt{font.id} 唯一。
%    \begin{macrocode}
\cs_new_nopar:Npn \@@_patch_external_font:w #1 ~ at
  { #1 \lua_now_x:n { luatexja.jfont.print_aftl_address() } ~ at }
%    \end{macrocode}
% \end{macro}
%
% \begin{macro}[int]{\ctex_ltj_select_alternate_font:}
% 在 \tn{selectfont} 中更新替代字体。
%    \begin{macrocode}
\cs_new_protected_nopar:Npn \ctex_ltj_select_alternate_font:
  {
    \ctex_ltj_if_alternate_shape_exist:nT { \l_@@_current_shape_tl }
      {
        \lua_now_x:n
          {
            luatexja.jfont.output_alt_font_cmd
              ('y', '\lua_escape_x:n { \l_@@_current_shape_tl }')
          }
        \lua_now_x:n { luatexja.jfont.pickup_alt_font_a ('\f@size') }
      }
  }
\tl_new:N \l_@@_current_shape_tl
\tl_set:Nn \l_@@_current_shape_tl
  { \CJK@encoding / \CJK@family / \f@series / \f@shape }
%    \end{macrocode}
% \end{macro}
%
% \begin{macro}[int]{\ltj@pickup@altfont@auxy}
% 被用在函数 |output_alt_font_cmd| 中,作用是定义替代字体。
%    \begin{macrocode}
\cs_new_protected_nopar:Npn \ltj@pickup@altfont@auxy #1
  {
    \cs_if_exist:cF { #1/\f@size }
      {
        \group_begin:
          \use:x { \exp_not:N \split@name #1 / \f@size } \@nil
          \@@_push_fontname:n { \use:c { \curr@fontshape / \f@size } }
          \ctex_ltj_pickup_font:
        \group_end:
        \@@_pop_fontname:
      }
  }
%    \end{macrocode}
% \end{macro}
%
%    \begin{macrocode}
%<@@=>
%    \end{macrocode}
%
% \begin{macro}[int]{\ltj@pickup@altfont@copy}
% 被用在函数 |pickup_alt_font_a| 中。\tn{ltj@@getjfontnumber} 的作用是将字体命令
% |#1| 对应的 \texttt{font.id} 保存到 \tn{ltj@tempcntc} 中。
%    \begin{macrocode}
\cs_new_protected_nopar:Npn \ltj@pickup@altfont@copy #1#2
  {
    \ltj@@getjfontnumber #1
    \lua_now_x:n
      {
        luatexja.jfont.pickup_alt_font_b
          ( \int_use:N \ltj@tempcntc, '\lua_escape_x:n {#2}' )
      }
  }
%    \end{macrocode}
% \end{macro}
%
% \paragraph{数学字体族}
%
%    \begin{macrocode}
%<@@=ctex_ltj>
%    \end{macrocode}
%
% 以下内容来自 \file{lltjfont.sty},目的是让汉字可以在数学环境中直接使用。
%
% \begin{macro}[int]{\ctex_ltj_if_jfont:nTF}
% 参数 |#1| 是一个 \LaTeXe{} 编码名称或者字体命令。\LaTeXe{} 字体命令的一般形式是:
% \begin{quote}\ttfamily\small
%   \textbackslash\meta{encoding}/\meta{family}/\meta{series}/\meta{shape}
% \end{quote}
% 通过截取名字中的 \meta{encoding} 来判断是否是 jfont。
% 最后会设置 \tn{ifin@} 为对应的 \tn{iftrue} 或者 \tn{iffalse}。
%    \begin{macrocode}
\cs_new:Npn \ctex_ltj_if_jfont:nTF #1
  {
    \lua_now_x:n
      { luatexja.jfont.is_kenc( string.match('\lua_escape_x:n {#1}', '[^/]+') ) }
    \ifin@ \exp_after:wN \use_i:nn \else: \exp_after:wN \use_ii:nn \fi:
  }
%    \end{macrocode}
% \end{macro}
%
% \begin{macro}[int]{\ctex_ltj_if_jfont_math:NTF}
% |#1| 是一个形式为 |\M@|\meta{encoding} 的命令,它由 \tn{DeclareFontEncoding} 的
% 第三个参数来定义。
%    \begin{macrocode}
\cs_new:Npn \ctex_ltj_if_jfont_math:NTF #1
  { \exp_after:wN \@@_if_jfont_math:w \token_to_str:N #1 \q_stop }
\group_begin:
  \char_set_catcode_other:N M
  \cs_new:Npn \@@_if_jfont_math:w #1 M #2#3 \q_stop
    { \ctex_ltj_if_jfont:nTF {#3} }
\group_end:
%    \end{macrocode}
% \end{macro}
%
%    \begin{macrocode}
%<@@=>
%    \end{macrocode}
%
% \begin{macro}[int]{\getanddefine@fonts,\ctex_ltj_get_and_define_fonts:nN}
% 在使用的场合,\tn{escapechar} 已经被设置成 $-1$,使用 \cs{token_to_str:N} 就
% 可以得到名字,不必使用 \cs{cs_to_str:N}。
%    \begin{macrocode}
\cs_new_protected_nopar:Npn \ctex_ltj_get_and_define_fonts:nN #1#2
  {
    \ctex_ltj_if_jfont:nTF { \token_to_str:N #2 }
      { \ctex_ltj_get_and_define_fonts_ja:nN }
      { \ctex_ltj_get_and_define_fonts_al:nN }
      {#1} #2
  }
\cs_new_eq:NN \ctex_ltj_get_and_define_fonts_al:nN \getanddefine@fonts
\cs_set_eq:NN \getanddefine@fonts \ctex_ltj_get_and_define_fonts:nN
\cs_new_protected_nopar:Npn \ctex_ltj_get_and_define_fonts_ja:nN #1#2
  {
    \tl_gset:Nx \font@name { \use:c { \token_to_str:N #2 / \tf@size } }
    \ctex_ltj_pickup_font: \tl_set_eq:NN \textfont@name \font@name
    \tl_gset:Nx \font@name { \use:c { \token_to_str:N #2 / \sf@size } }
    \ctex_ltj_pickup_font: \tl_set_eq:NN \scriptfont@name \font@name
    \tl_gset:Nx \font@name { \use:c { \token_to_str:N #2 / \ssf@size } }
    \ctex_ltj_pickup_font:
    \tl_put_right:Nx \math@fonts
      {
        \ltj@setpar@global
        \ltj@@set@stackfont #1 , \textfont@name   \c_colon_str { MJT }
        \ltj@@set@stackfont #1 , \scriptfont@name \c_colon_str { MJS }
        \ltj@@set@stackfont #1 , \font@name       \c_colon_str { MJSS }
      }
  }
%    \end{macrocode}
% \end{macro}
%
%    \begin{macrocode}
%<@@=ctex_ltj>
%    \end{macrocode}
%
% \begin{macro}[int]{\use@mathgroup,\ctex_ltj_use_math_group:Nn}
% 在使用 \pkg{unicode-math} 宏包时,\cs{ctex_ltj_math_group_hook:} 将被重定义。
%    \begin{macrocode}
\cs_new_protected_nopar:Npn \ctex_ltj_use_math_group:Nn #1#2
  {
    \mode_if_math:T
      {
        \math@bgroup
          \cs_if_eq:cNF { M@ \f@encoding } #1 {#1}
          \ctex_ltj_math_group_hook:
          \ctex_ltj_if_jfont_math:NTF #1
            { \jfam } { \mathgroup } #2 \scan_stop:
        \math@egroup
      }
  }
\cs_new_eq:NN \ctex_ltj_math_group_hook: \prg_do_nothing:
\cs_set_eq:NN \use@mathgroup \ctex_ltj_use_math_group:Nn
%    \end{macrocode}
% \end{macro}
%
% \changes{v2.4.3}{2016/08/25}{更新 \pkg{unicode-math} 补丁。}
% \changes{v2.4.9}{2017/02/27}{调整 \pkg{unicode-math} 补丁的代码顺序。}
% 对 \pkg{unicode-math} 的补丁主要是将 \file{unicode-math-table.tex} 中的数学符号设置为
% \pkg{luatexja} 中的数学字母。
% 本段代码应放在 \cs{ctex_ltj_math_group_hook:} 的定义之后,避免因宏包载入顺序而造成的编译错误。
%    \begin{macrocode}
\cs_new_protected:Npn \ctex_ltj_set_math_letter:NN #1#2
  {
    \group_begin:
      \cs_set_protected:Npn #1 ##1##2##3
        { \ltjsetmathletter { ##1 } }
      #2
    \group_end:
  }
\ctex_at_end_package:nn { unicode-math }
  {
    \cs_if_exist:NTF \um_input_math_symbol_table:
      {
        \ctex_ltj_set_math_letter:NN
          \um_sym:nnn
          \um_input_math_symbol_table:
      }
      {
        \cs_set_eq:NN \use@mathgroup \ctex_ltj_use_math_group:Nn
        \cs_set_protected_nopar:Npn \ctex_ltj_math_group_hook:
          { \__um_switchto_literal: }
        \ctex_ltj_set_math_letter:NN
          \__um_sym:nnn
          \__um_input_math_symbol_table:
      }
  }
%    \end{macrocode}
%
% \paragraph{字体族的定义与使用}
%
% \begin{macro}[int]{\ctex_mono_jfm:n,\l_@@_jfm_tl}
% \pkg{LuaTeX-ja} 中与标点格式 \opt{plain} 对应的 \texttt{JFM} 是 \opt{mono}。
%    \begin{macrocode}
\cs_new_protected_nopar:Npn \ctex_mono_jfm:n #1
  {
    \str_if_eq:nnTF {#1} { plain }
      { \tl_set:Nn \l_@@_jfm_tl { mono } }
      { \tl_set:Nn \l_@@_jfm_tl {#1} }
  }
\tl_new:N \l_@@_jfm_tl
\cs_generate_variant:Nn \ctex_mono_jfm:n { o }
\ctex_mono_jfm:o { \l__ctex_punct_tl }
%    \end{macrocode}
% \end{macro}
%
% \begin{macro}[int]{\CJK@encoding,\@@_change_encoding:}
% 在 \LaTeX 下,\pkg{LuaTeX-ja} 依赖字体编码来实现特殊设置。例如上述的
% \cs{ctex_ltj_if_jfont:nTF} 就是通过判断编码来实现的,它在设置数学字体时会用到。所以
% 不应该与西文共用 \texttt{EU2}。定义字体族 song 为 \tn{CJK@encoding} 的默认替换
% 字体。下划线 |_| 不在 \tn{nfss@catcodes} 里,可以放心使用。
%    \begin{macrocode}
\tl_const:Nn \CJK@encoding { LTJY3 }
\DeclareFontEncoding { \CJK@encoding } { } { }
\use:x
  {
    \exp_not:N \DeclareFontSubstitution
      { \CJK@encoding } { song } { \mddefault } { \updefault }
  }
\lua_now_x:n { luatexja.jfont.add_kyenc_list('\CJK@encoding') }
\cs_new_protected_nopar:Npn \@@_change_encoding:
  { \tl_set_eq:NN \g_fontspec_encoding_tl \CJK@encoding }
\DeclareFontFamily { \CJK@encoding } { song } { }
\DeclareFontShape { \CJK@encoding } { song } { \mddefault } { \updefault }
  { <-> psft:SimSun:cid=Adobe-GB1-5;jfm=\l_@@_jfm_tl } { }
\DeclareFontShape { \CJK@encoding } { song } { \bfdefault } { \updefault }
  { <-> psft:SimHei:cid=Adobe-GB1-5;jfm=\l_@@_jfm_tl } { }
\tl_const:Nn \c_@@_math_tl { CJKmath }
\DeclareSymbolFont { \c_@@_math_tl }
  { \CJK@encoding } { song } { \mddefault } { \updefault }
\SetSymbolFont { \c_@@_math_tl } { bold }
  { \CJK@encoding } { song } { \bfdefault } { \updefault }
\int_const:Nn \c_@@_math_fam_int { \use:c { sym \c_@@_math_tl } }
\jfam \c_@@_math_fam_int
%    \end{macrocode}
% \end{macro}
%
% 这是 \pkg{luatexja-fontspec} 中新增的一些字体选项。
%    \begin{macrocode}
\newfontfeature { CID }     {    cid = #1 }
\newfontfeature { JFM }     {    jfm = #1 }
\newfontfeature { JFM-var } { jfmvar = #1 }
%    \end{macrocode}
%
% 在新版本的 \pkg{fontspec} 中,\cs{__fontspec_fontname_wrap:n} 变成了私有函数。
%    \begin{macrocode}
\keys_define:nn { fontspec-preparse-external }
  {
    NoEmbed .code:n =
      { \cs_set_eq:NN \__fontspec_fontname_wrap:n \@@_noembed_wrap:n }
  }
\cs_new:Npn \@@_noembed_wrap:n #1 { psft: #1 }
%    \end{macrocode}
%
% \begin{macro}[int]{\ctex_ltj_set_family:nnn}
% 将自定义的字体族名与 \pkg{fontspec} 实际设置的名字对应起来。
%    \begin{macrocode}
\cs_new_protected:Npn \ctex_ltj_set_family:nnn #1#2#3
  {
    \group_begin:
    \clist_clear:N \l_@@_char_range_clist
    \seq_clear:N \l_@@_alternate_seq
    \tl_set:Nn \l_@@_base_CJKfamily_tl {#1}
    \keys_set_known:nnN { ctex_ltj / fontspec } {#2} \l_@@_tmp_tl
    \clist_set:No \l_@@_font_options_clist { \l_@@_tmp_tl }
    \ctex_ltj_set_alternate_family:nnF {#1} {#3}
      {
        \prop_gput:Nnn \g_@@_family_font_name_prop {#1} {#3}
        \prop_gput:Nno \g_@@_family_font_options_prop
          {#1} { \l_@@_font_options_clist }
        \@@_update_family_uid:N \l_@@_font_options_clist
        \@@_use_global_options:N \l_@@_font_options_clist
        \@@_gset_family_cs:nn {#1} {#3}
      }
    \group_end:
  }
\tl_new:N \l_@@_base_CJKfamily_tl
\clist_new:N \l_@@_font_options_clist
\cs_new_protected_nopar:Npn \@@_use_global_options:N #1
  {
    \clist_concat:NNN #1 \g_@@_default_features_clist #1
    \clist_put_left:Nx #1 { JFM = \l_@@_jfm_tl }
  }
%    \end{macrocode}
% \end{macro}
%
% \begin{variable}[int]
%  {\g_@@_family_name_prop,\g_@@_family_font_name_prop,\g_@@_family_font_options_prop}
% 分别保存 \pkg{fontspec} 设置的字体族名、字体名称和字体选项。
%    \begin{macrocode}
\prop_new:N \g_@@_family_name_prop
\prop_new:N \g_@@_family_font_name_prop
\prop_new:N \g_@@_family_font_options_prop
%    \end{macrocode}
% \end{variable}
%
% \begin{macro}[int]{\@@_check_family:n}
% 删除重复的定义,清除替代字体的先前设置。
%   \begin{macrocode}
\cs_new_protected_nopar:Npn \@@_check_family:n #1
  {
    \prop_gpop:NnNT \g_@@_family_font_name_prop {#1} \l_@@_tmp_tl
      {
        \cs_undefine:c { \@@_family_csname:n {#1} }
        \cs_undefine:c { \@@_alternate_cs:n {#1} }
        \prop_gpop:NnNT \g_@@_family_name_prop {#1} \l_@@_base_family_tl
          {
            \use:c { \@@_alternate_cs:n { clear / #1 } }
            \cs_undefine:c { \@@_alternate_cs:n { clear / #1 } }
            \cs_undefine:c { \@@_alternate_cs:n { reset / #1 } }
            \prop_gremove:Nn \g_@@_reset_alternate_prop {#1}
          }
        \msg_warning:nnxx { ctex } { redefine-family } {#1} { \l_@@_tmp_tl }
      }
  }
\tl_new:N \l_@@_tmp_tl
\msg_new:nnn { ctex } { redefine-family }
  { Redefining~CJKfamily~`\@@_msg_family_map:n {#1}'~(#2). }
%    \end{macrocode}
% \end{macro}
%
% \begin{macro}[int]{\@@_gset_family_cs:nn}
% 在设置字体时,实际上并不是马上就定义。而是只保存相关参数,在通过 \tn{CJKfamily}
% 第一次使用时才定义。需要注意将编码改为 \tn{CJK@encoding}。
%    \begin{macrocode}
\cs_new_protected_nopar:Npn \@@_gset_family_cs:nn #1#2
  {
    \cs_gset_protected_nopar:cpx { \@@_family_csname:n {#1} }
      {
        \group_begin:
        \@@_change_encoding:
        \exp_not:n { \cs_set_eq:NN \CJKfamily \use_none:n }
        \exp_not:n { \cs_set_eq:NN \CTEX@fontfamily \use_none:n }
        \exp_not:n { \fontspec_set_family:Nnn \g_@@_fontspec_family_tl }
          { \exp_not:o { \l_@@_font_options_clist } } {#2}
        \prop_gput:Nno \exp_not:N \g_@@_family_name_prop {#1}
          { \exp_not:N \g_@@_fontspec_family_tl }
        \tl_gset_eq:NN \exp_not:N \g_@@_fontspec_family_tl
          \exp_not:N \g_@@_fontspec_family_tl
        \@@_set_alternate_family:n {#1}
        \group_end:
      }
  }
\tl_new:N \l_@@_base_family_tl
\tl_new:N \g_@@_fontspec_family_tl
\cs_new_nopar:Npn \@@_family_csname:n #1 { ctex_ltj/family/#1 }
\cs_new_protected_nopar:Npn \@@_set_alternate_family:n #1
  {
    \tl_set:Nn \l_@@_base_CJKfamily_tl {#1}
    \tl_set_eq:NN \l_@@_base_family_tl \g_@@_fontspec_family_tl
    \cs_if_exist_use:c { \@@_alternate_cs:n { reset / #1 } }
    \cs_if_exist_use:c { \@@_alternate_cs:n {#1} }
  }
\cs_new:Npn \@@_alternate_cs:n #1 { ctex_ltj/alternate_family/#1 }
%    \end{macrocode}
% \end{macro}
%
% \begin{macro}[int]{\CJKfamily}
% 切换字体。
%    \begin{macrocode}
\NewDocumentCommand \CJKfamily { m }
  { \ctex_ltj_switch_family:x {#1} \tex_ignorespaces:D }
\cs_new_protected_nopar:Npn \ctex_ltj_switch_family:n #1
  {
    \ctex_ltj_family_if_exist:xNTF {#1} \CJK@family
      {
        \tl_set:Nn \l_ctex_ltj_family_tl {#1}
        \selectfont
      }
      { \@@_family_unknown_warning:n {#1} }
  }
\tl_new:N \l_ctex_ltj_family_tl
\cs_generate_variant:Nn \ctex_ltj_switch_family:n { x }
%    \end{macrocode}
% \end{macro}
%
% \begin{macro}[int,TF]{\ctex_ltj_family_if_exist:n}
% 判断 CJK 字体族 |#1| 是否存在,若存在则把实际族名保存到 |#2| 中。
%    \begin{macrocode}
\prg_new_protected_conditional:Npnn \ctex_ltj_family_if_exist:xN #1#2 { T , F , TF }
  {
    \prop_get:NxNTF \g_@@_family_name_prop {#1} #2
      { \prg_return_true: }
      {
        \cs_if_exist_use:cTF { \@@_family_csname:n {#1} }
          {
            \tl_set_eq:NN #2 \g_@@_fontspec_family_tl
            \prg_return_true:
          }
          { \prg_return_false: }
      }
  }
\cs_generate_variant:Nn \prop_get:NnNTF { Nx }
%    \end{macrocode}
% \end{macro}
%
% \begin{macro}[int]{\@@_family_unknown_warning:n}
%    \begin{macrocode}
\cs_new_protected_nopar:Npn \@@_family_unknown_warning:n #1
  {
    \prop_if_empty:NF \g_@@_family_font_name_prop
      {
        \seq_if_in:NnF \g_@@_unknown_family_seq {#1}
          {
            \seq_gput_right:Nn \g_@@_unknown_family_seq {#1}
            \msg_warning:nnn { ctex } { family-unknown } {#1}
          }
      }
  }
\seq_new:N \g_@@_unknown_family_seq
\msg_new:nnn { ctex } { family-unknown }
  {
    Unknown~CJK~family~`\@@_msg_family_map:n {#1}'~is~being~ignored.\\
    Try~to~use~`\@@_msg_def_family_map:n {#1}'~to~define~it.
  }
\cs_new_nopar:Npn \@@_msg_def_family_map:n #1
  {
    \str_case_x:nnF {#1}
      {
        \CJKrmdefault { \token_to_str:N \setCJKmainfont }
        \CJKsfdefault { \token_to_str:N \setCJKsansfont }
        \CJKttdefault { \token_to_str:N \setCJKmonofont }
      }
      { \token_to_str:N \setCJKfamilyfont \{ #1 \} }
    [...]\{...\}
  }
\cs_new_nopar:Npn \@@_msg_family_map:n #1
  {
    \str_case_x:nnF {#1}
      {
        \CJKrmdefault { \token_to_str:N \CJKrmdefault }
        \CJKsfdefault { \token_to_str:N \CJKsfdefault }
        \CJKttdefault { \token_to_str:N \CJKttdefault }
      }
      {#1}
  }
%    \end{macrocode}
% \end{macro}
%
% \begin{macro}[int]{\ctex_ltj_fontspec:nn}
%    \begin{macrocode}
\cs_new_protected_nopar:Npn \ctex_ltj_fontspec:nn #1#2
  {
    \prop_get:NnNTF \g_@@_fontspec_prop
      { CJKfontspec/#1/#2/id } \l_ctex_ltj_family_tl
      { \ctex_ltj_switch_family:x { \l_ctex_ltj_family_tl } }
      {
        \int_gincr:N \g_@@_family_int
        \@@_fontspec:xnn
          { CJKfontspec ( \int_use:N \g_@@_family_int ) }
          {#1} {#2}
      }
  }
\cs_new_protected_nopar:Npn \ctex_ltj_fontspec:xx #1#2
  { \use:x { \ctex_ltj_fontspec:nn {#1} {#2} } }
\cs_new_protected_nopar:Npn \@@_fontspec:nnn #1#2#3
  {
    \bool_if:NT \l_@@_add_alternate_bool
      {
        \cs_if_free:cF
          { \@@_alternate_cs:n { reset / \l_ctex_ltj_family_tl } }
          {
            \cs_gset_eq:cc
              { \@@_alternate_cs:n { reset / #1 } }
              { \@@_alternate_cs:n { reset / \l_ctex_ltj_family_tl } }
            \cs_gset_eq:cc
              { \@@_alternate_cs:n { clear / #1 } }
              { \@@_alternate_cs:n { clear / \l_ctex_ltj_family_tl } }
          }
        \bool_set_false:N \l_@@_add_alternate_bool
      }
    \prop_gput:Nnn \g_@@_fontspec_prop { CJKfontspec/#2/#3/id } {#1}
    \ctex_ltj_set_family:nnn {#1} {#2} {#3}
    \ctex_ltj_switch_family:n {#1}
  }
\cs_generate_variant:Nn \@@_fontspec:nnn { x }
\prop_new:N \g_@@_fontspec_prop
%    \end{macrocode}
% \end{macro}
%
% \begin{macro}[int]
% {\ctex_ltj_add_font_features:n,\ctex_ltj_add_font_features:nn}
%    \begin{macrocode}
\cs_new_protected_nopar:Npn \ctex_ltj_add_font_features:n #1
  { \ctex_ltj_add_font_features:xn { \l_ctex_ltj_family_tl } {#1} }
\cs_new_protected_nopar:Npn \ctex_ltj_add_font_features:nn #1#2
  {
    \prop_get:NnNTF \g_@@_family_font_name_prop
      {#1} \l_@@_tmp_tl
      {
        \prop_get:NnN \g_@@_family_font_options_prop
          {#1} \l_@@_font_options_clist
        \clist_put_right:Nn \l_@@_font_options_clist {#2}
        \bool_set_true:N \l_@@_add_alternate_bool
        \ctex_ltj_fontspec:xx
          { \exp_not:o { \l_@@_font_options_clist } }
          { \exp_not:o { \l_@@_tmp_tl } }
      }
      { \msg_warning:nn { ctex } { addCJKfontfeature-ignored } }
  }
\bool_new:N \l_@@_add_alternate_bool
\cs_generate_variant:Nn \ctex_ltj_add_font_features:n  { x }
\cs_generate_variant:Nn \ctex_ltj_add_font_features:nn { x }
\msg_new:nnn { ctex } { addCJKfontfeature-ignored }
  {
    \token_to_str:N \addCJKfontfeature (s)~ignored.\\
    It~cannot~be~used~with~a~font~that~wasn't~selected~by~ctex.
  }
%    \end{macrocode}
% \end{macro}
%
% \begin{macro}[int]{\@@_pass_args:nnnn}
% 为了支持字体属性可选项在前在后两种语法,给出两个辅助工具,自带展开功能。
%    \begin{macrocode}
\cs_new_protected:Npn \@@_pass_args:nnnn #1#2#3#4
  {
    \IfNoValueTF {#2}
      { \@@_post_arg:w {#1} {#3} {#4} }
      {
        \use:x { #1 {#2} {#3} }
        #4
      }
  }
\NewDocumentCommand \@@_post_arg:w { m m m O { } }
  {
    \use:x { #1 {#4} {#2} }
    #3
  }
%    \end{macrocode}
% \end{macro}
%
% \changes{v2.4.6}{2016/11/20}{支持字体属性可选项在后的新语法。}
%
% \begin{macro}[int]
% {\setCJKfamilyfont,\newCJKfontfamily,\CJKfontspec,\addCJKfontfeatures}
%    \begin{macrocode}
\NewDocumentCommand \setCJKfamilyfont { m o m }
  {
    \@@_pass_args:nnnn
      { \ctex_ltj_set_family:nnn {#1} } {#2} {#3}
      { }
  }
\NewDocumentCommand \newCJKfontfamily { o m o m }
  {
    \tl_set:Nx \l_@@_tmp_tl
      { \IfNoValueTF {#1} { \cs_to_str:N #2 } {#1} }
    \cs_new_protected_nopar:Npx #2
      { \ctex_ltj_switch_family:n { \l_@@_tmp_tl } }
    \@@_pass_args:nnnn
      { \ctex_ltj_set_family:nnn { \l_@@_tmp_tl } } {#3} {#4}
      { }
  }
\NewDocumentCommand \CJKfontspec { o m }
  {
    \@@_pass_args:nnnn
      { \ctex_ltj_fontspec:nn } {#1} {#2}
      { \tex_ignorespaces:D }
  }
\NewDocumentCommand \addCJKfontfeatures { m }
  {
    \ctex_ltj_add_font_features:x {#1}
    \tex_ignorespaces:D
  }
\cs_new_eq:NN \addCJKfontfeature \addCJKfontfeatures
%    \end{macrocode}
% \end{macro}
%
% \begin{macro}[int]
% {\setCJKmainfont,\setCJKsansfont,\setCJKmonofont,
%  \setCJKmathfont,\defaultCJKfontfeatures}
%    \begin{macrocode}
\NewDocumentCommand \setCJKmainfont { o m }
  {
    \@@_pass_args:nnnn
      { \ctex_ltj_set_family:nnn { \CJKrmdefault } } {#1} {#2}
      { \normalfont }
  }
\cs_new_eq:NN \setCJKromanfont \setCJKmainfont
\NewDocumentCommand \setCJKsansfont { o m }
  {
    \@@_pass_args:nnnn
      { \ctex_ltj_set_family:nnn { \CJKsfdefault } } {#1} {#2}
      { \normalfont }
  }
\NewDocumentCommand \setCJKmonofont { o m }
  {
    \@@_pass_args:nnnn
      { \ctex_ltj_set_family:nnn { \CJKttdefault } } {#1} {#2}
      { \normalfont }
  }
\NewDocumentCommand \setCJKmathfont { o m }
  {
    \@@_pass_args:nnnn
      { \ctex_ltj_set_family:nnn { \c_@@_math_tl } } {#1} {#2}
      { }
  }
\NewDocumentCommand \defaultCJKfontfeatures { m }
  { \clist_gset:Nn \g_@@_default_features_clist {#1} }
\clist_new:N \g_@@_default_features_clist
\@onlypreamble \setCJKmainfont
\@onlypreamble \setCJKsansfont
\@onlypreamble \setCJKmonofont
\@onlypreamble \setCJKmathfont
\@onlypreamble \setCJKromanfont
\@onlypreamble \defaultCJKfontfeatures
%    \end{macrocode}
% \end{macro}
%
%    \begin{macrocode}
\tl_if_exist:NF \CJKfamilydefault
  { \tl_const:Nn \CJKfamilydefault { \CJKrmdefault } }
\tl_if_exist:NF \CJKrmdefault { \tl_const:Nn \CJKrmdefault { rm } }
\tl_if_exist:NF \CJKsfdefault { \tl_const:Nn \CJKsfdefault { sf } }
\tl_if_exist:NF \CJKttdefault { \tl_const:Nn \CJKttdefault { tt } }
%    \end{macrocode}
%
% \begin{macro}[int]{\fontfamily}
% \begin{macro}[int]{\CTEX@fontfamily}
% \changes{v2.4.8}{2017/02/23}
%   {解决与 \pkg{fontspec} 2017/01/24 v2.5d 的字体族匹配兼容问题。}
% 修改 \tn{fontfamily},使主要 |CJK| 字体族能随西文主要字体更新。
%    \begin{macrocode}
\RenewDocumentCommand \fontfamily { m }
  {
    \tl_set:Nx \f@family {#1}
    \CTEX@fontfamily {#1}
  }
\cs_new_protected_nopar:Npn \CTEX@fontfamily #1
  {
    \str_if_eq:nnTF {#1} { \familydefault }
      { \CJKfamily { \CJKfamilydefault } }
      { \@@_update_family_aux: }
  }
\cs_new_protected_nopar:Npn \@@_update_family_aux:
  {
    \str_case_x:nn { \f@family }
      {
        { \rmdefault }     { \CJKfamily { \CJKrmdefault } }
        { \sfdefault }     { \CJKfamily { \CJKsfdefault } }
        { \ttdefault }     { \CJKfamily { \CJKttdefault } }
        { \familydefault } { \CJKfamily { \CJKfamilydefault } }
      }
  }
%    \end{macrocode}
% \end{macro}
% \end{macro}
%
% 在导言区末尾更新 \tn{CJKfamilydefault}。
%    \begin{macrocode}
\ctex_at_end_preamble:n { \ctex_update_default_family: }
%    \end{macrocode}
%
% \begin{macro}[int]{\ctex_ltj_ensure_default_family:}
% 在导言区结束确认 \tn{CJKfamilydefault} 确实存在。
%    \begin{macrocode}
\cs_new_protected_nopar:Npn \ctex_ltj_ensure_default_family:
  {
    \prop_if_empty:NF \g_@@_family_font_name_prop
      {
        \ctex_ltj_family_if_exist:xNF { \CJKfamilydefault } \l_@@_tmp_tl
          {
            \str_if_eq_x:nnTF { \CJKfamilydefault } { \CJKrmdefault }
              { \use:n }
              {
                \ctex_ltj_family_if_exist:xNTF { \CJKrmdefault } \l_@@_tmp_tl
                  { \tl_gset:Nn \CJKfamilydefault { \CJKrmdefault } \use_none:n }
                  { \use:n }
              }
              {
                \prop_map_inline:Nn \g_@@_family_font_name_prop
                  {
                    \prop_map_break:n
                      { \tl_gset_rescan:Nnn \CJKfamilydefault { } { ##1 } }
                  }
              }
          }
        \normalfont
        \ctex_ltj_update_mathfont:
      }
  }
%    \end{macrocode}
% \end{macro}
%
% \begin{macro}[int]{\ctex_ltj_update_mathfont:}
% 更新数学字体为实际的字体。
%    \begin{macrocode}
\cs_new_protected_nopar:Npn \ctex_ltj_update_mathfont:
  {
    \ctex_ltj_family_if_exist:xNTF { \c_@@_math_tl } \l_@@_tmp_tl
      { \ctex_ltj_update_mathfont:n { \l_@@_tmp_tl } }
      {
        \ctex_ltj_family_if_exist:xNT { \CJKfamilydefault } \l_@@_tmp_tl
          { \ctex_ltj_update_mathfont:n { \l_@@_tmp_tl } }
      }
  }
\cs_new_protected_nopar:Npn \ctex_ltj_update_mathfont:n #1
  {
    \tl_const:Nx \c_@@_math_family_tl {#1}
    \DeclareSymbolFont { \c_@@_math_tl } { \CJK@encoding }
      { \c_@@_math_family_tl } { \mddefault } { \updefault }
    \cs_if_free:cTF
      { \CJK@encoding/\c_@@_math_family_tl/\bfdefault/\updefault }
      {
        \SetSymbolFont { \c_@@_math_tl } { bold } { \CJK@encoding }
          { \c_@@_math_family_tl } { \mddefault } { \updefault }
      }
      {
        \SetSymbolFont { \c_@@_math_tl } { bold } { \CJK@encoding }
          { \c_@@_math_family_tl } { \bfdefault } { \updefault }
      }
  }
%    \end{macrocode}
% \end{macro}
%
% \paragraph{替代字体的设置}
%
% \begin{macro}{AlternateFont,CharRange}
% 设置替代字体的选项。
%    \begin{macrocode}
\keys_define:nn { ctex_ltj / fontspec }
  {
    AlternateFont  .code:n = \ctex_ltj_set_alternate_seq:n {#1} ,
    AlternateFont  .value_required:n = true ,
    CharRange .clist_set:N = \l_@@_char_range_clist ,
    CharRange .value_required:n = true
  }
%    \end{macrocode}
% \end{macro}
%
% \begin{macro}[int]{\ctex_ltj_set_alternate_seq:n}
% 我们使用 \verb=||= 作为替代字体序列的分隔标志。它可能被设置为活动字符,为此
% 需要先“消毒”,同时过滤掉空元素。
%    \begin{macrocode}
\group_begin:
  \char_set_catcode_other:N \|
  \cs_set:Npn \@@_tmp:w #1
    {
      \cs_new_protected:Npn \ctex_ltj_set_alternate_seq:n ##1
        {
          \clist_if_empty:NT \l_@@_char_range_clist
            {
              \tl_set:Nn \l_@@_tmp_tl { ##1 }
              \tl_replace_all:Nnn \l_@@_tmp_tl {#1} { || }
              \seq_set_split:NnV \l_@@_tmp_seq { || } \l_@@_tmp_tl
              \seq_set_filter:NNn \l_@@_tmp_seq \l_@@_tmp_seq
                { ! \tl_if_blank_p:n { ####1 } }
              \seq_concat:NNN \l_@@_alternate_seq
                \l_@@_alternate_seq \l_@@_tmp_seq
            }
        }
    }
  \char_set_catcode_active:N \|
  \@@_tmp:w { || }
\group_end:
\seq_new:N \l_@@_tmp_seq
\seq_new:N \l_@@_alternate_seq
%    \end{macrocode}
% \end{macro}
%
% \begin{macro}[int]{\ctex_ltj_set_alternate_family:nnF}
% 如果在字体的选项中设置了 \opt{CharRange},则只设置替代字体。
%    \begin{macrocode}
\cs_new_protected_nopar:Npn \ctex_ltj_set_alternate_family:nnF #1#2#3
  {
    \clist_if_empty:NTF \l_@@_char_range_clist
      {
        \@@_check_family:n {#1}
        \seq_if_empty:NF \l_@@_alternate_seq
          { \ctex_ltj_save_alternate_seq:cn { \@@_alternate_cs:n {#1} } {#2} }
        #3
      }
      { \ctex_ltj_set_alternate_family:nn {#1} {#2} }
  }
%    \end{macrocode}
% \end{macro}
%
% \begin{macro}[int]
% {\ctex_ltj_save_alternate_seq:Nn,\ctex_ltj_save_alternate_seq:Nnnwn}
% 保存由 \opt{AlternateFont} 设置的替代字体序列。
%    \begin{macrocode}
\cs_new_protected_nopar:Npn \ctex_ltj_save_alternate_seq:Nn #1#2
  {
    \seq_map_inline:Nn \l_@@_alternate_seq
      { \ctex_ltj_save_alternate_seq:Nnnwnw #1 {#2} ##1 { } \q_stop }
  }
\cs_generate_variant:Nn \ctex_ltj_save_alternate_seq:Nn { c }
\NewDocumentCommand \ctex_ltj_save_alternate_seq:Nnnwnw
  { m m m +O{ } m u{ \q_stop } }
  {
    \clist_set:Nn \l_@@_char_range_clist {#3}
    \clist_set:Nn \l_@@_alternate_options_clist {#4}
    \@@_use_global_options:N \l_@@_alternate_options_clist
    \tl_if_blank:nTF {#5}
      { \tl_set:Nn \l_@@_tmp_tl {#2} }
      {
        \tl_set:Nn \l_@@_tmp_tl {#5}
        \tl_replace_all:Nnn \l_@@_tmp_tl { * } {#2}
      }
    \use:x
      {
        \ctex_ltj_save_alternate_family:Nnnn \exp_not:N #1
          { \exp_not:o { \l_@@_char_range_clist } }
          { \exp_not:o { \l_@@_alternate_options_clist } }
          { \exp_not:o { \l_@@_tmp_tl } }
      }
  }
\clist_new:N \l_@@_alternate_options_clist
%    \end{macrocode}
% \end{macro}
%
% \begin{macro}[int]{\ctex_ltj_set_alternate_family:nn}
% 设置选项 \opt{CharRange} 范围内的替代字体。如果已经定义了主字体,我们也马上
% 定义替代字体,否则只保存起来备用。
%    \begin{macrocode}
\cs_new_protected_nopar:Npn \ctex_ltj_set_alternate_family:nn #1#2
  {
    \@@_update_family_uid:N \l_@@_font_options_clist
    \@@_use_global_options:N \l_@@_font_options_clist
    \ctex_ltj_set_alternate_family:coonn
      { \@@_alternate_cs:n {#1} }
      { \l_@@_char_range_clist }
      { \l_@@_font_options_clist } {#2} {#1}
  }
\cs_new_protected_nopar:Npn \ctex_ltj_set_alternate_family:Nnnnn #1#2#3#4#5
  {
    \prop_get:NnNT \g_@@_family_name_prop {#5} \l_@@_base_family_tl
      { \ctex_ltj_set_alternate_family:nnn {#2} {#3} {#4} }
    \ctex_ltj_save_alternate_family:Nnnn #1 {#2} {#3} {#4}
  }
\cs_generate_variant:Nn \ctex_ltj_set_alternate_family:Nnnnn { coo }
%    \end{macrocode}
% \end{macro}
%
% \begin{macro}[int]{\ctex_ltj_save_alternate_family:Nnnn}
% 保存替代字体序列的定义,以备定义主字体时使用。
%    \begin{macrocode}
\cs_new_protected_nopar:Npn \ctex_ltj_save_alternate_family:Nnnn #1#2#3#4
  {
    \cs_if_exist:NF #1 { \cs_set_eq:NN #1 \prg_do_nothing: }
    \cs_gset_protected_nopar:Npx #1
      { \exp_not:o { #1 \ctex_ltj_set_alternate_family:nnn {#2} {#3} {#4} } }
  }
%    \end{macrocode}
% \end{macro}
%
% \begin{macro}[int]{\ctex_ltj_set_alternate_family:nnn}
% 实际定义替代字体族。
%    \begin{macrocode}
\cs_new_protected_nopar:Npn \ctex_ltj_set_alternate_family:nnn #1#2#3
  {
    \group_begin:
    \@@_change_encoding:
    \cs_set_eq:NN \CJKfamily \use_none:n
    \ctex_ltj_swap_cs:NN
      \DeclareFontShape@ \ctex_ltj_declare_alternate_shape:nnnnnn
    \tl_set:Nn \l_@@_char_range_clist {#1}
    \fontspec_set_family:Nnn \l_@@_alternate_family_tl {#2} {#3}
    \group_end:
  }
\tl_new:N \l_@@_alternate_family_tl
%    \end{macrocode}
% \end{macro}
%
% \begin{macro}[int]{\ctex_ltj_swap_cs:NN}
% 交换两个控制序列的意义。
%    \begin{macrocode}
\cs_new_protected:Npn \ctex_ltj_swap_cs:NN #1#2
  {
    \cs_set_eq:NN \@@_tmp:w #1
    \cs_set_eq:NN #1 #2
    \cs_set_eq:NN #2 \@@_tmp:w
    \cs_undefine:N \@@_tmp:w
  }
%    \end{macrocode}
% \end{macro}
%
% \begin{macro}[int]{LTJFONTUID,\@@_update_family_uid:N}
% \pkg{fontspec} 在一个字体族的选项和字体名称相同的时候,就不定义新字体。为了
% 避免混淆替代字体的设置,我们新定义一个虚拟的选项 \opt{LTJFONTUID},确保
% \pkg{fontspec} 对 CJK 字体族总是定义新字体。
%    \begin{macrocode}
\keys_define:nn { fontspec } { LTJFONTUID .code:n = }
\cs_new_protected_nopar:Npn \@@_update_family_uid:N #1
  {
    \int_gincr:N \g_@@_family_int
    \clist_put_right:Nx #1 { LTJFONTUID = \int_use:N \g_@@_family_int }
  }
\int_new:N \g_@@_family_int
%    \end{macrocode}
% \end{macro}
%
% \begin{macro}[int]{\ctex_ltj_declare_alternate_shape:nnnnnn}
% 在定义替代字体的字形时,通过字符范围与主字体的对应字形关联起来。
% \tn{DeclareFontShape@} 一个有六个参数,我们只需要使用它的第三个参数 \meta{series}
% 和第四个参数 \meta{shape}。
%    \begin{macrocode}
\cs_new_protected:Npn \ctex_ltj_declare_alternate_shape:nnnnnn #1#2#3#4#5#6
  {
    \ctex_ltj_declare_alternate_shape:nnnnnn {#1} {#2} {#3} {#4} {#5} {#6}
    \ctex_ltj_set_alternate_shape:Nnnnnnn \l_@@_char_range_clist
      { \l_@@_base_family_tl } {#3} {#4}
      { \l_fontspec_family_tl } {#3} {#4}
  }
%    \end{macrocode}
% \end{macro}
%
% \begin{macro}[int]{\ctex_ltj_set_alternate_shape:Nnnnnnn}
% 与 \pkg{LuaTeX-ja} 的 \tn{DeclareAlternateKanjiFont} 的功能类似,区别是固定编码
% 为 \tn{CJK@encoding}。这个设置总是全局的。
%    \begin{macrocode}
\cs_new_protected_nopar:Npn \ctex_ltj_set_alternate_shape:Nnnnnnn #1#2#3#4#5#6#7
  {
    \clist_map_inline:Nn #1
      {
        \prop_get:NnNTF \g_@@_char_range_prop { ##1 } \l_@@_char_range_tl
          {
            \ctex_ltj_set_alternate_shape:nnN { #2/#3/#4 } { #5/#6/#7 }
              \l_@@_char_range_tl
          }
          { \ctex_ltj_set_alternate_shape:nnn { #2/#3/#4 } { #5/#6/#7 } { ##1 } }
      }
    \@@_save_alternate_shape:cnn
      { \@@_alternate_cs:n { clear / \l_@@_base_CJKfamily_tl } }
      { luatexja.jfont.clear_alt_font_latex }
      { '\lua_escape_x:n { \CJK@encoding/#2/#3/#4 }' }
  }
%    \end{macrocode}
% \end{macro}
%
% \begin{macro}[int]{\ctex_ltj_set_alternate_shape:nnn}
% 我们使用 \texttt{->} 而不是像 \pkg{LuaTeX-ja} 一样使用 \texttt{-} 作为区间的
% 分隔符。\pkg{LuaTeX-ja} 支持使用负数来引用由 \texttt{JFM} 设置的字符类。如果
% 使用 \texttt{-} 作为分隔符,那么负数单独使用时,就需要把它放在两层花括号之内
% (例如 |{{-1}}|),或者使用类似 |{-1}-{-1}| 的形式才不会解释错误。
%    \begin{macrocode}
\NewDocumentCommand \ctex_ltj_set_alternate_shape:nnn
  { m m > { \SplitArgument { \c_one } { -> } } m }
  { \ctex_ltj_set_alternate_shape:nnnn {#1} {#2} #3 }
\cs_new_protected_nopar:Npn \ctex_ltj_set_alternate_shape:nnnn #1#2#3#4
  {
    \ctex_ltj_set_alternate_shape:n
      {
        \IfNoValueTF {#4}
          { \int_eval:n {#3} , \int_eval:n {#3} , }
          {
            \int_eval:n { \tl_if_blank:nTF {#3} { "80 } {#3} } ,
            \int_eval:n { \tl_if_blank:nTF {#4} { "10FFFF } {#4} } ,
          }
        '\lua_escape_x:n { \CJK@encoding/#2 }' ,
        '\lua_escape_x:n { \CJK@encoding/#1 }'
      }
  }
\cs_new_protected_nopar:Npn \ctex_ltj_set_alternate_shape:n #1
  {
    \lua_now_x:n { luatexja.jfont.set_alt_font_latex ( #1 ) }
    \@@_save_alternate_shape:cnn
      { \@@_alternate_cs:n { reset / \l_@@_base_CJKfamily_tl } }
      { luatexja.jfont.set_alt_font_latex } {#1}
  }
%    \end{macrocode}
% \end{macro}
%
% \begin{macro}[int]{\ctex_ltj_set_alternate_shape:nnN}
% 若字符范围预先由 \texttt{declarecharrange} 声明,则可以直接使用。
%    \begin{macrocode}
\cs_new_protected_nopar:Npn \ctex_ltj_set_alternate_shape:nnN #1#2#3
  {
    \tl_map_inline:Nn #3
      {
        \ctex_ltj_set_alternate_shape:n
          {
            ##1 ,
            '\lua_escape_x:n { \CJK@encoding/#2 }' ,
            '\lua_escape_x:n { \CJK@encoding/#1 }'
          }
      }
  }
%    \end{macrocode}
% \end{macro}
%
% \begin{macro}[aux]{\@@_save_alternate_shape:Nnn}
% 将实际设置的替换字形保存起来用于清除或恢复。
% 暂时令 \cs{l_@@_base_family_tl} 为 \cs{scan_stop:} 是让它不被展开,使得替换
% 字体的设置可以在 \tn{addCJKfontfeature} 中直接使用。
%    \begin{macrocode}
\cs_new_protected_nopar:Npn \@@_save_alternate_shape:Nnn #1#2#3
  {
    \group_begin:
    \cs_if_exist:NF #1 { \cs_set_eq:NN #1 \prg_do_nothing: }
    \cs_set_eq:NN \l_@@_base_family_tl \scan_stop:
    \cs_set_eq:NN \lua_escape_x:n \scan_stop:
    \cs_gset_protected_nopar:Npx #1
      { \exp_not:o {#1} \exp_not:N \lua_now_x:n { #2 ( #3 ) } }
    \group_end:
  }
\cs_generate_variant:Nn \@@_save_alternate_shape:Nnn { c }
%    \end{macrocode}
% \end{macro}
%
% \begin{macro}{clearalternatefont,resetalternatefont}
% 清除和重置操作总是全局的。
%    \begin{macrocode}
\keys_define:nn { ctex }
  {
    clearalternatefont    .code:n =
      { \clist_map_function:xN {#1} \ctex_ltj_clear_alternate_font:n } ,
    resetalternatefont    .code:n =
      { \clist_map_function:xN {#1} \ctex_ltj_reset_alternate_font:n } ,
    clearalternatefont .default:n = \l_ctex_ltj_family_tl ,
    resetalternatefont .default:n = \l_ctex_ltj_family_tl
  }
\cs_new_protected_nopar:Npn \ctex_ltj_clear_alternate_font:n #1
  {
    \group_begin:
      \ctex_ltj_family_if_exist:xNTF {#1} \l_@@_base_family_tl
        {
          \cs_if_exist_use:cT { \@@_alternate_cs:n { clear / #1 } }
            {
              \prop_gput:Nno \g_@@_reset_alternate_prop
                {#1} { \l_@@_base_family_tl }
              \tl_set_eq:NN \CJK@family \l_@@_base_family_tl
              \selectfont
            }
        }
        { \@@_family_unknown_warning:n {#1} }
    \group_end:
  }
\cs_new_protected_nopar:Npn \ctex_ltj_reset_alternate_font:n #1
  {
    \group_begin:
      \prop_gpop:NnNT \g_@@_reset_alternate_prop {#1} \CJK@family
        {
          \tl_set_eq:NN \l_@@_base_family_tl \CJK@family
          \use:c { \@@_alternate_cs:n { reset / #1 } }
          \selectfont
        }
    \group_end:
  }
\prop_new:N \g_@@_reset_alternate_prop
\cs_generate_variant:Nn \clist_map_function:nN { x }
%    \end{macrocode}
% \end{macro}
%
% \begin{macro}{declarecharrange}
% 预先声明字符范围。
%    \begin{macrocode}
\keys_define:nn { ctex }
  {
    declarecharrange .code:n = \ctex_ltj_declare_char_range:x {#1} ,
    declarecharrange .value_required:n = true
  }
\cs_new_protected_nopar:Npn \ctex_ltj_declare_char_range:n #1
  { \clist_map_inline:nn {#1} { \@@_declare_char_range:nn ##1 } }
\cs_generate_variant:Nn \ctex_ltj_declare_char_range:n { x }
\cs_new_protected_nopar:Npn \@@_declare_char_range:nn #1#2
  { \use:x { \ctex_ltj_declare_char_range:nn { \tl_trim_spaces:n {#1} } } {#2} }
%    \end{macrocode}
% \end{macro}
%
% \begin{macro}[int]{\ctex_ltj_declare_char_range:nn,\g_@@_char_range_prop}
% |#1| 是名字,|#2| 是范围。
%    \begin{macrocode}
\cs_new_protected_nopar:Npn \ctex_ltj_declare_char_range:nn #1#2
  {
    \tl_clear:N \l_@@_char_range_tl
    \clist_map_function:nN {#2} \ctex_ltj_save_char_range:n
    \prop_gput:Nno \g_@@_char_range_prop {#1} { \l_@@_char_range_tl }
    \ctex_ltj_def_char_range_key:n {#1}
    \tl_clear:N \l_@@_char_range_tl
  }
\tl_new:N \l_@@_char_range_tl
\prop_new:N \g_@@_char_range_prop
%    \end{macrocode}
% \end{macro}
%
% \begin{macro}[int]{\ctex_ltj_save_char_range:n}
% 预先解释字符区间的意义。
%    \begin{macrocode}
\NewDocumentCommand \ctex_ltj_save_char_range:n
  { > { \SplitArgument { \c_one } { -> } } m }
  { \ctex_ltj_save_char_range:nn #1 }
\cs_new_protected_nopar:Npn \ctex_ltj_save_char_range:nn #1#2
  {
    \tl_put_right:Nx \l_@@_char_range_tl
      { {
          \IfNoValueTF {#2}
            { \int_eval:n {#1} , \int_eval:n {#1} }
            {
              \int_eval:n { \tl_if_blank:nTF {#1} { "80 } {#1} } ,
              \int_eval:n { \tl_if_blank:nTF {#2} { "10FFFF } {#2} }
            }
      } }
  }
%    \end{macrocode}
% \end{macro}
%
% \begin{macro}[int]{\ctex_ltj_def_char_range_key:n}
% 在字体设置选项中定义字符范围键。
%    \begin{macrocode}
\cs_new_protected_nopar:Npn \ctex_ltj_def_char_range_key:n #1
  {
    \keys_if_exist:nnF { ctex_ltj / fontspec } {#1}
      {
        \keys_define:nn { ctex_ltj / fontspec }
          { #1 .code:n = \ctex_ltj_char_range_key:nn {#1} { ##1 } }
      }
  }
%    \end{macrocode}
% \end{macro}
%
% \begin{macro}[int]{\ctex_ltj_char_range_key:nn}
% 如果字符范围键没有值,则只设置的这个字符范围内的替代字体。
%    \begin{macrocode}
\cs_new_protected:Npn \ctex_ltj_char_range_key:nn #1#2
  {
    \tl_if_blank:nTF {#2}
      { \tl_set:Nn \l_@@_char_range_clist {#1} }
      {
        \clist_if_empty:NT \l_@@_char_range_clist
          {
            \tl_set:Nn \l_@@_tmp_tl { {#1} }
            \@@_char_range_parse_feature:w #2 \q_stop
          }
      }
  }
%    \end{macrocode}
% \end{macro}
%
% \begin{macro}[int]{\@@_char_range_parse_feature:w}
% 可以使用加方括号的方式,通过文件名来调用字体。这容易与字体选项混淆。例如,需要
% 将 |[simsun.ttc]| 设置为 \opt{range} 的主字体,就需要使用
% |range={{[simsun.ttc]}}| 或者 |[]{[simsun.ttc]}|。下面的目的是,支持直接使用
% |[simsun.ttc]| 和 |[...][simsun.ttc]|。
%    \begin{macrocode}
\NewDocumentCommand \@@_char_range_parse_feature:w
  { +o o u { \q_stop } }
  {
    \exp_args:NNf \tl_put_right:Nn \l_@@_tmp_tl
      {
        \IfNoValueTF {#1} { {#3} }
          {
            \IfNoValueTF {#2}
              { \tl_if_blank:nTF {#3} { { [#1] } } { [ {#1} ] {#3} } }
              { [ {#1} ] { [#2] } }
          }
      }
    \seq_put_right:No \l_@@_alternate_seq { \l_@@_tmp_tl }
  }
%    \end{macrocode}
% \end{macro}
%
% \paragraph{其它设置}
%
% 在抄录环境中禁用 \opt{autospacing} 和 \opt{autoxspacing}。然而,\pkg{LuaTeX-ja}
% 还是会使 JAchar 自动折行。没有看到有简单的禁用折行的办法,可能需要设置所有的
% JAchar 的 \opt{prebreakpenalty} 或 \opt{postbreakpenalty} 为 \texttt{10000}:
% \begin{verbatim}
%   \directlua
%     {
%       luatexja.isglobal = tex.globaldefs > 0 and "global" or ""
%       for i = 0x80, 0x10FFFF do
%         if luatexja.charrange.jcr_table_main[i] > 0 and
%            luatexja.charrange.jcr_table_main[i] < 218 and
%            luatexja.charrange.is_japanese_char_curlist(i) then
%           luatexja.stack.set_stack_table(luatexja.stack_table_index.PRE + i, 10000)
%         end
%       end
%     }
% \end{verbatim}
%    \begin{macrocode}
\AtBeginDocument
  {
    \ctex_appto_cmd:NnnTF \verbatim@font
      { \char_set_catcode_letter:n { 64 } }
      { \CTEX@verbatim@font@hook }
      { }
      { \ctex_patch_failure:N \verbatim@font }
  }
\cs_new_protected_nopar:Npn \CTEX@verbatim@font@hook
  { \ltjsetparameter { autospacing = false , autoxspacing = false } }
%    \end{macrocode}
%
%    \begin{macrocode}
%<@@=>
%    \end{macrocode}
%
% \begin{macro}[int]{\@@italiccorr}
% \LaTeX{} 的倾斜校正也要重新定义。
%    \begin{macrocode}
\cs_set_eq:NN \@@italiccorr \/
%    \end{macrocode}
% \end{macro}
%
%    \begin{macrocode}
%<@@=ctex>
%    \end{macrocode}
%
%    \begin{macrocode}
%</luatex>
%    \end{macrocode}
%
% \subsubsection{\pkg{ctex-engine-uptex.def}}
%
%    \begin{macrocode}
%<*uptex|aptex>
%    \end{macrocode}
%
% \changes{v2.4}{2016/02/28}{初步支持 \upLaTeX。}
%
% 按 \pkg{CJK} 的命名习惯模拟部分命令,并设置默认字体。
%    \begin{macrocode}
\tl_set:Nn \CJKrmdefault { rm }
\tl_set:Nn \CJKsfdefault { sf }
\tl_set:Nn \CJKttdefault { tt }
\tl_set:Nn \CJKfamilydefault { \CJKrmdefault }
\ctex_preto_cmd:NnnTF \rmfamily { \ExplSyntaxOff }
  { \kanjifamily { \CJKrmdefault } }
  { }
  { \ctex_patch_failure:N \rmfamily }
\ctex_preto_cmd:NnnTF \sffamily { \ExplSyntaxOff }
  { \kanjifamily { \CJKsfdefault } }
  { }
  { \ctex_patch_failure:N \sffamily }
\ctex_preto_cmd:NnnTF \ttfamily { \ExplSyntaxOff }
  { \kanjifamily { \CJKttdefault } }
  { }
  { \ctex_patch_failure:N \ttfamily }
\ctex_preto_cmd:NnnTF \normalfont { \ExplSyntaxOff }
  { \kanjifamily { \CJKfamilydefault } }
  { \cs_set_eq:NN \reset@font \normalfont }
  { \ctex_patch_failure:N \normalfont }
\NewDocumentCommand \CJKfamily { m }
  {
    \kanjifamily { #1 }
    \selectfont
  }
\tl_set:Nn \kanjifamilydefault { \CJKfamilydefault }
%    \end{macrocode}
%
% 在导言区末尾更新 \tn{CJKfamilydefault}。
%    \begin{macrocode}
\ctex_at_end_preamble:n { \ctex_update_default_family: }
%    \end{macrocode}
%
% \begin{macro}[int]{\em}
% \changes{v2.4.2}{2016/05/15}{兼容 \upLaTeX{} 2016/05/07u00 的定义。}
% 取消 \upLaTeX{} 对 \tn{em} 使用 |\mcfamily|、|\gtfamily| 命令的重定义,恢复
% \LaTeXe{} 对 \tn{em} 的原始定义。如果用户已经重定义了 \tn{em},则新定义保持
% 不变。\upLaTeX{} 2016/05/07u00 的定义有所变化,这一行为可以由用户通过 \pkg{platexrelease}
% 包改变,需要分支处理。
%    \begin{macrocode}
\ctex_patch_cmd_once:NnnnTF \em
  { \ExplSyntaxOff }
  { \eminnershape \else \gtfamily \itshape }
  { \eminnershape \else \itshape }
  { }
  {
    \ctex_patch_cmd:Nnn \em
      { \mcfamily \upshape \else \gtfamily \itshape }
      { \eminnershape \else \itshape }
  }
\cs_set_nopar:Npn \eminnershape { \upshape }
%    \end{macrocode}
% \end{macro}
%
% \changes{v2.4}{2016/04/24}{正确设置 \upTeX{} 下字体命令。}
% \begin{macro}[int]{\ctex_set_upfamily:nnn}
% 将 NFSS 字体族 |#1| 设置为 JFM 字体名 |#2|,粗体形式字体名 |#3|。其中字体名
% 形如 |upzhserif|,不包括表示方向的后缀 |-h| 与 |-v|。粗体字体名为空时不设置该
% 字形。本命令不设置字体映射,需要复用已有的字体映射或另行设置。
%    \begin{macrocode}
\cs_new_protected_nopar:Npn \ctex_set_upfamily:nnn #1 #2 #3
  {
    \DeclareKanjiFamily{JY2}{#1}{}
    \DeclareKanjiFamily{JT2}{#1}{}
    \DeclareFontShape{JY2}{#1}{m}{n}{<->~ #2-h}{}
    \DeclareFontShape{JT2}{#1}{m}{n}{<->~ #2-v}{}
    \tl_if_empty:nF { #3 }
      {
        \DeclareFontShape{JY2}{#1}{bx}{n}{<->~ #3-h}{}
        \DeclareFontShape{JT2}{#1}{bx}{n}{<->~ #3-v}{}
      }
  }
\@onlypreamble \ctex_set_upfamily:nnn
%    \end{macrocode}
% \end{macro}
%
% \begin{macro}[int]{\ctex_set_upmap:nnn}
% 设置 \upTeX{} 字体映射。|#1| 是形如 |upserif| 的 PS TFM 字体名,不带表示粗体
% 的后缀 |b| 与表示排版方向的后缀 |-h| 与 |-v|。|#2| 与 |#3| 是普通与粗体的实际
% 字体名。
%    \begin{macrocode}
\cs_new_protected_nopar:Npn \ctex_set_upmap:nnn #1 #2 #3
  {
    \ctex_set_zhmap:n
      {
        \special{ pdf:mapline~ #1-h~   UniGB-UTF16-H~ #2 }
        \special{ pdf:mapline~ #1-v~   UniGB-UTF16-V~ #2 }
        \tl_if_empty:nF { #3 }
          {
            \special{ pdf:mapline~ #1b-h~  UniGB-UTF16-H~ #3 }
            \special{ pdf:mapline~ #1b-v~  UniGB-UTF16-V~ #3 }
          }
      }
  }
\@onlypreamble \ctex_set_upmap:nnn
%    \end{macrocode}
% \end{macro}
%
% \begin{macro}[int]{\ctex_set_upfonts:nnnnnn}
% 设置 \upTeX{} 基本字体映射,按 \pkg{zhmetrics-uptex} 的定义,依次设置衬线体
% 正、粗、意大利,无衬线体正、粗,等宽体正——共 6 种字体,并分横排及直排。
%    \begin{macrocode}
\cs_new_protected_nopar:Npn \ctex_set_upfonts:nnnnnn #1 #2 #3 #4 #5 #6
  {
    \ctex_set_upmap:nnn { upserif } { #1 } { #2 }
    \ctex_set_upmap:nnn { upserifit } { #3 } {}
    \ctex_set_upmap:nnn { upsans } { #4 } { #5 }
    \ctex_set_upmap:nnn { upmono } { #6 } {}
  }
\@onlypreamble \ctex_set_upfonts:nnnnnn
%    \end{macrocode}
% \end{macro}
%
% \changes{v2.4.7}{2016/12/27}{依赖 \pkg{pxeverysel} 宏包。}
%
% \pkg{everysel} 宏包(2011/10/28)未考虑 \upLaTeX{} 对 \tn{selectfont} 的修
% 改,需要引入 \pkg{pxeverysel} 宏包。
%    \begin{macrocode}
\RequirePackage { pxeverysel }
%    \end{macrocode}
%
%    \begin{macrocode}
%</uptex|aptex>
%    \end{macrocode}
%
% \subsubsection{调整 \tn{CJKfamilydefault}}
%
% \changes{v2.4}{2016/02/15}{正确更新 \pkg{CJK} 包的 \tn{CJKfamilydefault}。}
% \changes{v2.4.1}{2016/04/26}{正确更新 \upLaTeX{} 的 \tn{CJKfamilydefault}。}
%
% \begin{macro}[int]{\ctex_update_default_family:}
% 在导言区结束,如果 \tn{CJKfamilydefault} 没有被更改,则在此时根据西文字体的情况
% 更新 \tn{CJKfamilydefault}。\pkg{xeCJK} 已经有这个功能,不需要再调整。
%    \begin{macrocode}
%<*pdftex|luatex|uptex|aptex>
\cs_new_protected_nopar:Npn \ctex_update_default_family:
  {
    \tl_if_eq:NNT \CJKfamilydefault \l_@@_family_default_init_tl
      {
        \group_begin:
          \cs_set_eq:NN \@@_family_default_wrap:n \exp_not:n
          \tl_gset:Nx \CJKfamilydefault
            {
              \str_case:onF { \familydefault }
                {
                  { \rmdefault } { \exp_not:N \CJKrmdefault }
                  { \sfdefault } { \exp_not:N \CJKsfdefault }
                  { \ttdefault } { \exp_not:N \CJKttdefault }
                }
                { \CJKfamilydefault }
            }
        \group_end:
      }
%    \end{macrocode}
% 使用 \LuaLaTeX{} 时,自动调整得到的 \tn{CJKfamilydefault} 可能没有定义,需要
% 确认它的存在性。使用 \pkg{CJK} 宏包或 \upLaTeX{}
% 时,\texttt{C19rm}、\texttt{JY2rm} 等总是有定义的,不需要确认。
%    \begin{macrocode}
%<*luatex>
    \ctex_ltj_ensure_default_family:
%</luatex>
  }
%    \end{macrocode}
% \end{macro}
%
% \begin{macro}[int]{\l_@@_family_default_init_tl}
% 往 \tn{CJKfamilydefault} 中加入标志,用于判断它是否被更改。
%    \begin{macrocode}
\tl_new:N \l_@@_family_default_init_tl
\cs_new_eq:NN \@@_family_default_wrap:n \use:n
\tl_set:Nx \l_@@_family_default_init_tl
  {
    \exp_not:N \@@_family_default_wrap:n
      { \exp_not:o { \CJKfamilydefault } }
  }
\tl_gset_eq:NN \CJKfamilydefault \l_@@_family_default_init_tl
%</pdftex|luatex|uptex|aptex>
%    \end{macrocode}
% \end{macro}
%
% \subsubsection{操作系统的判断}
%
% \changes{v2.0}{2014/04/16}{自动检测操作系统,载入对应的字体配置。}
%
% \begin{macro}[int]{\ctex_detected_platform:}
% 在 \LuaTeX{} 下直接用调用 |os.name| 来判断。
%    \begin{macrocode}
%<*luatex>
\cs_new_protected_nopar:Npn \ctex_detected_platform:
  {
    \tl_gset:Nx \g_@@_fontset_tl
      {
        \lua_now_x:n
          {
            if ~ os.name == 'windows' then ~
              tex.sprint ( 'windows' )
            elseif ~ os.name == 'macosx' then ~
              tex.sprint ( 'mac' )
            else ~
              tex.sprint ( 'fandol' )
            end
          }
      }
  }
%</luatex>
%    \end{macrocode}
%
% \ApTeX{} 可以使用 \tn{ngostype} 来判断。
%    \begin{macrocode}
%<*aptex>
\cs_new_protected_nopar:Npn \ctex_detected_platform:
  {
    \tl_gset:Nx \g_@@_fontset_tl
      {
        \str_case:onF { \ngostype }
          {
            { Win32 }  { windows }
            { Win64 }  { windows }
            { Darwin } { mac }
          }
          { fandol }
      }
  }
%</aptex>
%    \end{macrocode}
%
% \pdfTeX{} 和 \XeTeX{} 下则依据 \file{/dev/null} 和 \file{nul:} 的存在性以及
% 文件系统的大小写敏感性来判断。Mac~OS~X 的大小写敏感性在安装时是可选的。为了
% 保险起见,这里的判断很繁琐,最多要进行 4 次文件操作!
%    \begin{macrocode}
%<*xetex|pdftex|uptex>
\cs_new_protected_nopar:Npn \ctex_detected_platform:
  {
    \file_if_exist:nTF { /dev/null }
      {
        \file_if_exist:nTF { nul: }
          {
            \file_if_exist:nTF { \c_@@_upper_case_file_str }
              { \ctex_if_macosx:TF { mac } { windows } }
              { \ctex_if_macosx:TF { mac } { fandol } }
          }
          { \ctex_if_macosx:TF { mac } { fandol } }
      }
      { \tl_gset:Nn \g_@@_fontset_tl { windows } }
  }
\str_const:Nx \c_@@_upper_case_file_str
  { \exp_args:No \str_upper_case:n { \g_file_current_name_tl } }
%    \end{macrocode}
% \end{macro}
%
% \begin{macro}[int]{\ctex_if_macosx:TF}
% \changes{v2.1}{2015/06/17}{改用 \file{/Library/Fonts/Songti.ttc} 为特征文件。}
% 以 \file{/Library/Fonts/Songti.ttc} 为特征文件判断 Mac~OS~X。
%    \begin{macrocode}
\cs_new_protected_nopar:Npn \ctex_if_macosx:TF #1#2
  {
    \file_if_exist:nTF { \c_@@_macosx_file_str }
      { \tl_gset:Nn \g_@@_fontset_tl {#1} }
      { \tl_gset:Nn \g_@@_fontset_tl {#2} }
  }
\str_const:Nn \c_@@_macosx_file_str { /Library/Fonts/Songti.ttc }
%</xetex|pdftex|uptex>
%    \end{macrocode}
% \end{macro}
%
% \subsubsection{\pkg{hyperref} 兼容性处理}
%
% \changes{v2.1}{2015/06/16}{不再设置 \pkg{hyperref} 宏包的 \opt{colorlinks} 选项。}
%
% 现在处理各个引擎下的 PDF 中文书签问题。根据编译引擎与文件编码的不
% 同,\pkg{ctex} 向 \pkg{hyperref} 传递适当的参数,完成中文书签的正确设置。用
% 户仍需要自己载入 \pkg{hyperref} 宏包。
%
% \begin{macro}[int]{\ctex_hypersetup:n}
% 如果已经载入 \pkg{hyperref} 宏包,则直接使用其定义设置选项;否则
% \cs{ctex_hypersetup:n} 的效果与 \tn{PassOptionsToPackage} 一致,只传递宏包参
% 数。如果用户不载入 \pkg{hyperref} 宏包,相关参数即被丢弃。
%    \begin{macrocode}
%<*class|ctex>
\@ifpackageloaded { hyperref }
  {
    \cs_new_protected_nopar:Npn \ctex_hypersetup:n #1
      { \hypersetup {#1} }
  }
  {
    \cs_new_protected_nopar:Npn \ctex_hypersetup:n #1
      { \PassOptionsToPackage {#1} { hyperref } }
  }
%</class|ctex>
%    \end{macrocode}
% \end{macro}
%
% 在 \pdfTeX{} 下使用 \texttt{GBK} 编码,\dvipdfmx{} 驱动可以直接用它的
% \tn{special} 命令,其它模式用 \pkg{xCJK2uni} 宏包处理。使用 \texttt{UTF-8} 编
% 码时,\pkg{CJKutf8} 已经处理了书签问题,但仍需要设置 \opt{pdfencoding} 为
% \opt{unicode},目的是在书签的开头写入 BOM (|\376\377|),提示这是
% \texttt{UTF-16BE} 字节流。
%    \begin{macrocode}
%<*pdftex>
\ctex_hypersetup:n { driverfallback = dvipdfmx }
\str_if_eq:onTF { \l_@@_encoding_tl } { GBK }
  {
    \ctex_hypersetup:n { CJKbookmarks = true }
    \sys_if_output_pdf:TF
      { \ctex_at_end_package:nn { hyperref } { \RequirePackage { xCJK2uni } } }
      {
        \ctex_at_end_package:nn { hyperref }
          {
            \str_if_eq:onTF { \Hy@driver } { hdvipdfm }
              {
                \AtBeginShipoutFirst
                  { \special { pdf:tounicode~GBK-EUC-UCS2 } }
              }
              { \RequirePackage { xCJK2uni } }
          }
      }
  }
  { \ctex_hypersetup:n { pdfencoding = unicode } }
%</pdftex>
%    \end{macrocode}
% 在 \XeTeX{} 下,\pkg{hyperref} 在处理带有非 ASCII 字符和 |\%| 的书签时有问题^^A
% \footnote{\url{https://github.com/CTeX-org/ctex-kit/issues/39}}。
% 事实上,\pkg{hyperref} 在驱动文件 \file{hxetex.def} 中设置了
% \tn{Hy@unicodetrue},从而书签总是会被 \tn{HyPsd@ConvertToUnicode} 转化成
% \texttt{UTF-16BE} 编码的形式(抄录自 \tn{pdfstringdef}的定义):
% \begin{verbatim}
%       \ifHy@unicode
%         \HyPsd@ConvertToUnicode#1%
%         \ifx\HyPsd@pdfencoding\HyPsd@pdfencoding@auto
%           \ltx@IfUndefined{StringEncodingConvertTest}{%
%           }{%
%             \EdefUnescapeString\HyPsd@temp#1%
%             \ifxetex
%               \let\HyPsd@UnescapedString\HyPsd@temp
%               \StringEncodingConvertTest\HyPsd@temp\HyPsd@temp
%                                         {utf16be}{ascii-print}{%
%                 \EdefEscapeString\HyPsd@temp\HyPsd@temp
%                 \global\let#1\HyPsd@temp
%                 \HyPsd@EscapeTeX#1%
%                 \Hy@unicodefalse
%               }{%
%                \HyPsd@ToBigChars#1%
%               }%
% \end{verbatim}
% 通过宏包选项 \opt{pdfencoding=unicode} 设置 \tn{HyPsd@pdfencoding} 为
% \opt{unicode},可以避免随后再将书签从 \texttt{UTF-16BE} 字节流转化回正常
% 字符(其中使用的 \tn{HyPsd@ToBigChars} 没有考虑书签中含有 |\%| 的情况)。
% Heiko Oberdiek 在 \file{README} 中说明了将书签转化回正常字符的意图:避免
% XDVIPDFMX 的警告^^A
% \footnote{\url{http://project.ktug.org/dvipdfmx/mailman/dvipdfmx/2009-December/000153.html}}:
% \begin{verbatim}
%   ** WARNING ** Failed to convert input string to UTF16...
% \end{verbatim}
% \XeTeX{} 的维护者 Khaled Hosny 已经注意到了这个问题^^A
% \footnote{\url{http://tug.org/pipermail/tex-live/2013-December/034613.html}}。
% 需要注意的是,\file{hxetex.def} 重载了宏包选项 \opt{unicode},目的是不能设置
% 它为 \opt{false},但也导致它不会改变 \tn{HyPsd@pdfencoding}。如果
% \pkg{hyperref} 先于 \CTeX{} 被载入,那么 \opt{unicode} 选项是没有意义的。
% 因此要通过意义相同但在 \XeTeX{} 下更保险的 \opt{pdfencoding} 选项来设置。
% 为了与 \XeTeX 下的行为一致(使用 \tn{HyPsd@LoadUnicode} 载入 \file{puenc.def}),
% 在 \LuaTeX{} 下也启用这个选项。
%    \begin{macrocode}
%<*xetex|luatex>
\ctex_hypersetup:n { pdfencoding = unicode }
%</xetex|luatex>
%    \end{macrocode}
%
% 我们假定 \upTeX{} 使用 \dvipdfmx{} 驱动输出,于是使用与 \pdfTeX{} 类似的设
% 置。注意 \upTeX{} 需要使用 UTF8-UTF16 的编码转换。
%    \begin{macrocode}
%<*uptex|aptex>
\ctex_hypersetup:n { driverfallback = dvipdfmx }
\ctex_at_end_package:nn { hyperref }
  { \AtBeginShipoutFirst { \special { pdf:tounicode~ UTF8-UTF16 } } }
%</uptex|aptex>
%    \end{macrocode}
%
%    \begin{macrocode}
%<*pdftex|xetex|luatex|uptex|aptex>
%    \end{macrocode}
%
% \subsubsection{\pkg{CJKfntef}、\pkg{xeCJKfntef} 相关设置}
%
% \CTeX{} 宏集对 \pdfTeX{} 与 \XeTeX{} 引擎,分别载入 \pkg{CJKfntef} 或
% \pkg{xeCJKfntef} 宏包,并关闭宏包默认的彩色等多余格式。
%
% \changes{v2.0}{2015/03/25}{默认关闭 \pkg{CJKfntef} 或 \pkg{xeCJKfntef} 的彩
% 色设置。}
% \changes{v2.0}{2015/03/25}{\tn{CTEXunderdot}, \tn{CTEXunderline},
% \tn{CTEXunderdblline}, \tn{CTEXunderwave}, \tn{CTEXsout}, \tn{CTEXxout} 是过
% 时命令;\env{CTEXfilltwosides} 是过时环境。}
%
% 载入 \pkg{CJKfntef} 或 \pkg{xeCJKfntef} 并做适当格式设置。有关 |\CTEX| 开头
% 的宏定义是过时命令,仅做兼容性保留。
%
%    \begin{macrocode}
%<*pdftex>
\RequirePackage { CJKfntef }
\normalem
\cs_new_protected_nopar:Npn \@@_clear_fntef_color:n #1
  { \tl_clear:c { CJK#1color } }
%</pdftex>
%<*xetex>
\RequirePackage { xeCJKfntef }
\@ifpackagelater { xeCJKfntef } { 2014/11/04 }
  {
    \cs_new_protected_nopar:Npn \@@_clear_fntef_color:n #1
      { \xeCJKsetup { #1 / format = { } } }
  }
  {
    \cs_new_protected_nopar:Npn \@@_clear_fntef_color:n #1
      { \tl_clear:c { CJK#1color } }
  }
%</xetex>
%<*luatex|uptex|aptex>
\msg_new:nnn { ctex } { fntef-not-available }
%<luatex>  { Functions~ of~ `CJKfntef'~ is~ not~ available~ in~ LuaLaTeX. }
%<uptex>  { Functions~ of~ `CJKfntef'~ is~ not~ available~ in~ upLaTeX. }
%<aptex>  { Functions~ of~ `CJKfntef'~ is~ not~ available~ in~ ApLaTeX. }
\msg_warning:nn { ctex } { fntef-not-available }
%</luatex|uptex|aptex>
\clist_map_inline:nn
  { underdot , underline , underdblline , underwave , sout , xout }
%<*pdftex|xetex>
  {
    \@@_clear_fntef_color:n {#1}
    \cs_new_protected_nopar:cpx { CTEX#1 }
      {
        \msg_warning:nnnn { ctex } { deprecated-command } { \exp_not:c { CTEX#1 } }
          { You~ can~ use~ the~ command~ with~ prefix~ \exp_not:N \CJK~ instead. }
        \exp_not:c { CJK#1 }
      }
  }
\cs_new_protected_nopar:Npn { \CTEXfilltwosides }
  {
    \msg_warning:nnnn { ctex } { deprecated-environment } { CTEXfilltwosides }
      { You~ can~ use~ `CJKfilltwosides'~ environment~ instead. }
    \CJKfilltwosides
  }
\cs_new_protected_nopar:Npn { \endCTEXfilltwosides } { \endCJKfilltwosides }
%</pdftex|xetex>
%<*luatex|uptex|aptex>
  { \cs_new_eq:cN { CTEX#1 } \use:n }
\cs_new_eq:NN \CTEXfilltwosides \use_none:n
\cs_new_eq:NN \endCTEXfilltwosides \prg_do_nothing:
%</luatex|uptex|aptex>
%<*pdftex>
\clist_map_inline:nn
  {
    underdotbasesep ,   underdotsep ,     underlinebasesep ,
    underlinesep ,      underdbllinesep , underdbllinebasesep ,
    underwavebasesep ,  underwavesep ,    southeight ,
    underdotcolor ,     underwavecolor ,  underlinecolor ,
    underdbllinecolor , soutcolor ,       xoutcolor
  }
  {
    \cs_new_eq:cc { CTEX#1 } { CJK#1 }
    \cs_set_nopar:cpx { CJK#1 } { \exp_not:c { CTEX#1 } }
  }
%</pdftex>
%    \end{macrocode}
%
% \subsubsection{\tn{ccwd} 的更新}
%
% \begin{macro}[int]{\ctex_update_ccwd:,\ccwd}
% \changes{v2.4.1}{2016/04/29}{正确设置 \upTeX{} 下的 \tn{ccwd}。}
%    \begin{macrocode}
\cs_new_protected_nopar:Npn \ctex_update_ccwd:
%<*pdftex|xetex>
  {
    \hbox_set:Nn \l_@@_tmp_box { \CJKglue }
    \dim_set:Nn \ccwd { \box_wd:N \l_@@_tmp_box + \f@size \p@ }
  }
%</pdftex|xetex>
%<*luatex>
  { \skip_set:Nn \ccwd { \ltjgetparameter { kanjiskip } + \zw } }
%</luatex>
%<*uptex|aptex>
  { \skip_set:Nn \ccwd { 1zw + \ptex_kanjiskip:D } }
%</uptex|aptex>
\dim_new:N \ccwd
%    \end{macrocode}
% \end{macro}
%
% \begin{macro}[int]{\ctex_update_ccglue:}
% 更新字间距。
%    \begin{macrocode}
\cs_new_protected_nopar:Npn \ctex_update_ccglue:
%<*pdftex>
  {
    \cs_set_protected_nopar:Npn \CJKglue
      { \skip_horizontal:N \l_@@_ccglue_skip }
  }
%</pdftex>
%<*xetex>
  { \xeCJKsetup { CJKglue = { \skip_horizontal:N \l_@@_ccglue_skip } } }
%</xetex>
%<*luatex>
  { \ltjsetkanjiskip \l_@@_ccglue_skip }
%</luatex>
%<*uptex|aptex>
  { \skip_set_eq:NN \ptex_kanjiskip:D \l_@@_ccglue_skip }
%</uptex|aptex>
\skip_new:N \l_@@_ccglue_skip
%    \end{macrocode}
% \end{macro}
%
% \begin{macro}[int,pTF]{\ctex_if_ccglue_touched:}
% 检查用户是否修改过汉字间距。
%    \begin{macrocode}
\prg_new_conditional:Npnn \ctex_if_ccglue_touched: { TF }
  {
%<*pdftex|xetex>
    \if_meaning:w \CJKglue \@@_ccglue:
      \prg_return_false: \else: \prg_return_true: \fi:
%</pdftex|xetex>
%<*luatex>
    \skip_if_eq:nnTF { \l_@@_ccglue_skip } { \ltjgetparameter { kanjiskip } }
      { \prg_return_false: } { \prg_return_true: }
%</luatex>
%<*uptex|aptex>
    \skip_if_eq:nnTF { \l_@@_ccglue_skip } { \ptex_kanjiskip:D }
      { \prg_return_false: } { \prg_return_true: }
%</uptex|aptex>
  }
%    \end{macrocode}
% 注意下面的标记不能用 \verb"%<pdftex|xetex>",它会导致旧版本的 \pkg{l3docstrip}
% 不能替换 \texttt{@@}。
%    \begin{macrocode}
%<*pdftex|xetex>
\ctex_at_end:n { \cs_new_eq:NN \@@_ccglue: \CJKglue }
%</pdftex|xetex>
%    \end{macrocode}
% \end{macro}
%
% \begin{macro}[int]{\ctex_update_em_unit:}
% 将当前汉字的宽度保存到 \tn{ccwd} 中备用。不采用 \texttt{1em},因为这时的
% \texttt{1em} 实际上来自西文字体的信息,未必等于汉字的宽度,这似乎在传统的
% \file{.tfm} 字体上表现更明显。在 \pdfTeX{} 和 \XeTeX{} 下,直接使用 |\f@size\p@|
% 作为汉字的宽度,这应该对大多数汉字字体都成立,但不适用于诸如“方正兰亭黑长”之类
% 的特殊字体。在 \XeTeX{} 可以用 \tn{fontcharwd} 来改进。而在 \pdfTeX{} 下,若使用
% \pkg{zhmetrics} 技术,所有的汉字共享同一个 \file{.tfm},\tn{fontcharwd} 也就没有
% 意义。在 \LuaTeX{} 下,\pkg{LuaTeX-ja} 总是按照 JFM 中的设置输出汉字的宽度,可以
% 直接用 \tn{zw} 作为汉字宽度。\upTeX{} 可以直接使用原生的长度单位 |zw|。
%    \begin{macrocode}
\cs_new_protected_nopar:Npn \ctex_update_em_unit:
%<pdftex|xetex>  { \dim_set:Nn \ccwd { \f@size \p@ } }
%<luatex>  { \dim_set:Nn \ccwd { \zw } }
%<uptex|aptex>  { \dim_set:Nn \ccwd { 1zw } }
%    \end{macrocode}
% \end{macro}
%
% \subsubsection{其它}
%
% \begin{macro}[int]{\ctex_add_to_selectfont:n,\CTEX@selectfont@hook}
% \tn{EverySelectfont} 直到文档开始时才有效。为了 \tn{ccwd} 和 \pkg{LuaTeX-ja} 的
% 字体设置在导言区也可用,我们还需要在这里手工修改 \tn{selectfont}。\pkg{everysel}
% 宏包会用 \tn{CheckCommand} 来检查 \tn{selectfont} 是否为标准定义。我们修改了
% \tn{selectfont},所以会给出一个警告。为了消除这个警告,在它检查之前,还原本来
% 定义。\pkg{pxeverysel} 宏包取消了检查,但也需要恢复定义,避免重复使用钩子。
%    \begin{macrocode}
\cs_new_protected:Npn \ctex_add_to_selectfont:n #1
  {
    \cs_set_protected_nopar:Npx \CTEX@selectfont@hook
      { \exp_not:o { \CTEX@selectfont@hook #1 } }
  }
\cs_new_eq:NN \CTEX@selectfont@hook \prg_do_nothing:
\if_cs_exist:N \@EverySelectfont@Init
  \group_begin:
    \cs_set:Npn \@@_tmp:N #1
      {
        \tl_set:Nn \l_@@_tmp_tl {#1}
        \cs_new_eq:NN \CTEX@selectfont@save #1
        \cs_new_protected_nopar:Npn \@@_restore_selectfont:
          {
            \tl_put_left:Nn \@EverySelectfont@Init
              { \let #1 \CTEX@selectfont@save }
            \cs_undefine:N \@@_restore_selectfont:
          }
      }
    \ctex_parse_name:NN \@@_tmp:N \selectfont
  \exp_last_unbraced:NNo \group_end:
  \ctex_preto_cmd:NnnTF { \l_@@_tmp_tl } { \ExplSyntaxOff }
    { \CTEX@selectfont@hook }
    { \@@_restore_selectfont: }
    { \ctex_patch_failure:N \selectfont }
\fi:
%    \end{macrocode}
% \end{macro}
%
% \tn{CJK@plane} 有定义,说明处于 \pkg{CJK} 宏包的 \tn{CJKsymbol} 之内,不必使用钩子。
%    \begin{macrocode}
%<*pdftex>
\EverySelectfont { \cs_if_exist:NF \CJK@plane { \CTEX@selectfont@hook } }
%</pdftex>
%<*xetex|luatex|uptex|aptex>
\EverySelectfont { \CTEX@selectfont@hook }
%</xetex|luatex|uptex|aptex>
%    \end{macrocode}
%
% Attribute 寄存器 \tn{ltj@curjfnt} 的初始值是 $-1$,必须把它设置为一个有效的
% \texttt{font.id},否则编译时会直接退出。
%    \begin{macrocode}
%<*luatex>
\ctex_add_to_selectfont:n
  {
    \ctex_ltj_select_font:
    \ctex_ltj_select_alternate_font:
  }
\tl_set:Nn \CJK@family { song } \selectfont
\tl_clear:N \CJK@family
%</luatex>
%    \end{macrocode}
%
% \changes{v2.4.1}{2016/05/01}{随字体更新 \upTeX{} 的 \tn{xkanjiskip}。}
%
% \begin{macro}[int]{\ctex_update_xkanjiskip:,\l_@@_xkanjiskip_skip}
% \upTeX{} 和 \pkg{LuaTeX-ja} 对 \tn{xkanjiskip} 都是即时赋值。单位 \opt{zw} 与字体相关,因此
% 需要每次 \tn{selectfont} 的时候更新一次 \tn{xkanjiskip}。如果用户设置过
% \tn{xkanjiskip},就不更新。注意,同 \TeX{} 的 \tn{baselineskip} 一样,如果在
% 一个段落内多次设置了 \tn{kanjiskip} 或 \tn{xkanjiskip},只有最后的设置会影响
% 全段。
%    \begin{macrocode}
%<*luatex|uptex|aptex>
\cs_new_protected_nopar:Npn \ctex_update_xkanjiskip:
  {
    \skip_if_eq:nnT
%<luatex>      { \ltjgetparameter { xkanjiskip } } { \l_@@_xkanjiskip_skip }
%<uptex|aptex>      { \ptex_xkanjiskip:D } { \l_@@_xkanjiskip_skip }
      {
        \skip_set:Nn \l_@@_xkanjiskip_skip { \l_@@_xkanjiskip_tl }
%<luatex>        \ltjsetxkanjiskip \l_@@_xkanjiskip_skip
%<uptex|aptex>        \skip_set_eq:NN \ptex_xkanjiskip:D \l_@@_xkanjiskip_skip
      }
  }
\tl_new:N \l_@@_xkanjiskip_tl
\tl_set:Nn \l_@@_xkanjiskip_tl
%<luatex>  { .25\zw plus 1pt minus 1pt }
%<uptex|aptex>  { .25zw plus 1pt minus 1pt }
\skip_new:N \l_@@_xkanjiskip_skip
\skip_set:Nn \l_@@_xkanjiskip_skip
%<luatex>  { \ltjgetparameter { xkanjiskip } }
%<uptex|aptex>  { \ptex_xkanjiskip:D }
%    \end{macrocode}
% \end{macro}
%
%    \begin{macrocode}
\ctex_add_to_selectfont:n { \ctex_update_xkanjiskip: }
%</luatex|uptex|aptex>
%    \end{macrocode}
%
% \begin{macro}{space}
% 在导言区或正文中设置忽略空格方式。
% \pdfTeX{} 和 \XeTeX{} 下初始设置为 \opt{auto},\LuaTeX{}、\upTeX{} 下是无效
% 选项。
%    \begin{macrocode}
\keys_define:nn { ctex }
  {
%<*pdftex|xetex>
    space .choice: ,
    space / true  .code:n =
%<pdftex>      { \ctex_ignorespaces_case:N \prg_do_nothing: } ,
%<xetex>      { \xeCJKsetup { CJKspace = true } } ,
    space / auto  .code:n =
%<pdftex>      { \ctex_ignorespaces_case:N \ctex_auto_ignorespaces: } ,
%<xetex>      { \xeCJKsetup { CJKspace = false } } ,
    space / false .code:n =
%<pdftex>      { \ctex_ignorespaces_case:N \tex_ignorespaces:D } ,
%<xetex>      { \xeCJKsetup { CJKspace = false } } ,
    space .default:n = { true } ,
    space .initial:n = { auto }
%</pdftex|xetex>
%<*luatex|uptex|aptex>
    space .code:n =
      { \msg_warning:nn { ctex } { invalid-option } }
%</luatex|uptex|aptex>
  }
%    \end{macrocode}
% \end{macro}
%
% \begin{macro}{punct}
% 在导言区或正文中设置标点符号输出格式。\pkg{LuaTeX-ja} 设置的是字体的默认 \texttt{JFM},
% 只会影响到之后设置的字体。\upTeX{} 暂时无效。
%    \begin{macrocode}
\keys_define:nn { ctex }
  {
    punct .code:n =
      {
        \tl_set:Nx \l_@@_punct_tl { #1 }
%<pdftex>        \punctstyle { \l_@@_punct_tl }
%<xetex>        \xeCJKsetup { PunctStyle = \l_@@_punct_tl }
%<luatex>        \ctex_mono_jfm:o { \l_@@_punct_tl }
%<uptex|aptex>        \msg_warning:nn { ctex } { invalid-option }
      } ,
    punct .default:n = { quanjiao } ,
  }
%    \end{macrocode}
% \end{macro}
%
% \XeLaTeX{}、\LuaLaTeX{} 和 \upLaTeX{} 总是使用 \texttt{UTF8} 编码。
%    \begin{macrocode}
%<*xetex|luatex|uptex|aptex>
\tl_set:Nn \l_@@_encoding_tl { UTF8 }
%</xetex|luatex|uptex|aptex>
%    \end{macrocode}
%
%    \begin{macrocode}
%</pdftex|xetex|luatex|uptex|aptex>
%<*class|ctex>
%    \end{macrocode}
%
% \subsubsection{载入引擎定义文件}
%
% 最后载入各个编译引擎的定义文件。
%    \begin{macrocode}
\ctex_file_input:n { \c_@@_engine_file_str }
%    \end{macrocode}
%
% \subsection{用户设置接口}
%
% \changes{v2.0}{2014/03/18}{新增统一设置接口 \tn{ctexset}。}
%
% \begin{macro}{\ctexset}
%    \begin{macrocode}
%<*class|ctex|ctexheading>
\NewDocumentCommand \ctexset { } { \keys_set:nn { ctex } }
%</class|ctex|ctexheading>
%    \end{macrocode}
% \end{macro}
%
% \changes{v2.0}{2015/03/21}{\tn{CTEXsetup}, \tn{CTEXoptions} 是过时命令。}
% \begin{macro}{\CTEXsetup,\CTEXoptions}
% 过时命令。出于历史原因,\tn{CTEXoptions} 需要在 \pkg{pxeverysel} 宏包之后定义。
%    \begin{macrocode}
%<*class|ctex>
\NewDocumentCommand \CTEXsetup { +o > { \TrimSpaces } m }
  {
    \msg_warning:nnnn { ctex } { deprecated-command } { \CTEXsetup }
      { \ctexset~ {~ #2~ =~ {~ #1~ }~ }~ is~ set. }
    \IfNoValueF {#1} { \keys_set:nn { ctex / #2 } {#1} }
  }
\NewDocumentCommand \CTEXoptions { +o }
  {
    \msg_warning:nnnn { ctex } { deprecated-command } { \CTEXoptions }
      { \ctexset~ {~ #1~ }~ is~ set. }
    \IfNoValueF {#1} { \keys_set:nn { ctex } {#1} }
  }
%</class|ctex>
%    \end{macrocode}
% \end{macro}
%
% \subsection{字距与缩进}
%
% \begin{macro}{autoindent}
% \opt{autoindent} 也是可以用在正文中的选项,意义与宏包选项 |option/autoindent| 相同。
%    \begin{macrocode}
\keys_define:nn { ctex }
  {
    autoindent .choice: ,
    autoindent .default:n = { true } ,
    autoindent / true    .code:n =
      {
        \tl_set:Nn \l_@@_autoindent_tl { 2 \ccwd }
        \ctex_select_size:
      } ,
    autoindent / false   .code:n =
      { \tl_clear:N \l_@@_autoindent_tl } ,
    autoindent / unknown .code:n =
      {
        \ctex_set_default_ccwd:Nn \l_@@_autoindent_tl {#1}
        \ctex_select_size:
      }
  }
%    \end{macrocode}
% \end{macro}
%
% \begin{macro}{\CTEXsetfont}
% 无论字体大小是否变化都更新相关信息。
%    \begin{macrocode}
\NewDocumentCommand \CTEXsetfont { } { \ctex_select_size: }
\cs_new_protected_nopar:Npn \ctex_select_size:
  { \cs_if_free:NTF \size@update { \ctex_update_size: } { \selectfont } }
%    \end{macrocode}
% \end{macro}
%
% \begin{macro}[int]{\ctex_update_size:}
% 在字号变化时更新 \tn{ccwd}、\tn{parindent} 和汉字间距。字距为零则恢复正常设置。
%    \begin{macrocode}
\cs_new_protected_nopar:Npn \ctex_update_size:
  {
    \tl_if_eq:NNTF \l_@@_ziju_tl \c_@@_zero_tl
      {
        \ctex_update_stretch:
        \ctex_update_parindent:
      }
      { \ctex_update_ziju: }
  }
\tl_const:Nx \c_@@_zero_tl { \fp_use:N \c_zero_fp }
\tl_new:N \l_@@_ziju_tl
\tl_set_eq:NN \l_@@_ziju_tl \c_@@_zero_tl
%    \end{macrocode}
% 在 \tn{selectfont} 中,若 \tn{size@update} 为 \tn{relax},说明字体大小没有变化,
% 我们也就不用更新相关参数。
%    \begin{macrocode}
\ctex_add_to_selectfont:n
  { \cs_if_free:NF \size@update { \ctex_update_size: } }
%    \end{macrocode}
% \end{macro}
%
% \changes{v2.0}{2014/03/26}{新增 \opt{linestretch} 选项。}
%
% \begin{macro}{linestretch}
% 若行宽不是汉字宽度的整数倍,自然要求伸展它们之间的差。这里设置的是在此基础上的
% 额外伸展量。初始化为一个汉字的宽度。若设置为 \tn{maxdimen},则禁用此功能。
% 参数的默认单位是汉字的宽度 \tn{ccwd}。
%    \begin{macrocode}
\keys_define:nn { ctex }
  {
    linestretch .code:n =
      {
        \ctex_set_default_ccwd:Nn \l_@@_line_stretch_tl {#1}
        \ctex_select_size:
      } ,
    linestretch .value_required:n = true
  }
\tl_new:N \l_@@_line_stretch_tl
\tl_set:Nn \l_@@_line_stretch_tl { \ccwd }
%    \end{macrocode}
% \end{macro}
%
% \begin{macro}[int]{\ctex_update_stretch:}
% 首先计算一行上汉字的字数,\tn{CJKglue} 相当于将 \tn{linewidth} 与汉字总宽度之差
% 均匀地填充到汉字之间。\hologo{eTeX} 的除法是四舍五入,而我们这里应该用截断。由于
% 没有可展性的要求,直接用原语 \cs{tex_divide:D} 要比 \cs{int_div_truncate:nn}
% 快一些。下面的算法还兼顾到了 \tn{linewidth} 不为汉字字宽的整数倍的情况。
% 若用户禁用 \opt{linestretch} 并且修改过 \tn{CJKglue},则只更新
% \tn{ccwd},否则设置伸展量为 $0.08$ 倍 \tn{baselineskip}。注意 \pkg{everysel} 的
% 钩子位于 \tn{size@update} 之前,\tn{baselineskip} 还未更新,不能直接使用它。
%    \begin{macrocode}
\cs_new_protected_nopar:Npn \ctex_update_stretch:
  {
    \ctex_update_em_unit:
    \dim_set:Nn \l_@@_tmp_dim { \l_@@_line_stretch_tl }
    \dim_compare:nNnTF \l_@@_tmp_dim = \c_max_dim
      {
        \ctex_if_ccglue_touched:TF
          { \ctex_update_ccwd: }
          {
            \dim_set:Nn \l_@@_tmp_dim
              { \baselinestretch \etex_glueexpr:D \f@baselineskip \scan_stop: }
            \skip_set:Nn \l_@@_ccglue_skip
              { \c_zero_dim plus .08 \l_@@_tmp_dim }
            \ctex_update_ccglue:
          }
      }
      {
        \int_set:Nn \l_@@_tmp_int
          { \etex_dimexpr:D \linewidth - \ccwd - \l_@@_tmp_dim \scan_stop: }
        \tex_divide:D \l_@@_tmp_int \ccwd
        \int_compare:nNnTF \l_@@_tmp_int > \c_zero
          {
            \skip_set:Nn \l_@@_ccglue_skip
              {
                \c_zero_dim plus \dim_eval:n
                  {
                    ( \linewidth - \ccwd - \l_@@_tmp_int \ccwd ) /
                    \l_@@_tmp_int
                  }
              }
          }
          { \skip_zero:N \l_@@_ccglue_skip }
        \ctex_update_ccglue:
      }
  }
%    \end{macrocode}
% \end{macro}
%
% \begin{macro}[int]{\ctex_update_parindent:}
% 更新段落首行缩进。此函数在字号变化时调用。
%    \begin{macrocode}
\cs_new_protected_nopar:Npn \ctex_update_parindent:
  {
    \tl_if_empty:NF \l_@@_autoindent_tl
      {
        \dim_compare:nNnF \parindent = \c_zero_dim
          { \dim_set:Nn \parindent { \l_@@_autoindent_tl } }
      }
  }
%    \end{macrocode}
% \end{macro}
%
% \begin{macro}{\ziju}
% 若参数为 $0$,则恢复正常间距。
%    \begin{macrocode}
\NewDocumentCommand \ziju { m }
  { \exp_args:Nx \ctex_ziju:n {#1} \tex_ignorespaces:D }
\cs_new_protected_nopar:Npn \ctex_ziju:n #1
  {
    \tl_set:Nx \l_@@_ziju_tl { \fp_eval:n {#1} }
    \ctex_select_size:
  }
%    \end{macrocode}
% \end{macro}
%
% \begin{macro}[int]{\ctex_update_ziju:}
% 更新字距。若字距不大于 $-1$,即 \tn{ccwd} 为非正值,则不计算伸缩值。
% 否则,首先假定汉字的宽度为正常宽度加上字距,看一行上能正常放下多少个汉字。
%    \begin{macrocode}
\cs_new_protected_nopar:Npn \ctex_update_ziju:
  {
    \ctex_update_em_unit:
    \dim_set:Nn \l_@@_ziju_dim { \l_@@_ziju_tl \ccwd }
    \dim_add:Nn \ccwd { \l_@@_ziju_dim }
    \dim_compare:nNnTF \ccwd > \c_zero_dim
%    \end{macrocode}
% 伸展量保证行内的剩余空白能够被均匀地填充到汉字之间,收缩的最大限度是让当前行
% 还能够再挤下一个汉字并且不会出现负间距。由 \TeX{} 决定伸展还是收缩。
%    \begin{macrocode}
      {
        \dim_set:Nn \l_@@_tmp_dim
          { \linewidth - \ccwd + \l_@@_ziju_dim }
        \int_set:Nn \l_@@_tmp_int { \l_@@_tmp_dim }
        \tex_divide:D \l_@@_tmp_int \ccwd
        \dim_sub:Nn \l_@@_tmp_dim { \l_@@_tmp_int \ccwd }
%    \end{macrocode}
% 由于 \tn{parindent} 是一个固定值,并不参与伸缩,容易导致第一行出现坏盒子。
% 我们在这里将字数减去 $2$,以此放大伸缩值。
%    \begin{macrocode}
        \dim_compare:nNnF \parindent = \c_zero_dim
          {
            \int_compare:nNnF \l_@@_tmp_int < \c_three
              { \int_sub:Nn \l_@@_tmp_int { \c_two } }
          }
        \skip_set:Nn \l_@@_ccglue_skip
          {
            \l_@@_ziju_dim
            plus  \dim_eval:n { \l_@@_tmp_dim / \l_@@_tmp_int }
            minus \dim_min:nn { \dim_abs:n { \l_@@_ziju_dim } }
              { ( \ccwd - \l_@@_tmp_dim ) / ( \l_@@_tmp_int + \c_one ) }
          }
      }
      { \skip_set:Nn \l_@@_ccglue_skip { \l_@@_ziju_dim } }
    \ctex_update_ccglue:
%    \end{macrocode}
% 字距设置得比较大时,为了尽量保证段首缩进能够与下一行对齐,应该需要相应地加上
% 或者减去伸缩值。但是这里并不清楚 \TeX{} 是伸展还是收缩,之前以“当前行是否还
% 放得下一个汉字”为标准加上或减去伸缩值的做法也未必与实际结果一致,所以只好还
% 是设置为 |2\ccwd|。
%    \begin{macrocode}
    \ctex_update_parindent:
  }
\dim_new:N \l_@@_ziju_dim
%    \end{macrocode}
% \end{macro}
%
% \changes{v2.0}{2015/03/21}{\tn{CTEXindent}, \tn{CTEXnoindent} 是过时命令。}
% \begin{macro}{\CTEXindent,\CTEXnoindent}
% 过时命令。
%    \begin{macrocode}
\NewDocumentCommand \CTEXindent { }
  {
    \msg_warning:nnnn { ctex } { deprecated-command } { \CTEXindent }
      { \parindent is~ set~ to~ 2\ccwd. }
    \ctex_update_ccwd: \dim_set:Nn \parindent { 2 \ccwd }
  }
\NewDocumentCommand \CTEXnoindent { }
  {
    \msg_warning:nnnn { ctex } { deprecated-command } { \CTEXnoindent }
      { \parindent is~ set~ to~ 0pt. }
    \dim_zero:N \parindent
  }
%    \end{macrocode}
% \end{macro}
%
% \subsection{中文数字与日期}
%
%    \begin{macrocode}
\PassOptionsToPackage { encoding = \l_@@_encoding_tl } { zhnumber }
\RequirePackage { zhnumber }
%    \end{macrocode}
%
% \begin{macro}{\chinese}
% \changes{v2.4.1}{2016/05/01}{支持 \tn{pagenumbering}。}
%    \begin{macrocode}
\cs_new_nopar:Npn \chinese { \zhnum_counter:n }
\cs_new_eq:NN \@chinese \@zhnum
\cs_new_eq:NN \Chinese \chinese
\cs_new_eq:NN \CTEXcounter \use_none:n
%    \end{macrocode}
% \end{macro}
%
% \changes{v2.2}{2015/06/29}{给 \pkg{enumitem} 宏包注册 \tn{chinese} 和
%   \tn{zhnum}。}
%
% 给 \pkg{enumitem} 宏包注册 \tn{chinese}、\tn{zhnum} 和 \tn{zhdig}。
%    \begin{macrocode}
\ctex_at_end_package:nn { enumitem }
  {
    \cs_if_free:NF \AddEnumerateCounter
      {
        \AddEnumerateCounter * { \zhnum }   { \@zhnum } { 1 }
        \AddEnumerateCounter * { \zhdig }   { \@zhdig } { 1 }
        \AddEnumerateCounter * { \chinese } { \@chinese } { 1 }
      }
  }
%    \end{macrocode}
%
% \begin{macro}{\CTEXnumber,\CTEXdigits}
%    \begin{macrocode}
\NewDocumentCommand \CTEXnumber { m m }
  { \protected@edef #1 { \zhnumber {#2} } }
\NewDocumentCommand \CTEXdigits { m m }
  { \protected@edef #1 { \zhdigits {#2} } }
%    \end{macrocode}
% \end{macro}
%
% \begin{macro}{today}
%    \begin{macrocode}
\cs_set_eq:NN \CTEX@todayold \today
\keys_define:nn { ctex }
  {
    today .choice: ,
    today / old     .code:n =
      { \cs_set_eq:NN \today \CTEX@todayold } ,
    today / small   .code:n =
      {
        \cs_set_eq:NN \today \zhtoday
        \zhnumsetup { time = Arabic }
      } ,
    today / big     .code:n =
      {
        \cs_set_eq:NN \today \zhtoday
        \zhnumsetup { time = Chinese }
      } ,
    today / unknown .code:n =
      { \msg_error:nnx { ctex } { today-undef } {#1} }
  }
\msg_new:nnnn { ctex } { today-undef }
  { Today~format~`#1'~is~undefined. }
  { Available~today~formats~are~`old',~`small',~and~`big'. }
%    \end{macrocode}
% \end{macro}
%
% \subsection{其它中文标题定义}
%
% \changes{v2.0}{2014/03/08}{将标题汉化功能加入 \pkg{ctex.sty}。}
% \changes{v2.4.3}{2016/08/18}{确保 \tn{proofname} 非空。}
%
% \begin{macro}[int]{\proofname}
% \tn{proofname} 未在标准文档类中定义,需要确保它非空。
%    \begin{macrocode}
\tl_if_exist:NF \proofname
  {
    \tl_new:N \proofname
    \tl_set:Nn \proofname { Proof }
  }
%    \end{macrocode}
% \end{macro}
%
%    \begin{macrocode}
\keys_define:nn { ctex }
  {
    contentsname   .tl_set:N = \contentsname ,
    listfigurename .tl_set:N = \listfigurename ,
    listtablename  .tl_set:N = \listtablename ,
    figurename     .tl_set:N = \figurename ,
    tablename      .tl_set:N = \tablename ,
    abstractname   .tl_set:N = \abstractname ,
    indexname      .tl_set:N = \indexname ,
    appendixname   .tl_set:N = \appendixname ,
    proofname      .tl_set:N = \proofname ,
%<article>    bibname        .tl_set:N = \refname
%<book|report>    bibname        .tl_set:N = \bibname
%<*beamer>
    algorithmname  .tl_set:N = \algorithmname ,
    bibname        .tl_set:N = \bibname ,
    refname        .tl_set:N = \refname ,
    continuation   .tl_set:N = \insertcontinuationtext
%</beamer>
  }
%    \end{macrocode}
%
%    \begin{macrocode}
%<*ctex>
\msg_new:nnn { ctex } { ctexbibname }
  {
    Neither~`\token_to_str:N \bibname'~nor~`\token_to_str:N \refname'~can~be~found.\\
    The~key~`bibname'~will~set~`\token_to_str:N \ctexbibname'~to~the~given~value.
  }
\tl_if_exist:NTF \insertcontinuationtext
  {
    \keys_define:nn { ctex }
      {
        algorithmname .tl_set:N = \algorithmname ,
        bibname       .tl_set:N = \bibname ,
        refname       .tl_set:N = \refname ,
        continuation  .tl_set:N = \insertcontinuationtext
      }
  }
  {
    \tl_if_exist:NTF \bibname
      { \keys_define:nn { ctex } { bibname .tl_set:N = \bibname } }
      {
        \tl_if_exist:NTF \refname
          { \keys_define:nn { ctex } { bibname .tl_set:N = \refname } }
          {
            \msg_warning:nn { ctex } { ctexbibname }
            \keys_define:nn { ctex } { bibname .tl_set:N = \ctexbibname }
          }
      }
  }
%</ctex>
%    \end{macrocode}
%
%    \begin{macrocode}
%</class|ctex>
%    \end{macrocode}
%
% \subsection{中文化的标题结构}
%
% 本节内容在 \CTeX{} 文档类或打开 \opt{heading} 选项下生效。
%    \begin{macrocode}
%<*class|heading>
%    \end{macrocode}
%
% \subsubsection{定义标题格式选项}
%
% \begin{variable}[int]{\c_@@_section_headings_seq}
% 保存 \tn{section} 级以下标题名字。
%    \begin{macrocode}
%<*article|book|report>
\seq_new:N \c_@@_section_headings_seq
\seq_gset_from_clist:Nn \c_@@_section_headings_seq
  { section , subsection , subsubsection , paragraph , subparagraph }
%</article|book|report>
%    \end{macrocode}
% \end{variable}
%
% \begin{variable}[int]{\c_@@_headings_seq}
%    \begin{macrocode}
\seq_new:N \c_@@_headings_seq
%<*article|book|report>
\seq_gset_eq:NN \c_@@_headings_seq \c_@@_section_headings_seq
%<book|report>\seq_gput_left:Nn \c_@@_headings_seq { chapter }
\seq_gput_left:Nn \c_@@_headings_seq { part }
%</article|book|report>
%<*beamer>
\seq_gset_from_clist:Nn \c_@@_headings_seq
  { part , section , subsection }
%</beamer>
%    \end{macrocode}
% \end{variable}
%
% \changes{v2.1}{2015/06/19}{\opt{nameformat} 可以接受章节名字为参数。}
% \changes{v2.3}{2016/01/05}{修复 \opt{nameformat} 作用域问题。}
%
% \begin{macro}[int]{\@@_initial_heading:n}
%    \begin{macrocode}
\cs_new_protected_nopar:Npn \@@_initial_heading:n #1
  {
    \tl_new:c { CTEX@pre#1 }
    \tl_new:c { CTEX@post#1 }
    \tl_const:cx { CTEXthe#1 }
      {
        \exp_not:c { CTEX@pre#1 }
        \exp_not:c { CTEX@the#1 }
        \exp_not:c { CTEX@post#1 }
      }
    \tl_const:cx { CTEX@#1name }
      {
        \group_begin:
          \exp_not:c { CTEX@#1@nameformat }
            {
              \exp_not:c { CTEX@pre#1 }
              \exp_not:N \tl_if_empty:NTF
              \exp_not:c { CTEX@#1@numberformat }
                { \exp_not:c { CTEX@the#1 } }
                {
                  \group_begin:
                    \exp_not:c { CTEX@#1@numberformat }
                    \exp_not:c { CTEX@the#1 }
                  \group_end:
                }
              \exp_not:c { CTEX@post#1 }
            }
        \group_end:
      }
  }
%    \end{macrocode}
% \end{macro}
%
% \changes{v2.1}{2015/05/22}{\opt{format+}, \opt{nameformat+} 等带加号的选项,
%   加号与前面的文字之间可以有可选的空格。}
% \changes{v2.1}{2015/06/19}{新的标题格式选项 \opt{aftertitle}。}
% \changes{v2.2}{2015/06/21}{新的标题格式选项 \opt{numbering}。}
% \changes{v2.2}{2015/06/27}{\opt{beforeskip} 和 \opt{afterskip} 选项的符号
%   不再有特殊意义。}
% \changes{v2.2}{2015/06/27}{新的标题格式选项 \opt{afterindent}。}
% \changes{v2.2}{2015/06/27}{新的标题格式选项 \opt{runin}。}
% \changes{v2.4.3}{2016/06/03}{新的标题格式选项 \opt{fixskip}。}
% \changes{v2.4.4}{2016/09/19}{新的标题格式选项 \opt{break}。}
% \changes{v2.4.5}{2016/10/22}{新的标题格式选项 \opt{hang}。}
% \changes{v2.4.5}{2016/10/25}{新的标题格式选项 \opt{tocline}。}
%
% \begin{macro}[int]{\@@_def_heading_keys:n}
%    \begin{macrocode}
\cs_new_protected_nopar:Npn \@@_def_heading_keys:n #1
  {
    \tl_put_right:Nx \l_@@_tmp_tl
      {
        #1                  .meta:nn = { ctex / #1 } { ####1 } ,
        #1 / name            .code:n =
          { \ctex_assign_heading_name:nn {#1} { ####1 } } ,
        #1 / number        .tl_set:N = \exp_not:c { CTEX@the#1 } ,
        #1 / format        .tl_set:N = \exp_not:c { CTEX@#1@format } ,
        #1 / nameformat    .tl_set:N = \exp_not:c { CTEX@#1@nameformat } ,
        #1 / numberformat  .tl_set:N = \exp_not:c { CTEX@#1@numberformat } ,
        #1 / aftername     .tl_set:N = \exp_not:c { CTEX@#1@aftername } ,
        #1 / titleformat   .tl_set:N = \exp_not:c { CTEX@#1@titleformat } ,
        #1 / aftertitle    .tl_set:N = \exp_not:c { CTEX@#1@aftertitle } ,
        #1 / beforeskip    .tl_set:N = \exp_not:c { CTEX@#1@beforeskip } ,
        #1 / afterskip     .tl_set:N = \exp_not:c { CTEX@#1@afterskip} ,
        #1 / indent        .tl_set:N = \exp_not:c { CTEX@#1@indent } ,
        #1 / numbering   .bool_set:N = \exp_not:c { CTEX@#1@numbering } ,
%<*article|book|report>
        #1 / afterindent .bool_set:N = \exp_not:c { CTEX@#1@afterindent } ,
        #1 / fixskip     .bool_set:N = \exp_not:c { CTEX@#1@fixskip } ,
        #1 / hang        .bool_set:N = \exp_not:c { CTEX@#1@hang } ,
        #1 / hang         .initial:n = true ,
        #1 / runin       .bool_set:N = \exp_not:c { CTEX@#1@runin } ,
        #1 / break         .tl_set:N = \exp_not:c { CTEX@#1@break } ,
        #1 / break+          .code:n =
          { \tl_put_right:Nn \exp_not:c { CTEX@#1@break } { ####1 } } ,
        #1 / tocline         .code:n =
          {
            \cs_set:Npn \exp_not:c { CTEX@#1@tocline}
                        \exp_not:n { ####1####2 } { ####1 }
          } ,
%</article|book|report>
        #1 / format+         .code:n =
          { \tl_put_right:Nn \exp_not:c { CTEX@#1@format } { ####1 } } ,
        #1 / nameformat+     .code:n =
          { \tl_put_right:Nn \exp_not:c { CTEX@#1@nameformat } { ####1 } } ,
        #1 / numberformat+   .code:n =
          { \tl_put_right:Nn \exp_not:c { CTEX@#1@numberformat } { ####1 } } ,
        #1 / aftername+      .code:n =
          { \tl_put_right:Nn \exp_not:c { CTEX@#1@aftername } { ####1 } } ,
        #1 / titleformat+    .code:n =
          { \tl_put_right:Nn \exp_not:c { CTEX@#1@titleformat } { ####1 } } ,
        #1 / aftertitle+     .code:n =
          { \tl_put_right:Nn \exp_not:c { CTEX@#1@aftertitle } { ####1 } } ,
        #1 / numbering    .initial:n = true ,
        #1 / beforeskip   .initial:n = \c_zero_skip ,
        #1 / afterskip    .initial:n = \c_zero_skip ,
        #1 / indent       .initial:n = \c_zero_dim ,
        #1 / beforeskip   .value_required:n = true ,
        #1 / afterskip    .value_required:n = true ,
        #1 / indent       .value_required:n = true ,
      }
  }
%    \end{macrocode}
% \end{macro}
%
% \begin{macro}[int]{\ctex_assign_heading_name:nn,\@@_assign_heading_name:nnn}
% \opt{name} 的值是一个至多两个元素的逗号分隔列表。由于 \LaTeXiii{} 的
% \texttt{clist} 总是会自动忽略空元素,所以设置 |name={,章}| 后,第一个元素将会
% 是“章”,必须用空的分组保护空元素:|name={{},章}|,这在使用中有些许不便。我们
% 可以改用 \texttt{seq} 或者手写函数解析参数来加以改进。为实现的简单起见,这里用
% 了 \pkg{xparse} 的 \tn{SplitArgument},它带有参数的长度检查。
%    \begin{macrocode}
\NewDocumentCommand \ctex_assign_heading_name:nn
  { m > { \SplitArgument { \c_one } { , } } +m }
  { \@@_assign_heading_name:nnn {#1} #2 }
\cs_new_protected:Npn \@@_assign_heading_name:nnn #1#2#3
  {
    \tl_set:cn { CTEX@pre#1 } {#2}
    \IfNoValueTF {#3}
      { \tl_clear:c { CTEX@post#1 } }
      { \tl_set:cn { CTEX@post#1 } {#3} }
  }
%    \end{macrocode}
% \end{macro}
%
% \changes{v2.0}{2014/03/21}{标题设置新增 \opt{pagestyle} 选项。}
% \changes{v2.4.1}{2016/05/10}{新的标题格式选项 \opt{part/fixbeforeskip} 和
%   \opt{chapter/fixbeforeskip}。}
% \changes{v2.4.3}{2016/06/03}{删除选项 \opt{part/fixbeforeskip} 和
%   \opt{chapter/fixbeforeskip}。}
% \changes{v2.4.5}{2016/10/01}{新的标题格式选项 \opt{chapter/lofskip} 和
%   \opt{chapter/lotskip}。}
%
% \begin{macro}{part/pagestyle,chapter/pagestyle,chapter/lofskip,chapter/lotskip}
% 只在 \cls{ctexbook} 和 \cls{ctexrep} 下有定义。
%    \begin{macrocode}
\group_begin:
%<*book|report>
\tl_set:Nn \l_@@_tmp_tl
  {
    part    / pagestyle .tl_set:N = \CTEX@part@pagestyle ,
    chapter / pagestyle .tl_set:N = \CTEX@chapter@pagestyle ,
    chapter / lofskip   .tl_set:N = \CTEX@chapter@lofskip ,
    chapter / lotskip   .tl_set:N = \CTEX@chapter@lotskip ,
    chapter / lofskip  .initial:n = \c_zero_skip ,
    chapter / lotskip  .initial:n = \c_zero_skip ,
    chapter / lofskip  .value_required:n = true ,
    chapter / lotskip  .value_required:n = true ,
  }
%</book|report>
%<*article|beamer>
\tl_clear:N \l_@@_tmp_tl
%</article|beamer>
%    \end{macrocode}
% \end{macro}
%
% 定义标题键值选项。
%    \begin{macrocode}
\seq_map_inline:Nn \c_@@_headings_seq
  {
    \@@_initial_heading:n {#1}
    \@@_def_heading_keys:n {#1}
  }
\use:x
  {
    \group_end:
    \keys_define:nn { ctex } { \exp_not:o { \l_@@_tmp_tl } }
  }
%    \end{macrocode}
%
%    \begin{macrocode}
%<*article|book|report>
%    \end{macrocode}
%
% \subsubsection{标准标题命令的修改}
%
% \begin{macro}[int]{\CTEX@fixtopskip}
% 修正 \cls{book} 和 \cls{report} 类的 \tn{part} 和 \tn{chapter} 标题之前的多余空行。
%    \begin{macrocode}
%<*book|report>
\cs_new_protected_nopar:Npn \CTEX@fixtopskip
  {
    \CTEX@fixheadingskip
    \dim_compare:nNnF \tex_pagegoal:D < \c_max_dim
      { \skip_sub:Nn \l_@@_heading_skip { \tex_topskip:D } }
  }
%</book|report>
%    \end{macrocode}
% \end{macro}
%
% \begin{macro}[int]{\CTEX@fixheadingskip}
% 抑制行间粘连,修正标题前后的多余间距。事实上,减掉 \tn{parskip},有一定的风险。
% 如果接下来的内容不会进入水平模式(例如在 \opt{format} 选项中使用 \tn{hrule} 或者 \tn{hbox}),
% \TeX{} 就不会加上 \tn{parskip}。这时候就需要用户把 \tn{parskip} 加到 \opt{beforeskip}
% 或者 \opt{afterskip} 作为修正。
%    \begin{macrocode}
\cs_new_protected_nopar:Npn \CTEX@fixheadingskip
  {
    \par
    \dim_set:Nn \tex_prevdepth:D { -1000pt }
    \skip_sub:Nn \l_@@_heading_skip { \tex_parskip:D }
  }
\skip_new:N \l_@@_heading_skip
\cs_new_protected_nopar:Npn \CTEX@setheadingskip
  { \skip_set:Nn \l_@@_heading_skip }
\cs_new_eq:NN \CTEX@headingskip \l_@@_heading_skip
%    \end{macrocode}
% \end{macro}
%
% \changes{v2.4.4}{2016/09/18}{提供 \tn{partmark}。}
% \begin{macro}[int]{\partmark}
% 提供 \tn{partmark}。
%    \begin{macrocode}
\ProvideDocumentCommand \partmark { m }
  { \markboth { } { } }
%    \end{macrocode}
% \end{macro}
%
% \changes{v2.4.4}{2016/09/19}{提供 \tn{CTEXifname}。}
% \changes{v2.4.6}{2016/10/31}{\tn{CTEXifname} 初始为假。}
% \begin{macro}{\CTEXifname}
% \begin{macro}[int]{\CTEX@ifnametrue,\CTEX@ifnamefalse}
% 用于判断当前标题是否有编号。
%    \begin{macrocode}
\cs_new_eq:NN \CTEXifname \use_ii:nn
\cs_new_protected_nopar:Npn \CTEX@ifnametrue
  { \cs_set_eq:NN \CTEXifname \use_i:nn }
\cs_new_protected_nopar:Npn \CTEX@ifnamefalse
  { \cs_set_eq:NN \CTEXifname \use_ii:nn }
%    \end{macrocode}
% \end{macro}
% \end{macro}
%
% \begin{macro}[int]{\CTEX@addloflotskip}
% 往插图和表格目录中加入额外间距。如果间距为零,则不加入。
%    \begin{macrocode}
%<*book|report>
\cs_new_protected_nopar:Npn \CTEX@addloflotskip #1
  {
    \skip_set:Nn \l_@@_heading_skip { \use:c { CTEX@#1@lofskip } }
    \skip_if_eq:nnF { \l_@@_heading_skip } { \c_zero_skip }
      {
        \addtocontents { lof }
          { \protect \addvspace { \skip_use:N \l_@@_heading_skip } }
      }
    \skip_set:Nn \l_@@_heading_skip { \use:c { CTEX@#1@lotskip } }
    \skip_if_eq:nnF { \l_@@_heading_skip } { \c_zero_skip }
      {
        \addtocontents { lot }
          { \protect \addvspace { \skip_use:N \l_@@_heading_skip } }
      }
  }
%</book|report>
%    \end{macrocode}
% \end{macro}
%
% \begin{macro}[int]{\CTEX@addtocline}
%    \begin{macrocode}
\cs_new_protected:Npn \CTEX@addtocline #1#2
  { \addcontentsline { toc } {#1} { \use:c { CTEX@#1@tocline } {#1} {#2} } }
%    \end{macrocode}
% \end{macro}
%
% \changes{v2.2}{2015/06/27}{\opt{beforeskip}、\opt{afterskip} 和 \opt{indent}
%   选项支持表达式。}
%
% \paragraph{part 的标题}
%
%    \begin{macrocode}
%<@@=>
%    \end{macrocode}
%
% \changes{v2.2}{2015/06/27}{非 \cls{ctexart} 类的 \opt{part/beforeskip} 和
%   \opt{part/afterskip} 选项有意义。}
%
% \begin{macro}[int]{\part}
%    \begin{macrocode}
%<*article>
\renewcommand\part{%
   \if@noskipsec \leavevmode \fi
   \par
   \CTEX@part@break
%  \addvspace{4ex}%
   \CTEX@setheadingskip \CTEX@part@beforeskip
   \ifodd \CTEX@part@fixskip \CTEX@fixheadingskip \fi
   \addvspace \CTEX@headingskip
   \ifodd \CTEX@part@afterindent
     \@afterindenttrue
   \else
     \@afterindentfalse
   \fi
   \secdef\@part\@spart}
%</article>
%<*book|report>
\renewcommand\part{%
% \if@openright
%   \cleardoublepage
% \else
%   \clearpage
% \fi
  \CTEX@part@break
% \thispagestyle{plain}%
  \thispagestyle{\CTEX@part@pagestyle}%
  \if@twocolumn
    \onecolumn
    \@tempswatrue
  \else
    \@tempswafalse
  \fi
% \null\vfil
  \CTEX@setheadingskip \CTEX@part@beforeskip
  \ifodd \CTEX@part@fixskip \CTEX@fixtopskip \fi
  \vspace*{\CTEX@headingskip}%
  \secdef\@part\@spart}
%</book|report>
%    \end{macrocode}
% \end{macro}
%
% \begin{macro}[int]{\@part}
%    \begin{macrocode}
%<*article>
\def\@part[#1]#2{%
  \ifnum \c@secnumdepth >\m@ne
    \ifodd \CTEX@part@numbering
      \CTEX@ifnametrue
      \refstepcounter{part}%
%     \addcontentsline{toc}{part}{\thepart\hspace{1em}#1}%
    \else
      \CTEX@ifnamefalse
      \CTEX@makeanchor{part*}%
%     \addcontentsline{toc}{part}{#1}%
    \fi
  \else
    \CTEX@ifnamefalse
    \CTEX@makeanchor{part*}%
%   \addcontentsline{toc}{part}{#1}%
  \fi
  \CTEX@addtocline{part}{#1}%
  {\interlinepenalty \@M
%  \normalfont \parindent \z@ \raggedright
   \normalfont \parindent \dimexpr \CTEX@part@indent \relax \CTEX@part@format
   \ifnum \c@secnumdepth >\m@ne \ifodd \CTEX@part@numbering
%    \Large\bfseries\partname\nobreakspace\thepart\par\nobreak
     \CTEX@partname \CTEX@part@aftername
   \fi \fi
%  \huge\bfseries #2%
   \CTEX@part@titleformat{#2}%
%  \markboth{}{}%
   \partmark{#1}%
   \CTEX@part@aftertitle}%
  \nobreak
% \vskip 3ex
  \CTEX@setheadingskip \CTEX@part@afterskip
  \ifodd \CTEX@part@fixskip \CTEX@fixheadingskip \fi
  \vskip \CTEX@headingskip
  \@afterheading}
%</article>
%<*book|report>
\def\@part[#1]#2{%
  \ifnum \c@secnumdepth >-2\relax
    \ifodd \CTEX@part@numbering
      \CTEX@ifnametrue
      \refstepcounter{part}%
%     \addcontentsline{toc}{part}{\thepart\hspace{1em}#1}%
    \else
      \CTEX@ifnamefalse
      \CTEX@makeanchor{part*}%
%     \addcontentsline{toc}{part}{#1}%
    \fi
  \else
    \CTEX@ifnamefalse
    \CTEX@makeanchor{part*}%
%   \addcontentsline{toc}{part}{#1}%
  \fi
  \CTEX@addtocline{part}{#1}%
%  \markboth{}{}%
   \partmark{#1}%
  {\interlinepenalty \@M
%  \normalfont \centering
   \normalfont \CTEX@part@format
   \ifnum \c@secnumdepth >-2\relax \ifodd \CTEX@part@numbering
%    \huge\bfseries\partname\nobreakspace\thepart\par\vskip 20\p@
     \CTEX@partname \CTEX@part@aftername
   \fi \fi
%  \Huge\bfseries #2\par}%
   \CTEX@part@titleformat{#2}%
   \CTEX@part@aftertitle}%
  \@endpart}
%</book|report>
%    \end{macrocode}
% \end{macro}
%
% \begin{macro}[int]{\@spart}
%    \begin{macrocode}
%<*article>
\def\@spart#1{%
    \CTEX@ifnamefalse
    \CTEX@makeanchor@spart{part*}%
    {\interlinepenalty \@M
%    \normalfont \parindent \z@ \raggedright
     \normalfont \parindent \dimexpr \CTEX@part@indent \relax
     \CTEX@part@format
%    \huge \bfseries #1\par}%
     \CTEX@part@titleformat{#1}%
     \CTEX@part@aftertitle}%
     \nobreak
%    \vskip 3ex
     \CTEX@setheadingskip \CTEX@part@afterskip
     \ifodd \CTEX@part@fixskip \CTEX@fixheadingskip \fi
     \vskip \CTEX@headingskip
     \@afterheading}
%</article>
%<*book|report>
\def\@spart#1{%
    \CTEX@ifnamefalse
    \CTEX@makeanchor@spart{part*}%
    {\interlinepenalty \@M
%    \normalfont \centering
     \normalfont \CTEX@part@format
%    \Huge \bfseries #1\par}%
     \CTEX@part@titleformat{#1}%
     \CTEX@part@aftertitle}%
    \@endpart}
%</book|report>
%    \end{macrocode}
% \end{macro}
%
% \begin{macro}[int]{\@endpart}
%    \begin{macrocode}
%<*book|report>
\def\@endpart{%
%             \vfil
              \CTEX@setheadingskip \CTEX@part@afterskip
              \ifodd \CTEX@part@fixskip \CTEX@fixheadingskip \fi
              \vskip \CTEX@headingskip
              \newpage
              \if@twoside
               \if@openright
                \null
                \thispagestyle{empty}%
                \newpage
               \fi
              \fi
              \if@tempswa
                \twocolumn
              \fi}
%</book|report>
%    \end{macrocode}
% \end{macro}
%
% \paragraph{chapter 的标题}
%
%    \begin{macrocode}
%<*book|report>
%    \end{macrocode}
%
% \begin{macro}[int]{\chapter}
%    \begin{macrocode}
\renewcommand\chapter{%
%                   \if@openright\cleardoublepage\else\clearpage\fi
%                   \thispagestyle{plain}%
                    \CTEX@chapter@break
                    \thispagestyle{\CTEX@chapter@pagestyle}%
                    \global\@topnum\z@
%                   \@afterindentfalse
                    \ifodd \CTEX@chapter@afterindent
                      \@afterindenttrue
                    \else
                      \@afterindentfalse
                    \fi
                    \secdef\@chapter\@schapter}
%    \end{macrocode}
% \end{macro}
%
% \begin{macro}[int]{\@chapter}
%    \begin{macrocode}
\def\@chapter[#1]#2{%
  \ifnum \c@secnumdepth >\m@ne
%<*book>
    \if@mainmatter
%</book>
      \ifodd \CTEX@chapter@numbering
        \CTEX@ifnametrue
        \refstepcounter{chapter}%
%       \typeout{\@chapapp\space\thechapter.}%
        \typeout{\CTEXthechapter}%
%       \addcontentsline{toc}{chapter}
%         {\protect\numberline{\thechapter}#1}%
      \else
        \CTEX@ifnamefalse
        \CTEX@makeanchor{\Hy@chapapp*}%
%       \addcontentsline{toc}{chapter}{#1}%
      \fi
%<*book>
    \else
      \CTEX@ifnamefalse
      \CTEX@makeanchor@chapter{\Hy@chapapp*}%
%     \addcontentsline{toc}{chapter}{#1}%
    \fi
%</book>
  \else
    \CTEX@ifnamefalse
    \CTEX@makeanchor@chapter{\Hy@chapapp*}%
%   \addcontentsline{toc}{chapter}{#1}%
  \fi
  \CTEX@addtocline{chapter}{#1}%
  \chaptermark{#1}%
% \addtocontents{lof}{\protect\addvspace{10\p@}}%
% \addtocontents{lot}{\protect\addvspace{10\p@}}%
  \CTEX@addloflotskip{chapter}%
  \if@twocolumn
    \@topnewpage[\@makechapterhead{#2}]%
  \else
    \@makechapterhead{#2}%
  \@afterheading
  \fi}
%    \end{macrocode}
% \end{macro}
%
% \begin{macro}[int]{\@schapter}
%    \begin{macrocode}
\def\@schapter#1{%
  \CTEX@ifnamefalse
  \CTEX@makeanchor@schapter{\Hy@chapapp*}%
  \if@twocolumn
    \@topnewpage[\@makeschapterhead{#1}]%
  \else
    \@makeschapterhead{#1}%
    \@afterheading
  \fi}
%    \end{macrocode}
% \end{macro}
%
% \begin{macro}[int]{\@makechapterhead}
%    \begin{macrocode}
\def\@makechapterhead#1{%
% \vspace*{50\p@}%
  \CTEX@setheadingskip \CTEX@chapter@beforeskip
  \ifodd \CTEX@chapter@fixskip \CTEX@fixtopskip \fi
  \vspace*{\CTEX@headingskip}%
% {\normalfont \parindent \z@ \raggedright
  {\normalfont \parindent \dimexpr \CTEX@chapter@indent \relax
   \CTEX@chapter@format
%  \ifnum \c@secnumdepth >\m@ne
%    \if@mainmatter
%      \huge\bfseries\@chapapp\space\thechapter\par\nobreak\vskip 20\p@
%    \fi
%  \fi
   \CTEXifname{\CTEX@chaptername\CTEX@chapter@aftername}{}%
   \interlinepenalty\@M
%  \Huge \bfseries #1\par\nobreak
   \CTEX@chapter@titleformat{#1}%
   \CTEX@chapter@aftertitle
   \nobreak
%  \vskip 40\p@
   \CTEX@setheadingskip \CTEX@chapter@afterskip
   \ifodd \CTEX@chapter@fixskip \CTEX@fixheadingskip \fi
   \vskip \CTEX@headingskip
  }}
%    \end{macrocode}
% \end{macro}
%
% \begin{macro}[int]{\@makeschapterhead}
%    \begin{macrocode}
\def\@makeschapterhead#1{%
% \vspace*{50\p@}%
  \CTEX@setheadingskip \CTEX@chapter@beforeskip
  \ifodd \CTEX@chapter@fixskip \CTEX@fixtopskip \fi
  \vspace*{\CTEX@headingskip}%
% {\normalfont \parindent \z@ \raggedright
  {\normalfont \parindent \dimexpr \CTEX@chapter@indent \relax
   \CTEX@chapter@format
   \interlinepenalty\@M
%  \Huge \bfseries  #1\par\nobreak
   \CTEX@chapter@titleformat{#1}
   \CTEX@chapter@aftertitle
   \nobreak
%  \vskip 40\p@
   \CTEX@setheadingskip \CTEX@chapter@afterskip
   \ifodd \CTEX@chapter@fixskip \CTEX@fixheadingskip \fi
   \vskip \CTEX@headingskip
  }}
%    \end{macrocode}
% \end{macro}
%
%    \begin{macrocode}
%</book|report>
%    \end{macrocode}
%
% \paragraph{section 类的标题}
%
% \begin{macro}[int]{\@startsection}
%    \begin{macrocode}
\def\@startsection#1#2#3#4#5#6{%
  \if@noskipsec \leavevmode \fi
  \par
% \@tempskipa #4\relax
% \@afterindenttrue
% \ifdim \@tempskipa <\z@
%   \@tempskipa -\@tempskipa \@afterindentfalse
% \fi
  \CTEX@update@sectionformat@n{#1}%
  \ifodd \CTEX@afterindent
    \@afterindenttrue
  \else
    \@afterindentfalse
  \fi
  \if@nobreak
    \everypar{}%
  \else
%   \addpenalty\@secpenalty\addvspace\@tempskipa
    \csname CTEX@#1@break\endcsname
    \CTEX@setheadingskip{#4}%
    \ifodd \CTEX@fixskip \CTEX@fixheadingskip \fi
    \addvspace \CTEX@headingskip
  \fi
  \@ifstar
    {\CTEX@makeanchor@ssect{#1*}\@ssect{#3}{#4}{#5}{#6}}%
    {\@dblarg{\@sect{#1}{#2}{#3}{#4}{#5}{#6}}}}
%    \end{macrocode}
% \end{macro}
%
% \begin{macro}[int]{\@seccntformat}
%    \begin{macrocode}
\def\@seccntformat#1{%
% \csname the#1\endcsname\quad}%
  \csname CTEX@#1name\endcsname
  \csname CTEX@#1@aftername\endcsname}
%    \end{macrocode}
% \end{macro}
%
% \begin{macro}[int]{\@sect}
%    \begin{macrocode}
\def\@sect#1#2#3#4#5#6[#7]#8{%
  \ifnum #2>\c@secnumdepth
    \CTEX@ifnamefalse
    \CTEX@makeanchor@sect{#1*}%
    \let\@svsec\@empty
  \else
    \ifodd \csname CTEX@#1@numbering\endcsname
      \CTEX@ifnametrue
      \refstepcounter{#1}%
      \protected@edef\@svsec{\@seccntformat{#1}\relax}%
    \else
      \CTEX@ifnamefalse
      \CTEX@makeanchor{#1*}%
      \let\@svsec\@empty
    \fi
  \fi
% \@tempskipa #5\relax
% \ifdim \@tempskipa>\z@
  \unless \ifodd \CTEX@runin
    \begingroup
      #6{%
        \CTEX@hangfrom{\hskip\glueexpr #3\relax\@svsec}%
%       \interlinepenalty \@M #8\@@par}%
        \interlinepenalty \@M
        \csname CTEX@#1@titleformat\endcsname{#8}%
        \csname CTEX@#1@aftertitle\endcsname}%
    \endgroup
    \csname #1mark\endcsname{#7}%
%   \addcontentsline{toc}{#1}{%
%     \ifnum #2>\c@secnumdepth \else
%       \protect\numberline{\csname the#1\endcsname}%
%     \fi
%     #7}%
    \CTEX@addtocline{#1}{#7}%
  \else
    \def\@svsechd{%
    #6{\hskip\glueexpr #3\relax
%     \@svsec #8}%
      \@svsec
      \csname CTEX@#1@titleformat\endcsname{#8}%
      \csname CTEX@#1@aftertitle\endcsname}%
    \csname #1mark\endcsname{#7}%
%   \addcontentsline{toc}{#1}{%
%     \ifnum #2>\c@secnumdepth \else
%       \protect\numberline{\csname the#1\endcsname}%
%     \fi
%     #7}%
    \CTEX@addtocline{#1}{#7}}%
  \fi
  \@xsect{#5}}
%    \end{macrocode}
% \end{macro}
%
% \begin{macro}[int]{\@ssect}
%    \begin{macrocode}
\def\@ssect#1#2#3#4#5{%
  \CTEX@ifnamefalse
% \@tempskipa #3\relax
% \ifdim \@tempskipa>\z@
  \unless \ifodd \CTEX@runin
    \begingroup
      #4{%
        \CTEX@hangfrom{\hskip\glueexpr #1\relax}%
%         \interlinepenalty \@M #5\@@par}%
          \interlinepenalty \@M
          \CTEX@titleformat@n{#5}%
          \CTEX@aftertitle}%
    \endgroup
  \else
%   \def\@svsechd{#4{\hskip\glueexpr #1\relax #5}}%
    \def\@svsechd{#4{\hskip\glueexpr #1\relax
                     \CTEX@titleformat@n{#5}\CTEX@aftertitle}}%
  \fi
  \@xsect{#3}}
%    \end{macrocode}
% \end{macro}
%
% \begin{macro}[int]{\@xsect}
%    \begin{macrocode}
\def\@xsect#1{%
% \@tempskipa #1\relax
% \ifdim \@tempskipa>\z@
  \unless \ifodd \CTEX@runin
    \par \nobreak
%   \vskip \@tempskipa
    \CTEX@setheadingskip{#1}%
    \ifodd \CTEX@fixskip \CTEX@fixheadingskip \fi
    \vskip \CTEX@headingskip
    \@afterheading
  \else
    \@nobreakfalse
    \global\@noskipsectrue
    \everypar{%
      \if@noskipsec
        \global\@noskipsecfalse
       {\setbox\z@\lastbox}%
        \clubpenalty\@M
        \begingroup \@svsechd \endgroup
        \unskip
%       \@tempskipa #1\relax
%       \hskip -\@tempskipa
        \hskip\glueexpr #1\relax
      \else
        \clubpenalty \@clubpenalty
        \everypar{}%
      \fi}%
  \fi
  \ignorespaces}
%    \end{macrocode}
% \end{macro}
%
%    \begin{macrocode}
%<@@=ctex>
%    \end{macrocode}
%
% \begin{macro}[int]{\CTEX@hangfrom}
% \opt{hang} 选项控制是否采用悬挂缩进。
%    \begin{macrocode}
\cs_new_protected_nopar:Npn \CTEX@hangfrom
  {
    \bool_if:NTF \CTEX@hang
      { \@hangfrom }
      { \noindent \use:n }
  }
%    \end{macrocode}
% \end{macro}
%
% \begin{macro}[int]{\CTEX@update@sectionformat@n}
% 在 \tn{@startsection} 中设置 \tn{CTEX@titleformat@n} 等为相应函数。
%    \begin{macrocode}
\cs_new_protected_nopar:Npn \CTEX@update@sectionformat@n #1
  {
    \cs_set_eq:Nc \CTEX@titleformat@n { CTEX@#1@titleformat }
    \cs_set_eq:Nc \CTEX@aftertitle { CTEX@#1@aftertitle }
    \cs_set_eq:Nc \CTEX@afterindent { CTEX@#1@afterindent }
    \cs_set_eq:Nc \CTEX@fixskip { CTEX@#1@fixskip }
    \cs_set_eq:Nc \CTEX@hang { CTEX@#1@hang }
    \cs_set_eq:Nc \CTEX@runin { CTEX@#1@runin }
  }
\cs_new_eq:NN \CTEX@titleformat@n \use:n
\cs_new_eq:NN \CTEX@aftertitle \prg_do_nothing:
\cs_new_eq:NN \CTEX@afterindent \c_true_bool
\cs_new_eq:NN \CTEX@fixskip \c_false_bool
\cs_new_eq:NN \CTEX@hang \c_true_bool
\cs_new_eq:NN \CTEX@runin \c_false_bool
%    \end{macrocode}
% \end{macro}
%
% \begin{macro}[int]{\CTEX@part@tocline, \CTEX@chapter@tocline}
%    \begin{macrocode}
\cs_new:Npn \CTEX@part@tocline #1#2
  {
    \CTEXifname
      { \CTEXthepart \hspace { 1em } }
      { }
    #2
  }
%<*book|report>
\cs_new:Npn \CTEX@chapter@tocline #1#2
  {
    \CTEXifname
      { \protect \numberline { \CTEXthechapter \hspace { .3em } } }
      { }
    #2
  }
%</book|report>
%    \end{macrocode}
% \end{macro}
%
% \begin{macro}[int]{\CTEXnumberline}
%    \begin{macrocode}
\cs_new_nopar:Npn \CTEXnumberline #1
  {
    \CTEXifname
      { \protect \numberline { \use:c { CTEXthe #1 } } }
      { }
  }
%    \end{macrocode}
% \end{macro}
%
%    \begin{macrocode}
\int_zero:N \l_@@_tmp_int
\seq_map_inline:Nn \c_@@_section_headings_seq
  {
    \int_incr:N \l_@@_tmp_int
    \cs_gset_protected_nopar:cpx  {#1}
      {
        \exp_not:N \@startsection {#1}
          { \int_use:N \l_@@_tmp_int }
          { \exp_not:c { CTEX@#1@indent } }
          { \exp_not:c { CTEX@#1@beforeskip } }
          { \exp_not:c { CTEX@#1@afterskip } }
          { \exp_not:N \normalfont \exp_not:c { CTEX@#1@format } }
      }
    \cs_new:cpn { CTEX@#1@tocline } ##1##2
      { \CTEXnumberline { ##1 } ##2 }
  }
%    \end{macrocode}
%
% \paragraph{附录标题}
%
% \begin{macro}[int]{appendix/name,appendix/number,appendix/numbering}
%    \begin{macrocode}
\keys_define:nn { ctex }
  {
    appendix                .meta:nn = { ctex / appendix } {#1} ,
    appendix / name          .code:n =
      { \ctex_assign_heading_name:nn { appendix } {#1} } ,
    appendix / number      .tl_set:N = \CTEX@appendix@number ,
    appendix / numbering .bool_set:N = \CTEX@appendix@numbering ,
    appendix / numbering  .initial:n = true
  }
\tl_new:N \CTEX@preappendix
\tl_new:N \CTEX@postappendix
%    \end{macrocode}
% \end{macro}
%
% \begin{macro}[int]{\appendix}
%    \begin{macrocode}
\cs_new_eq:NN \CTEX@save@appendix \appendix
\cs_gset_protected_nopar:Npn \appendix
  {
    \CTEX@save@appendix
%<*article>
    \gdef \CTEX@presection { \CTEX@preappendix }
    \gdef \CTEX@thesection { \CTEX@appendix@number }
    \gdef \CTEX@postsection { \CTEX@postappendix }
    \gdef \CTEX@section@numbering { \CTEX@appendix@numbering }
%</article>
%<*book|report>
    \gdef \CTEX@prechapter { \CTEX@preappendix }
    \gdef \CTEX@thechapter { \CTEX@appendix@number }
    \gdef \CTEX@postchapter { \CTEX@postappendix }
    \gdef \CTEX@chapter@numbering { \CTEX@appendix@numbering }
%</book|report>
  }
%    \end{macrocode}
% \end{macro}
%
% \paragraph{设置 \pkg{hyperref} 宏包的标题锚点}
%
% \changes{v2.4.4}{2016/09/12}{改进 \pkg{hyperref} 宏包的标题锚点设置。}
%
% \begin{macro}[int]{\CTEX@makeanchor}
% 设置超链接跳转锚点,在 \pkg{hyperref} 载入后才有意义。
%    \begin{macrocode}
\cs_new_protected_nopar:Npn \CTEX@makeanchor #1
  { }
%    \end{macrocode}
% \end{macro}
%
% \begin{variable}[int]{\c_@@_headings_cs_seq}
% 保存内部标题命令的 \CTeX{} 定义,用于随后比较。
%    \begin{macrocode}
\seq_new:N \c_@@_headings_cs_seq
\seq_gset_from_clist:Nn \c_@@_headings_cs_seq
%<article>  { part , spart , sect , ssect }
%<book|report>  { part , spart , chapter , schapter , sect , ssect }
\seq_map_inline:Nn \c_@@_headings_cs_seq
  {
    \cs_new_eq:cc { CTEX@ #1 } { @ #1 }
    \cs_new_eq:cN { CTEX@makeanchor@ #1 } \CTEX@makeanchor
  }
%    \end{macrocode}
% \end{variable}
%
% \begin{macro}[int]{\CTEX@hyperheadinghook}
% \pkg{hyperref} 会重定义内部标题命令,目的在于为没有编号的标题设置锚点(这一功能受他的
% \opt{implicit} 选项的控制)。我们在上面对标题命令的修改已经包含这一功能,如果这些标题命令在
% \pkg{hyperref} 载入之前没有被修改过,则恢复 \CTeX{} 的定义。
%    \begin{macrocode}
\cs_new_protected_nopar:Npn \CTEX@hyperheadinghook
  {
    \group_begin:
      \ifHy@implicit
        \cs_set_eq:NN \H@old@chapter \Hy@org@chapter
        \seq_map_inline:Nn \c_@@_headings_cs_seq
          {
            \cs_if_eq:ccT { H@old@ ##1 } { CTEX@ ##1 }
              {
                \cs_gset_eq:cc { @ ##1 } { CTEX@ ##1 }
                \cs_gset_eq:cN { CTEX@makeanchor@ ##1 } \CTEX@makeanchor
              }
          }
      \else:
        \seq_map_inline:Nn \c_@@_headings_cs_seq
          { \cs_gset_eq:cN { CTEX@makeanchor@ ##1 } \CTEX@makeanchor }
      \fi:
    \group_end:
  }
%    \end{macrocode}
% \end{macro}
%
%    \begin{macrocode}
\ctex_at_end_package:nn { hyperref }
  {
    \cs_gset_protected_nopar:Npn \CTEX@makeanchor #1
      {
        \Hy@MakeCurrentHrefAuto {#1}
        \Hy@raisedlink
          {
            \hyper@anchorstart { \@currentHref }
            \hyper@anchorend
          }
      }
    \CTEX@hyperheadinghook
  }
%    \end{macrocode}
%
% \paragraph{兼容 \pkg{titlesec} 宏包}
%
% 我们修改了 \tn{@startsection} 的定义,它的第四个(\meta{beforeskip})和
% 第五个(\meta{afterskip})参数的符号不再有特殊意义,改由相应的选项
% \opt{afterindent} 和 \opt{runin} 来控制。
%
% 引入 \pkg{titlesec} 宏包,并且未设置它的 \opt{loadonly} 选项时,\pkg{titlesec}
% 会展开 section 类标题获取它们的参数,进行初始设置。我们需要进行一些调整。
%
% \begin{macro}[int]{\ctex_titlesec_hook:}
% \tn{titleformat} 的设置保存在名为 |\ttlf@|\meta{section} 的宏中备用,它的内容是
% \begin{quote}\small
%   |\ttlh@|\meta{shape}\Arg{format}\Arg{label}\Arg{sep}\Arg{before}\Arg{after}
% \end{quote}
% 我们这里的 \meta{shape} 为 |hang| 或者 |runin|。\tn{titlespacing} 的设置保存在
% |\ttls@|\meta{section} 之中,它的内容是
% \begin{quote}\small
%   \Arg{left}\Arg{right}\Arg{before}\Arg{after}\Arg{afterindent}
% \end{quote}
% 其中 \meta{afterindent} 为 |1| 或 |0|,分别对应是否保留段首缩进。
% 我们需要根据 \CTeX{} 的 \opt{runin} 和 \opt{afterindent} 选项调整
% |\ttlh@|\meta{shape} 和 \meta{afterindent}。注意,由 \tn{ttl@extract} 得的
% \meta{before} 和 \meta{after} 的值总是非负的,而 \CTeX{} 的 \opt{beforeskip}
% 和 \opt{afterskip} 是可以取负值的,但我们不打算调整它们了。
% 如果使用了 \pkg{titlesec} 的 \opt{indentafter} 等选项,也不需要调整
% |\ttls@|\meta{section}。
%    \begin{macrocode}
\cs_new_protected_nopar:Npn \ctex_titlesec_hook:
  {
    \@ifpackagewith { titlesec } { explicit }
      {
        \cs_set_eq:NN \@@_titlesec_format:Nn
                      \@@_titlesec_format_explicit:Nn
      }
      { }
    \clist_map_inline:nn
      { indentafter , noindentafter , indentfirst , nonindentfirst }
      {
        \@ifpackagewith { titlesec } { ##1 }
          {
            \clist_map_break:n
              { \cs_set_eq:NN \@@_titlesec_hook:n \@@_titlesec_format:n }
          }
          { }
      }
    \seq_map_function:NN \c_@@_section_headings_seq \@@_titlesec_hook:n
  }
\cs_new_protected_nopar:Npn \@@_titlesec_hook:n #1
  {
    \@@_titlesec_format:n {#1}
    \exp_args:Nc \@@_titlesec_spacing:Nn { ttls@#1 } {#1}
  }
\cs_new_protected_nopar:Npn \@@_titlesec_format:n #1
  {
    \cs_if_free:cF { ttlf@#1 }
      { \exp_args:Nc \@@_titlesec_format:Nn { ttlf@#1 } {#1} }
  }
\cs_new_protected_nopar:Npn \@@_titlesec_format:Nn #1#2
  {
    \tl_set:Nx #1
      {
        \bool_if:cTF { CTEX@#2@runin }
          { \exp_not:N \ttlh@runin }
          { \exp_not:N \ttlh@hang }
        \tl_tail:N #1
      }
  }
\cs_new_protected_nopar:Npn \@@_titlesec_format_explicit:Nn #1#2
  {
    \cs_set_nopar:Npx #1 ##1
      {
        \bool_if:cTF { CTEX@#2@runin }
          { \exp_not:N \ttlh@runin }
          { \exp_not:N \ttlh@hang }
        \exp_args:No \tl_tail:n { #1 { } }
      }
  }
\cs_new_protected_nopar:Npn \@@_titlesec_spacing:Nn #1#2
  { \tl_set:Nx #1 { \exp_after:wN \@@_titlesec_spacing:nnnnnn #1 {#2} } }
\cs_new:Npn \@@_titlesec_spacing:nnnnnn #1#2#3#4#5#6
  {
    \exp_not:n { {#1} {#2} {#3} {#4} }
    { \bool_if:cTF { CTEX@#6@afterindent } { \@ne } { \z@ } }
  }
%    \end{macrocode}
% \end{macro}
%
%    \begin{macrocode}
\@ifpackageloaded { titlesec }
  { }
  {
    \ctex_at_end_package:nn { titlesec }
      {
        \@ifpackagewith { titlesec } { loadonly }
          { }
          { \ctex_titlesec_hook: }
      }
  }
%    \end{macrocode}
%
% \changes{v2.4.4}{2016/09/13}{使用 \pkg{titlesec} 时,章节目录也使用 \CTeX{} 的编号。}
% 让编译时终端显示  \tn{CTEXthechapter},目录使用 |\CTEXtheXXX| 编号。
%    \begin{macrocode}
\ctex_at_end_package:nn { titlesec }
  {
%<*book|report>
    \tl_set:Nn \ttl@chapterout { \typeout { \CTEXthechapter } }
%</book|report>
    \cs_if_free:NF \ttl@tocpart
      {
        \cs_set_protected_nopar:Npn \ttl@tocpart
          { \tl_set:Nn \ttl@a { \CTEXthepart \hspace { 1em } } }
      }
    \seq_map_inline:Nn \c_@@_headings_seq
      {
        \cs_if_exist:cF { ttl@toc #1 }
          {
            \cs_new_protected_nopar:cpx { ttl@toc #1 }
              {
                \tl_set:Nn \exp_not:N \ttl@a
                  {
                    \exp_not:N \protect
                    \exp_not:N \numberline { \exp_not:c { CTEXthe #1 } }
                  }
              }
          }
      }
  }
%    \end{macrocode}
%
% \paragraph{兼容 \pkg{titleps} 宏包}
%
% \changes{v2.3}{2015/12/25}{兼容 \pkg{titleps} 宏包。}
%
% 按照 \pkg{titleps} 宏包的实现机制,|\CTEXtheXXX| 等宏直到页眉排版时才会被展开,
% 这可能会造成问题\footnote{\url{https://github.com/CTeX-org/ctex-kit/issues/217}}。
%
% \begin{macro}[int]{\ctex_titleps_hook:}
% 我们修改 \pkg{titleps} 包的内部命令 \tn{ttl@settopmark} 和 \tn{ttl@setsubmark},
% 将 |\CTEXtheXXX| 等加入更新队列中。
%    \begin{macrocode}
\group_begin:
\char_set_catcode_other:N \#
\cs_new_protected_nopar:Npn \ctex_titleps_hook:
  {
    \ctex_patch_cmd:Nnn \ttl@settopmark
      { \protect \@namedef { the#1 } { \@nameuse { the#1 } } }
      {
        \protect \@namedef { the#1 } { \@nameuse { the#1 } }
        \CTEX@titlepslabel@set {#1}
      }
    \ctex_patch_cmd:Nnn \ttl@setsubmark
      { \protect \@namedef { the#1 } { } }
      {
        \protect \@namedef { the#1 } { }
        \CTEX@titlepslabel@clear {#1}
      }
    \ctex_patch_cmd:Nnn \ttl@setsubmark
      { \protect \@namedef { the#2 } { \@nameuse { the#2 } } }
      {
        \protect \@namedef { the#2 } { \@nameuse { the#2 } }
        \CTEX@titlepslabel@set {#2}
      }
  }
\group_end:
%    \end{macrocode}
% \end{macro}
%
% \begin{macro}[aux]{\CTEX@titlepslabel@set,\CTEX@titlepslabel@clear}
% 这两个函数要在随后被 \tn{xdef} 展开来获得 |\CTEXtheXXX| 的内容,不应该用
% \tn{protected} 来定义。
%    \begin{macrocode}
\cs_new_nopar:Npn \CTEX@titlepslabel@set #1
  {
    \cs_if_free:cF { CTEXthe#1 }
      { \protect \@namedef { CTEXthe#1 } { \@nameuse { CTEXthe#1 } } }
  }
\cs_new_nopar:Npn \CTEX@titlepslabel@clear #1
  {
    \cs_if_free:cF { CTEXthe#1 }
      { \protect \@namedef { CTEXthe#1 } { } }
  }
%    \end{macrocode}
% \end{macro}
%
% \pkg{titleps} 宏包的功能可以由 \pkg{titlesec} 的选项 \opt{pagestyles} 引入。
%    \begin{macrocode}
\ctex_at_end_package:nn { titlesec }
  { \cs_if_free:NF \ttl@settopmark { \ctex_titleps_hook: } }
\ctex_at_end_package:nn { titleps } { \ctex_titleps_hook: }
%    \end{macrocode}
%
% 除此之外,也可以使用 \pkg{titleps} 提供的命令 \tn{newtitlemark} 来完成:
% \begin{verbatim}
%   \newtitlemark { \CTEXthechapter }
%   \newtitlemark { \CTEXthesection }
% \end{verbatim}
% 但 \tn{newtitlemark} 不包含章节间的层次信息,功能上不及修改内部命令完整。
%
% \changes{v2.4.6}{2016/10/31}{重新初始化 \tn{ifthechapter} 等。}
% \begin{macro}[int]{\ttl@setifthe}
% 使 |\iftheXXX| 等命令在页眉设置中可用。
%    \begin{macrocode}
\ctex_at_end_package:nn { titleps }
  {
    \cs_set_protected_nopar:Npn \ttl@setifthe #1
      {
        \exp_args:Nco \cs_set_nopar:Npn { ifthe #1 }
          {
            \CTEXifname
              { \protect \@firstoftwo }
              { \protect \@secondoftwo }
          }
      }
    \seq_map_function:NN \c_@@_headings_seq \ttl@setifthe
  }
%    \end{macrocode}
% \end{macro}
%
%
% \subsubsection{目录标签的宽度}
%
% \begin{macro}[int]{\numberline}
%   \begin{macrocode}
\cs_new_protected:Npn \CTEX@toc@width@n #1
  {
    \hbox_set:Nn \l_@@_tmp_box {#1}
    \dim_set:Nn \@tempdima
      {
        \dim_max:nn { \@tempdima }
          { \box_wd:N \l_@@_tmp_box + \f@size \p@ / \c_two }
      }
  }
\group_begin:
\char_set_catcode_other:N \#
\use:n
  {
    \group_end:
    \ctex_preto_cmd:NnnTF \numberline { \ExplSyntaxOff }
      { \CTEX@toc@width@n {#1} }
      { }
      { \ctex_patch_failure:N \numberline }
    \@ifpackageloaded { tocloft }
      { }
      {
        \ctex_at_end_package:nn { tocloft }
          {
            \ctex_preto_cmd:NnnTF \numberline
              { \char_set_catcode_letter:n { 64 } }
              { \CTEX@toc@width@n {#1} }
              { }
              { \ctex_patch_failure:N \numberline }
          }
      }
  }
%    \end{macrocode}
% \end{macro}
%
% \subsubsection{页眉信息的修改}
%
% \begin{macro}[int]{\ps@headings}
% \changes{v2.4.5}{2016/10/01}{修复补丁失败。}
% \changes{v2.4.7}{2016/12/23}{修复 \cls{ctexrep} 类的 \tn{chaptermark} 汉化错误。}
%    \begin{macrocode}
%<*article>
\if@twoside
  \ctex_patch_cmd:Nnn \ps@headings
    { \ifnum \c@secnumdepth > \z@ \thesection \quad \fi }
    { \CTEXifname { \CTEXthesection \quad } { } }
  \ctex_patch_cmd:Nnn \ps@headings
    { \ifnum \c@secnumdepth > \@ne \thesubsection \quad \fi }
    { \CTEXifname { \CTEXthesubsection \quad } { } }
\else:
%    \end{macrocode}
% 不知为何,标准文档类此处对 \texttt{secnumdepth} 的判断为 $0$,与 \tn{section} 的层次 $1$ 不符。
%    \begin{macrocode}
  \ctex_patch_cmd:Nnn \ps@headings
    { \ifnum \c@secnumdepth > \m@ne \thesection \quad \fi }
    { \CTEXifname { \CTEXthesection \quad } { } }
\fi:
%</article>
%<*book|report>
\ctex_patch_cmd:Nnn \ps@headings
  {
%<book>    \ifnum \c@secnumdepth > \m@ne \if@mainmatter
%<report>    \ifnum \c@secnumdepth > \m@ne
      \@chapapp \ \thechapter . ~ \ %
%<report>    \fi
%<book>    \fi \fi
  }
  { \CTEXifname { \CTEXthechapter \quad } { } }
\if@twoside
  \ctex_patch_cmd:Nnn \ps@headings
    { \ifnum \c@secnumdepth > \z@ \thesection . ~ \ \fi }
    { \CTEXifname { \CTEXthesection } { } }
\fi:
%</book|report>
%    \end{macrocode}
% \end{macro}
%
%
% \begin{macro}[int]{\ps@fancy}
% 这里对 \pkg{fancyhdr} 宏包打补丁。原来 \pkg{fancyhdr} 宏包中使用
% \tn{thesection} 等宏表示页眉中的章节编号,这里改用 \pkg{ctex} 包所用的
% \tn{CTEXthesection} 系列宏。
%    \begin{macrocode}
\ctex_at_end_package:nn { fancyhdr }
  {
%<*article>
    \ctex_patch_cmd:Nnn \ps@fancy
      { \ifnum \c@secnumdepth > \z@ \thesection \hskip 1em \relax \fi }
      { \CTEXifname { \CTEXthesection \quad } { } }
    \ctex_patch_cmd:Nnn \ps@fancy
      { \ifnum \c@secnumdepth > \@ne \thesubsection \hskip 1em \relax \fi }
      { \CTEXifname { \CTEXthesubsection \quad } { } }
%</article>
%<*book|report>
    \ctex_patch_cmd:Nnn \ps@fancy
      { \ifnum \c@secnumdepth > \m@ne \@chapapp \ \thechapter . ~ \ \fi }
      { \CTEXifname { \CTEXthechapter \quad } { } }
    \ctex_patch_cmd:Nnn \ps@fancy
      { \ifnum \c@secnumdepth > \z@ \thesection . ~ \ \fi }
      { \CTEXifname { \CTEXthesection } { } }
%</book|report>
  }
%    \end{macrocode}
% \end{macro}
%
%    \begin{macrocode}
%</article|book|report>
%    \end{macrocode}
%
% \subsubsection{\cls{beamer} 标题页模板的修改}
%
%    \begin{macrocode}
%<*beamer>
%    \end{macrocode}
%
%    \begin{macrocode}
\ExplSyntaxOff
%    \end{macrocode}
%
% 对应 \tn{partpage}。
%    \begin{macrocode}
\defbeamertemplate*{part page}{CTEX}[1][]{
  \begingroup
%    \centering
%    {\usebeamerfont{part name}\usebeamercolor[fg]{part name}\partname~\insertromanpartnumber}
%    \vskip1em\par
    \par \addvspace{\glueexpr\CTEX@part@beforeskip\relax}%
    \parindent \dimexpr \CTEX@part@indent \relax
    \CTEX@part@format
    \ifodd \CTEX@part@numbering
      \CTEX@partname \CTEX@part@aftername
    \fi
    \begin{beamercolorbox}[sep=16pt,center,#1]{part title}
%      \usebeamerfont{part title}\insertpart\par
      \CTEX@part@titleformat \insertpart \CTEX@part@aftertitle
    \end{beamercolorbox}
    \par \addvspace{\glueexpr\CTEX@part@afterskip\relax}%
  \endgroup
}
%    \end{macrocode}
%
% 对应 \tn{sectionpage}。
%    \begin{macrocode}
\defbeamertemplate*{section page}{CTEX}[1][]{
  \begingroup
%    \centering
%    {\usebeamerfont{section name}\usebeamercolor[fg]{section name}\sectionname~\insertsectionnumber}
%    \vskip1em\par
    \par \addvspace{\glueexpr\CTEX@section@beforeskip\relax}%
    \parindent \dimexpr \CTEX@section@indent \relax
    \CTEX@section@format
    \ifodd \CTEX@section@numbering
      \CTEX@sectionname \CTEX@section@aftername
    \fi
    \begin{beamercolorbox}[sep=12pt,center,#1]{part title}
%      \usebeamerfont{section title}\insertsection\par
      \CTEX@section@titleformat \insertsection \CTEX@section@aftertitle
    \end{beamercolorbox}
    \par \addvspace{\glueexpr\CTEX@section@afterskip\relax}%
  \endgroup
}
%    \end{macrocode}
%
% 对应 \tn{subsectionpage}。
%    \begin{macrocode}
\defbeamertemplate*{subsection page}{CTEX}[1][]{
  \begingroup
%    \centering
%    {\usebeamerfont{subsection name}\usebeamercolor[fg]{subsection name}\subsectionname~\insertsubsectionnumber}
%    \vskip1em\par
    \par \addvspace{\glueexpr\CTEX@subsection@beforeskip\relax}%
    \parindent \dimexpr \CTEX@subsection@indent \relax
    \CTEX@subsection@format
    \ifodd \CTEX@subsection@numbering
      \CTEX@subsectionname \CTEX@subsection@aftername
    \fi
    \begin{beamercolorbox}[sep=8pt,center,#1]{part title}
%      \usebeamerfont{subsection title}\insertsubsection\par
      \CTEX@subsection@titleformat \insertsubsection \CTEX@subsection@aftertitle
    \end{beamercolorbox}
    \par \addvspace{\glueexpr\CTEX@subsection@afterskip\relax}%
  \endgroup
}
%    \end{macrocode}
%
% 将 \cls{beamer} 的默认模板重定向为 \texttt{CTEX} 模板。
%    \begin{macrocode}
\defbeamertemplatealias{part page}{default}{CTEX}
\defbeamertemplatealias{section page}{default}{CTEX}
\defbeamertemplatealias{subsection page}{default}{CTEX}
%    \end{macrocode}
%
%    \begin{macrocode}
\ExplSyntaxOn
%    \end{macrocode}
%
%    \begin{macrocode}
%</beamer>
%    \end{macrocode}
%
% \subsubsection{标签引用数字的汉化}
%
% \begin{macro}[int]{\refstepcounter}
% 对标题进行引用时,设置标签为通过 \opt{number} 选项设置的形式。
%    \begin{macrocode}
\cs_new_protected_nopar:Npn \CTEX@setcurrentlabel@n #1
  {
    \protected@edef \@currentlabel
      {
        \cs_if_exist:cTF { CTEX@the#1 }
          { \exp_args:cc { p@#1 } { CTEX@the#1 } }
          { \exp_not:o { \@currentlabel } }
      }
  }
%    \end{macrocode}
% \end{macro}
%
% \begin{macro}[int]{\ctex_varioref_hook:}
% 关于标签引用的宏包可能会修改 \tn{refstepcounter}。其中 \pkg{cleveref} 和
% \pkg{hyperref} 宏包都会保存之前的定义,并且它们都要求尽可能晚的被载入,所以
% 对我们上述的修改影响不大。需要注意的是 \pkg{varioref} 宏包,如果它在
% \CTeX{} 之后被载入,我们之前的修改将会被覆盖。
%    \begin{macrocode}
\cs_new_protected_nopar:Npn \ctex_varioref_hook:
  {
    \seq_map_inline:Nn \c_@@_headings_seq
      { \ctex_fix_varioref_label:n { ##1 } }
  }
%    \end{macrocode}
% \end{macro}
%
% \begin{macro}[int]{\@@_fix_varioref_label:n}
% \pkg{varioref} 宏包的 \tn{labelformat} 实际上是定义一个以 |\the<#1>| 为参数的宏
% |\p@<#1>|。\LaTeX{} 在定义计数器 |<#1>| 时,都会将 |\p@<#1>| 初始化为 \tn{@empty}。
% 如果这个宏非空,说明用户自定义了标签格式,我们就不再修改。这里不能使用
% \cs{exp_args:Nnc},因为 \texttt{c} 这种展开格式不会将参数放在花括号内。而
% \tn{labelformat} 的定义是
% \begin{verbatim}
%   \def\labelformat#1{\expandafter\def\csname p@#1\endcsname##1}
% \end{verbatim}
% 它的第二个参数必须放在花括号内,否则将会被作为宏的定界符号。
%    \begin{macrocode}
\cs_new_protected_nopar:Npn \ctex_fix_varioref_label:n #1
  {
    \tl_if_empty:cT { p@#1 }
      { \exp_args:Nno \labelformat {#1} { \cs:w CTEX@the#1 \cs_end: } }
  }
%    \end{macrocode}
% \end{macro}
%
% 如果 \pkg{varioref} 已经被载入,则使用它来设置。
%    \begin{macrocode}
\@ifpackageloaded { varioref }
  { \ctex_varioref_hook: }
  {
    \cs_new_eq:NN \CTEX@save@refstepcounter \refstepcounter
    \RenewDocumentCommand \refstepcounter { m }
      {
        \CTEX@save@refstepcounter {#1}
        \CTEX@setcurrentlabel@n {#1}
      }
    \ctex_at_end_package:nn { varioref } { \ctex_varioref_hook: }
  }
%    \end{macrocode}
%
% \subsubsection{载入 \meta{scheme} 文件}
%
%    \begin{macrocode}
\ctex_scheme_input:o { \l_@@_scheme_tl }
%    \end{macrocode}
%
%    \begin{macrocode}
%</class|heading>
%    \end{macrocode}
%
% \subsubsection{标题格式的 \opt{scheme} 定义}
%
% 下面使用 \CTeX 文档类的设置方式,\opt{plain} 模拟标准文档类直接定义或以
% \tn{@startsection} 设定的章节标题格式,\opt{chinese} 汉化的标题格式。
%
%    \begin{macrocode}
%<*scheme&(article|book|report|beamer)>
%    \end{macrocode}
%
% \changes{v2.1}{2015/05/30}{修复 \cls{ctexbook} 和 \cls{ctexrep} 类的中文
% \opt{part/number} 选项初值为空的错误。}
% \changes{v2.2}{2015/06/30}{将中文版式下的 \texttt{part} 和 \texttt{chapter}
% 标题的 \opt{nameformat} 和 \opt{titleformat} 选项的初值合并到 \opt{format} 中。}
%
%    \begin{macrocode}
\keys_set:nn { ctex / part }
  {
    aftertitle  = \par ,
%<*plain>
    name        = \partname \space ,
%<*article|book|report>
    number      = \thepart ,
%</article|book|report>
%<*beamer>
    number      = \insertromanpartnumber ,
%</beamer>
%</plain>
%<*chinese>
    number      = \chinese { part } ,
%</chinese>
%<*article>
    beforeskip  = 4ex ,
    afterskip   = 3ex ,
%<*plain>
    format      = \raggedright ,
    nameformat  = \Large \bfseries ,
    aftername   = \par \nobreak ,
    titleformat = \huge \bfseries ,
    afterindent = false
%</plain>
%<*chinese>
    format      = \Large \bfseries \centering ,
    aftername   = \quad ,
    afterindent = true
%</chinese>
%</article>
%<*book|report>
    aftername   = \par \vskip 20 \p@ ,
    beforeskip  = 0pt \@plus 1fil ,
    afterskip   = 0pt \@plus 1fil ,
    pagestyle   = plain ,
    break       = \if@openright \cleardoublepage \else \clearpage \fi ,
%<*plain>
    format      = \centering ,
    nameformat  = \huge \bfseries ,
    titleformat = \Huge \bfseries
%</plain>
%<*chinese>
    format      = \huge \bfseries \centering
%</chinese>
%</book|report>
%<*beamer>
    format      = \centering ,
    nameformat  = \usebeamerfont { part ~ name }
                  \usebeamercolor [fg] { part ~ name } ,
    aftername   = \vskip 1em \par ,
    titleformat = \usebeamerfont { part ~ title }
%</beamer>
  }
%    \end{macrocode}
%
%    \begin{macrocode}
%<*book|report>
\keys_set:nn { ctex / chapter }
  {
    pagestyle   = plain ,
    aftertitle  = \par ,
    beforeskip  = 50 \p@ ,
    afterskip   = 40 \p@ ,
    lofskip     = 10 \p@ ,
    lotskip     = 10 \p@ ,
    break       = \if@openright \cleardoublepage \else \clearpage \fi ,
%<*plain>
    name        = \chaptername \space ,
    number      = \thechapter ,
    format      = \raggedright ,
    nameformat  = \huge \bfseries ,
    aftername   = \par \nobreak \vskip 20 \p@ ,
    titleformat = \Huge \bfseries ,
    afterindent = false ,
    tocline     = \CTEXnumberline {#1} #2
%</plain>
%<*chinese>
    number      = \chinese { chapter } ,
    format      = \huge \bfseries \centering ,
    aftername   = \quad ,
    afterindent = true
%</chinese>
  }
%</book|report>
%    \end{macrocode}
%
%    \begin{macrocode}
%<@@=>
%    \end{macrocode}
%
%    \begin{macrocode}
\keys_set:nn { ctex / section }
  {
%<*article|book|report>
    number      = \thesection ,
    aftername   = \quad ,
    aftertitle  = \@@par ,
    beforeskip  = 3.5ex \@plus 1ex \@minus .2ex ,
    afterskip   = 2.3ex \@plus .2ex ,
    runin       = false ,
    break       = \addpenalty \@secpenalty ,
%<*plain>
    format      = \Large \bfseries ,
    afterindent = false
%</plain>
%<*chinese>
    format      = \Large \bfseries \centering ,
    afterindent = true
%</chinese>
%</article|book|report>
%<*beamer>
%<*plain>
    name        = \sectionname \space ,
%</plain>
    format      = \centering ,
    number      = \insertsectionnumber ,
    nameformat  = \usebeamerfont { section ~ name }
                  \usebeamercolor [fg] { section ~ name } ,
    aftername   = \vskip 1em \par ,
    titleformat = \usebeamerfont { section ~ title } ,
    aftertitle  = \par
%</beamer>
  }
%    \end{macrocode}
%
%    \begin{macrocode}
\keys_set:nn { ctex / subsection }
  {
%<*article|book|report>
    number      = \thesubsection ,
    format      = \large \bfseries ,
    aftername   = \quad ,
    aftertitle  = \@@par ,
    beforeskip  = 3.25ex \@plus 1ex \@minus .2ex ,
    afterskip   = 1.5ex  \@plus .2ex ,
    runin       = false ,
    break       = \addpenalty \@secpenalty ,
%<*plain>
    afterindent = false
%</plain>
%<*chinese>
    afterindent = true
%</chinese>
%</article|book|report>
%<*beamer>
%<*plain>
    name        = \subsectionname \space ,
    number      = \insertsubsectionnumber ,
%</plain>
%<*chinese>
    number      = \arabic { section } . \arabic { subsection } ,
%</chinese>
    format      = \centering ,
    nameformat  = \usebeamerfont { subsection ~ name }
                  \usebeamercolor [fg] { subsection ~ name } ,
    aftername   = \vskip 1em \par ,
    titleformat = \usebeamerfont { subsection ~ title } ,
    aftertitle  = \par
%</beamer>
  }
%    \end{macrocode}
%
%    \begin{macrocode}
%<*article|book|report>
%    \end{macrocode}
%
%    \begin{macrocode}
\keys_set:nn { ctex / subsubsection }
  {
    number      = \thesubsubsection ,
    format      = \normalsize \bfseries ,
    aftername   = \quad ,
    aftertitle  = \@@par ,
    beforeskip  = 3.25ex \@plus 1ex \@minus .2ex ,
    afterskip   = 1.5ex \@plus .2ex ,
    runin       = false ,
    break       = \addpenalty \@secpenalty ,
%<*plain>
    afterindent = false
%</plain>
%<*chinese>
    afterindent = true
%</chinese>
  }
%    \end{macrocode}
%
%    \begin{macrocode}
\keys_set:nn { ctex / paragraph }
  {
    number      = \theparagraph ,
    format      = \normalsize \bfseries ,
    aftername   = \quad ,
    beforeskip  = 3.25ex \@plus 1ex \@minus .2ex ,
    break       = \addpenalty \@secpenalty ,
%<*plain>
    afterindent = false
%</plain>
%<*chinese>
    afterindent = true
%</chinese>
  }
%    \end{macrocode}
%
%    \begin{macrocode}
\keys_set:nn { ctex / subparagraph }
  {
    number      = \thesubparagraph ,
    format      = \normalsize \bfseries ,
    aftername   = \quad ,
    beforeskip  = 3.25ex \@plus 1ex \@minus .2ex ,
    break       = \addpenalty \@secpenalty ,
%<*plain>
    afterindent = false
%</plain>
%<*chinese>
    afterindent = true
%</chinese>
  }
%    \end{macrocode}
%
% 处理 \opt{sub3section} 与 \opt{sub4section} 的格式。
%    \begin{macrocode}
\int_compare:nNnTF \g__ctex_section_depth_flag > \c_two
  {
    \keys_set:nn { ctex / paragraph }
      {
        aftertitle  = \@@par ,
        afterskip   = 1ex \@plus .2ex ,
        runin       = false
      }
  }
  {
    \keys_set:nn { ctex / paragraph }
      {
        afterskip   = 1em ,
        runin       = true
      }
  }
\int_compare:nNnTF \g__ctex_section_depth_flag > \c_three
  {
    \keys_set:nn { ctex / subparagraph }
      {
        aftertitle  = \@@par ,
        afterskip   = 1ex \@plus .2ex ,
        runin       = false
      }
  }
  {
    \keys_set:nn { ctex / subparagraph }
      {
        afterskip   = 1em ,
        runin       = true
      }
  }
\int_compare:nNnTF \g__ctex_section_depth_flag > \c_two
  { \keys_set:nn { ctex / subparagraph } { indent = \c_zero_dim } }
  { \keys_set:nn { ctex / subparagraph } { indent = \parindent } }
%    \end{macrocode}
%
%    \begin{macrocode}
%<@@=ctex>
%    \end{macrocode}
%
% 处理附录的格式。
%    \begin{macrocode}
\keys_set:nn { ctex / appendix }
%<*article>
  { number      = \@Alph \c@section }
%</article>
%<*book|report>
  {
    name        = \appendixname \space ,
    number      = \@Alph \c@chapter
  }
%</book|report>
%    \end{macrocode}
%
%    \begin{macrocode}
%</article|book|report>
%    \end{macrocode}
%
%    \begin{macrocode}
%</scheme&(article|book|report|beamer)>
%    \end{macrocode}
%
% \subsubsection{\pkg{ctex.sty} 的 \opt{heading} 选项}
%
%    \begin{macrocode}
%<*ctex|ctexheading>
%    \end{macrocode}
%
% \begin{variable}[int]{\c_@@_std_class_tl}
% 用于记录被引入的标准文档类。
%    \begin{macrocode}
\clist_map_inline:nn { article , book , report , beamer }
  {
    \@ifclassloaded {#1}
      { \clist_map_break:n { \tl_const:Nn \c_@@_std_class_tl {#1} } }
      { }
  }
%    \end{macrocode}
% \end{variable}
%
% 若标准文档类被引入,则载入对应的标题定义文件。否则视 \tn{chapter} 是否有定义来
% 引入 \cls{book} 或者 \cls{article}。
%    \begin{macrocode}
\msg_new:nnn { ctex } { not-standard-class }
  {
    None~of~the~standard~document~classes~was~loaded.\\
    Heading~`#1'~is~selected.\\
    ctex~may~not~work~as~expected.
  }
%<ctex>\bool_if:NTF \l_@@_heading_bool
%<ctexheading>\use:n
  {
    \tl_if_exist:NTF \c_@@_std_class_tl
      { \cs_new_eq:NN \c_@@_class_tl \c_@@_std_class_tl }
      {
        \cs_if_exist:NTF \chapter
          {
            \cs_if_exist:NF \if@mainmatter
              { \cs_new_eq:NN \if@mainmatter \tex_iftrue:D }
            \tl_const:Nn \c_@@_class_tl { book }
          }
          { \tl_const:Nn \c_@@_class_tl { article } }
        \msg_warning:nnx { ctex } { not-standard-class } { \c_@@_class_tl }
      }
    \ctex_file_input:n { ctex- \c_@@_class_tl .def }
  }
%<ctex>  { \ctex_scheme_input:o { \l_@@_scheme_tl } }
%    \end{macrocode}
%
%    \begin{macrocode}
%</ctex|ctexheading>
%    \end{macrocode}
%
% \subsubsection{标题配置文件}
%
%    \begin{macrocode}
%<*name>
%    \end{macrocode}
%
%    \begin{macrocode}
\keys_set_known:nn { ctex }
  {
    contentsname   = 目录 ,
    listfigurename = 插图 ,
    listtablename  = 表格 ,
    figurename     = 图 ,
    tablename      = 表 ,
    abstractname   = 摘要 ,
    indexname      = 索引 ,
    bibname        = 参考文献 ,
    appendixname   = 附录 ,
    proofname      = 证明 ,
    algorithmname  = 算法 ,
    refname        = 参考文献 ,
    continuation   = (续) ,
    part    / name = { 第 , 部分 } ,
    chapter / name = { 第 , 章 }
  }
%    \end{macrocode}
%
%    \begin{macrocode}
%</name>
%    \end{macrocode}
%
% \subsection{\opt{chinese} 方案的其他设置}
%
%    \begin{macrocode}
%<*scheme&chinese>
%    \end{macrocode}
%
% \opt{chinese} 在标准文档类下的页面格式总采用 \texttt{headings}。
%    \begin{macrocode}
%<article|book|report>\pagestyle { headings }
%    \end{macrocode}
%
% 日期格式。
%    \begin{macrocode}
\keys_set:nn { ctex } { today = small }
%    \end{macrocode}
%
% 若用户未设置宏包选项 \opt{autoindent},则自动调整首行缩进。
%    \begin{macrocode}
\ctex_if_autoindent_touched:F
  { \keys_set:nn { ctex } { autoindent = true } }
%    \end{macrocode}
%
% 使用标题定义时的设置。首先是命题名字汉化。\cls{beamer} 需要汉化定理名称。
%    \begin{macrocode}
%<*!generic>
\str_if_eq:onTF { \l_@@_encoding_tl } { GBK }
%<*beamer>
  {
    \uselanguage { ChineseGBK }
    \languagealias { chinese } { ChineseGBK }
    \ctex_file_input:n { ctex-name-gbk.cfg }
  }
  {
    \uselanguage { ChineseUTF8 }
    \languagealias { chinese } { ChineseUTF8 }
    \ctex_file_input:n { ctex-name-utf8.cfg }
  }
%    \end{macrocode}
% 让 \pkg{translator} 包优先查找中文翻译。
%    \begin{macrocode}
\clist_put_left:Nn \trans@languagepath { chinese }
%</beamer>
%<*!beamer>
  { \ctex_file_input:n { ctex-name-gbk.cfg } }
  { \ctex_file_input:n { ctex-name-utf8.cfg } }
%    \end{macrocode}
%
% \changes{v2.4.1}{2016/05/09}{\cls{beamer} 不调整默认字体大小。}
% 对 \cls{beamer} 以外的文档类,若用户未设置宏包选项 \opt{zihao},则设置 \tn{normalsize}
% 为五号字。\cls{beamer} 不调整默认字体大小。
%    \begin{macrocode}
\int_compare:nNnF \g_@@_font_size_flag > { -1 }
  { \cs_gset_eq:NN \g_@@_font_size_flag \c_zero }
%    \end{macrocode}
%
% \changes{v2.4.1}{2016/05/09}{\cls{beamer} 不调整默认行距。}
% 对 \cls{beamer} 以外的文档类,若用户未设置宏包选项 \opt{linespread},则设置行
% 距初始值为 $1.3\times 1.2=1.56$ 倍字体大小。\cls{beamer} 不调整行距。
%    \begin{macrocode}
\fp_compare:nNnT { \l_@@_line_spread_fp } ? { \c_zero_fp }
  { \fp_set:Nn \l_@@_line_spread_fp { 1.3 } }
%</!beamer>
%</!generic>
%    \end{macrocode}
%
% 不使用标题定义时的通用设置。
%    \begin{macrocode}
%<*generic>
\tl_set:Nn \l_@@_tmp_tl { beamer }
\tl_if_eq:NNTF \c_@@_std_class_tl \l_@@_tmp_tl
  {
    \str_if_eq:onTF { \l_@@_encoding_tl } { GBK }
      {
        \uselanguage { ChineseGBK }
        \languagealias { chinese } { ChineseGBK }
        \ctex_file_input:n { ctex-name-gbk.cfg }
      }
      {
        \uselanguage { ChineseUTF8 }
        \languagealias { chinese } { ChineseUTF8 }
        \ctex_file_input:n { ctex-name-utf8.cfg }
      }
    \clist_put_left:Nn \trans@languagepath { chinese }
  }
  {
    \str_if_eq:onTF { \l_@@_encoding_tl } { GBK }
      { \ctex_file_input:n { ctex-name-gbk.cfg } }
      { \ctex_file_input:n { ctex-name-utf8.cfg } }
    \int_compare:nNnF \g_@@_font_size_flag > { -1 }
      { \cs_gset_eq:NN \g_@@_font_size_flag \c_zero }
    \fp_compare:nNnT { \l_@@_line_spread_fp } ? { \c_zero_fp }
      { \fp_set:Nn \l_@@_line_spread_fp { 1.3 } }
%    \end{macrocode}
% \changes{v2.0.2}{2015/05/16}{修复加载 \pkg{ctex} 宏包后章节标题后第一段
% 无段首缩进的问题。}
% 若 \pkg{ctex} 宏包与标准文档类及其衍生文档类联用,则将载入 \pkg{indentfirst} 宏包,
% 实现章节标题后首个段落的段首缩进。
%    \begin{macrocode}
    \tl_if_exist:NT \c_@@_std_class_tl
      { \RequirePackage { indentfirst } }
  }
%</generic>
%    \end{macrocode}
%
%    \begin{macrocode}
%</scheme&chinese>
%    \end{macrocode}
%
% \subsection{中文字号}
%
%    \begin{macrocode}
%<*class|ctex|ctexsize>
%    \end{macrocode}
%
% \changes{v2.0}{2014/03/08}{将中文字号功能提取到可以独立使用的 \pkg{ctexsize}。}
%
% \begin{macro}{\zihao}
%    \begin{macrocode}
\NewDocumentCommand \zihao { m }
  { \exp_args:Nx \ctex_zihao:n {#1} \tex_ignorespaces:D }
%    \end{macrocode}
% \end{macro}
%
% \begin{macro}[int]{\ctex_zihao:n}
%    \begin{macrocode}
\cs_new_protected_nopar:Npn \ctex_zihao:n #1
  {
    \prop_get:NnNTF \c_@@_font_size_prop {#1} \l_@@_font_size_tl
      { \exp_after:wN \fontsize \l_@@_font_size_tl \selectfont }
      { \msg_error:nnn { ctex } { fontsize } {#1} }
  }
\msg_new:nnnn { ctex } { fontsize }
  { Undefined~Chinese~font~size~`#1'~in~command~\token_to_str:N \zihao.}
  {
    The~old~font~size~is~used~if~you~continue.\\
    The~available~font~sizes~are~listed~as~follow.\\
    \seq_use:Nnnn \c_@@_font_size_seq { ~and~ } { ,~ } { ,~and~ }.
  }
%    \end{macrocode}
% \end{macro}
%
% \subsubsection{定义中文字号}
%
% \changes{v2.0}{2014/03/08}{中文字号不再采用近似值。}
%
% \begin{variable}[int]{\c_@@_font_size_prop}
% \begin{macro}[aux]{\@@_save_font_size:nn}
% 基础行距是字号的 $1.2$ 倍,采用 \hologo{eTeX} 的 scaling 运算得到的结果
% 要比简单的 |1.2\dimexpr| 精确^^A
% \footnote{\url{http://thread.gmane.org/gmane.comp.tex.latex.latex3/3190}}。
%    \begin{macrocode}
\prop_new:N \c_@@_font_size_prop
\seq_new:N \c_@@_font_size_seq
\cs_new_protected_nopar:Npn \@@_save_font_size:nn #1#2
  {
    \use:x
      {
        \prop_gput:Nnn \exp_not:N \c_@@_font_size_prop {#1}
          {
            { \dim_to_decimal:n {#2} }
            { \dim_to_decimal:n { (#2) * \c_six / \c_five } }
          }
      }
    \seq_gput_right:Nn \c_@@_font_size_seq {#1}
  }
\clist_map_inline:nn
  {
    {  8 } { 5    bp } ,
    {  7 } { 5.5  bp } ,
    { -6 } { 6.5  bp } ,
    {  6 } { 7.5  bp } ,
    { -5 } { 9    bp } ,
    {  5 } { 10.5 bp } ,
    { -4 } { 12   bp } ,
    {  4 } { 14   bp } ,
    { -3 } { 15   bp } ,
    {  3 } { 16   bp } ,
    { -2 } { 18   bp } ,
    {  2 } { 22   bp } ,
    { -1 } { 24   bp } ,
    {  1 } { 26   bp } ,
    { -0 } { 36   bp } ,
    {  0 } { 42   bp }
  }
  { \@@_save_font_size:nn #1 }
%    \end{macrocode}
% \end{macro}
% \end{variable}
%
% \begin{macro}[int]{\ctex_declare_math_sizes:nnnn}
%    \begin{macrocode}
\cs_new_protected_nopar:Npn \ctex_declare_math_sizes:nnnn #1#2#3#4
  {
    \@@_get_font_sizes:Nn \l_@@_font_size_tl { {#1} {#2} {#3} {#4} }
    \exp_after:wN \DeclareMathSizes \l_@@_font_size_tl
  }
%    \end{macrocode}
% \end{macro}
%
% \begin{macro}[aux]{\@@_get_font_sizes:Nn}
%    \begin{macrocode}
\cs_new_protected_nopar:Npn \@@_get_font_sizes:Nn #1#2
  {
    \tl_clear:N #1
    \tl_map_inline:nn {#2}
      {
        \prop_get:NnNTF \c_@@_font_size_prop {##1} \l_@@_tmp_tl
          { \tl_put_right:Nx #1 { { \tl_head:N \l_@@_tmp_tl } } }
          { \tl_put_right:Nx #1 { { \dim_to_decimal:n { ##1 } } } }
      }
  }
%    \end{macrocode}
% \end{macro}
%
%    \begin{macrocode}
\clist_map_inline:nn
  {
    {  8 }{  8 }{ 5pt }{ 5pt } ,
    {  7 }{  7 }{ 5pt }{ 5pt } ,
    { -6 }{ -6 }{ 5pt }{ 5pt } ,
    {  6 }{  6 }{ 5pt }{ 5pt } ,
    { -5 }{ -5 }{ 6pt }{ 5pt } ,
    {  5 }{  5 }{ 7pt }{ 5pt } ,
    { -4 }{ -4 }{ 8pt }{ 6pt } ,
    {  4 }{  4 }{  5 }{  6 } ,
    { -3 }{ -3 }{ -4 }{ -5 } ,
    {  3 }{  3 }{  4 }{  5 } ,
    { -2 }{ -2 }{ -3 }{ -4 } ,
    {  2 }{  2 }{  3 }{  4 } ,
    { -1 }{ -1 }{ -2 }{ -3 } ,
    {  1 }{  1 }{  2 }{  3 } ,
    { -0 }{ -0 }{ -1 }{ -2 } ,
    {  0 }{  0 }{  1 }{  2 }
  }
  { \ctex_declare_math_sizes:nnnn #1 }
%    \end{macrocode}
%
% \subsubsection{修改默认字号大小}
%
% \begin{macro}[int]{\ctex_set_font_size:Nnn}
%    \begin{macrocode}
\cs_new_protected:Npn \ctex_set_font_size:Nnn #1#2#3
  {
    \prop_get:NnNTF \c_@@_font_size_prop {#2} \l_@@_font_size_tl
      { \exp_after:wN \@@_set_font_size:nnNn \l_@@_font_size_tl #1 {#3} }
      { \msg_error:nnn { ctex } { fontsize } {#2} }
  }
\cs_new_protected:Npn \@@_set_font_size:nnNn #1#2#3#4
  { \cs_set_protected_nopar:Npn #3 { \@setfontsize #3 {#1} {#2} #4 } }
%    \end{macrocode}
% \end{macro}
%
%    \begin{macrocode}
\if_case:w \g_@@_font_size_flag
  \ctex_file_input:n { ctex-c5size.clo }
\or:
  \ctex_file_input:n { ctex-cs4size.clo }
\fi:
%    \end{macrocode}
%
%    \begin{macrocode}
%</class|ctex|ctexsize>
%    \end{macrocode}
%
%    \begin{macrocode}
%<*c5size>
\ctex_set_font_size:Nnn \normalsize { 5 }
  {
    \abovedisplayskip 10\p@ \@plus2\p@ \@minus5\p@
    \abovedisplayshortskip \z@ \@plus3\p@
    \belowdisplayshortskip 6\p@ \@plus3\p@ \@minus3\p@
    \belowdisplayskip \abovedisplayskip
    \let\@listi\@listI
  }
\ctex_set_font_size:Nnn \small { -5 }
  {
    \abovedisplayskip 8.5\p@ \@plus3\p@ \@minus4\p@
    \abovedisplayshortskip \z@ \@plus2\p@
    \belowdisplayshortskip 4\p@ \@plus2\p@ \@minus2\p@
    \def\@listi{\leftmargin\leftmargini
                \topsep 4\p@ \@plus2\p@ \@minus2\p@
                \parsep 2\p@ \@plus\p@ \@minus\p@
                \itemsep \parsep}
    \belowdisplayskip \abovedisplayskip
  }
\ctex_set_font_size:Nnn \footnotesize { 6 }
  {
    \abovedisplayskip 6\p@ \@plus2\p@ \@minus4\p@
    \abovedisplayshortskip \z@ \@plus\p@
    \belowdisplayshortskip 3\p@ \@plus\p@ \@minus2\p@
    \def\@listi{\leftmargin\leftmargini
                \topsep 3\p@ \@plus\p@ \@minus\p@
                \parsep 2\p@ \@plus\p@ \@minus\p@
                \itemsep \parsep}
    \belowdisplayskip \abovedisplayskip
  }
\ctex_set_font_size:Nnn \scriptsize { -6 } { }
\ctex_set_font_size:Nnn \tiny  {  7 } { }
\ctex_set_font_size:Nnn \large { -4 } { }
\ctex_set_font_size:Nnn \Large { -3 } { }
\ctex_set_font_size:Nnn \LARGE { -2 } { }
\ctex_set_font_size:Nnn \huge  {  2 } { }
\ctex_set_font_size:Nnn \Huge  {  1 } { }
%</c5size>
%<*cs4size>
\ctex_set_font_size:Nnn \normalsize { -4 }
  {
    \abovedisplayskip 12\p@ \@plus3\p@ \@minus7\p@
    \abovedisplayshortskip \z@ \@plus3\p@
    \belowdisplayshortskip 6.5\p@ \@plus3.5\p@ \@minus3\p@
    \belowdisplayskip \abovedisplayskip
    \let\@listi\@listI
  }
\ctex_set_font_size:Nnn \small { 5 }
  {
    \abovedisplayskip 11\p@ \@plus3\p@ \@minus6\p@
    \abovedisplayshortskip \z@ \@plus3\p@
    \belowdisplayshortskip 6.5\p@ \@plus3.5\p@ \@minus3\p@
    \def\@listi{\leftmargin\leftmargini
                \topsep 9\p@ \@plus3\p@ \@minus5\p@
                \parsep 4.5\p@ \@plus2\p@ \@minus\p@
                \itemsep \parsep}
    \belowdisplayskip \abovedisplayskip
  }
\ctex_set_font_size:Nnn \footnotesize { -5 }
  {
    \abovedisplayskip 10\p@ \@plus2\p@ \@minus5\p@
    \abovedisplayshortskip \z@ \@plus3\p@
    \belowdisplayshortskip 6\p@ \@plus3\p@ \@minus3\p@
    \def\@listi{\leftmargin\leftmargini
                \topsep 6\p@ \@plus2\p@ \@minus2\p@
                \parsep 3\p@ \@plus2\p@ \@minus\p@
                \itemsep \parsep}
    \belowdisplayskip \abovedisplayskip
  }
\ctex_set_font_size:Nnn \scriptsize { 6 } { }
\ctex_set_font_size:Nnn \tiny  { -6 } { }
\ctex_set_font_size:Nnn \large { -3 } { }
\ctex_set_font_size:Nnn \Large { -2 } { }
\ctex_set_font_size:Nnn \LARGE {  2 } { }
\ctex_set_font_size:Nnn \huge  { -1 } { }
\ctex_set_font_size:Nnn \Huge  {  1 } { }
%</cs4size>
%    \end{macrocode}
%
%    \begin{macrocode}
%<ctexsize>\normalsize
%    \end{macrocode}
%
%    \begin{macrocode}
%<*class|ctex>
%    \end{macrocode}
%
% \subsection{更新行距}
%
% \cs{l_@@_line_spread_fp} 被设置了才有必要更新行距和 \tn{footnotesep}。
%    \begin{macrocode}
\fp_compare:nNnF { \l_@@_line_spread_fp } ? { \c_zero_fp }
  {
    \exp_args:Nx \linespread { \fp_use:N \l_@@_line_spread_fp }
%    \end{macrocode}
%
% \changes{v2.0}{2014/04/23}{调整 \tn{footnotesep} 的大小,以适合行距的变化。}
%
% \begin{variable}[int]{\footnotesep}
% 我们调整了行距,可能导致脚注的间距与行距不协调,需要调整 \tn{footnotesep}。标准
% 文档类对 \tn{footnotesep} 的设置是,字体大小为 \tn{footnotesize} 时 \tn{strutbox}
% 的高度(默认值是 |.7\baselineskip|)。我们沿用这个设置方法,只需要更新具体的大小。
%    \begin{macrocode}
    \group_begin: \footnotesize \exp_args:NNNo \group_end:
    \dim_set:Nn \footnotesep { \dim_use:N \box_ht:N \strutbox }
  }
%    \end{macrocode}
% \end{variable}
%
% 激活默认字体大小,更新行距、\tn{parindent} 和 \tn{CJKglue}。
%    \begin{macrocode}
\normalsize
%    \end{macrocode}
%
% \subsection{其它功能}
%
% \begin{macro}{\CTeX}
% \file{ctex-faq.sty} 中的定义是
% \begin{verbatim}
%   \DeclareRobustCommand\CTeX{$\mathbb{C}$\kern-.05em\TeX}
% \end{verbatim}
% 然而 \tn{mathbb} 未必有定义,这里就不采用它了,只定义最简单的形式。
% 同 \pkg{hologo} 宏包的设置类似,\CTeX{} 可以用在 \tn{csname} 和 PDF 书签中。
%    \begin{macrocode}
\NewDocumentCommand \CTeX { }
  { \ifincsname CTeX \else: C \TeX \fi: }
\ctex_at_end_package:nn { hyperref }
  { \pdfstringdefDisableCommands { \tl_set:Nn \CTeX { CTeX } } }
%    \end{macrocode}
% \end{macro}
%
% \changes{v2.0}{2014/03/28}{\opt{captiondelimiter} 是过时选项。}
% \begin{macro}{captiondelimiter}
% 过时选项。
%    \begin{macrocode}
\keys_define:nn { ctex }
  {
    captiondelimiter .code:n =
      {
        \msg_warning:nnn { ctex } { deprecated-option }
          { You~can~load~the~package~`caption'~to~get~its~functionality. }
      }
  }
%    \end{macrocode}
% \end{macro}
%
%    \begin{macrocode}
%</class|ctex>
%    \end{macrocode}
%
% \subsubsection{列表环境的缩进}
%
% \begin{macro}[int]{\verse,\quotation}
% 只在使用文档类的时候修改诗歌和引用环境的缩进。
%    \begin{macrocode}
%<*scheme&chinese&(article|book|report)>
\ctex_patch_cmd:Nnn \verse { -1.5em } { -2 \ccwd }
\ctex_patch_cmd:Nnn \verse {  1.5em } {  2 \ccwd }
\ctex_patch_cmd:Nnn \quotation { 1.5em } { 2 \ccwd }
%</scheme&chinese&(article|book|report)>
%    \end{macrocode}
% \end{macro}
%
%    \begin{macrocode}
%<*class|ctex>
%    \end{macrocode}
%
% \changes{v2.0}{2014/03/09}
% {解决 \pkg{etoolbox} 与 \pkg{breqn} 关于 \tn{end} 的冲突。}
% \changes{v2.2}{2015/06/23}{删去 \pkg{etoolbox} 与 \pkg{breqn} 的兼容补丁。}
%
% \subsection{载入中文字体}
%
% \begin{macro}[int]{\ctex_fontset_error:n}
% 字库不可用时给出紧急错误信息,停止读取定义文件。
%    \begin{macrocode}
\cs_new_protected_nopar:Npn \ctex_fontset_error:n #1
  { \msg_critical:nnn { ctex } { fontset-unavailable } {#1} }
\msg_new:nnn { ctex } { fontset-unavailable }
  { CTeX~fontset~`#1'~is~unavailable~in~current~mode. }
%    \end{macrocode}
% \end{macro}
%
% \begin{macro}[int]{\ctex_load_fontset:}
% 如果用户没有指定字体,则探测操作系统,载入相应的字体配置。
%    \begin{macrocode}
\cs_new_protected_nopar:Npn \ctex_load_fontset:
  {
    \tl_if_empty:NTF \g_@@_fontset_tl
      {
        \ctex_detected_platform:
        \ctex_file_input:n { ctex-fontset- \g_@@_fontset_tl .def }
      }
      {
        \file_if_exist:nTF { ctex-fontset- \g_@@_fontset_tl .def }
          { \ctex_file_input:n { ctex-fontset- \g_@@_fontset_tl .def } }
          {
            \use:x
              {
                \ctex_detected_platform:
                \msg_error:nnxx { ctex } { fontset-not-found }
                  { \g_@@_fontset_tl } { \exp_not:N \g_@@_fontset_tl }
              }
            \ctex_file_input:n { ctex-fontset- \g_@@_fontset_tl .def }
          }
      }
  }
\@onlypreamble \ctex_load_fontset:
\msg_new:nnnn { ctex } { fontset-not-found }
  {
    CTeX~fontset~`#1'~could~not~be~found.\\
    Fontset~`#2'~will~be~used~instead.
  }
  { You~may~run~`mktexlsr'~firstly. }
%    \end{macrocode}
% \end{macro}
%
% \begin{macro}{fontset}
% 在导言区通过 \tn{ctexset} 载入中文字库的选项。
%    \begin{macrocode}
\keys_define:nn { ctex }
  {
    fontset .code:n =
      {
        \ctex_if_preamble:TF
          {
            \str_if_eq_x:nnTF {#1} { none }
              { \msg_warning:nnn { ctex } { invalid-value } {#1} }
              {
                \str_if_eq:onTF { \g_@@_fontset_tl } { none }
                  {
                    \tl_gset:Nx \g_@@_fontset_tl {#1}
                    \ctex_load_fontset:
                  }
                  {
                    \msg_error:nnxx { ctex } { fontset-loaded }
                      { \g_@@_fontset_tl } {#1}
                  }
              }
          }
          { \msg_error:nn { ctex } { fontset-only-preamble } }
      }
  }
\msg_new:nnnn { ctex } { fontset-loaded }
  {
    CTeX~fontset~`#1'~has~been~loaded.
    \str_if_eq:nnF {#1} {#2} { \\ Fontset~`#2'~will~be~ignored. }
  }
  { Only~one~fontset~can~be~loaded~in~the~preamble. }
\msg_new:nnn { ctex } { fontset-only-preamble }
  {
    The~ `fontset'~ option~ can~ be~ used~ only~ in~ preamble.
  }
%    \end{macrocode}
% \end{macro}
%
% 载入中文字库。
%    \begin{macrocode}
\str_if_eq:onF { \g_@@_fontset_tl } { none }
  { \ctex_load_fontset: }
%    \end{macrocode}
%
% \subsection{宏包配置文件}
%
% \subsubsection{\pkg{ctex.cfg}}
%
%    \begin{macrocode}
\ctex_at_end:n { \ctex_file_input:n { ctex.cfg } }
%    \end{macrocode}
%
%    \begin{macrocode}
%</class|ctex>
%    \end{macrocode}
%
%    \begin{macrocode}
%<*config>
%%
%</config>
%    \end{macrocode}
%
% \subsubsection{\pkg{ctexopts.cfg}}
%
% 这里仅为配置文件示例:使用 Windows Vista 或以后版本的字体设置。
%    \begin{macrocode}
%<*ctexopts>
%%
%% \keys_set:nn { ctex / option } { fontset = windowsnew }
%</ctexopts>
%    \end{macrocode}
%
% \subsection{字体定义文件}
%
% \subsubsection{传统定义方式}
%
%    \begin{macrocode}
%<*c19|c70>
%%
%% Chinese characters
%%
%<c19>%% character set: GBK (extension of GB 2312)
%<c70>%% character set: Unicode
%% font encoding: Unicode
%%
%</c19|c70>
%    \end{macrocode}
%
% \pkg{CJK} 宏包使用的字体族。
%    \begin{macrocode}
%<rm&c19>\DeclareFontFamily{C19}{rm}{\hyphenchar\font\m@ne}
%<rm&c70>\DeclareFontFamily{C70}{rm}{\hyphenchar\font\m@ne}
%<sf&c19>\DeclareFontFamily{C19}{sf}{\hyphenchar\font\m@ne}
%<sf&c70>\DeclareFontFamily{C70}{sf}{\hyphenchar\font\m@ne}
%<tt&c19>\DeclareFontFamily{C19}{tt}{\hyphenchar\font\m@ne}
%<tt&c70>\DeclareFontFamily{C70}{tt}{\hyphenchar\font\m@ne}
%    \end{macrocode}
%
% \changes{v2.4}{2016/04/25}{提供 \upLaTeX{} 的 NFSS 字体定义。}
% \upTeX{} 使用的字体族。\upLaTeX 在 NFSS 下使用字体编码 |JY2| 和 |JT2| 来分别
% 表示横排与直排的日文。
%    \begin{macrocode}
%<*uptex|aptex>
\DeclareKanjiFamily{JY2}{rm}{}
\DeclareKanjiFamily{JY2}{sf}{}
\DeclareKanjiFamily{JY2}{tt}{}
\DeclareKanjiFamily{JT2}{rm}{}
\DeclareKanjiFamily{JT2}{sf}{}
\DeclareKanjiFamily{JT2}{tt}{}
%</uptex|aptex>
%    \end{macrocode}
%
%
%    \begin{macrocode}
%<*rm>
%<*c19>
\DeclareFontShape{C19}{rm}{m}{n}{<-> CJK * gbksong}{\CJKnormal}
\DeclareFontShape{C19}{rm}{b}{n}{<-> CJK * gbkhei}{\CJKnormal}
\DeclareFontShape{C19}{rm}{bx}{n}{<-> CJK * gbkhei}{\CJKnormal}
\DeclareFontShape{C19}{rm}{m}{sl}{<-> CJK * gbksongsl}{\CJKnormal}
\DeclareFontShape{C19}{rm}{b}{sl}{<-> CJK * gbkheisl}{\CJKnormal}
\DeclareFontShape{C19}{rm}{bx}{sl}{<-> CJK * gbkheisl}{\CJKnormal}
\DeclareFontShape{C19}{rm}{m}{it}{<-> CJK * gbkkai}{\CJKnormal}
\DeclareFontShape{C19}{rm}{b}{it}{<-> CJKb * gbkkai}{\CJKbold}
\DeclareFontShape{C19}{rm}{bx}{it}{<-> CJKb * gbkkai}{\CJKbold}
%</c19>
%<*c70>
\DeclareFontShape{C70}{rm}{m}{n}{<-> CJK * unisong}{\CJKnormal}
\DeclareFontShape{C70}{rm}{b}{n}{<-> CJK * unihei}{\CJKnormal}
\DeclareFontShape{C70}{rm}{bx}{n}{<-> CJK * unihei}{\CJKnormal}
\DeclareFontShape{C70}{rm}{m}{sl}{<-> CJK * unisongsl}{\CJKnormal}
\DeclareFontShape{C70}{rm}{b}{sl}{<-> CJK * uniheisl}{\CJKnormal}
\DeclareFontShape{C70}{rm}{bx}{sl}{<-> CJK * uniheisl}{\CJKnormal}
\DeclareFontShape{C70}{rm}{m}{it}{<-> CJK * unikai}{\CJKnormal}
\DeclareFontShape{C70}{rm}{b}{it}{<-> CJKb * unikai}{\CJKbold}
\DeclareFontShape{C70}{rm}{bx}{it}{<-> CJKb * unikai}{\CJKbold}
%</c70>
%</rm>
%<*uptex|aptex>
\DeclareFontShape{JY2}{rm}{m}{n}{<-> upzhserif-h}{}
\DeclareFontShape{JY2}{rm}{m}{it}{<-> upzhserifit-h}{}
\DeclareFontShape{JY2}{rm}{bx}{n}{<-> upzhserifb-h}{}
\DeclareFontShape{JT2}{rm}{m}{n}{<-> upzhserif-v}{}
\DeclareFontShape{JT2}{rm}{m}{it}{<-> upzhserifit-v}{}
\DeclareFontShape{JT2}{rm}{bx}{n}{<-> upzhserifb-v}{}
%</uptex|aptex>
%    \end{macrocode}
%
%    \begin{macrocode}
%<*sf>
%<*c19>
\DeclareFontShape{C19}{sf}{m}{n}{<-> CJK * gbkyou}{\CJKnormal}
\DeclareFontShape{C19}{sf}{b}{n}{<-> CJKb * gbkyou}{\CJKbold}
\DeclareFontShape{C19}{sf}{bx}{n}{<-> CJKb * gbkyou}{\CJKbold}
\DeclareFontShape{C19}{sf}{m}{sl}{<-> CJK * gbkyousl}{\CJKnormal}
\DeclareFontShape{C19}{sf}{b}{sl}{<-> CJKb * gbkyousl}{\CJKbold}
\DeclareFontShape{C19}{sf}{bx}{sl}{<-> CJKb * gbkyousl}{\CJKbold}
\DeclareFontShape{C19}{sf}{m}{it}{<-> CJK * gbkyou}{\CJKnormal}
\DeclareFontShape{C19}{sf}{b}{it}{<-> CJKb * gbkyou}{\CJKbold}
\DeclareFontShape{C19}{sf}{bx}{it}{<-> CJKb * gbkyou}{\CJKbold}
%</c19>
%<*c70>
\DeclareFontShape{C70}{sf}{m}{n}{<-> CJK * uniyou}{\CJKnormal}
\DeclareFontShape{C70}{sf}{b}{n}{<-> CJKb * uniyou}{\CJKbold}
\DeclareFontShape{C70}{sf}{bx}{n}{<-> CJKb * uniyou}{\CJKbold}
\DeclareFontShape{C70}{sf}{m}{sl}{<-> CJK * uniyousl}{\CJKnormal}
\DeclareFontShape{C70}{sf}{b}{sl}{<-> CJKb * uniyousl}{\CJKbold}
\DeclareFontShape{C70}{sf}{bx}{sl}{<-> CJKb * uniyousl}{\CJKbold}
\DeclareFontShape{C70}{sf}{m}{it}{<-> CJK * uniyou}{\CJKnormal}
\DeclareFontShape{C70}{sf}{b}{it}{<-> CJKb * uniyou}{\CJKbold}
\DeclareFontShape{C70}{sf}{bx}{it}{<-> CJKb * uniyou}{\CJKbold}
%</c70>
%</sf>
%<*uptex|aptex>
\DeclareFontShape{JY2}{sf}{m}{n}{<-> upzhsans-h}{}
\DeclareFontShape{JY2}{sf}{bx}{n}{<-> upzhsansb-h}{}
\DeclareFontShape{JT2}{sf}{m}{n}{<-> upzhsans-v}{}
\DeclareFontShape{JT2}{sf}{bx}{n}{<-> upzhsansb-v}{}
%</uptex|aptex>
%    \end{macrocode}
%
%    \begin{macrocode}
%<*tt>
%<*c19>
\DeclareFontShape{C19}{tt}{m}{n}{<-> CJK * gbkfs}{\CJKnormal}
\DeclareFontShape{C19}{tt}{b}{n}{<-> CJKb * gbkfs}{\CJKbold}
\DeclareFontShape{C19}{tt}{bx}{n}{<-> CJKb * gbkfs}{\CJKbold}
\DeclareFontShape{C19}{tt}{m}{sl}{<-> CJK * gbkfssl}{\CJKnormal}
\DeclareFontShape{C19}{tt}{b}{sl}{<-> CJKb * gbkfssl}{\CJKbold}
\DeclareFontShape{C19}{tt}{bx}{sl}{<-> CJKb * gbkfssl}{\CJKbold}
\DeclareFontShape{C19}{tt}{m}{it}{<-> CJK * gbkfs}{\CJKnormal}
\DeclareFontShape{C19}{tt}{b}{it}{<-> CJKb * gbkfs}{\CJKbold}
\DeclareFontShape{C19}{tt}{bx}{it}{<-> CJKb * gbkfs}{\CJKbold}
%</c19>
%<*c70>
\DeclareFontShape{C70}{tt}{m}{n}{<-> CJK * unifs}{\CJKnormal}
\DeclareFontShape{C70}{tt}{b}{n}{<-> CJKb * unifs}{\CJKbold}
\DeclareFontShape{C70}{tt}{bx}{n}{<-> CJKb * unifs}{\CJKbold}
\DeclareFontShape{C70}{tt}{m}{sl}{<-> CJK * unifssl}{\CJKnormal}
\DeclareFontShape{C70}{tt}{b}{sl}{<-> CJKb * unifssl}{\CJKbold}
\DeclareFontShape{C70}{tt}{bx}{sl}{<-> CJKb * unifssl}{\CJKbold}
\DeclareFontShape{C70}{tt}{m}{it}{<-> CJK * unifs}{\CJKnormal}
\DeclareFontShape{C70}{tt}{b}{it}{<-> CJKb * unifs}{\CJKbold}
\DeclareFontShape{C70}{tt}{bx}{it}{<-> CJKb * unifs}{\CJKbold}
%</c70>
%</tt>
%<*uptex|aptex>
\DeclareFontShape{JY2}{tt}{m}{n}{<-> upzhmono-h}{}
\DeclareFontShape{JT2}{tt}{m}{n}{<-> upzhmono-v}{}
%</uptex|aptex>
%    \end{macrocode}
%
%    \begin{macrocode}
%<*fontset>
%    \end{macrocode}
%
% \subsubsection{\pkg{ctex-fontset-windows.def},
% \pkg{ctex-fontset-windowsnew.def}, \pkg{ctex-fontset-windowsold.def}}
%
% \changes{v2.4.1}{2016/05/14}{使用 \file{bootfont.bin} 判断 Windows XP 以避免
% 权限问题。}
% \pkg{ctex} 包利用 |C:\bootfont.bin| 文件的存在性来判断是否使用 Windows XP 版
% 本,分别载入新旧字体设置。(注:\pkg{ctex} 包不支持 Windows 2000 以前使用
% \file{simsun.ttf} 的旧宋体文件名。)
%    \begin{macrocode}
%<*windows>
\file_if_exist:nTF { C:/bootfont.bin }
  { \ctex_file_input:n { ctex-fontset-windowsold.def } }
  { \ctex_file_input:n { ctex-fontset-windowsnew.def } }
%</windows>
%    \end{macrocode}
%
% 旧的 Windows 字体设置使用黑体作为无衬线体,楷体和仿宋是 GB2312 编码;新的
% Windows 字体设置使用微软雅黑作为无衬线体,楷体和仿宋是大字库。
% Windows 8 以后,微软雅黑由原来的 \file{.ttf} 后缀改为 \file{.ttc} 后缀,需要
% 加以区分。
%    \begin{macrocode}
%<*windowsnew>
\tl_new:N \l_@@_msyh_suffix_tl
\tl_set:Nn \l_@@_msyh_suffix_tl { .ttc }
\file_if_exist:nF { C:/Windows/Fonts/msyh.ttc }
  {
    \file_if_exist:nF { msyh.ttc }
      { \tl_set:Nn \l_@@_msyh_suffix_tl { .ttf } }
  }
%</windowsnew>
%<*windowsnew|windowsold>
\sys_if_engine_pdftex:TF
  {
    \ctex_zhmap_case:nnn
      {
        \ctex_punct_set:n { windows }
        \setCJKmainfont
          [ BoldFont = simhei.ttf , ItalicFont = simkai.ttf ] { simsun.ttc }
%<*windowsold>
        \setCJKsansfont { simhei.ttf }
        \ctex_punct_map_family:nn { \CJKsfdefault } { zhhei }
%</windowsold>
%<*windowsnew>
        \setCJKsansfont [ BoldFont = msyhbd\l_@@_msyh_suffix_tl ] { msyh\l_@@_msyh_suffix_tl }
        \setCJKfamilyfont { zhyahei }
          [ BoldFont = msyhbd\l_@@_msyh_suffix_tl ] { msyh\l_@@_msyh_suffix_tl }
        \ctex_punct_map_family:nn { \CJKsfdefault } { zhyahei }
        \ctex_punct_map_bfseries:nn { \CJKsfdefault , zhyahei } { zhyaheib }
%</windowsnew>
        \setCJKmonofont { simfang.ttf }
        \setCJKfamilyfont { zhkai }  { simkai.ttf }
        \setCJKfamilyfont { zhfs }   { simfang.ttf }
        \setCJKfamilyfont { zhsong } { simsun.ttc }
        \setCJKfamilyfont { zhhei }  { simhei.ttf }
        \setCJKfamilyfont { zhli }   { simli.ttf }
        \setCJKfamilyfont { zhyou }  { simyou.ttf }
        \ctex_punct_map_family:nn { \CJKrmdefault } { zhsong }
        \ctex_punct_map_family:nn { \CJKttdefault } { zhfs }
        \ctex_punct_map_itshape:nn { \CJKrmdefault } { zhkai }
        \ctex_punct_map_bfseries:nn { \CJKrmdefault } { zhhei }
      }
      {
        \ctex_load_zhmap:nnnn { rm } { zhhei } { zhfs } { zhwindowsfonts }
        \ctex_punct_set:n { windows }
        \ctex_punct_map_family:nn { \CJKrmdefault } { zhsong }
        \ctex_punct_map_bfseries:nn { \CJKrmdefault } { zhhei }
        \ctex_punct_map_itshape:nn { \CJKrmdefault } { zhkai }
      }
      {
        \tl_set:Nn \CJKrmdefault { rm }
        \tl_set:Nn \CJKsfdefault { sf }
        \tl_set:Nn \CJKttdefault { tt }
      }
  }
  {
    \sys_if_engine_uptex:TF
      {
        \ctex_set_upfonts:nnnnnn
          {simsun.ttc} {simhei.ttf} {simkai.ttf}
%<windowsnew>          {msyh\l_@@_msyh_suffix_tl} {msyhbd\l_@@_msyh_suffix_tl}
%<windowsold>          {simhei.ttf} {simhei.ttf}
          {simfang.ttf}
        \ctex_set_upfamily:nnn { zhsong } { upzhserif } {}
        \ctex_set_upfamily:nnn { zhhei } { upzhserifb } {}
        \ctex_set_upfamily:nnn { zhfs } { upzhmono} {}
        \ctex_set_upfamily:nnn { zhkai } { upzhserifit } {}
%<windowsnew>        \ctex_set_upfamily:nnn { zhyahei } { upzhsans } { upzhsansb }
        \ctex_set_upfamily:nnn { zhli } { upschrm } {}
        \ctex_set_upmap:nnn { upstsl } { simli.ttf } {}
        \ctex_set_upfamily:nnn { zhyou } { upschgt } {}
        \ctex_set_upmap:nnn { upstht } { simyou.ttf } {}
      }
      {
%<*windowsold>
        \setCJKmainfont
          [ BoldFont = SimHei , ItalicFont = KaiTi_GB2312 ] { SimSun }
        \setCJKsansfont { SimHei }
        \setCJKmonofont { FangSong_GB2312 }
        \setCJKfamilyfont { zhkai } { KaiTi_GB2312 }
        \setCJKfamilyfont { zhfs }  { FangSong_GB2312 }
%</windowsold>
%<*windowsnew>
        \setCJKmainfont
          [ BoldFont = SimHei , ItalicFont = KaiTi ] { SimSun }
        \setCJKsansfont
          [ BoldFont = { *~Bold } ] { Microsoft~YaHei }
        \setCJKmonofont { FangSong }
        \setCJKfamilyfont { zhkai } { KaiTi }
        \setCJKfamilyfont { zhfs }  { FangSong }
%</windowsnew>
        \setCJKfamilyfont { zhsong }  { SimSun }
        \setCJKfamilyfont { zhhei }   { SimHei }
        \setCJKfamilyfont { zhli }    { LiSu }
        \setCJKfamilyfont { zhyou }   { YouYuan }
%<*windowsnew>
        \setCJKfamilyfont { zhyahei }
          [ BoldFont = { *~Bold } ] { Microsoft~YaHei }
%</windowsnew>
      }
  }
%</windowsnew|windowsold>
%    \end{macrocode}
%
% \subsubsection{\pkg{ctex-fontset-adobe.def}}
%
%    \begin{macrocode}
%<*adobe>
%    \end{macrocode}
%
% \tn{pdfmapline} 不支持 OpenType 字体,因而 \opt{adobe} 字体集在 pdf 模式下
% 就没有定义。\opt{fandol} 的情况类似。
%    \begin{macrocode}
\sys_if_engine_pdftex:TF
  {
    \sys_if_output_pdf:TF
      { \ctex_fontset_error:n { adobe } }
      {
        \ctex_zhmap_case:nnn
          {
            \setCJKmainfont
              [
                      cmap = UniGB-UTF16-H ,
                  BoldFont = AdobeHeitiStd-Regular.otf ,
                ItalicFont = AdobeKaitiStd-Regular.otf
              ] { AdobeSongStd-Light.otf }
            \setCJKsansfont [ cmap = UniGB-UTF16-H ] { AdobeHeitiStd-Regular.otf }
            \setCJKmonofont [ cmap = UniGB-UTF16-H ] { AdobeFangsongStd-Regular.otf }
            \setCJKfamilyfont { zhsong }
              [ cmap = UniGB-UTF16-H ] { AdobeSongStd-Light.otf }
            \setCJKfamilyfont { zhhei }
              [ cmap = UniGB-UTF16-H ] { AdobeHeitiStd-Regular.otf }
            \setCJKfamilyfont { zhkai }
              [ cmap = UniGB-UTF16-H ] { AdobeKaitiStd-Regular.otf }
            \setCJKfamilyfont { zhfs }
              [ cmap = UniGB-UTF16-H ] { AdobeFangsongStd-Regular.otf }
            \ctex_punct_set:n { adobe }
            \ctex_punct_map_family:nn { \CJKrmdefault } { zhsong }
            \ctex_punct_map_family:nn { \CJKsfdefault } { zhhei }
            \ctex_punct_map_family:nn { \CJKttdefault } { zhfs }
            \ctex_punct_map_itshape:nn { \CJKrmdefault } { zhkai }
            \ctex_punct_map_bfseries:nn { \CJKrmdefault } { zhhei }
          }
          {
            \ctex_load_zhmap:nnnn { rm } { zhhei } { zhfs } { zhadobefonts }
            \ctex_punct_set:n { adobe }
            \ctex_punct_map_family:nn { \CJKrmdefault } { zhsong }
            \ctex_punct_map_bfseries:nn { \CJKrmdefault } { zhhei }
            \ctex_punct_map_itshape:nn { \CJKrmdefault } { zhkai }
          }
          { \ctex_fontset_error:n { adobe } }
      }
  }
  {
    \sys_if_engine_uptex:TF
      {
        \ctex_set_upfonts:nnnnnn
          {AdobeSongStd-Light.otf} {AdobeHeitiStd-Regular.otf} {AdobeKaitiStd-Regular.otf}
          {AdobeHeitiStd-Regular.otf} {AdobeHeitiStd-Regular.otf}
          {AdobeFangsongStd-Regular.otf}
        \ctex_set_upfamily:nnn { zhsong } { upzhserif } {}
        \ctex_set_upfamily:nnn { zhhei } { upzhsans } {}
        \ctex_set_upfamily:nnn { zhfs } { upzhmono} {}
        \ctex_set_upfamily:nnn { zhkai } { upzhserifit } {}
      }
      {
        \setCJKmainfont
          [
              BoldFont = AdobeHeitiStd-Regular ,
            ItalicFont = AdobeKaitiStd-Regular
          ] { AdobeSongStd-Light }
        \setCJKsansfont { AdobeHeitiStd-Regular}
        \setCJKmonofont { AdobeFangsongStd-Regular}
        \setCJKfamilyfont { zhsong } { AdobeSongStd-Light }
        \setCJKfamilyfont { zhhei }  { AdobeHeitiStd-Regular }
        \setCJKfamilyfont { zhfs }   { AdobeFangsongStd-Regular }
        \setCJKfamilyfont { zhkai }  { AdobeKaitiStd-Regular }
      }
  }
%    \end{macrocode}
%
%    \begin{macrocode}
%</adobe>
%<*fandol>
%    \end{macrocode}
%
% \subsubsection{\pkg{ctex-fontset-fandol.def}}
%
%    \begin{macrocode}
\sys_if_engine_pdftex:TF
  {
    \sys_if_output_pdf:TF
      { \ctex_fontset_error:n { fandol } }
      {
        \ctex_zhmap_case:nnn
          {
            \setCJKmainfont
              [
                      cmap = UniGB-UTF16-H ,
                  BoldFont = FandolSong-Bold.otf ,
                ItalicFont = FandolKai-Regular.otf
              ] { FandolSong-Regular.otf }
            \setCJKsansfont
              [
                    cmap = UniGB-UTF16-H ,
                BoldFont = FandolHei-Bold.otf
              ] { FandolHei-Regular.otf }
            \setCJKmonofont [ cmap = UniGB-UTF16-H ] { FandolFang-Regular.otf }
            \setCJKfamilyfont { zhsong }
              [
                    cmap = UniGB-UTF16-H ,
                BoldFont = FandolSong-Bold.otf
              ] { FandolSong-Regular.otf }
            \setCJKfamilyfont { zhhei }
              [
                    cmap = UniGB-UTF16-H ,
                BoldFont = FandolHei-Bold.otf
              ] { FandolHei-Regular.otf }
            \setCJKfamilyfont { zhfs }
              [ cmap = UniGB-UTF16-H ] { FandolFang-Regular.otf }
            \setCJKfamilyfont { zhkai }
              [ cmap = UniGB-UTF16-H ] { FandolKai-Regular.otf }
            \ctex_punct_set:n { fandol }
            \ctex_punct_map_family:nn { \CJKrmdefault } { zhsong }
            \ctex_punct_map_family:nn { \CJKsfdefault } { zhhei }
            \ctex_punct_map_family:nn { \CJKttdefault } { zhfs }
            \ctex_punct_map_itshape:nn { \CJKrmdefault } { zhkai }
            \ctex_punct_map_bfseries:nn { \CJKrmdefault , zhsong } { zhsongb }
            \ctex_punct_map_bfseries:nn { \CJKsfdefault , zhhei } { zhheib }
          }
          {
            \ctex_load_zhmap:nnnn { rm } { zhhei } { zhfs } { zhfandolfonts }
            \ctex_punct_set:n { fandol }
            \ctex_punct_map_family:nn { \CJKrmdefault } { zhsong }
            \ctex_punct_map_bfseries:nn { \CJKrmdefault } { zhhei }
            \ctex_punct_map_itshape:nn { \CJKrmdefault } { zhkai }
          }
          { \ctex_fontset_error:n { fandol } }
      }
  }
  {
    \sys_if_engine_uptex:TF
      {
        \ctex_set_upfonts:nnnnnn
          {FandolSong-Regular.otf} {FandolSong-Bold.otf} {FandolKai-Regular.otf}
          {FandolHei-Regular.otf} {FandolHei-Bold.otf}
          {FandolFang-Regular.otf}
        \ctex_set_upfamily:nnn { zhsong } { upzhserif } { upzhserifb }
        \ctex_set_upfamily:nnn { zhhei } { upzhsans } { upzhsansb }
        \ctex_set_upfamily:nnn { zhfs } { upzhmono} {}
        \ctex_set_upfamily:nnn { zhkai } { upzhserifit } {}
      }
      {
        \setCJKmainfont
          [ Extension = .otf , BoldFont = FandolSong-Bold , ItalicFont = FandolKai-Regular ]
          { FandolSong-Regular }
        \setCJKsansfont [ Extension = .otf , BoldFont = FandolHei-Bold ] { FandolHei-Regular }
        \setCJKmonofont [ Extension = .otf ] { FandolFang-Regular }
        \setCJKfamilyfont { zhsong }
          [ Extension = .otf , BoldFont = FandolSong-Bold ] { FandolSong-Regular }
        \setCJKfamilyfont { zhhei }
          [ Extension = .otf , BoldFont = FandolHei-Bold ] { FandolHei-Regular }
        \setCJKfamilyfont { zhfs }  [ Extension = .otf ] { FandolFang-Regular }
        \setCJKfamilyfont { zhkai } [ Extension = .otf ] { FandolKai-Regular }
      }
  }
%    \end{macrocode}
%
%    \begin{macrocode}
%</fandol>
%<*mac>
%    \end{macrocode}
%
% \subsubsection{\pkg{ctex-fontset-mac.def}}
%
% OS X Mavericks (10.9) 预装的主要简体中文字体如下^^A
% \footnote{\url{http://support.apple.com/kb/HT5944}}:
% \begin{verbatim}
%   /Library/Fonts/Baoli.ttc: 报隶-简,Baoli SC:style=Regular
%   /Library/Fonts/Hannotate.ttc: 手札体-简,Hannotate SC:style=Bold
%   /Library/Fonts/Hannotate.ttc: 手札体-简,Hannotate SC:style=Regular
%   /Library/Fonts/Hanzipen.ttc: 翩翩体-简,HanziPen SC:style=Bold
%   /Library/Fonts/Hanzipen.ttc: 翩翩体-简,HanziPen SC:style=Regular
%   /Library/Fonts/Hei.ttf: Hei
%   /Library/Fonts/Hiragino Sans GB W3.otf: 冬青黑体简体中文 W3,Hiragino Sans GB W3
%   /Library/Fonts/Hiragino Sans GB W6.otf: 冬青黑体简体中文 W6,Hiragino Sans GB W6
%   /Library/Fonts/Kai.ttf: Kai
%   /Library/Fonts/Kaiti.ttc: STKaiti
%   /Library/Fonts/Kaiti.ttc: 楷体-简,Kaiti SC:style=Black
%   /Library/Fonts/Kaiti.ttc: 楷体-简,Kaiti SC:style=Bold
%   /Library/Fonts/Kaiti.ttc: 楷体-简,Kaiti SC:style=Regular
%   /Library/Fonts/Lantinghei.ttc: 兰亭黑-简,Lantinghei SC:style=Demibold
%   /Library/Fonts/Lantinghei.ttc: 兰亭黑-简,Lantinghei SC:style=Heavy
%   /Library/Fonts/Lantinghei.ttc: 兰亭黑-简,Lantinghei SC:style=Extralight
%   /Library/Fonts/Libian.ttc: 隶变-简,Libian SC
%   /Library/Fonts/Songti.ttc: STSong
%   /Library/Fonts/Songti.ttc: 宋体-简,Songti SC:style=Black
%   /Library/Fonts/Songti.ttc: 宋体-简,Songti SC:style=Bold
%   /Library/Fonts/Songti.ttc: 宋体-简,Songti SC:style=Light
%   /Library/Fonts/Songti.ttc: 宋体-简,Songti SC:style=Regular
%   /Library/Fonts/WawaSC-Regular.otf: 娃娃体-简,Wawati SC
%   /Library/Fonts/WeibeiSC-Bold.otf: 魏碑-简,Weibei SC
%   /Library/Fonts/Xingkai.ttc: 行楷-简,Xingkai SC:style=Bold
%   /Library/Fonts/Xingkai.ttc: 行楷-简,Xingkai SC:style=Light
%   /Library/Fonts/Yuanti.ttc: 圆体-简,Yuanti SC:style=Bold
%   /Library/Fonts/Yuanti.ttc: 圆体-简,Yuanti SC:style=Light
%   /Library/Fonts/Yuanti.ttc: 圆体-简,Yuanti SC:style=Regular
%   /Library/Fonts/YuppySC-Regular.otf: 雅痞-简,Yuppy SC
%   /Library/Fonts/华文仿宋.ttf: STFangsong
%   /Library/Fonts/华文细黑.ttf: STHeiti:style=Light
%   /Library/Fonts/华文黑体.ttf: STHeiti:style=Regular
%   /System/Library/Fonts/STHeiti Light.ttc: 黑体-简,Heiti SC:style=Light
%   /System/Library/Fonts/STHeiti Medium.ttc: 黑体-简,Heiti SC:style=Medium
% \end{verbatim}
% 在 \dvipdfmx{} 下,可以通过下述方式使用华文宋体和华文楷体:
% \begin{verbatim}
%   \special{pdf:mapline unisong@Unicode@ unicode :4:Songti.ttc}
%   \special{pdf:mapline unikai@Unicode@  unicode :4:Kaiti.ttc}
% \end{verbatim}
% 而 \tn{pdfmapline} 似乎不支持带索引的 \texttt{ttc} 字体,\file{Songti.ttc} 默认
% 使用的是 Songti SC Black,\file{Kaiti.ttc} 默认使用的是 Kaiti SC Black。
% 华文黑体不能通过这种方式使用:
% \begin{verbatim}
%   \special{pdf:mapline unihei@Unicode@ unicode \detokenize{华文黑体}.ttf}
% \end{verbatim}
% \dvipdfmx{} 不能生成 PDF,报下述错误:
% \begin{verbatim}
%   ** WARNING ** UCS-4 TrueType cmap table...
%   ** ERROR ** Unable to read OpenType/TrueType Unicode cmap table.
% \end{verbatim}
% 如果将 CMap 改为 UniGB-UTF16-H,错误信息是
% \begin{verbatim}
%   ** WARNING ** No usable TrueType cmap table found for font "华文黑体.ttf".
%   ** WARNING ** CID character collection for this font is set to "Adobe-GB1"
%   ** ERROR ** Cannot continue without this...
% \end{verbatim}
% 在 \pdfTeX{} 下生成的 PDF 只有方框^^A
% \footnote{\url{http://www.newsmth.net/bbscon.php?bid=460&id=312640}}。
% 华文细黑和华文仿宋的情况类似。
% 这确认为 \dvipdfmx{} 的 bug^^A
% \begingroup\makeatletter\@makeother\#\@firstofone{\endgroup
% \footnote{\url{https://github.com/clerkma/ptex-ng/blob/master/texk/libdpx/cidtype2.c#L597}}}。
%    \begin{macrocode}
\sys_if_engine_pdftex:TF
  { \ctex_fontset_error:n { mac } }
  {
    \sys_if_engine_uptex:TF
      { \ctex_fontset_error:n { mac } }
      {
        \setCJKmainfont [ BoldFont = STHeiti , ItalicFont = STKaiti ]  { STSong }
        \setCJKsansfont [ BoldFont = STHeiti ] { STXihei }
        \setCJKmonofont { STFangsong }
        \setCJKfamilyfont { zhsong } { STSong }
        \setCJKfamilyfont { zhhei }  { STHeiti }
        \setCJKfamilyfont { zhfs }   { STFangsong }
        \setCJKfamilyfont { zhkai }  { STKaiti }
      }
  }
%    \end{macrocode}
%
%    \begin{macrocode}
%</mac>
%<*founder>
%    \end{macrocode}
%
% \subsubsection{\pkg{ctex-fontset-founder.def}}
%
%    \begin{macrocode}
\sys_if_engine_pdftex:TF
  {
    \ctex_zhmap_case:nnn
      {
        \setCJKmainfont
          [ BoldFont = FZXBSK.TTF , ItalicFont = FZKTK.TTF ] { FZSSK.TTF }
        \setCJKsansfont [ BoldFont = FZHTK.TTF ] { FZXH1K.TTF }
        \setCJKmonofont { FZFSK.TTF }
        \setCJKfamilyfont { zhsong } [ BoldFont = FZXBSK.TTF ] { FZSSK.TTF }
        \setCJKfamilyfont { zhhei }  { FZHTK.TTF }
        \setCJKfamilyfont { zhkai }  { FZKTK.TTF }
        \setCJKfamilyfont { zhfs }   { FZFSK.TTF }
        \setCJKfamilyfont { zhli }   { FZLSK.TTF }
        \setCJKfamilyfont { zhyou } [ BoldFont = FZY3K.TTF ] { FZY1K.TTF }
        \ctex_punct_set:n { founder }
        \ctex_punct_map_family:nn { \CJKrmdefault } { zhsong }
        \ctex_punct_map_family:nn { \CJKsfdefault } { zhheil }
        \ctex_punct_map_family:nn { \CJKttdefault } { zhfs }
        \ctex_punct_map_itshape:nn { \CJKrmdefault } { zhkai }
        \ctex_punct_map_bfseries:nn { \CJKrmdefault , zhsong } { zhsongb }
        \ctex_punct_map_bfseries:nn { \CJKsfdefault } { zhhei }
        \ctex_punct_map_bfseries:nn { zhyou } { zhyoub }
      }
      {
        \ctex_load_zhmap:nnnn { rm } { zhhei } { zhfs } { zhfounderfonts }
        \ctex_punct_set:n { founder }
        \ctex_punct_map_family:nn { \CJKrmdefault } { zhsong }
        \ctex_punct_map_bfseries:nn { \CJKrmdefault } { zhhei }
        \ctex_punct_map_itshape:nn { \CJKrmdefault } { zhkai }
      }
      { \ctex_fontset_error:n { founder } }
  }
  {
    \sys_if_engine_uptex:TF
      {
        \ctex_set_upfonts:nnnnnn
          { FZSSK.TTF } { FZXBSK.TTF } { FZKTK.TTF }
          { FZXH1K.TTF } { FZHTK.TTF }
          { FZFSK.TTF }
        \ctex_set_upfamily:nnn { zhsong } { upzhserif } { upzhserifb }
        \ctex_set_upfamily:nnn { zhhei } { upzhsans } { upzhsansb }
        \ctex_set_upfamily:nnn { zhfs } { upzhmono} {}
        \ctex_set_upfamily:nnn { zhkai } { upzhserifit } {}
        \ctex_set_upfamily:nnn { zhli } { upschrm } {}
        \ctex_set_upmap:nnn { upstsl } { FZLSK.TTF } {}
        \ctex_set_upfamily:nnn { zhyou } { upschgt } {}
        \ctex_set_upmap:nnn { upstht } { FZY1K.TTF } {}
      }
      {
        \setCJKmainfont
          [ BoldFont = FZXiaoBiaoSong-B05 , ItalicFont = FZKai-Z03 ] { FZShuSong-Z01 }
%    \end{macrocode}
%
% 在 WPS For Linux 附带的 5.00 版和家庭版 5.20 版的“方正细黑一\_GBK”的字体名称
% 是 |FZXiHeiI-Z08|。但在网上发现不少文档和资料都是 \verb*|FZXiHei I-Z08|,而在
% 官方资料《2013 方正字库字体样张》中对应的英文名字是 \verb*|FZXiHei I|。可以用
% Postscript 名字 |FZXH1K--GBK1-0| 来统一。经测试时发现(WPS 中的字体),\XeTeX{}
% 找该字体时会出现明显的延迟,而用字体文件名 |FZXH1K.TTF| 又可能会出现大小写问题,
% 遂采用汉字名称。由于 \pkg{luaotfload} 不支持汉字名称,故使用 Postscript 名字,
% 速度不受影响。
%    \begin{macrocode}
        \setCJKsansfont [ BoldFont = FZHei-B01 ]
          { \sys_if_engine_xetex:TF { 方正细黑一_GBK } { FZXH1K--GBK1-0 } }
        \setCJKmonofont { FZFangSong-Z02 }
        \setCJKfamilyfont { zhsong } [ BoldFont = FZXiaoBiaoSong-B05 ] { FZShuSong-Z01 }
        \setCJKfamilyfont { zhhei }  { FZHei-B01 }
        \setCJKfamilyfont { zhkai }  { FZKai-Z03 }
        \setCJKfamilyfont { zhfs }   { FZFangSong-Z02 }
        \setCJKfamilyfont { zhli }   { FZLiShu-S01 }
        \setCJKfamilyfont { zhyou } [ BoldFont = FZZhunYuan-M02 ] { FZXiYuan-M01 }
      }
  }
%    \end{macrocode}
%
%    \begin{macrocode}
%</founder>
%<*ubuntu>
%    \end{macrocode}
%
% \subsubsection{\pkg{ctex-fontset-ubuntu.def}}
%
% 以下根据 Ubuntu 12.04 的中文字体情况设置。CMap 不清楚应该是什么,指定为
% UniGB-UTF16-H 还是有警告:
% \begin{verbatim}
%   ** WARNING ** UCS-4 TrueType cmap table...
% \end{verbatim}
% 需要注意的是 \file{uming.ttc} 和 \file{ukai.ttc} 看起来像有四种字形的样子,但
% 其实只有“令”和“骨”这区区两个字有新字形,其余都取旧字形^^A
% \footnote{\url{http://www.freedesktop.org/wiki/Software/CJKUnifonts/Download/}}。
%    \begin{macrocode}
\sys_if_engine_pdftex:TF
  {
    \ctex_zhmap_case:nnn
      {
        \setCJKmainfont
          [ BoldFont = wqy-zenhei.ttc , ItalicFont = ukai.ttc ] { uming.ttc }
        \setCJKsansfont { wqy-zenhei.ttc }
        \setCJKmonofont { uming.ttc }
        \setCJKfamilyfont { zhsong } { uming.ttc }
        \setCJKfamilyfont { zhhei }  { wqy-zenhei.ttc }
        \setCJKfamilyfont { zhkai }  { ukai.ttc }
        \ctex_punct_set:n { ubuntu }
        \ctex_punct_map_family:nn { \CJKrmdefault } { zhsong }
        \ctex_punct_map_family:nn { \CJKsfdefault } { zhhei }
        \ctex_punct_map_family:nn { \CJKttdefault } { zhsong }
        \ctex_punct_map_itshape:nn { \CJKrmdefault } { zhkai }
        \ctex_punct_map_bfseries:nn { \CJKrmdefault } { zhhei }
      }
      {
        \ctex_load_zhmap:nnnn { rm } { zhhei } { zhsong } { zhubuntufonts }
        \ctex_punct_set:n { ubuntu }
        \ctex_punct_map_family:nn { \CJKrmdefault } { zhsong }
        \ctex_punct_map_bfseries:nn { \CJKrmdefault } { zhhei }
        \ctex_punct_map_itshape:nn { \CJKrmdefault } { zhkai }
      }
      { \ctex_fontset_error:n { ubuntu } }
  }
  {
    \sys_if_engine_uptex:TF
      {
        \ctex_set_upfonts:nnnnnn
          { uming.ttc } { wqy-zenhei.ttc } { ukai.ttc }
          { wqy-zenhei.ttc } { wqy-zenhei.ttc }
          { uming.ttc }
        \ctex_set_upfamily:nnn { zhsong } { upzhserif } {}
        \ctex_set_upfamily:nnn { zhhei } { upzhsans } {}
        \ctex_set_upfamily:nnn { zhkai } { upzhserifit } {}
      }
      {
        \setCJKmainfont
          [ BoldFont = WenQuanYi~Zen~Hei , ItalicFont = AR~PL~UKai~CN ] { AR~PL~UMing~CN }
        \setCJKsansfont { WenQuanYi~Zen~Hei }
        \setCJKmonofont { AR~PL~UMing~CN }
        \setCJKfamilyfont { zhsong } { AR~PL~UMing~CN }
        \setCJKfamilyfont { zhhei }  { WenQuanYi~Zen~Hei }
        \setCJKfamilyfont { zhkai }  { AR~PL~UKai~CN }
      }
  }
%    \end{macrocode}
%
%    \begin{macrocode}
%</ubuntu>
%    \end{macrocode}
%
% \subsubsection{中文字体命令}
%
%    \begin{macrocode}
%<*!windows>
%    \end{macrocode}
%
%    \begin{macrocode}
\NewDocumentCommand \songti   { } { \CJKfamily { zhsong } }
\NewDocumentCommand \heiti    { } { \CJKfamily { zhhei } }
%<!ubuntu>\NewDocumentCommand \fangsong { } { \CJKfamily { zhfs } }
\NewDocumentCommand \kaishu   { } { \CJKfamily { zhkai } }
%<*windowsnew|windowsold|founder>
\NewDocumentCommand \lishu    { } { \CJKfamily { zhli } }
\NewDocumentCommand \youyuan  { } { \CJKfamily { zhyou } }
%</windowsnew|windowsold|founder>
%<windowsnew>\NewDocumentCommand \yahei    { } { \CJKfamily { zhyahei } }
%    \end{macrocode}
%
%    \begin{macrocode}
%</!windows>
%</fontset>
%<*zhmap>
%    \end{macrocode}
%
% \subsubsection{\pkg{zhmetrics} 的字体映射}
%
% 确认 \tn{catcode},没有重复载入检查。
%    \begin{macrocode}
\begingroup\catcode61\catcode48\catcode32=10\relax%
  \catcode 35=6 % #
  \catcode123=1 % {
  \catcode125=2 % }
  \toks0{\endlinechar=\the\endlinechar\relax}%
  \toks2{\endlinechar=13 }%
  \def\x#1 #2 {%
    \toks0\expandafter{\the\toks0 \catcode#1=\the\catcode#1\relax}%
    \toks2\expandafter{\the\toks2 \catcode#1=#2 }}%
  \x  13  5 % carriage return
  \x  32 10 % space
  \x  35  6 % #
  \x  40 12 % (
  \x  41 12 % )
  \x  45 12 % -
  \x  46 12 % .
  \x  47 12 % /
  \x  58 12 % :
  \x  60 12 % <
  \x  61 12 % =
  \x  64 11 % @
  \x  91 12 % [
  \x  93 12 % ]
  \x 123  1 % {
  \x 125  2 % }
  \edef\x#1{\endgroup%
    \edef\noexpand#1{%
      \the\toks0 %
      \let\noexpand\noexpand\noexpand#1%
          \noexpand\noexpand\noexpand\undefined%
      \noexpand\noexpand\noexpand\endinput}%
    \the\toks2}%
\expandafter\x\csname ctex@zhmap@endinput\endcsname
%    \end{macrocode}
%
%    \begin{macrocode}
\begingroup\expandafter\endgroup
\expandafter\let\csname ifzhmappdf\expandafter\endcsname\csname
  \expandafter\ifx\csname ifctexpdf\endcsname\relax
    \expandafter\ifx\csname pdfoutput\endcsname\relax
      iffalse\else\ifnum\pdfoutput < 1 iffalse\else iftrue\fi\fi
  \else ifctexpdf\fi
\endcsname
%    \end{macrocode}
%
% 提供非 \LaTeX{} 格式下的 \tn{ProvidesFile}。
%    \begin{macrocode}
\begingroup
\expandafter\ifx\csname ProvidesFile\endcsname\relax
  \long\def\x#1\ProvidesFile#2[#3]{%
    #1%
    \immediate\write-1{File: #2 #3}%
    \expandafter\xdef\csname ver@#2\endcsname{#3}}
  \expandafter\x%
\fi
\endgroup
%    \end{macrocode}
%
% \paragraph{\pkg{zhwindowsfonts.tex}}
%
%    \begin{macrocode}
%<*windows>
\ProvidesFile{zhwindowsfonts.tex}%
  [2017/04/01 v2.4.9 Windows font map loader for pdfTeX and DVIPDFMx (CTEX)]

\ifzhmappdf
  \pdfmapline{=gbk@UGBK@     <simsun.ttc}
  \pdfmapline{=gbksong@UGBK@ <simsun.ttc}
  \pdfmapline{=gbkkai@UGBK@  <simkai.ttf}
  \pdfmapline{=gbkhei@UGBK@  <simhei.ttf}
  \pdfmapline{=gbkfs@UGBK@   <simfang.ttf}
  \pdfmapline{=gbkli@UGBK@   <simli.ttf}
  \pdfmapline{=gbkyou@UGBK@  <simyou.ttf}

  \pdfmapline{=cyberb@Unicode@  <simsun.ttc}
  \pdfmapline{=unisong@Unicode@ <simsun.ttc}
  \pdfmapline{=unikai@Unicode@  <simkai.ttf}
  \pdfmapline{=unihei@Unicode@  <simhei.ttf}
  \pdfmapline{=unifs@Unicode@   <simfang.ttf}
  \pdfmapline{=unili@Unicode@   <simli.ttf}
  \pdfmapline{=uniyou@Unicode@  <simyou.ttf}

  \pdfmapline{=gbksongsl@UGBK@ <simsun.ttc}
  \pdfmapline{=gbkkaisl@UGBK@  <simkai.ttf}
  \pdfmapline{=gbkheisl@UGBK@  <simhei.ttf}
  \pdfmapline{=gbkfssl@UGBK@   <simfang.ttf}
  \pdfmapline{=gbklisl@UGBK@   <simli.ttf}
  \pdfmapline{=gbkyousl@UGBK@  <simyou.ttf}

  \pdfmapline{=unisongsl@Unicode@ <simsun.ttc}
  \pdfmapline{=unikaisl@Unicode@  <simkai.ttf}
  \pdfmapline{=uniheisl@Unicode@  <simhei.ttf}
  \pdfmapline{=unifssl@Unicode@   <simfang.ttf}
  \pdfmapline{=unilisl@Unicode@   <simli.ttf}
  \pdfmapline{=uniyousl@Unicode@  <simyou.ttf}

\else
  \special{pdf:mapline gbk@UGBK@     unicode :0:simsun.ttc -v 50}
  \special{pdf:mapline gbksong@UGBK@ unicode :0:simsun.ttc -v 50}
  \special{pdf:mapline gbkkai@UGBK@  unicode simkai.ttf -v 70}
  \special{pdf:mapline gbkhei@UGBK@  unicode simhei.ttf -v 150}
  \special{pdf:mapline gbkfs@UGBK@   unicode simfang.ttf -v 50}
  \special{pdf:mapline gbkli@UGBK@   unicode simli.ttf -v 150}
  \special{pdf:mapline gbkyou@UGBK@  unicode simyou.ttf -v 60}

  \special{pdf:mapline cyberb@Unicode@  unicode :0:simsun.ttc -v 50}
  \special{pdf:mapline unisong@Unicode@ unicode :0:simsun.ttc -v 50}
  \special{pdf:mapline unikai@Unicode@  unicode simkai.ttf -v 70}
  \special{pdf:mapline unihei@Unicode@  unicode simhei.ttf -v 150}
  \special{pdf:mapline unifs@Unicode@   unicode simfang.ttf -v 50}
  \special{pdf:mapline unili@Unicode@   unicode simli.ttf -v 150}
  \special{pdf:mapline uniyou@Unicode@  unicode simyou.ttf -v 60}

  \special{pdf:mapline gbksongsl@UGBK@ unicode :0:simsun.ttc -s .167 -v 50}
  \special{pdf:mapline gbkkaisl@UGBK@  unicode simkai.ttf -s .167 -v 70}
  \special{pdf:mapline gbkheisl@UGBK@  unicode simhei.ttf -s .167 -v 150}
  \special{pdf:mapline gbkfssl@UGBK@   unicode simfang.ttf -s .167 -v 50}
  \special{pdf:mapline gbklisl@UGBK@   unicode simli.ttf -s .167 -v 150}
  \special{pdf:mapline gbkyousl@UGBK@  unicode simyou.ttf -s .167 -v 60}

  \special{pdf:mapline unisongsl@Unicode@ unicode :0:simsun.ttc -s .167 -v 50}
  \special{pdf:mapline unikaisl@Unicode@  unicode simkai.ttf -s .167 -v 70}
  \special{pdf:mapline uniheisl@Unicode@  unicode simhei.ttf -s .167 -v 150}
  \special{pdf:mapline unifssl@Unicode@   unicode simfang.ttf -s .167 -v 50}
  \special{pdf:mapline unilisl@Unicode@   unicode simli.ttf -s .167 -v 150}
  \special{pdf:mapline uniyousl@Unicode@  unicode simyou.ttf -s .167 -v 60}

%</windows>
%    \end{macrocode}
%
% \paragraph{\pkg{zhadobefonts.tex}}
%
%    \begin{macrocode}
%<*adobe>
\ProvidesFile{zhadobefonts.tex}%
  [2017/04/01 v2.4.9 Adobe font map loader for DVIPDFMx (CTEX)]

\ifzhmappdf
%% pdfTeX does not support OTF fonts

\else
  \special{pdf:mapline gbk@UGBK@     UniGB-UTF16-H AdobeSongStd-Light.otf}
  \special{pdf:mapline gbksong@UGBK@ UniGB-UTF16-H AdobeSongStd-Light.otf}
  \special{pdf:mapline gbkkai@UGBK@  UniGB-UTF16-H AdobeKaitiStd-Regular.otf}
  \special{pdf:mapline gbkhei@UGBK@  UniGB-UTF16-H AdobeHeitiStd-Regular.otf}
  \special{pdf:mapline gbkfs@UGBK@   UniGB-UTF16-H AdobeFangsongStd-Regular.otf}

  \special{pdf:mapline cyberb@Unicode@  UniGB-UTF16-H AdobeSongStd-Light.otf}
  \special{pdf:mapline unisong@Unicode@ UniGB-UTF16-H AdobeSongStd-Light.otf}
  \special{pdf:mapline unikai@Unicode@  UniGB-UTF16-H AdobeKaitiStd-Regular.otf}
  \special{pdf:mapline unihei@Unicode@  UniGB-UTF16-H AdobeHeitiStd-Regular.otf}
  \special{pdf:mapline unifs@Unicode@   UniGB-UTF16-H AdobeFangsongStd-Regular.otf}

  \special{pdf:mapline gbksongsl@UGBK@ UniGB-UTF16-H AdobeSongStd-Light.otf -s .167}
  \special{pdf:mapline gbkkaisl@UGBK@  UniGB-UTF16-H AdobeKaitiStd-Regular.otf -s .167}
  \special{pdf:mapline gbkheisl@UGBK@  UniGB-UTF16-H AdobeHeitiStd-Regular.otf -s .167}
  \special{pdf:mapline gbkfssl@UGBK@   UniGB-UTF16-H AdobeFangsongStd-Regular.otf -s .167}

  \special{pdf:mapline unisongsl@Unicode@ UniGB-UTF16-H AdobeSongStd-Light.otf -s .167}
  \special{pdf:mapline unikaisl@Unicode@  UniGB-UTF16-H AdobeKaitiStd-Regular.otf -s .167}
  \special{pdf:mapline uniheisl@Unicode@  UniGB-UTF16-H AdobeHeitiStd-Regular.otf -s .167}
  \special{pdf:mapline unifssl@Unicode@   UniGB-UTF16-H AdobeFangsongStd-Regular.otf -s .167}

%</adobe>
%    \end{macrocode}
%
% \paragraph{\pkg{zhfandolfonts.tex}}
%
%    \begin{macrocode}
%<*fandol>
\ProvidesFile{zhfandolfonts.tex}%
  [2017/04/01 v2.4.9 Fandol font map loader for DVIPDFMx (CTEX)]

\ifzhmappdf
%% pdfTeX does not support OTF fonts

\else
  \special{pdf:mapline gbk@UGBK@     UniGB-UTF16-H FandolSong-Regular.otf}
  \special{pdf:mapline gbksong@UGBK@ UniGB-UTF16-H FandolSong-Regular.otf}
  \special{pdf:mapline gbkkai@UGBK@  UniGB-UTF16-H FandolKai-Regular.otf}
  \special{pdf:mapline gbkhei@UGBK@  UniGB-UTF16-H FandolHei-Regular.otf}
  \special{pdf:mapline gbkfs@UGBK@   UniGB-UTF16-H FandolFang-Regular.otf}

  \special{pdf:mapline cyberb@Unicode@  UniGB-UTF16-H FandolSong-Regular.otf}
  \special{pdf:mapline unisong@Unicode@ UniGB-UTF16-H FandolSong-Regular.otf}
  \special{pdf:mapline unikai@Unicode@  UniGB-UTF16-H FandolKai-Regular.otf}
  \special{pdf:mapline unihei@Unicode@  UniGB-UTF16-H FandolHei-Regular.otf}
  \special{pdf:mapline unifs@Unicode@   UniGB-UTF16-H FandolFang-Regular.otf}

  \special{pdf:mapline gbksongsl@UGBK@ UniGB-UTF16-H FandolSong-Regular.otf -s .167}
  \special{pdf:mapline gbkkaisl@UGBK@  UniGB-UTF16-H FandolKai-Regular.otf -s .167}
  \special{pdf:mapline gbkheisl@UGBK@  UniGB-UTF16-H FandolHei-Regular.otf -s .167}
  \special{pdf:mapline gbkfssl@UGBK@   UniGB-UTF16-H FandolFang-Regular.otf -s .167}

  \special{pdf:mapline unisongsl@Unicode@ UniGB-UTF16-H FandolSong-Regular.otf -s .167}
  \special{pdf:mapline unikaisl@Unicode@  UniGB-UTF16-H FandolKai-Regular.otf -s .167}
  \special{pdf:mapline uniheisl@Unicode@  UniGB-UTF16-H FandolHei-Regular.otf -s .167}
  \special{pdf:mapline unifssl@Unicode@   UniGB-UTF16-H FandolFang-Regular.otf -s .167}

%</fandol>
%    \end{macrocode}
%
% \paragraph{\pkg{zhfounderfonts.tex}}
%
%    \begin{macrocode}
%<*founder>
\ProvidesFile{zhfounderfonts.tex}%
  [2017/04/01 v2.4.9 Founder font map loader for pdfTeX and DVIPDFMx (CTEX)]

\ifzhmappdf
  \pdfmapline{=gbk@UGBK@     <FZSSK.TTF}
  \pdfmapline{=gbksong@UGBK@ <FZSSK.TTF}
  \pdfmapline{=gbkkai@UGBK@  <FZKTK.TTF}
  \pdfmapline{=gbkhei@UGBK@  <FZHTK.TTF}
  \pdfmapline{=gbkfs@UGBK@   <FZFSK.TTF}
  \pdfmapline{=gbkli@UGBK@   <FZLSK.TTF}
  \pdfmapline{=gbkyou@UGBK@  <FZY1K.TTF}

  \pdfmapline{=cyberb@Unicode@  <FZSSK.TTF}
  \pdfmapline{=unisong@Unicode@ <FZSSK.TTF}
  \pdfmapline{=unikai@Unicode@  <FZKTK.TTF}
  \pdfmapline{=unihei@Unicode@  <FZHTK.TTF}
  \pdfmapline{=unifs@Unicode@   <FZFSK.TTF}
  \pdfmapline{=unili@Unicode@   <FZLSK.TTF}
  \pdfmapline{=uniyou@Unicode@  <FZY1K.TTF}

  \pdfmapline{=gbksongsl@UGBK@ <FZSSK.TTF}
  \pdfmapline{=gbkkaisl@UGBK@  <FZKTK.TTF}
  \pdfmapline{=gbkheisl@UGBK@  <FZHTK.TTF}
  \pdfmapline{=gbkfssl@UGBK@   <FZFSK.TTF}
  \pdfmapline{=gbklisl@UGBK@   <FZLSK.TTF}
  \pdfmapline{=gbkyousl@UGBK@  <FZY1K.TTF}

  \pdfmapline{=unisongsl@Unicode@ <FZSSK.TTF}
  \pdfmapline{=unikaisl@Unicode@  <FZKTK.TTF}
  \pdfmapline{=uniheisl@Unicode@  <FZHTK.TTF}
  \pdfmapline{=unifssl@Unicode@   <FZFSK.TTF}
  \pdfmapline{=unilisl@Unicode@   <FZLSK.TTF}
  \pdfmapline{=uniyousl@Unicode@  <FZY1K.TTF}

\else
  \special{pdf:mapline gbk@UGBK@     unicode FZSSK.TTF}
  \special{pdf:mapline gbksong@UGBK@ unicode FZSSK.TTF}
  \special{pdf:mapline gbkkai@UGBK@  unicode FZKTK.TTF}
  \special{pdf:mapline gbkhei@UGBK@  unicode FZHTK.TTF}
  \special{pdf:mapline gbkfs@UGBK@   unicode FZFSK.TTF}
  \special{pdf:mapline gbkli@UGBK@   unicode FZLSK.TTF}
  \special{pdf:mapline gbkyou@UGBK@  unicode FZY1K.TTF}

  \special{pdf:mapline cyberb@Unicode@  unicode FZSSK.TTF}
  \special{pdf:mapline unisong@Unicode@ unicode FZSSK.TTF}
  \special{pdf:mapline unikai@Unicode@  unicode FZKTK.TTF}
  \special{pdf:mapline unihei@Unicode@  unicode FZHTK.TTF}
  \special{pdf:mapline unifs@Unicode@   unicode FZFSK.TTF}
  \special{pdf:mapline unili@Unicode@   unicode FZLSK.TTF}
  \special{pdf:mapline uniyou@Unicode@  unicode FZY1K.TTF}

  \special{pdf:mapline gbksongsl@UGBK@ unicode FZSSK.TTF -s .167}
  \special{pdf:mapline gbkkaisl@UGBK@  unicode FZKTK.TTF -s .167}
  \special{pdf:mapline gbkheisl@UGBK@  unicode FZHTK.TTF -s .167}
  \special{pdf:mapline gbkfssl@UGBK@   unicode FZFSK.TTF -s .167}
  \special{pdf:mapline gbklisl@UGBK@   unicode FZLSK.TTF -s .167}
  \special{pdf:mapline gbkyousl@UGBK@  unicode FZY1K.TTF -s .167}

  \special{pdf:mapline unisongsl@Unicode@ unicode FZSSK.TTF -s .167}
  \special{pdf:mapline unikaisl@Unicode@  unicode FZKTK.TTF -s .167}
  \special{pdf:mapline uniheisl@Unicode@  unicode FZHTK.TTF -s .167}
  \special{pdf:mapline unifssl@Unicode@   unicode FZFSK.TTF -s .167}
  \special{pdf:mapline unilisl@Unicode@   unicode FZLSK.TTF -s .167}
  \special{pdf:mapline uniyousl@Unicode@  unicode FZY1K.TTF -s .167}

%</founder>
%    \end{macrocode}
%
% \paragraph{\pkg{zhubuntufonts.tex}}
%
%    \begin{macrocode}
%<*ubuntu>
\ProvidesFile{zhubuntufonts.tex}%
  [2017/04/01 v2.4.9 Ubuntu font map loader for pdfTeX and DVIPDFMx (CTEX)]

\ifzhmappdf
  \pdfmapline{=gbk@UGBK@     <uming.ttc}
  \pdfmapline{=gbksong@UGBK@ <uming.ttc}
  \pdfmapline{=gbkkai@UGBK@  <ukai.ttc}
  \pdfmapline{=gbkhei@UGBK@  <wqy-zenhei.ttc}
  \pdfmapline{=gbkfs@UGBK@   <uming.ttc}
  \pdfmapline{=gbkyou@UGBK@  <wqy-zenhei.ttc}

  \pdfmapline{=cyberb@Unicode@  <uming.ttc}
  \pdfmapline{=unisong@Unicode@ <uming.ttc}
  \pdfmapline{=unikai@Unicode@  <ukai.ttc}
  \pdfmapline{=unihei@Unicode@  <wqy-zenhei.ttc}
  \pdfmapline{=unifs@Unicode@   <uming.ttc}
  \pdfmapline{=uniyou@Unicode@  <wqy-zenhei.ttc}

  \pdfmapline{=gbksongsl@UGBK@ <uming.ttc}
  \pdfmapline{=gbkkaisl@UGBK@  <ukai.ttc}
  \pdfmapline{=gbkheisl@UGBK@  <wqy-zenhei.ttc}
  \pdfmapline{=gbkfssl@UGBK@   <uming.ttc}
  \pdfmapline{=gbkyousl@UGBK@  <wqy-zenhei.ttc}

  \pdfmapline{=unisongsl@Unicode@ <uming.ttc}
  \pdfmapline{=unikaisl@Unicode@  <ukai.ttc}
  \pdfmapline{=uniheisl@Unicode@  <wqy-zenhei.ttc}
  \pdfmapline{=unifssl@Unicode@   <uming.ttc}
  \pdfmapline{=uniyousl@Unicode@  <wqy-zenhei.ttc}

\else
  \special{pdf:mapline gbk@UGBK@     unicode :0:uming.ttc}
  \special{pdf:mapline gbksong@UGBK@ unicode :0:uming.ttc}
  \special{pdf:mapline gbkkai@UGBK@  unicode :0:ukai.ttc}
  \special{pdf:mapline gbkhei@UGBK@  unicode :0:wqy-zenhei.ttc}
  \special{pdf:mapline gbkfs@UGBK@   unicode :0:uming.ttc}

  \special{pdf:mapline cyberb@Unicode@  unicode :0:uming.ttc}
  \special{pdf:mapline unisong@Unicode@ unicode :0:uming.ttc}
  \special{pdf:mapline unikai@Unicode@  unicode :0:ukai.ttc}
  \special{pdf:mapline unihei@Unicode@  unicode :0:wqy-zenhei.ttc}
  \special{pdf:mapline unifs@Unicode@   unicode :0:uming.ttc}

  \special{pdf:mapline gbksongsl@UGBK@ unicode :0:uming.ttc -s .167}
  \special{pdf:mapline gbkkaisl@UGBK@  unicode :0:ukai.ttc -s .167}
  \special{pdf:mapline gbkheisl@UGBK@  unicode :0:wqy-zenhei.ttc -s .167}
  \special{pdf:mapline gbkfssl@UGBK@   unicode :0:uming.ttc -s .167}

  \special{pdf:mapline unisongsl@Unicode@ unicode :0:uming.ttc -s .167}
  \special{pdf:mapline unikaisl@Unicode@  unicode :0:ukai.ttc -s .167}
  \special{pdf:mapline uniheisl@Unicode@  unicode :0:wqy-zenhei.ttc -s .167}
  \special{pdf:mapline unifssl@Unicode@   unicode :0:uming.ttc -s .167}

%</ubuntu>
%    \end{macrocode}
%
%    \begin{macrocode}
\fi

\ctex@zhmap@endinput
%</zhmap>
%    \end{macrocode}
%
% \subsubsection{制作 \texttt{spa} 文件}
%
% 我们通过 \XeTeX{} 的 \tn{XeTeXglyphbounds} 取得字体中标点符号的边界信息,为
% \pkg{CJKpunct} 宏包制作 \file{spa}。
%
%    \begin{macrocode}
%<*spa>
%<*macro>
\input expl3-generic %
\ExplSyntaxOn
\sys_if_engine_xetex:F
  {
    \msg_new:nn { ctex } { xetex }
      { XeTeX~is~required~to~compile~this~document! }
    \msg_fatal:nn { ctex } { xetex }
  }
%    \end{macrocode}
%
% \pkg{CJKpunct} 定义的标点符号是:
% \begin{verbatim}
%   ‘“「『〔([{〈《〖【
%   —…、。,.:;!?%〕)]}〉》〗】’”」』
% \end{verbatim}
% 注意顺序不能改变。
%    \begin{macrocode}
\seq_new:N \c_@@_punct_seq
\seq_gset_from_clist:Nn \c_@@_punct_seq
  {
    "2018 , "201C , "300C , "300E , "3014 , "FF08 , "FF3B , "FF5B ,
    "3008 , "300A , "3016 , "3010 ,
    "2014 , "2026 , "3001 , "3002 , "FF0C , "FF0E , "FF1A , "FF1B ,
    "FF01 , "FF1F , "FF05 , "3015 , "FF09 , "FF3D , "FF5D , "3009 ,
    "300B , "3017 , "3011 , "2019 , "201D , "300D , "300F
  }
%    \end{macrocode}
%
% \begin{macro}[int]{\ctex_make_spa:nn}
% |#1| 是 \file{spa} 文件名,|#2| 是由 CJK 族名与字体构成的逗号列表。
%    \begin{macrocode}
\cs_new_protected_nopar:Npn \ctex_make_spa:nn #1#2
  {
    \iow_open:Nn \g_@@_spa_iow {#1}
    \clist_map_inline:nn {#2}
      { \@@_write_family:nn ##1 }
    \iow_close:N \g_@@_spa_iow
  }
\iow_new:N \g_@@_spa_iow
\cs_new_eq:NN \MAKESPA \ctex_make_spa:nn
%    \end{macrocode}
% \end{macro}
%
%    \begin{macrocode}
\cs_new_protected:Npn \@@_write_family:nn #1#2
  {
    \group_begin:
      \tex_font:D \l_@@_punct_font = "#2" ~ at ~ 100 pt \scan_stop:
      \l_@@_punct_font
      \clist_clear:N \l_@@_punct_bounds_clist
      \seq_map_inline:Nn \c_@@_punct_seq
        { \exp_args:No \@@_save_bounds:n { \int_use:N \xetex_charglyph:D ##1 } }
      \iow_now:Nx \g_@@_spa_iow
        {
          \token_to_str:N \ctexspadef {#1}
%    \end{macrocode}
% 最后这三个逗号对 \pkg{CJKpunct} 来说是必要的。
%    \begin{macrocode}
            { \l_@@_punct_bounds_clist , , , }
        }
    \group_end:
  }
\cs_new_protected_nopar:Npn \@@_save_bounds:n #1
  {
    \clist_put_right:Nx \l_@@_punct_bounds_clist
      {
        \@@_calc_bounds:nn { \c_one }   {#1} ,
        \@@_calc_bounds:nn { \c_three } {#1}
      }
  }
\clist_new:N \l_@@_punct_bounds_clist
%    \end{macrocode}
%
% \pkg{CJKpunct} 要求的格式是边界空白宽度与 1\,em 的比值的一百倍。
%    \begin{macrocode}
\cs_new_nopar:Npn \@@_calc_bounds:nn #1#2
  {
    \fp_eval:n
      {
        round
          (
            \dim_to_decimal_in_unit:nn
              { 100 \xetex_glyphbounds:D #1 ~ #2 }
              { 1 em }
          )
      }
  }
\ExplSyntaxOff
%</macro>
%    \end{macrocode}
%
% 下面是 \CTeX{} 定义的一些字体。
%    \begin{macrocode}
%<*make>
\input ctexspamacro %

\MAKESPA {ctexpunct.spa}
  {
    {adobezhsong}     {AdobeSongStd-Light} ,
    {adobezhhei}      {AdobeHeitiStd-Regular} ,
    {adobezhkai}      {AdobeKaitiStd-Regular} ,
    {adobezhfs}       {AdobeFangsongStd-Regular} ,
    {fandolzhsong}    {FandolSong} ,
    {fandolzhsongb}   {FandolSong-Bold} ,
    {fandolzhhei}     {FandolHei} ,
    {fandolzhheib}    {FandolHei-Bold} ,
    {fandolzhkai}     {FandolKai} ,
    {fandolzhfs}      {FandolFang} ,
    {founderzhsong}   {FZShuSong-Z01} ,
    {founderzhsongb}  {FZXiaoBiaoSong-B05} ,
    {founderzhhei}    {FZHei-B01} ,
    {founderzhheil}   {FZXiHeiI-Z08} ,
    {founderzhkai}    {FZKai-Z03} ,
    {founderzhfs}     {FZFangSong-Z02} ,
    {founderzhli}     {FZLiShu-S01} ,
    {founderzhyou}    {FZXiYuan-M01} ,
    {founderzhyoub}   {FZZhunYuan-M02} ,
    {ubuntuzhsong}    {AR PL UMing CN} ,
    {ubuntuzhhei}     {WenQuanYi Zen Hei} ,
    {ubuntuzhkai}     {AR PL UKai CN} ,
    {windowszhsong}   {SimSun} ,
    {windowszhhei}    {SimHei} ,
    {windowszhkai}    {KaiTi} ,
    {windowszhfs}     {FangSong} ,
    {windowszhli}     {LiSu} ,
    {windowszhyou}    {YouYuan} ,
    {windowszhyahei}  {Microsoft YaHei} ,
    {windowszhyaheib} {Microsoft YaHei Bold}
  }

\primitive\end
%</make>
%</spa>
%    \end{macrocode}
%
% \subsection{\pkg{translator} 宏包的中文字典}
%
%    \begin{macrocode}
%<*dict>
%    \end{macrocode}
%
% \changes{v2.4}{2016/02/19}{提供 \pkg{translator} 宏包的中文定理名称翻译。}
%
% 包括 \pkg{ChineseGBK} 和 \pkg{ChineseUTF8} 两种形式,目前只翻译 \pkg{beamer}
% 宏包需要的定理环境名称。
%
%    \begin{macrocode}
%<*theorem>
\providetranslation{Comments}{评论}
\providetranslation{comments}{评论}
\providetranslation{Comment}{评论}
\providetranslation{comment}{评论}
\providetranslation{Corollaries}{推论}
\providetranslation{corollaries}{推论}
\providetranslation{Corollary}{推论}
\providetranslation{corollary}{推论}
\providetranslation{Definitions}{定义}
\providetranslation{definitions}{定义}
\providetranslation{Definition}{定义}
\providetranslation{definition}{定义}
\providetranslation{Examples}{例}
\providetranslation{examples}{例}
\providetranslation{Example}{例}
\providetranslation{example}{例}
\providetranslation{Exercises}{练习}
\providetranslation{exercises}{练习}
\providetranslation{Exercise}{练习}
\providetranslation{exercise}{练习}
\providetranslation{Facts}{事实}
\providetranslation{facts}{事实}
\providetranslation{Fact}{事实}
\providetranslation{fact}{事实}
\providetranslation{Key Lemmas}{关键引理}
\providetranslation{key lemmas}{关键引理}
\providetranslation{Key Lemma}{关键引理}
\providetranslation{key lemma}{关键引理}
\providetranslation{Key Observations}{关键观察}
\providetranslation{key observations}{关键观察}
\providetranslation{Key Observation}{关键观察}
\providetranslation{key observation}{关键观察}
\providetranslation{Lemmas}{引理}
\providetranslation{lemmas}{引理}
\providetranslation{Lemma}{引理}
\providetranslation{lemma}{引理}
\providetranslation{Main Theorems}{主要定理}
\providetranslation{main theorems}{主要定理}
\providetranslation{Main Theorem}{主要定理}
\providetranslation{main theorem}{主要定理}
\providetranslation{Observations}{观察}
\providetranslation{observations}{观察}
\providetranslation{Observation}{观察}
\providetranslation{observation}{观察}
\providetranslation{Problems}{问题}
\providetranslation{problems}{问题}
\providetranslation{Problem}{问题}
\providetranslation{problem}{问题}
\providetranslation{Proofs}{证明}
\providetranslation{proofs}{证明}
\providetranslation{Proof}{证明}
\providetranslation{proof}{证明}
\providetranslation{Proof Sketch}{证明提要}
\providetranslation{Proof sketch}{证明提要}
\providetranslation{proof sketch}{证明提要}
\providetranslation{Proof Sketches}{证明提要}
\providetranslation{Proof sketches}{证明提要}
\providetranslation{proof sketches}{证明提要}
\providetranslation{Sketch of Proof}{证明提要}
\providetranslation{Sketch of Proofs}{证明提要}
\providetranslation{Sketch of proof}{证明提要}
\providetranslation{Sketch of proofs}{证明提要}
\providetranslation{sketch of proof}{证明提要}
\providetranslation{sketch of proofs}{证明提要}
\providetranslation{Propositions}{命题}
\providetranslation{propositions}{命题}
\providetranslation{Proposition}{命题}
\providetranslation{proposition}{命题}
\providetranslation{Remarks}{注}
\providetranslation{remarks}{注}
\providetranslation{Remark}{注}
\providetranslation{remark}{注}
\providetranslation{Solutions}{解}
\providetranslation{solutions}{解}
\providetranslation{Solution}{解}
\providetranslation{solution}{解}
\providetranslation{Theorems}{定理}
\providetranslation{theorems}{定理}
\providetranslation{Theorem}{定理}
\providetranslation{theorem}{定理}
%</theorem>
%    \end{macrocode}
%
%    \begin{macrocode}
%</dict>
%    \end{macrocode}
%
% \subsection{\pkg{ctexcap} 宏包}
%
%    \begin{macrocode}
%<*ctexcap>
%    \end{macrocode}
%
% \pkg{ctexcap} 是过时宏包。
%    \begin{macrocode}
\clist_new:N \l_@@_ctexcap_options_clist
\clist_set:Nx \l_@@_ctexcap_options_clist
  { \exp_not:v { opt@ \@currname . \@currext } , heading }
\msg_new:nnn { ctexcap } { deprecated }
  {
    Package~`ctexcap'~is~deprecated.\\
    Please~use~package~`ctex'~with~option~`#1'~instead: \\\\
    \iow_indent:n { \token_to_str:N \usepackage [#1] \{ ctex \} } \\
  }
\msg_warning:nnx { ctexcap } { deprecated }
  { \clist_use:Nn \l_@@_ctexcap_options_clist { , ~ } }
%    \end{macrocode}
%
% \pkg{ctexcap} 是默认打开 \opt{heading} 选项的 \pkg{ctex}。
%    \begin{macrocode}
\PassOptionsToPackage { heading = true } { ctexcap }
\RequirePackageWithOptions { ctex }
%    \end{macrocode}
%
%    \begin{macrocode}
%</ctexcap>
%    \end{macrocode}
%
%
% \subsection{\pkg{ctexhook} 宏包}
%
% \changes{v2.2}{2015/06/26}{将文档开头和宏包末尾钩子提取到 \pkg{ctexhook} 宏包中。}
%
%    \begin{macrocode}
%<*ctexhook>
%    \end{macrocode}
%
% \begin{macro}[int]{\ctex_at_end_preamble:n,\ctex_after_end_preamble:n}
% 实现 \pkg{etoolbox} 宏包的 \tn{AtEndPreamble} 和 \tn{AfterEndPreamble}。
%    \begin{macrocode}
\cs_new_protected:Npn \ctex_at_end_preamble:n #1
  { \tl_gput_right:Nn \g_@@_end_preamble_hook_tl {#1} }
\cs_new_protected:Npn \ctex_after_end_preamble:n #1
  { \tl_gput_right:Nn \g_@@_after_end_preamble_hook_tl {#1} }
\cs_new_protected_nopar:Npn \CTEX@document@left@hook
  { \group_end: \g_@@_end_preamble_hook_tl \group_begin: }
\cs_new_protected_nopar:Npn \CTEX@document@right@hook
  { \scan_stop: \g_@@_after_end_preamble_hook_tl \tex_ignorespaces:D }
\cs_set_nopar:Npx \document
  {
    \CTEX@document@left@hook
    \exp_not:o { \document }
    \CTEX@document@right@hook
  }
\tl_new:N \g_@@_end_preamble_hook_tl
\tl_new:N \g_@@_after_end_preamble_hook_tl
%    \end{macrocode}
% \end{macro}
%
% \begin{macro}[int]{\ctex_at_end_package:nn}
% 与 \pkg{filehook} 的 \tn{AtEndOfPackageFile*} 类似,如果原来没有在载入宏包则
% 在宏包末尾执行语句,否则立即执行。
%    \begin{macrocode}
\cs_new_protected:Npn \ctex_at_end_package:nn #1#2
  {
    \@ifpackageloaded {#1}
      {#2}
      { \ctex_gadd_hook:cn { g_@@_at_end_ #1 _hook_tl } {#2} }
  }
%    \end{macrocode}
% \end{macro}
%
% \begin{macro}[int]{\ctex_gadd_hook:Nn, \ctex_gadd_hook:cn}
% 给钩子附加内容。
%    \begin{macrocode}
\cs_new_protected:Npn \ctex_gadd_hook:Nn #1#2
  {
    \tl_if_exist:NF #1 { \tl_new:N #1 }
    \tl_gput_right:Nn #1 {#2}
  }
\cs_generate_variant:Nn \ctex_gadd_hook:Nn { c }
%    \end{macrocode}
% \end{macro}
%
% \begin{macro}[int]{\ctex_package_end_hook:n, \ctex_package_end_hook:o}
% 宏包末尾钩子,只执行一次,用后清除。
%    \begin{macrocode}
\cs_new_protected_nopar:Npn \ctex_package_end_hook:n #1
  {
    \cs_if_exist_use:cT { g_@@_at_end_ #1 _hook_tl }
      { \cs_undefine:c { g_@@_at_end_ #1 _hook_tl } }
  }
\cs_generate_variant:Nn \ctex_package_end_hook:n { o }
%    \end{macrocode}
% \end{macro}
%
% 对 \tn{@popfilename} 做补丁来实现 \cs{ctex_at_end_package:nn} 的功能。
%    \begin{macrocode}
\tl_put_left:Nn \@popfilename
  {
    \cs_if_eq:NNT \@currext \@pkgextension
      { \ctex_package_end_hook:o { \@currname } }
  }
%    \end{macrocode}
%
%    \begin{macrocode}
%</ctexhook>
%    \end{macrocode}
%
%
% \subsection{\pkg{ctexpatch} 宏包}
%
% \changes{v2.2}{2015/06/23}{新增子宏包 \pkg{ctexpatch} 实现给宏打补丁功能。}
%
%    \begin{macrocode}
%<*ctexpatch>
%    \end{macrocode}
%
%    \begin{macrocode}
\cs_if_exist:NF \str_new:N { \RequirePackage { l3str } }
%    \end{macrocode}
%
% \begin{macro}[int]{\ctex_patch_cmd_once:NnnnTF}
% 只进行第一次匹配进行替换。参数 |#2| 是宏重建时的 \tn{catcode} 设置。
%    \begin{macrocode}
\cs_new_protected:Npn \ctex_patch_cmd_once:NnnnTF #1#2
  {
    \ctex_patch_boot:NNnnTF \@@_patch_cmd:Nnnnnw #1
      { once } {#2} { \use_i:nn } { \use_ii:nn }
  }
%    \end{macrocode}
% \end{macro}
%
% \begin{macro}[int]{\ctex_patch_cmd_all:NnnnTF}
% 替换所有匹配到的文本。
%    \begin{macrocode}
\cs_new_protected:Npn \ctex_patch_cmd_all:NnnnTF #1#2
  {
    \ctex_patch_boot:NNnnTF \@@_patch_cmd:Nnnnnw #1
      { all } {#2} { \use_i:nn } { \use_ii:nn }
  }
%    \end{macrocode}
% \end{macro}
%
% \begin{macro}[int]{\ctex_patch_cmd:Nnn}
% 快捷方式,在补丁的时候关闭 \LaTeXiii{} 语法和设置 |@| 为字母类,补丁失败时给出警告。
%    \begin{macrocode}
\cs_new_protected:Npn \ctex_patch_cmd:Nnn #1
  {
    \ctex_patch_boot:NNnnTF \@@_patch_cmd:Nnnnnw #1
      { once }
      {
        \ExplSyntaxOff
        \char_set_catcode_letter:n { 64 }
      }
      { }
      { \ctex_patch_failure:N #1 }
  }
\cs_new_protected:Npn \ctex_patch_failure:N #1
  { \msg_warning:nnx { ctex } { patch-failure } { \token_to_str:N #1 } }
\msg_new:nnn { ctex } { patch-failure }
  { Oops!~Command~`#1'~is~NOT~patchable.\\ }
%    \end{macrocode}
% \end{macro}
%
% \begin{macro}[int]{\ctex_preto_cmd:NnnTF}
% 在宏的原本定义前面增加钩子。
%    \begin{macrocode}
\cs_new_protected:Npn \ctex_preto_cmd:NnnTF #1#2
  {
    \ctex_patch_boot:NNnnTF \@@_hookto_cmd:Nnnnw #1
      { left } {#2} { \use_i:nn } { \use_ii:nn }
  }
%    \end{macrocode}
% \end{macro}
%
% \begin{macro}[int]{\ctex_appto_cmd:NnnTF}
% 在宏的原本定义后面追加钩子。
%    \begin{macrocode}
\cs_new_protected:Npn \ctex_appto_cmd:NnnTF #1#2
  {
    \ctex_patch_boot:NNnnTF \@@_hookto_cmd:Nnnnw #1
      { right } {#2} { \use_i:nn } { \use_ii:nn }
  }
%    \end{macrocode}
% \end{macro}
%
% \begin{macro}[int]{\ctex_patch_boot:NNnnTF}
% 参数记号 |#| 作为宏的参数被读入时,总是会双写,会影响随后的字符串替换。需要先
% 将它转换为普通符号。
%    \begin{macrocode}
\cs_new_protected:Npn \ctex_patch_boot:NNnnTF #1#2#3#4#5#6
  {
    \tl_set:Nn \@@_patch_true:w {#5}
    \tl_set:Nn \@@_patch_false:w {#6}
    \group_begin:
      \char_set_catcode_other:n { 35 }
      \ctex_parse_name:NN #1 #2 {#3} {#4}
  }
\tl_new:N \@@_patch_true:w
\tl_new:N \@@_patch_false:w
%    \end{macrocode}
% \end{macro}
%
% \begin{macro}[int]{\ctex_parse_name:NN}
% \changes{v2.4}{2016/04/11}{修复宏名解析错误。}
% 用 \tn{DeclareRobustCommand} 定义的宏或者由 \tn{newcommand} 或 \tn{newrobustcmd}
% 定义的带一个可选参数的宏第一次展开的结果都不是其实际定义,实际定义被保存在另外的
% 宏中。由这些命令定义的宏的第一次展开结果可以有下面的形式(细节可查阅 \pkg{xpatch}
% 的文档):
% \begin{verbatim}[numbers=left,gobble=4]
%   \protect␣\xaa␣␣                    % \DeclareRobustCommand\xaa[1]{...}
%   \protect␣\xab␣␣                    % \DeclareRobustCommand\xab[1][]{...}
%   \@protected@testopt␣\xac␣\\xac␣{}  % \newcommand\xac[1][]{...}
%   \@testopt␣\\xad␣{}                 % \newrobustcmd\xad[1][]{...}
%   \x@protect␣\1\protect␣\1␣␣         % \DeclareRobustCommand\1[1]{...}
%   \x@protect␣\2\protect␣\2␣␣         % \DeclareRobustCommand\2[1][]{...}
%   \@protected@testopt␣\3\\3␣{}       % \newcommand\3[1][]{...}
%   \@testopt␣\\4␣{}                   % \newrobustcmd\4[1][]{...}
% \end{verbatim}
% \pkg{ctexpatch} 的主要原理是先对宏的 \tn{meaning} 作字符串替换,然后再用
% \tn{scantokens} 来重建它。我们希望对宏的实际定义打补丁,为此需要先得到
% 对应的名字。\pkg{letltxmacro}、\pkg{show2e} 和 \pkg{xpatch} 宏包中都有
% 类似的工作。
%    \begin{macrocode}
\cs_new_protected:Npn \ctex_parse_name:NN #1#2
  { \ctex_parse_name:NNx #1#2 { \cs_to_str:N #2 } }
\group_begin:
\cs_set_protected:Npn \@@_tmp:w #1#2#3
  {
    \cs_new_protected:Npn \ctex_parse_name:NNn ##1##2##3
      {
        \bool_lazy_or:nnTF
          { \cs_if_exist_p:c { ##3 ~ } }
          { \cs_if_exist_p:c { #1##3 } }
          {
            \group_begin:
            \use:x
              {
                \group_end:
                \@@_parse_name:nNNNnN
                  { \token_get_replacement_spec:N ##2 }
                  \exp_not:N ##2
                  \exp_not:c { ##3 ~ }
                  \exp_not:c { #1##3 }
              } {##3} ##1
          }
          { ##1##2 }
      }
    \cs_new_protected:Npn \@@_parse_name:nNNNnN ##1##2##3##4##5##6
      {
        \exp_args:Nc ##6
          {
            \str_case:nnTF {##1}
              {
                { \protect ##3 } { }
                { \x@protect ##2 \protect ##3 } { }
              }
              {
                \str_if_eq_x:nnTF
                  { \exp_not:n { #1@protected@ ##3 #1##3 } }
                  {
                    \exp_last_unbraced:Nf \@@_parse_name:w
                      \token_get_replacement_spec:N ##3 #3 ~ #2 \q_stop
                  }
                  { #1##5 ~ } { ##5 ~ }
              }
              {
                \str_case:onTF { \@@_parse_name:w ##1 #3 ~ #2 \q_stop }
                  {
                    { #1@protected@ ##2 ##4 } { }
                    { #1@ ##4 } { }
                  }
                  { #1##5 } {##5}
              }
          }
      }
    \cs_new:Npn \@@_parse_name:w ##1 #3 ~ ##2 #2 ##3 \q_stop { ##1##2 }
  }
\use:x
  {
    \@@_tmp:w
      { \c_backslash_str }
      { \c_left_brace_str }
      { \tl_to_str:n { testopt } }
  }
\group_end:
\cs_generate_variant:Nn \ctex_parse_name:NNn { NNx }
%    \end{macrocode}
% \end{macro}
%
% \begin{variable}[int]{\l_@@_prefix_str,\l_@@_parameter_str,\l_@@_replacement_str}
% 分别保存宏的 \tn{meaning} 中的前缀、参数文本和替换文本。
%    \begin{macrocode}
\str_new:N \l_@@_prefix_str
\str_new:N \l_@@_parameter_str
\str_new:N \l_@@_replacement_str
%    \end{macrocode}
% \end{variable}
%
% \begin{macro}[int]{\ctex_get_macro_meaning:NTF}
% \begin{macro}[aux]{\@@_get_macro_meaning:w}
% 解构待补丁宏的 \tn{meaning}。若命令不是宏,则走向 |false| 分支。
%    \begin{macrocode}
\group_begin:
  \cs_set_protected:Npn \@@_tmp:w #1
    {
      \prg_new_protected_conditional:Npnn
        \ctex_get_macro_meaning:N ##1 { TF }
        {
          \exp_after:wN \@@_get_macro_meaning:w
            \token_to_meaning:N ##1 \q_mark #1 -> \q_mark \q_stop
        }
      \cs_new_protected:Npn \@@_get_macro_meaning:w
          ##1 #1 ##2 -> ##3 \q_mark ##4 \q_stop
        {
          \tl_if_empty:nTF { ##4 }
            { \prg_return_false: }
            {
              \str_set:Nn \l_@@_prefix_str      { ##1 }
              \str_set:Nn \l_@@_parameter_str   { ##2 }
              \str_set:Nn \l_@@_replacement_str { ##3 }
              \prg_return_true:
            }
        }
    }
  \exp_args:No \@@_tmp:w { \tl_to_str:n { macro: } }
\group_end:
%    \end{macrocode}
% \end{macro}
% \end{macro}
%
% \begin{macro}[int]{\ctex_if_rescanable:NnTF}
% 检查宏是否可以重建。
%    \begin{macrocode}
\cs_new_protected:Npn \ctex_if_rescanable:NnTF #1#2#3#4
  {
    \ctex_get_macro_meaning:NTF #1
      {
        \@@_patch_rebuild:Nn \@@_rebuild_cmd:w {#2}
        \cs_if_eq:NNTF #1 \@@_rebuild_cmd:w {#3} {#4}
      }
      {#4}
  }
\cs_new_eq:NN \@@_rebuild_cmd:w \prg_do_nothing:
%    \end{macrocode}
% \end{macro}
%
% \begin{macro}[int]{\@@_patch_rebuild:Nn}
% 使用 \cs{tl_rescan:nn} 来重新记号化 \tn{meaning} 字符串。
%    \begin{macrocode}
\cs_new_protected:Npn \@@_patch_rebuild:Nn #1#2
  {
    \@@_patch_rescan:NNn \l_@@_prefix_tl      \l_@@_prefix_str      {#2}
    \@@_patch_rescan:NNn \l_@@_parameter_tl   \l_@@_parameter_str   {#2}
    \@@_patch_rescan:NNn \l_@@_replacement_tl \l_@@_replacement_str {#2}
    \use:x
      {
        \exp_not:o { \l_@@_prefix_tl } \tex_def:D \exp_not:N #1
          \exp_not:o { \l_@@_parameter_tl }
            { \exp_not:o { \l_@@_replacement_tl } }
      }
  }
\cs_new_protected:Npn \@@_patch_rescan:NNn #1#2#3
  {
    \str_if_empty:NTF #2
      { \tl_clear:N #1 }
      { \tl_set_rescan:Nno #1 {#3} {#2} }
  }
\tl_new:N \l_@@_prefix_tl
\tl_new:N \l_@@_parameter_tl
\tl_new:N \l_@@_replacement_tl
%    \end{macrocode}
% \end{macro}
%
% \begin{macro}[int]{\@@_patch_cmd:Nnnnnw}
% 对宏的替换文本进行字符串替换,然后重建。
%    \begin{macrocode}
\cs_new_protected:Npn \@@_patch_cmd:Nnnnnw #1#2#3#4#5
  {
    \group_end:
    \ctex_if_rescanable:NnTF #1 {#3}
      {
        \use:x
          {
            \@@_patch_replace:nnnTF {#2}
              { \tl_to_str:n {#4} }
              { \tl_to_str:n {#5} }
          }
          {
            \@@_patch_rebuild:Nn #1 {#3}
            \@@_patch_true:w
          }
          { \@@_patch_false:w }
      }
      { \@@_patch_false:w }
  }
%    \end{macrocode}
% \end{macro}
%
% \begin{macro}[int]{\@@_patch_replace:nnnTF}
% 替换前先检查原文本是否存在。
%    \begin{macrocode}
\cs_new_protected:Npn \@@_patch_replace:nnnTF #1#2#3#4
  {
    \tl_if_in:NnTF \l_@@_replacement_str {#2}
      { \use:c { tl_replace_ #1 :Nnn } \l_@@_replacement_str {#2} {#3} #4 }
  }
%    \end{macrocode}
% \end{macro}
%
% \begin{macro}[int]{\@@_hookto_cmd:Nnnnw}
% 在宏的前/后附加钩子。
%    \begin{macrocode}
\cs_new_protected:Npn \@@_hookto_cmd:Nnnnw #1#2#3#4
  {
    \group_end:
    \ctex_get_macro_meaning:NTF #1
      {
        \str_if_empty:NTF \l_@@_parameter_str
          { \@@_hookto_cmd_parameterless:Nnnnw }
          { \@@_hookto_cmd_parameter:Nnnnw }
          #1 {#2} {#3} {#4}
      }
      { \@@_patch_false:w }
  }
%    \end{macrocode}
% \end{macro}
%
% \begin{macro}[aux]{\@@_hookto_cmd_parameterless:Nnnnw}
% 如果宏没有参数,可以直接进行附加操作。注意保持宏的前缀。
%    \begin{macrocode}
\cs_new_protected:Npn \@@_hookto_cmd_parameterless:Nnnnw #1#2#3#4
  {
    \str_if_empty:NF \l_@@_prefix_str
      { \tl_rescan:no {#3} { \l_@@_prefix_str } }
    \tex_edef:D #1
      {
        \use:c { @@_ #2 _hook_aux:nn }
          { \exp_not:o {#1} }
          { \exp_not:n {#4} }
      }
    \@@_patch_true:w
  }
\cs_generate_variant:Nn \tl_rescan:nn { no }
\cs_new:Npn \@@_left_hook_aux:nn #1#2 { #2#1 }
\cs_new_eq:NN \@@_right_hook_aux:nn \use:nn
%    \end{macrocode}
% \end{macro}
%
% \begin{macro}[aux]{\@@_hookto_cmd_parameter:Nnnnw}
% 如果宏有参数,需要在字符串中进行附加,然后再重建。
%    \begin{macrocode}
\cs_new_protected:Npn \@@_hookto_cmd_parameter:Nnnnw #1#2#3#4
  {
    \@@_patch_rebuild:Nn \@@_rebuild_cmd:w {#3}
    \cs_if_eq:NNTF #1 \@@_rebuild_cmd:w
      {
        \use:c { str_put_ #2 :Nn } \l_@@_replacement_str {#4}
        \@@_patch_rebuild:Nn #1 {#3}
        \@@_patch_true:w
      }
      { \@@_patch_false:w }
  }
%    \end{macrocode}
% \end{macro}
%
%    \begin{macrocode}
%</ctexpatch>
%    \end{macrocode}
%
% \end{implementation}
%
% \Finale
%
% \endinput
%
% \DisableImplementation
%
% \begin{implementation}
%
% \subsection{\cls{ctxdoc}}
%
%    \begin{macrocode}
%<*ctxdoc>
%    \end{macrocode}
%
%    \begin{macrocode}
%<@@=ctxdoc>
%    \end{macrocode}
%
%    \begin{macrocode}
\ExplSyntaxOff
\let\pdfmdfivesum\mdfivesum
\DeclareOption*{\PassOptionsToClass{\CurrentOption}{l3doc}}
\PassOptionsToClass{a4paper,full}{l3doc}
\ProcessOptions
\@namedef{ver@thumbpdf.sty}{9999/99/99}
\LoadClass{l3doc}
\RequirePackage[UTF8, punct = kaiming, heading, fontset = none,
  linespread = 1.2, sub3section]{ctex}
\ifxetex
  \xeCJKsetup{AutoFakeBold=false}
\fi
\ctexset{
  fontset,
  abstractname   = 简介,
  indexname      = 代码索引,
  section        = {
    format = \Large\bfseries\raggedright,
    name   = {第,节},
  },
}
\RequirePackage[toc]{multitoc}
\RequirePackage{geometry}
\RequirePackage{tabularx}
\RequirePackage{makecell}
\RequirePackage{threeparttable}
\RequirePackage{siunitx}
\RequirePackage{unicode-math}
\RequirePackage{xcolor}
\RequirePackage{caption}
\RequirePackage{fvrb-ex}
\RequirePackage{zref-base}
\geometry{includemp,hmargin={0mm,15mm},vmargin={25mm,15mm},footskip=7mm}
\hypersetup{pdfstartview=FitH,bookmarksdepth=subparagraph}
\setcounter{secnumdepth}{4}
\setcounter{tocdepth}{2}
\newcommand*\email{\nolinkurl}
\setmainfont{TeX Gyre Pagella}
\setsansfont{TeX Gyre Heros}
\setmonofont[
  HyphenChar = None ,
  UprightFont=* Light, BoldFont=* Bold,
  SlantedFont=* Light Oblique]{CMU Typewriter Text}
\setmathfont{texgyrepagella-math.otf}
\captionsetup{strut=off, labelsep=quad, labelfont+=bf}
%% <--- http://tex.stackexchange.com/a/40896
\patchcmd{\@addtocurcol}%
    {\vskip \intextsep}%
    {\edef\save@first@penalty{\the\lastpenalty}\unpenalty
     \ifnum \lastpenalty = \@M  % hopefully the OR penalty
        \unpenalty
     \else
        \penalty \save@first@penalty \relax % put it back
     \fi
      \ifnum\outputpenalty <-\@Mii
                         \addvspace\intextsep
                         \vskip\parskip
      \else
                         \addvspace\intextsep
      \fi}%
    {\typeout{*** SUCCESS ***}}{\typeout{*** FAIL ***}}
\patchcmd{\@addtocurcol}%
    {\vskip\intextsep \ifnum\outputpenalty <-\@Mii \vskip -\parskip\fi}%
    {\ifnum\outputpenalty <-\@Mii
       \aftergroup\vskip\aftergroup\intextsep
       \aftergroup\nointerlineskip
     \else
       \vskip\intextsep
     \fi}%
    {\typeout{*** SUCCESS ***}}{\typeout{*** FAIL ***}}
\patchcmd{\@getpen}{\@M}{\@Mi}
  {\typeout{*** SUCCESS ***}}{\typeout{*** FAIL ***}}
%% --->
%% 不对代码实现的 \section 以下标题编目录。
\AtBeginEnvironment{implementation}{%
  \ifnum\value{tocdepth}>\@ne
    \addtocontents{toc}{\protect\value{tocdepth}=1\relax}%
  \fi}
\ifxetex
  \let\ctexdocverbaddon\xeCJKVerbAddon
  \def\ctexdisableecglue{\xeCJKsetup{CJKecglue}}
  \def\ctexplainps{\xeCJKsetup{PunctStyle=plain}}
  \appto\meta@font@select{\ifinner\ctexdisableecglue\fi}
\else
  \let\ctexdocverbaddon\relax
  \let\ctexplainps\relax
  \def\ctexdisableecglue{\ltjsetparameter{autoxspacing=false}}
  \appto\meta@font@select{\ctexdisableecglue}
\fi
\setlist{noitemsep,topsep=\smallskipamount}
\setlist[1]{labelindent=\parindent}
\setlist[enumerate]{leftmargin=*}
\setlist[itemize]{leftmargin=*}
\newlist{optdesc}{description}{3}
%% 设置间距为 \marginparsep,与 l3doc 一致
\setlist[optdesc]{%
  font=\mdseries\small\ttfamily,align=right,listparindent=\parindent,
  labelsep=\marginparsep,labelindent=-\marginparsep,leftmargin=*}
%% 重新定义 threeparttable 包的 tablenotes 环境
\renewlist{tablenotes}{description}{1}
\setlist[tablenotes]{%
  format=\normalfont\tnote@item,align=right,listparindent=\parindent,
  labelindent=\tabcolsep,leftmargin=*,rightmargin=\tabcolsep,
  after=\@noparlisttrue}
\AtBeginEnvironment{tablenotes}{%
  \setlength\parindent{2\ccwd}%
  \normalfont\footnotesize}
\AtBeginEnvironment{threeparttable}{%
  \stepcounter{tpt@id}%
  \edef\curr@tpt@id{tpt@\arabic{tpt@id}}}
\newcounter{tpt@id}
\def\tnote@item#1{%
  \Hy@raisedlink{\hyper@anchor{\curr@tpt@id-#1}}#1}
\def\TPTtagStyle#1{\hyperlink{\curr@tpt@id-#1}{#1}}
\fvset{
  fontsize=\small,baselinestretch=1,numbersep=5pt,
  formatcom=\ctexdocverbaddon,
  listparameters=\setlength\topsep{\MacrocodeTopsep}}
\DefineVerbatimEnvironment{frameverb}{Verbatim}{%
  gobble=4,
  frame=single,framesep=8pt,
  listparameters=
    \setlength\topsep{\medskipamount}%
    \appto\FV@EndList{\nointerlineskip}}
\DefineVerbatimEnvironment{ctexexam}{Verbatim}{%
  gobble=4,
  frame=single,framesep=10pt,
  label=\rule{0pt}{12pt}\textnormal{\bfseries 例 \arabic{ctexexam}},
  listparameters=
    \setlength\topsep{\bigskipamount}%
    \refstepcounter{ctexexam}\ctexexamlabelref
    \appto\FV@EndList{\nointerlineskip}}
\define@key{FV}{labelref}{\def\ctexexamlabelref{\label{#1}}}
\let\ctexexamlabelref\empty
\newcounter{ctexexam}
\BeforeBeginEnvironment{function}{\par\nointerlineskip}
\AtEndEnvironment{function}{%
  \par\xdef\ctexfixprevdepth{\prevdepth=\the\prevdepth\space}}
\AfterEndEnvironment{function}{\ctexfixprevdepth}
\AtBeginEnvironment{syntax}{\linespread{1}\ctexplainps\ctexdisableecglue}
\BeforeBeginEnvironment{SideBySideExample}{\par\addvspace{\medskipamount}}
\newcommand*\exptarget{\Hy@raisedlink{\hypertarget{expstar}{}}}
\newcommand*\rexptarget{\Hy@raisedlink{\hypertarget{rexpstar}{}}}
\newcommand*\expstar{\hyperlink{expstar}{$\star$}}
\newcommand*\rexpstar{\hyperlink{rexpstar}{\ding{73}}}
\ExplSyntaxOn
%% l3doc 会设置列表环境中 \listparindent=\z@,我们在这里恢复它。
\cs_set_eq:NN \list \__codedoc_oldlist:nn
\cs_new_nopar:Npn \zihaopt #1
  {
    \exp_last_unbraced:Nf \tl_head:w
      { \prop_item:Nn \c__ctex_font_size_prop {#1} } { } \q_stop
  }
%% 抑制首段的 \parskip
\ctex_patch_cmd_once:NnnnTF \__codedoc_function_descr_start:w
  { }
  { \noindent }
  { \skip_vertical:n { -\parskip } \noindent }
  { \iow_term:n { *** ~ SUCCESS ~ *** } }
  { \iow_term:n { *** ~ FAIL ~ *** } }
%% l3doc 会在 function 环境的 syntax 和 descr 盒子中间加上 \medskipamount 的距离。
%% 但是若 syntax 盒子为空(未使用 syntax 环境),就会显得不好看。
%% 此时我们通过将 \medskipamount 设置为零来修正。若盒子非空,则将 \parskip 还回去。
\ctex_preto_cmd:NnnTF \__codedoc_function_assemble:
  { }
  { \ctex_doc_fix_yoffset: }
  { \iow_term:n { *** ~ SUCCESS ~ *** } }
  { \iow_term:n { *** ~ FAIL ~ *** } }
\cs_new_protected_nopar:Npn \ctex_doc_fix_yoffset:
  {
    \box_if_empty:NTF \g__codedoc_syntax_box
      { \skip_zero:N \medskipamount }
      { \skip_add:Nn \medskipamount { \parskip } }
  }
%% 左侧边注的函数列表采用单倍行距
\ctex_preto_cmd:NnnTF \__codedoc_typeset_functions:
  { }
  { \MacroFont }
  { \iow_term:n { *** ~ SUCCESS ~ *** } }
  { \iow_term:n { *** ~ FAIL ~ *** } }
\ctex_patch_cmd_once:NnnnTF \__codedoc_macro_init:
  { }
  { \hbox:n }
  { \MacroFont \hbox:n }
  { \iow_term:n { *** ~ SUCCESS ~ *** } }
  { \iow_term:n { *** ~ FAIL ~ *** } }
\ctex_patch_cmd_once:NnnnTF \__codedoc_macro_dump:
  { }
  { \hbox_unpack_clear:N }
  { \MacroFont \hbox_unpack_clear:N }
  { \iow_term:n { *** ~ SUCCESS ~ *** } }
  { \iow_term:n { *** ~ FAIL ~ *** } }
\cs_set_eq:NN \__codedoc_macro_end_style:n \use_none:n
\cs_set_protected:Npn \__codedoc_macro_typeset_one:nN #1#2
  {
    \vbox_set:Nn \l__codedoc_macro_box
      {
        \MacroFont
        \vbox_unpack_clear:N \l__codedoc_macro_box
        \hbox_set:Nn \l_tmpa_box
          { \__codedoc_print_macroname:nN {#1} #2 }
        \dim_set:Nn \l_tmpa_dim { \marginparwidth - \labelsep }
        \dim_compare:nNnT { \box_wd:N \l_tmpa_box } > \l_tmpa_dim
          {
            \box_resize_to_wd_and_ht:Nnn \l_tmpa_box
              { \l_tmpa_dim }
              { \box_ht:N \l_tmpa_box }
          }
        \hbox_overlap_left:n
          {
            \box_use:N \l_tmpa_box
            \skip_horizontal:n { \marginparsep - \labelsep }
          }
      }
    \int_incr:N \l__codedoc_macro_int
  }
\cs_set_protected:Npn \__codedoc_print_macroname:nN #1#2
  {
    \strut
    \__codedoc_get_hyper_target:xN
      {
        \exp_not:n {#1}
        \bool_if:NT #2 { \tl_to_str:n {TF} }
      }
      \l__codedoc_tmpa_tl
    \cs_if_exist:cTF { r@ \l__codedoc_tmpa_tl }
      { \exp_args:NNo \label@hyperref [ \l__codedoc_tmpa_tl ] }
      { \use:n }
      {
        \tl_set:Nn \l__codedoc_tmpa_tl {#1}
        \tl_replace_all:Non \l__codedoc_tmpa_tl
          { \c_catcode_other_space_tl }
          { \fontspec_visible_space: }
        \__codedoc_macroname_prefix:o \l__codedoc_tmpa_tl
        \__codedoc_macroname_suffix:N #2
      }
  }
\AtBeginEnvironment { syntax }
  {
    \char_set_catcode_active:N \|
    \char_set_active_eq:NN \| \orbar
    \char_set_catcode_active:N \(
    \char_set_active_eq:NN \( \defaultvalaux
  }
%% 不对目录中出现的 \cs 和 \tn 等编索引
\DeclareDocumentCommand \StopSpecialIndexModule { }
  { \cs_set_eq:NN \__codedoc_special_index_module:nnnnN \use_none:nnnnn }
\tl_map_inline:nn { \actualchar \encapchar \levelchar }
  { \exp_args:Nx \DoNotIndex { \bslash \tl_to_str:N #1 } }
\DeclareDocumentCommand \package { o m }
  {
    \exp_args:Nx \href
      {
        http \c_colon_str //www.ctan.org/pkg/
        \IfNoValueTF {#1} { \str_fold_case:n {#2} } {#1}
      }
      { \pkg {#2} }
  }
\DeclareDocumentCommand \GetFileId { m }
  {
    \GetFileInfo {#1}
    \ior_open:NnTF \g_@@_id_ior { \c_sys_jobname_str .id }
      {
        \ior_get:NN \g_@@_id_ior \l_@@_tmp_tl
        \ior_close:N \g_@@_id_ior
        \exp_after:wN \GetIdInfo \l_@@_tmp_tl
      }
      { \GetIdInfo $Id$ }
      { \fileinfo }
  }
\ior_new:N \g_@@_id_ior
\cs_new_eq:NN \@@_ltx_changes:nnn \changes@
\cs_set_protected:Npn \changes@ #1#2
  {
    \@@_save_version_date:nn {#1} {#2}
    \tl_if_empty:nTF {#1}
      { \@@_ltx_changes:nnn }
      { \@@_version_zfill:wnnn #1 \q_stop }
      {#1} {#2}
  }
\cs_new_protected:Npn \@@_version_zfill:wnnn #1#2 \q_stop
  {
    \str_if_eq:nnTF {#1} { v }
      { \@@_version_zfill:nnnn {#2} }
      { \@@_ltx_changes:nnn }
  }
\cs_new_protected:Npn \@@_version_zfill:nnnn #1#2
  {
    \tl_clear:N \l_@@_tmp_tl
    \int_zero:N \l_tmpa_int
    \seq_set_split:Nnn \l_tmpa_seq { . } {#1}
    \seq_map_function:NN \l_tmpa_seq \@@_version_zfill:n
    \int_compare:nNnF \l_tmpa_int > \c_two
      {
        \tl_put_right:Nx \l_@@_tmp_tl
          { \prg_replicate:nn { \c_three - \l_tmpa_int } { 00000 } }
      }
    \@@_ltx_changes:nnn { \l_@@_tmp_tl \actualchar #2 }
  }
\tl_new:N \l_@@_tmp_tl
\cs_new_protected:Npn \@@_version_zfill:n #1
  {
    \int_incr:N \l_tmpa_int
    \tl_put_right:Nx \l_@@_tmp_tl
      {
        \prg_replicate:nn
          { \int_max:nn { 0 } { 5 - \tl_count:n {#1} } } { 0 }
        \exp_not:n {#1}
      }
  }
\cs_new_protected:Npn \@@_save_version_date:nn #1#2
  {
    \prop_get:NnNTF \g_@@_version_date_prop {#1} \l_@@_tmp_tl
      { \exp_after:wN \@@_save_version_date:nnnn \l_@@_tmp_tl {#2} {#1} }
      { \@@_save_version_date:nnn {#1} {#2} {#2} }
  }
\cs_new_protected:Npn \@@_save_version_date:nnnn #1#2#3#4
  {
    \@@_if_date_later:nnTF {#1} {#3}
      { \@@_save_version_date:nnn {#4} {#3} {#2} }
      {
        \@@_if_date_later:nnT {#3} {#2}
          { \@@_save_version_date:nnn {#4} {#1} {#3} }
      }
  }
\prg_new_conditional:Npnn \@@_if_date_later:nn #1#2 { TF , T }
  {
    \if_int_compare:w \@@_parse_date:w #1 / / / 0 \q_stop >
                      \@@_parse_date:w #2 / / / 0 \q_stop \exp_stop_f:
      \prg_return_true: \else: \prg_return_false: \fi:
  }
\cs_new:Npn \@@_parse_date:w #1/#2/#3/ #4 \q_stop
  { #1#2#3 }
\cs_new_protected:Npn \@@_save_version_date:nnn #1#2#3
  { \prop_gput:Nnn \g_@@_version_date_prop {#1} { {#2} {#3} } }
\cs_new_protected:Npn \CTEX@versionitem #1 \efill
  {
    \@idxitem
    \prop_get:NnNTF \g_@@_version_date_prop {#1} \l_@@_tmp_tl
      { \exp_after:wN \@@_version_item:nnn \l_@@_tmp_tl {#1} }
      { \BOOM }
  }
\cs_new_protected:Npn \@@_version_item:nnn #1#2#3
  {
    \noindent
    \Hy@raisedlink { \belowpdfbookmark {#3} { HD.#3 } }
    \textbf {#3} \hfill
    \hbox:n
      {
        \footnotesize
        \str_if_eq:nnTF {#1} {#2}
          { ( #1 ) }
          { ( #1 ~ -- ~ #2 ) }
      }
    \par \nopagebreak
  }
\prop_new:N \g_@@_version_date_prop
\ctex_patch_cmd:Nnn \HDorg@theglossary
  { \let \item \@idxitem }
  { \let \item \CTEX@versionitem }
\ctex_patch_cmd:Nnn \l@section    { 2.5em } { 1.5em }
\ctex_patch_cmd:Nnn \l@subsection { 2.5em } { 1.5em }
\ctex_patch_cmd:Nnn \@wrglossary
  { hdpindex }
  {
    \ifnum \c@HD@hypercount = \z@
      hdpindex
    \else
      hdclindex { \the \c@HD@hypercount }
    \fi
  }
%% 重定义 macrocode 环境的实现,逐行处理
\cs_set_protected_nopar:Npn \xmacro@code
  { \@@_marco_code:w }
\cs_set_protected_nopar:Npn \sxmacro@code
  {
    \fontspec_print_visible_spaces:
    \xmacro@code
  }
\cs_new_protected_nopar:Npn \@@_marco_code:w
  {
    \ifcodeline@index
      \@@_marco_every_par:n { \@@_code_line_no: }
    \else:
      \@@_marco_every_par:n { }
    \fi:
    \exp_args:Nx \@@_make_finish_tag:n { \@currenvir }
    \@@_verbatim_start:w
  }
\cs_new_protected:Npn \@@_marco_every_par:n #1
  {
    \everypar
      {
        \everypar {#1}
        \if@inlabel
          \global \@inlabelfalse
          \@noparlistfalse
          \llap { \box \@labels \hskip \leftskip }
        \fi
        #1
      }
  }
\ctex_patch_cmd:Nnn \macro@code
  { \if@inlabel \leavevmode \fi }
  { \partopsep \z@skip }
\group_begin:
  \int_set:Nn \tex_endlinechar:D { -1 }
  \use:n
    {
      \char_set_catcode_active:n { 32 }
      \tl_const:Nn \c_@@_active_space_tl
    }
    { }
\group_end:
\group_begin:
  \char_set_catcode_active:n { 13 }
  \cs_new_protected:Npx \@@_make_finish_tag:n #1
    {
      \tl_set:Nn \exp_not:N \l_@@_verbatim_finish_tl
        {
          \c_percent_str
          \prg_replicate:nn { 4 }
            { \exp_not:o { \c_@@_active_space_tl } }
          \exp_not:o { \active@escape@char } end
          \c_left_brace_str #1 \c_right_brace_str
          \exp_not:N ^^M
        }
    }
  \cs_new_protected:Npn \@@_verbatim_start:w #1
    {
      \str_if_eq:nnTF {#1} { ^^M }
        { \@@_verbatim_read_line:w }
        { \@@_verbatim_read_line:w #1 }
    }
  \cs_new_protected:Npn \@@_verbatim_read_line:w #1 ^^M
    {
      \tl_set:Nn \l_@@_verbatim_line_tl { #1 ^^M }
      \tl_if_eq:NNTF \l_@@_verbatim_line_tl \l_@@_verbatim_finish_tl
        { \exp_args:Nx \end { \@currenvir } }
        {
          \@@_verbatim_process_line:
          \@@_verbatim_read_line:w
        }
    }
  \tl_const:Nn \c_@@_active_cr_tl { ^^M }
\group_end:
\tl_new:N \l_@@_verbatim_line_tl
\tl_new:N \l_@@_verbatim_finish_tl
\tl_new:N \g_@@_verbatim_verb_stop_tl
\cs_new_protected_nopar:Npn \@@_process_normal_line:
  {
    \str_if_eq_x:nnTF
      { \str_head:N \l_@@_verbatim_line_tl } { \c_percent_str }
      { \@@_check_angle:x { \tl_tail:N \l_@@_verbatim_line_tl } }
      { \@@_output_line: }
  }
\cs_new_protected_nopar:Npn \@@_process_verb_line:
  {
    \tl_if_eq:NNTF \l_@@_verbatim_line_tl \g_@@_verbatim_verb_stop_tl
      {
        \tl_gclear:N \g_@@_verbatim_verb_stop_tl
        \cs_gset_eq:NN \@@_verbatim_process_line: \@@_process_normal_line:
        \@@_output_module:nn
          { \color { verb@guard } }
          { \@@_module_pop:n { \l_@@_verbatim_line_tl } }
      }
      { \tl_use:N \l_@@_verbatim_line_tl }
  }
\cs_new_eq:NN \@@_verbatim_process_line: \@@_process_normal_line:
\DeclareDocumentCommand \CheckModules { }
  { \cs_set_eq:NN \@@_verbatim_process_line: \@@_process_normal_line: }
\DeclareDocumentCommand \DontCheckModules { }
  { \cs_set_eq:NN \@@_verbatim_process_line: \@@_output_line: }
\cs_new_protected:Npn \@@_check_angle:n #1
  {
     \str_if_eq_x:nnTF { \str_head:n {#1} } { < }
      { \@@_check_module:x { \tl_tail:n {#1} } }
      { \@@_output_percent_line: }
  }
\cs_generate_variant:Nn \@@_check_angle:n { x }
\cs_new_protected:Npn \@@_check_module:n #1
  {
    \exp_args:Nx \str_case:nnF { \str_head:n {#1} }
      {
        { * } { \@@_module_star:w }
        { / } { \@@_module_slash:w }
        { @ } { \@@_module_at:w }
        { < } { \@@_module_verb:w }
      }
      { \@@_module_pm:w }
    #1 \q_stop
  }
\cs_generate_variant:Nn \@@_check_module:n { x }
\group_begin:
  \char_set_catcode_active:N \>
  \cs_new_protected:Npn \@@_module_star:w #1 > #2 \q_stop
    {
      \@@_output_module:nn
        { \@@_star_color: }
        { \@@_module_push:n { \@@_module_angle:n {#1} } }
      \@@_output_line:n {#2}
      \@@_star_format:
    }
  \cs_new_protected:Npn \@@_module_slash:w #1 > #2 \q_stop
    {
      \@@_output_module:nn
        { \@@_slash_color: }
        { \@@_module_pop:n { \@@_module_angle:n {#1} } }
      \@@_output_line:n {#2}
      \@@_slash_format:
    }
  \cs_new_protected:Npn \@@_module_at:w @ @ = #1 > #2 \q_stop
    {
      \@@_output_module:nn
        { \color { at@guard } }
        { \@@_module_angle:n { @ @ = #1 } }
      \tl_gset:Nn \g__codedoc_module_name_tl {#1}
      \@@_output_line:n {#2}
    }
  \cs_new_protected:Npn \@@_module_pm:w #1 > #2 \q_stop
    {
      \tex_noindent:D
      \hbox_overlap_left:n
        {
          \@@_output_module:nn
            { \@@_pm_color: }
            { \@@_module_angle:n {#1} }
          \skip_horizontal:n { \leftskip + \smallskipamount }
        }
      \group_begin:
        \@@_pm_format:
        \@@_output_line:n {#2}
      \group_end:
    }
  \cs_new_protected:Npn \@@_module_verb:w #1 \q_stop
    {
      \cs_gset_eq:NN \@@_verbatim_process_line: \@@_process_verb_line:
      \tl_gset:Nx \g_@@_verbatim_verb_stop_tl
        { \c_percent_str \tl_tail:n {#1} }
      \@@_output_module:nn
        { \color { verb@guard } }
        { \@@_module_push:n { \l_@@_verbatim_line_tl } }
    }
\group_end:
\cs_new_protected_nopar:Npn \@@_output_line:
  {
    \tex_noindent:D
    \@@_replace_at_at:N \l_@@_verbatim_line_tl
    \tl_use:N \l_@@_verbatim_line_tl
  }
\cs_new_protected:Npn \@@_replace_at_at:N #1
  {
    \tl_if_empty:NF \g__codedoc_module_name_tl
      {
        \exp_args:NNo \@@_replace_at_at_aux:Nn
          #1 \g__codedoc_module_name_tl
      }
  }
\cs_new_protected:Npn \@@_replace_at_at_aux:Nn #1#2
  {
    \tl_replace_all:Nnn #1 { _ @ @ } { _ _ #2 }
    \tl_replace_all:Nnn #1 {   @ @ } { _ _ #2 }
  }
\cs_new_protected:Npn \@@_output_line:n #1
  {
    \tl_set:Nn \l_@@_verbatim_line_tl {#1}
    \tl_if_eq:NNTF \l_@@_verbatim_line_tl \c_@@_active_cr_tl
      { \tl_use:N \l_@@_verbatim_line_tl }
      {
        \str_if_eq_x:nnTF
          { \str_head:N \l_@@_verbatim_line_tl }
          { \c_percent_str }
          { \@@_output_percent_line: }
          { \@@_output_line: }
      }
  }
\cs_new_protected:Npn \@@_output_percent_line:
  {
    \tex_noindent:D
    \group_begin:
      \color { code@gray }
      \str_if_eq_x:nnTF { \f@shape } { \updefault }
        { \slshape }
        { \upshape }
      \@@_output_line:
    \group_end:
  }
\cs_new_protected_nopar:Npn \@@_module_push:n
  { \exp_args:No \@@_module_push_aux:nn { \int_use:N \c@HD@hypercount } }
\cs_new_protected:Npn \@@_module_push_aux:nn #1
  {
    \seq_gpush:Nn \g_@@_module_dest_seq {#1}
    \hypersetup { hidelinks }
    \exp_args:Nx \hdclindex
      { \zref@extractdefault { HD.#1 } { guard@end } { 1 } } { }
  }
\cs_new_protected_nopar:Npn \@@_module_pop:n
  {
    \seq_gpop:NNTF \g_@@_module_dest_seq \l_@@_tmp_tl
      { \exp_args:No \@@_module_pop_aux:nn { \l_@@_tmp_tl } }
      { \BOOM \use:n }
  }
\cs_new_protected:Npn \@@_module_pop_aux:nn #1
  {
    \zref@labelbylist { HD.#1 } { ctxdoc }
    \hypersetup { hidelinks }
    \hdclindex {#1} { }
  }
\seq_new:N \g_@@_module_dest_seq
\zref@newlist { ctxdoc }
\zref@newprop { guard@end } [ 1 ]
  { \int_eval:n { \c@HD@hypercount - 1 } }
\zref@addprop { ctxdoc } { guard@end }
\cs_new_protected_nopar:Npn \@@_star_format:
  {
    \seq_gpush:No \g_@@_slash_format_seq { \macro@font }
    \seq_gpop:NNF \g_@@_star_format_seq \l_@@_format_tl
      { \@@_pop_format: }
    \@@_select_format:
  }
\cs_new_protected_nopar:Npn \@@_slash_format:
  {
    \seq_gpop:NNTF \g_@@_slash_format_seq \l_@@_format_tl
      {
        \seq_gpush:No \g_@@_star_format_seq { \macro@font }
        \@@_select_format:
      }
      { \BOOM }
  }
\cs_new_protected_nopar:Npn \@@_pm_format:
  {
    \seq_get:NNF \g_@@_star_format_seq \l_@@_format_tl
      {
        \@@_pop_format:
        \seq_gpush:No \g_@@_star_format_seq { \l_@@_format_tl }
      }
    \cs_if_eq:NNF \macro@font \l_@@_format_tl
      { \l_@@_format_tl }
  }
\cs_new_protected_nopar:Npn \@@_pop_format:
  {
    \seq_gpop_left:NN \g_@@_format_seq \l_@@_format_tl
    \seq_gput_right:No \g_@@_format_seq { \l_@@_format_tl }
  }
\cs_new_protected_nopar:Npn \@@_select_format:
  {
    \cs_if_eq:NNF \macro@font \l_@@_format_tl
      {
        \cs_gset_eq:NN \macro@font \l_@@_format_tl
        \macro@font
      }
  }
\tl_new:N \l_@@_format_tl
\seq_new:N \g_@@_format_seq
\seq_new:N \g_@@_star_format_seq
\seq_new:N \g_@@_slash_format_seq
\seq_gput_right:Nn \g_@@_format_seq { \MacroFont }
\seq_gput_right:Nn \g_@@_format_seq { \AltMacroFont }
\cs_set_protected:Npn \MacroFont
  {
    \linespread { 1 }
    \small
    \fontseries { \mddefault }
    \fontshape  { \updefault }
    \ttfamily
    \ctexdocverbaddon
  }
\cs_set_protected:Npn \AltMacroFont
  {
    \linespread { 1 }
    \small
    \fontseries { \mddefault }
    \fontshape  { \sldefault }
    \ttfamily
    \ctexdocverbaddon
  }
\AtBeginDocument
  {
    \tl_gset:Nx \macro@font
      { \seq_item:Nn \g_@@_format_seq { 1 } }
  }
\cs_new_protected:Npn \@@_output_module:nn #1#2
  {
    \tex_noindent:D
    \group_begin:
      #1
      \footnotesize \normalfont \sffamily #2
    \group_end:
  }
\cs_new_protected_nopar:Npn \@@_star_color:
  {
    \seq_gpop:NNTF \g_@@_star_color_seq \current@color
      { \set@color }
      { \@@_select_color: }
    \seq_gpush:No \g_@@_slash_color_seq { \current@color }
  }
\cs_new_protected_nopar:Npn \@@_slash_color:
  {
    \seq_gpop:NNTF \g_@@_slash_color_seq \current@color
      {
        \set@color
        \seq_gpush:No \g_@@_star_color_seq { \current@color }
      }
      { \BOOM }
  }
\cs_new_protected_nopar:Npn \@@_pm_color:
  {
    \seq_get:NNTF \g_@@_star_color_seq \current@color
      { \set@color }
      {
        \@@_select_color:
        \seq_gpush:No \g_@@_star_color_seq { \current@color }
      }
  }
\seq_new:N \g_@@_star_color_seq
\seq_new:N \g_@@_slash_color_seq
\cs_new_protected_nopar:Npn \@@_select_color:
  { \color { guard@series!!+ } }
\definecolorseries { guard@series }
  { cmyk } { last } { blue } { purple }
\resetcolorseries [ 3 ] { guard@series }
\definecolor { verb@guard } { rgb }  { 0.5  , 0.5 , 0 }
\definecolor { at@guard }   { rgb }  { 0.5  , 0   , 0.5 }
\definecolor { code@gray }  { gray } { 0.5 }
\cs_new_protected:Npn \@@_module_angle:n #1
  { \textlangle #1 \textrangle }
\cs_new_protected_nopar:Npn \@@_code_line_no:
  {
    \int_gincr:N \c@CodelineNo
    \hbox_overlap_left:n
      {
        \hbox_to_wd:nn
          { \MacroIndent }
          {
            \HD@target
            \tex_hss:D
            \@@_code_line_no_style:
            \theCodelineNo \enspace
          }
        \tex_kern:D \@totalleftmargin
      }
  }
\tl_set:Nn \theCodelineNo
  { \arabic { CodelineNo } }
\cs_new_protected_nopar:Npn \@@_code_line_no_style:
  { \color { code@gray } \normalfont \sffamily \tiny }
\cs_set_protected:Npn \HD@SetMacroIndent #1
  {
    \group_begin:
      \settowidth \MacroIndent
        {
          \@@_code_line_no_style:
          \prg_replicate:nn { \tl_count:n {#1} } { 0 }
          \enspace
        }
      \dim_gset_eq:NN \MacroIndent \MacroIndent
    \group_end:
  }
\ExplSyntaxOff
\AtBeginDocument{\addtocontents{toc}{\StopSpecialIndexModule}}
\pdfstringdefDisableCommands{%
  \let\path\meta
  \let\opt\@firstofone}
\preto\@thehead{\cslet{MakeUppercase\space}{\@iden}}
\def\orbar{\textup{\textbar}}
\def\defaultval#1{\textbf{\textup{#1}}}
\def\defaultvalaux#1){\defaultval{#1}}
\def\TF{true\orbar false}
\def\TTF{\defaultval{true}\orbar false}
\def\TFF{true\orbar\defaultval{false}}
\protected\def\opt{\texttt}
\def\pdfTeX{\hologo{pdfTeX}}
\def\XeTeX{\hologo{XeTeX}}
\def\XeLaTeX{\hologo{XeLaTeX}}
\def\LuaLaTeX{\hologo{LuaLaTeX}}
\def\pdfLaTeX{\hologo{pdfLaTeX}}
\def\LaTeX{\hologo{LaTeX}}
\def\LaTeXe{\hologo{LaTeX2e}}
\def\LaTeXiii{\hologo{LaTeX3}}
\def\dvipdfmx{DVIPDFM\textit{x}}
\def\TeX{\hologo{TeX}}
\def\ApTeX{Ap\TeX}
\def\ApLaTeX{Ap\LaTeX}
\def\upTeX{up\TeX}
\def\upLaTeX{up\LaTeX}
\def\bashcmd{\texttt}
\def\TeXLive{\TeX\ Live}
\def\MiKTeX{\hologo{MiKTeX}}
\def\BSTACK{\begin{tabular}[t]{@{}l@{}}}
\def\ESTACK{\end{tabular}}
\newenvironment{defaultcapconfig}{%
  \MakePercentComment
  \input{ctex-name-utf8.cfg}%
  \ExplSyntaxOff
  \MakePercentIgnore}{}
\def\ctexkit{\href{https://github.com/CTeX-org/ctex-kit/}{\texttt{ctex-kit}}}
\def\ctexkitrev#1{%
  \href{https://github.com/CTeX-org/ctex-kit/commit/#1}{\texttt{ctex-kit} rev. #1}}
\appto\GlossaryParms{%
  \raggedcolumns
  \let\Hy@writebookmark\HDorg@writebookmark
  \def\@idxitem{\par\hangindent 2em }%
  \def\subitem{\@idxitem\hspace*{1em}}%
  \def\subsubitem{\@idxitem\hspace*{2em}}}
\def\glossaryname{版本历史}
\GlossaryPrologue{\section{\glossaryname}}
\IndexPrologue{%
  \section{\indexname}
  \textit{意大利体的数字表示描述对应索引项的页码;
  带下划线的数字表示定义对应索引项的代码行号;
  罗马字体的数字表示使用对应索引项的代码行号。}}
\def\IndexLayout{%
  \newgeometry{hmargin=15mm,vmargin={25mm,15mm},footskip=7mm}%
  \setlength\IndexMin{.5\textheight}%
  \ctexset{section/numbering=false}%
  \StopSpecialIndexModule}
\EnableCrossrefs
\CodelineIndex
\RecordChanges
%    \end{macrocode}
%
%    \begin{macrocode}
%</ctxdoc>
%    \end{macrocode}
%
% \subsection{\pkg{ctxdocstrip}}
%
%    \begin{macrocode}
%<*docstrip>
\csname ctxdocstriploaded\endcsname
\let\ctxdocstriploaded\endinput
\input l3docstrip %
\edef\ctxresetcatcode{%
  \catcode\number`\$=\the\catcode`\$\relax}
\makeOther\$
\def\plusOption+#1>#2$#3$\endLine{%
  \maybeMsg{<+#1 . >}%
  \Evaluate{#1}%
  \def\ctxCMD{#2}%
  \def\ctxID{$#3$}%
  \def\do##1##2##3{%
    \if1\Expr{##2}%
      \readIDFile
      \StreamPut##1{\inLine}%
    \fi
  }%
  \activefiles
}
\def\readIDFile{%
  \openin\inputcheck=\jobname.id\relax
  \ifeof\inputcheck\else
    \read \inputcheck to \inLine
    \closein\inputcheck
    \ifx\inLine\empty\else
      \let\ctxID\inLine
    \fi
  \fi
  \maybeMsg{^^J\ctxID^^J}%
  \def\inLine{\ctxCMD\ctxID}%
}
\keepsilent
\askforoverwritefalse
\declarepreamble\emptypreamble
\endpreamble
\declarepostamble\emptypostamble
\endpostamble
\ctxresetcatcode
%</docstrip>
%    \end{macrocode}
%
% \end{implementation}
%
\endinput
