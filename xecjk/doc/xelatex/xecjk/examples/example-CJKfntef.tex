%%
%% This is file `example-CJKfntef.tex',
%% generated with the docstrip utility.
%%
%% The original source files were:
%%
%% xeCJK.dtx  (with options: `example-CJKfntef')
%% 
%%  Version 2.3.13 (24-May-2010)
%% 
%%  Copyright (C) Wenchang Sun <sunwch@hotmail.com>
%% 
%%  This file may be distributed and/or modified under the
%%  conditions of the LaTeX Project Public License, either version 1.3
%%  of this license or (at your option) any later version.
%%  The latest version of this license is in
%%    http://www.latex-project.org/lppl.txt
%%  and version 1.3 or later is part of all distributions of LaTeX
%%  version 2005/12/01 or later.
%% 

\documentclass[11pt]{article}
\textheight 220mm
\textwidth 150mm
\oddsidemargin 0pt
\evensidemargin 0pt
\usepackage[slantfont,boldfont]{xeCJK}
\usepackage{xcolor}
\usepackage{CJKfntef}

\begin{document}
\setCJKmainfont{AR PLBaosong2GBK Light}% 设置缺省中文字体
\setCJKmonofont{AR PLBaosong2GBK Light}% 设置缺省中文字体

\baselineskip 16pt
\parindent 2em

 \section{举例}
\begin{verbatim}
标点。
\end{verbatim}

\CJKunderline{汉字}\CJKunderline{加下划线加下划线加下划线加下划线%
加下划线加下划线加下划线加下划线加下划线加下划线加下划线}

\CJKunderwave{波浪线}

\ifcsname CJKunderanyline\endcsname
  \CJKunderanyline{0.5em}{\sixly \kern-.021em\char58 \kern-.021em}{自定义下划线}

  \CJKunderanyline{0.2em}{\kern-.03em\vtop{\kern.2ex\hrule width.2em\kern 0.11em
  \hrule height .1em}\kern-.03em}{自定义下划线}

  \CJKunderanysymbol{0.2em}{$\cdot$}{汉字加点}
\fi
\end{document}
\endinput
%%
%% End of file `example-CJKfntef.tex'.
