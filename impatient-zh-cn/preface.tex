% This is part of the book TeX for the Impatient.
% Copyright (C) 2003 Paul W. Abrahams, Kathryn A. Hargreaves, Karl Berry.
% See file fdl.tex for copying conditions.

\input macros
\frontchapter{前言}

{\tighten
Donald Knuth 開發的計算機排版系統 \TeX, 
提供了幾乎所有排版高質量數學符號和普通文本的功能.
它因為它獨有的易用性, 超羣的分詞處理和美觀的斷行而聞名天下.
因為這些非凡的功能, \TeX\ 已經成為數學, 自然科學, 工程領域領先的排版系統, 
並且被美國數學學會定為標準. 
同時, 它的成對軟件, ^{\Metafont}, 可以用來設計任意的字體, 尤其是數學排版所需要的符號.
\TeX\ 和 \Metafont\ 在科學和工程領域有很廣泛的應用, 
也被移殖到各種不同的計算機架構上.
\TeX 當然不是全能的, 它缺少對於圖片的完整支持, 
同時一些功能, 比如修訂線, 在 \TeX\ 中實現起來也比較麻煩.
但是, 這些缺點和它的優點比較起來, 是微不足道的.
\par}

\thisbook\/ 是為科學工作者, 數學工作者和專業排印者而寫的.
對於這些人而言, \TeX\ 不是一個興趣愛好, 而是一個非常有用的工具.
本書同時也面向那些對 \TeX\ 有很強烈興趣的計算機行業工作者.
我們希望不管是新手們還是熟悉 \TeX\ 的人, 都能從本書獲益.
我們假定我們的讀者羣體己經熟悉了基本的計算機操作, 
並且他們希望用最快的速度得到他們想要的信屲.
因此, 我們的目標是提供簡明的信息, 並且讓讀者能夠方便地獲取它們.

{\tighten This book therefore provides a bright searchlight, a stout
walking-stick, and detailed maps for exploring and using \TeX.  It will
enable you to master \TeX\ at a rapid pace through inquiry and
experiment, but it will not lead you by the hand through the entire
\TeX\ system.  Our approach is to provide you with a handbook for \TeX\
that makes it easy for you to retrieve whatever information you need.
We explain both the full repertoire of \TeX\ commands and the concepts
that underlie them.  You won't have to waste your time plowing through
material that you neither need nor want.  \par}

In the early sections we also provide you with enough orientation so
that you can get started if you haven't used \TeX\ before.  We assume
that you have access to a \TeX\ implementation and that you know how to
use a text editor, but we don't assume much else about your background.
Because this book is organized for ready reference, you'll continue to
find it useful as you become more familiar with \TeX.  If you prefer to
start with a carefully guided tour, we recommend that you first read
Knuth's ^{\texbook} (see \xrefpg{resources} for a citation), passing
over the ``dangerous bend'' sections, and then return to this book for
additional information and for reference as you start to use \TeX.  (The
dangerous bend sections of \texbook\ cover advanced topics.)

The structure of \TeX\ is really quite simple: a \TeX\ input document
consists of ordinary text interspersed with commands that give \TeX\
further instructions on how to typeset your document.  Things like math
formulas contain many such commands, while expository text contains
relatively few of them.

The time-consuming part of learning \TeX\ is learning the commands and
the concepts underlying their descriptions.  Thus we've devoted most of
the book to defining and explaining the commands and the concepts.
We've also provided examples showing \TeX\ typeset output and the
corresponding input, hints on solving common problems, information about
error messages, and so forth.  We've supplied extensive cross-references
by page number and a complete index.

We've arranged the descriptions of the commands so that you can look
them up either by function or alphabetically.  The functional
arrangement is what you need when you know what you want to do but you
don't know what command might do it for you.  The alphabetical arrangement
is what you need when you know the name of a command but you don't know exactly
what it does.

We must caution you that we haven't tried to provide a complete
definition of \TeX.  For that you'll need ^{\texbook}, which is the
original source of information on \TeX.  \texbook\ also contains a lot
of information about the fine points of using \TeX, particularly on the
subject of composing math formulas.  We recommend it highly.

In 1989 Knuth made a major revision to \TeX\ in order to adapt it to
$8$-bit character sets, needed to support typesetting for languages
other than English.  The description of \TeX\ in this book incorporates
that revision (see \xref{newtex}).

{\tighten 你可能正在使用一個專門的 \TeX\ 封裝形式, 
比如 ^{\LaTeX} 或者 ^{\AMSTeX} (見 \xref{resources}).
雖然這些封裝形式是完整的, 你仍可能為更好地控製 \TeX\ 而去使用 \TeX\ 本身獨有的一些功能, 
你可以使用這本書來學習你想要知道的這些功能, 
而把其它你不感興趣的東西扔在一邊.\par}

Two of us (K.A.H. and K.B.) were generously supported by the
University of Massachusetts at Boston during the preparation of this
book.  In particular, Rick Martin kept the machines running, and
Robert~A. Morris and Betty O'Neil made the machines available.  Paul
English of Interleaf helped us produce proofs for a cover design.

We wish to thank the reviewers of our book: Richard Furuta of the
University of Maryland, John Gourlay of Arbortext, Inc., Jill Carter
Knuth, and Richard Rubinstein of the Digital Equipment Corporation. We
took to heart their perceptive and unsparing criticisms of the original
manuscript, and the book has benefitted greatly from their insights.

We are particularly grateful to our editor, Peter Gordon of
Addison-Wesley.  This book was really his idea, and throughout its
development he has been a source of encouragement and valuable
advice.  We thank his assistant at Addison-Wesley, Helen Goldstein, for
her help in so many ways, and Loren Stevens of Addison-Wesley for her
skill and energy in shepherding this book through the production
process.  Were it not for our copyeditor, Janice Byer, a number of small
but irritating errors would have remained in this book.  We appreciate
her sensitivity and taste in correcting what needed to be corrected
while leaving what did not need to be corrected alone.  Finally, we wish
to thank Jim Byrnes of Prometheus Inc. for making this collaboration
possible by introducing us to each other.
\vskip1.5\baselineskip

\line{\it Deerfield, Massachusetts\hfil\rm P.\thinspace W.\thinspace A.}
\line{\it Manomet, Massachusetts\hfil\rm K.\thinspace A.\thinspace H.,
       K.\thinspace B.}

\vskip2\baselineskip

\noindent {\bf Preface to the free edition:} This book was originally
published in 1990 by Addison-Wesley.  In 2003, it was declared out of
print and Addison-Wesley generously reverted all rights to us, the
authors.  We decided to make the book available in source form, under
the GNU Free Documentation License, as our way of supporting the
community which supported the book in the first place.  See the
copyright page for more information on the licensing.

The illustrations which were part of the original book are not included
here.  Some of the fonts have also been changed; now, only
freely-available fonts are used.  We left the cropmarks and galley
information on the pages, to serve as identification.  An old version of
Eplain was used to produce it; see the {\tt eplain.tex} file for
details.

We don't plan to make any further changes or additions to the book
ourselves, except for correction of any outright errors reported to us,
and perhaps inclusion of the illustrations.

Our distribution of the book is at {\tt
ftp://tug.org/tex/impatient}.  You can reach us by email at {\tt
impatient@tug.org}.

\pagebreak
\byebye
